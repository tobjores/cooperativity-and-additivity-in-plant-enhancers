%%% set compatibility level
\pgfplotsset{compat = 1.18}

%%% length, counters, booleans, ...
\newlength{\plotwidth}
\newlength{\plotheight}
\newlength{\plotylabelwidth}
\newlength{\plotxlabelheight}
\newlength{\groupplotsep}

\setlength{\plotylabelwidth}{2.8em}
\setlength{\plotxlabelheight}{2.8em}
\setlength{\groupplotsep}{1mm}

\newcounter{groupplot}
\newcounter{groupcol}
\newcounter{grouprow}
\newcounter{tmpct}

\newif\iflastplot

%%% project-specific settings and macros
\pgfplotsset{
	axis limits/.is choice,
	axis limits/AB80 A/.style = {xmin = 1, xmax = 169, xtick = {1, 20, 40, ..., 170}},
	axis limits/AB80 B/.style = {xmin = 79, xmax = 247, xtick = {79, 100, 120, ..., 250}},
	axis limits/Cab-1 A/.style = {xmin = 1, xmax = 169, xtick = {1, 20, 40, ..., 170}},
	axis limits/Cab-1 B/.style = {xmin = 100, xmax = 268, xtick = {100, 120, ..., 270}},
	axis limits/rbcS-E9 A/.style = {xmin = 1, xmax = 169, xtick = {1, 20, 40, ..., 170}},
	axis limits/rbcS-E9 B/.style = {xmin = 66, xmax = 234, xtick = {66, 80, 100, ..., 240}},
	shade right/.style 2 args = {execute at begin axis/.append = {\fill[#2, fill opacity = .1] (#1, \ymin) rectangle (\xmax, \ymax);}},
	shade left/.style 2 args = {execute at begin axis/.append = {\fill[#2, fill opacity = .1] (\xmin, \ymin) rectangle (#1, \ymax);}},
	shade overlap A/.is choice,
	shade overlap A/AB80/.style = {shade right = {79}{AB80}},
	shade overlap A/Cab-1/.style = {shade right = {100}{Cab-1}},
	shade overlap A/rbcS-E9/.style = {shade right = {66}{rbcS-E9}},
	shade overlap B/.style = {shade left = {169}{#1}}
}

%% draw enhancer fragments
% \enhFrag[<yshift (multiples of \baselineskip)>]{<center>}{<width>}{<color>}{<name>}
\newcommand{\enhFrag}[5][1]{
	\node[anchor = north, font = \bfseries\vphantom{abcdectrl}, inner sep = .15em, yshift = #1 * \baselineskip, fill = #4, fill opacity = .5, text opacity = 1, minimum width = #3 * \pgfplotsunitxlength] (#4 #5) at (#2, \ymin) {#5};
}

%% colorbar for distance between deletions in PEVdouble
	\pgfplotsset{
		colorbar distance/.style = {
			colorbar horizontal,
			colorbar style = {
				anchor = south west,
				at = {(.025, .025)},
				height = .25cm,
				width = .4 * \pgfkeysvalueof{/pgfplots/parent axis height},
				tick pos = upper,
				xticklabel pos = right,
				xticklabel style = {node font = \figtiny, inner ysep = .1em, text depth = 0pt},
				title = distance (bp)\\\vphantom{1},
				title style = {fill = none, draw = none, minimum width = 0, node font = \figsmall, align = center, anchor = south},
				extra x ticks = {8},
			},
		}
	}

%%% general setup
\usepgfplotslibrary{statistics, groupplots, colormaps}

\colorlet{titlecol}{gray}

\pgfplotsset{
	table/col sep = tab,
	every axis/.append style = {
		alias = last plot,
		thin,
		scale only axis = true,
		width = \fourcolumnwidth - \plotylabelwidth,
		height = 3.5cm,
		legend style = {node font = \figsmall, fill = none, draw = none, /tikz/every even column/.append style = {column sep = \groupplotsep}},
		tick label style = {node font = \figsmall},
		tick align = outside,
		label style = {node font = \fignormal, text depth = 0pt, align = center},
		title style = {node font = \fignormal},
		every axis title shift = 0pt,
		max space between ticks = 20,
		ticklabel style = {
			/pgf/number format/fixed,
		},
		every tick/.append style = {black, thin},
		tickwidth = .75mm,
		every x tick label/.append style = {align = center, inner xsep = 0pt},
		every y tick label/.append style = {inner ysep = 0pt},
		scaled ticks = false,
		mark/.default = solido,
		mark size = .75,
		grid = both,
		grid style = {very thin, gray!20},
		tick pos = lower,
		x grids/.is choice,
		x grids/true/.style = {xmajorgrids = true, xminorgrids = true},
		x grids/false/.style = {xmajorgrids = false, xminorgrids = false},
		y grids/.is choice,
		y grids/true/.style = {ymajorgrids = true, yminorgrids = true},
		y grids/false/.style = {ymajorgrids = false, yminorgrids = false},
		boxplot/whisker extend = {\pgfkeysvalueof{/pgfplots/boxplot/box extend}*0.5},
		boxplot/every median/.style = thick,
		title style = {minimum width = \plotwidth, fill = titlecol!20, draw = black, yshift = -.5\pgflinewidth, name = title, text depth = 0pt, align = center},
		legend cell align = left,
		axis background/.style = {fill = white},
		line plot/.style = {thick, line join = round},
		unbounded coords = jump,
		boxplot/draw direction = y,
		xticklabel style = {align = center}
	},
	x decimals/.style = {x tick label style = {/pgf/number format/.cd, fixed, zerofill, precision = #1}},
	y decimals/.style = {y tick label style = {/pgf/number format/.cd, fixed, zerofill, precision = #1}},
	zero line/.style = {execute at begin axis/.append = {\draw[thin, #1] (\xmin, 0) -- (\xmax, 0);}},
	zero line/.default = black,
	show diagonal/.style = {execute at begin axis/.append = {\draw[thin, #1] (-10, -10) -- (10, 10);}},
	show diagonal/.default = {gray, dashed},
	colormap = {blue-red}{color(-1) = (RoyalBlue1) color(0) = (white), color(1) = (IndianRed1)},
	title color/.code = {\colorlet{titlecol}{#1}}
}

%%% extra markers
% round marker without border
\pgfdeclareplotmark{solido}{%
  \pgfpathcircle{\pgfpointorigin}{\pgfplotmarksize + .5\pgflinewidth}%
  \pgfusepathqfill
}

% hexagonal marker without border
\pgfdeclareplotmark{hexagon}
{%
	\pgftransformyscale{\yscale}
	\pgfmathsetlength{\templength}{0.5\pgfplotmarksize / cos(30)}
  \pgfpathmoveto{\pgfqpoint{0pt}{\templength}}
  \pgfpathlineto{\pgfqpointpolar{150}{\templength}}
  \pgfpathlineto{\pgfqpointpolar{210}{\templength}}
  \pgfpathlineto{\pgfqpointpolar{270}{\templength}}
  \pgfpathlineto{\pgfqpointpolar{330}{\templength}}
  \pgfpathlineto{\pgfqpointpolar{30}{\templength}}
  \pgfpathclose
	\pgfusepathqfill
}
\def\yscale{1}

% hexagonal marker with border
\pgfdeclareplotmark{hexagonb}
{%
	\pgftransformyscale{\yscale}
	\pgfmathsetlength{\templength}{0.5\pgfplotmarksize / cos(30)}
  \pgfpathmoveto{\pgfqpoint{0pt}{\templength}}
  \pgfpathlineto{\pgfqpointpolar{150}{\templength}}
  \pgfpathlineto{\pgfqpointpolar{210}{\templength}}
  \pgfpathlineto{\pgfqpointpolar{270}{\templength}}
  \pgfpathlineto{\pgfqpointpolar{330}{\templength}}
  \pgfpathlineto{\pgfqpointpolar{30}{\templength}}
  \pgfpathclose
	\pgfusepathqfillstroke
}


%%% commands for min and max coordinates
\def\xmin{\pgfkeysvalueof{/pgfplots/xmin}}
\def\xmax{\pgfkeysvalueof{/pgfplots/xmax}}
\def\ymin{\pgfkeysvalueof{/pgfplots/ymin}}
\def\ymax{\pgfkeysvalueof{/pgfplots/ymax}}


%%% add jitter to points of a scatter plot
\pgfplotsset{
	x jitter/.style = {
		x filter/.expression = {x + rand * #1}
	},
	y jitter/.style = {
		y filter/.expression = {y + rand * #1}
	},
	x jitter/.default = 0.25,
	y jitter/.default = 0.25
}


%%% set x axis limits and tick labels (depending on sample number)
% use:	x limits = {<number of samples>}
%				x tick table = {<table>}{<column>}
% for half violin plots use:	x tick table half = {<table>}{<column>}
\pgfplotsset{
	x limits/.style = {xmin = 1 - 0.6, xmax = #1 + 0.6},
	y limits/.style = {ymin = 1 - 0.6, ymax = #1 + 0.6},
	x tick table/.code 2 args = {\getrows{#1}},
	x tick table/.append style = {%
		x limits = {\datarows},
		xtick = {1, ..., \datarows},
		xticklabels from table = {#1}{#2}
	},
	x tick table segment/.code 2 args = {\getrows{#1}},
	x tick table segment/.append style = {%
		x limits = {\datarows},
		xtick = {1, ..., \datarows},
		xticklabel = {\pgfmathint{\tick - 1}\pgfplotstablegetelem{\pgfmathresult}{#2}\of{#1}\segment{\pgfplotsretval}}
	},
	x tick table enhancer/.code 2 args = {\getrows{#1}},
	x tick table enhancer/.append style = {%
		x limits = {\datarows},
		xtick = {1, ..., \datarows},
		xticklabel = {\pgfmathint{\tick - 1}\pgfplotstablegetelem{\pgfmathresult}{#2}\of{#1}\enhancer{\pgfplotsretval}}
	},
	x tick table half/.code 2 args = {\gethalfrows{#1}},
	x tick table half/.append style = {%
		x limits = {\halfrows},
		xtick = {1, ..., \halfrows},
		xticklabel = {\pgfmathint{2 * (\tick - 1)}\pgfplotstablegetelem{\pgfmathresult}{#2}\of{#1}\pgfplotsretval}
	},
	x tick table half segment/.code 2 args = {\gethalfrows{#1}},
	x tick table half segment/.append style = {%
		x limits = {\halfrows},
		xtick = {1, ..., \halfrows},
		xticklabel = {\pgfmathint{2 * (\tick - 1)}\pgfplotstablegetelem{\pgfmathresult}{#2}\of{#1}\segment{\pgfplotsretval}}
	},
	x tick table half enhancer/.code 2 args = {\gethalfrows{#1}},
	x tick table half enhancer/.append style = {%
		x limits = {\halfrows},
		xtick = {1, ..., \halfrows},
		xticklabel = {\pgfmathint{2 * (\tick - 1)}\pgfplotstablegetelem{\pgfmathresult}{#2}\of{#1}\enhancer{\pgfplotsretval}}
	},
	y tick table/.code 2 args= {\getrows{#1}},
	y tick table/.append style = {%
		y limits = {\datarows},
		ytick = {1, ..., \datarows},
		yticklabels from table = {#1}{#2}
	},
	y tick table half/.code 2 args= {\gethalfrows{#1}},
	y tick table half/.append style = {%
		y limits = {\halfrows},
		ytick = {1, ..., \halfrows},
		yticklabel = {\pgfmathint{2 * (\tick - 1)}\pgfplotstablegetelem{\pgfmathresult}{#2}\of{#1}\pgfplotsretval}
	},
	xymin/.style = {xmin = #1, ymin = #1},
	xymax/.style = {xmax = #1, ymax = #1},
	xytick/.style = {xtick = {#1}, ytick = {#1}},
}
\makeatletter
	\pgfplotsset{
		set limit from table/.code = {
			\pgfplotstablegetelem{0}{#1}\of{\axes@datatable}%
			\pgfplotsset{#1/.expand once={\pgfplotsretval}}%
		},
		axis limits from table/.code = {
			\pgfplotstable@isloadedtable{#1}%
				{\pgfplotstablecopy{#1}\to\axes@datatable}%
				{\pgfplotstableread{#1}\axes@datatable}%
			\pgfplotstableforeachcolumn{\axes@datatable}\as{\colname}{%
				\pgfplotsset{set limit from table/.expand once = {\colname}}%
			}%
		},
		option from table/.code 2 args = {
			\pgfplotstablegetelem{0}{#1}\of{#2}%
			\pgfplotsset{#1/.expand once={\pgfplotsretval}}%
		},
		option from table row/.code n args = {3}{
			\pgfplotstablegetelem{#3}{#1}\of{#2}%
			\pgfplotsset{#1/.expand once={\pgfplotsretval}}%
		},
	}
\makeatother


%%% horizontal groupplot
% use: \begin{hgroupplot}[<axis options>]{<total width>}{<number of columns>}{<xlabel>} <plot commands> \end{hgroupplot}
\newenvironment{hgroupplot}[4][]{%
	\setcounter{groupcol}{0}\setcounter{grouprow}{1}%
	\pgfmathsetlength{\plotwidth}{(#2 - \plotylabelwidth - ((#3 - 1) * \groupplotsep) ) / #3}%
	\ifthenelse{\isempty{#4}}{%
		\def\xlabel{}%
	}{%
		\def\xlabel{%
			\path let \p1 = (group c1r1.west), \p2 = (group c#3r1.east), \p3 = (xlabel.base) in node[anchor = base, align = center] (plot xlabel) at ($(\x1, \y3)!.5!(\x2, \y3)$) {#4};%
		}%
	}
	\begin{groupplot}[
		width = \plotwidth,
		group style = {
			horizontal sep = \groupplotsep,
			y descriptions at = edge left,
			x descriptions at = edge bottom,
			columns = #3,
			rows = 1,
			every plot/.append style = {no inner x ticks, no inner y ticks, first plot, groupplot labels = {#3}{1}}
		},
		xlabel = \vphantom{#4},
		xlabel style = {name = xlabel},
		#1
	]
}{%
	\end{groupplot}%
	\xlabel%
}


%%% vertical groupplot
% use: \begin{vgroupplot}[<axis options>]{<total width>}{<number of rows>}{<ylabel>} <plot commands> \end{vgroupplot}
\newenvironment{vgroupplot}[4][]{%
	\setcounter{groupcol}{0}\setcounter{grouprow}{1}%
	\pgfmathsetlength{\plotheight}{(#2 - \plotxlabelheight - ((#3 - 1) * \groupplotsep) ) / #3}%
	\ifthenelse{\isempty{#4}}{%
		\def\ylabel{}%
	}{%
		\def\ylabel{%
			\path let \p1 = (group c1r1.north), \p2 = (group c1r#3.south), \p3 = (ylabel.base) in node[anchor = base, rotate = 90, align = center] (plot ylabel) at ($(\x3, \y1)!.5!(\x3, \y2)$) {#4};%
		}%
	}
	\setcounter{groupplot}{0}
	\begin{groupplot}[
		height = \plotheight,
		group style = {
			vertical sep = \groupplotsep,
			y descriptions at = edge left,
			x descriptions at = edge bottom,
			columns = 1,
			rows = #3,
			every plot/.append style = {no inner x ticks, no inner y ticks, first plot, groupplot labels = {1}{#3}}
		},
		ylabel = \vphantom{#4},
		ylabel style = {name = ylabel},
		#1
	]
}{%
	\end{groupplot}%
	\ylabel%
}


%%% horizontal and vertical groupplot
% use: \begin{hvgroupplot}[<axis options>]{<total width>}{<total height>}{<number of columns>}{<number of rows>}{<xlabel>}{<ylabel>} <plot commands> \end{hvgroupplot}
\newenvironment{hvgroupplot}[7][]{%
	\setcounter{groupcol}{0}\setcounter{grouprow}{1}%
	\pgfmathsetlength{\plotwidth}{(#2 - \plotylabelwidth - ((#4 - 1) * \groupplotsep) ) / #4}%
	\pgfmathsetlength{\plotheight}{(#3 - \plotxlabelheight - ((#5 - 1) * \groupplotsep) ) / #5}%
	\ifthenelse{\isempty{#6}}{%
		\def\xlabel{}%
	}{%
		\def\xlabel{%
			\path let \p1 = (group c1r1.west), \p2 = (group c#4r1.east), \p3 = (xlabel.base) in node[anchor = base, align = center] (plot xlabel) at ($(\x1, \y3)!.5!(\x2, \y3)$) {#6};%
		}%
	}
	\ifthenelse{\isempty{#7}}{%
		\def\ylabel{}%
	}{%
		\def\ylabel{%
			\path let \p1 = (group c1r1.north), \p2 = (group c1r#5.south), \p3 = (ylabel.base) in node[anchor = base, rotate = 90, align = center] (plot ylabel) at ($(\x3, \y1)!.5!(\x3, \y2)$) {#7};%
		}%
	}
	\begin{groupplot}[
		width = \plotwidth,
		height = \plotheight,
		group style = {
			horizontal sep = \groupplotsep,
			vertical sep = \groupplotsep,
			y descriptions at = edge left,
			x descriptions at = edge bottom,
			columns = #4,
			rows = #5,
			every plot/.append style = {no inner x ticks, no inner y ticks, first plot, groupplot labels = {#4}{#5}}
		},
		xlabel = \vphantom{#6},
		xlabel style = {name = xlabel},
		ylabel = \vphantom{#7},
		ylabel style = {name = ylabel},
		#1
	]
}{%
	\end{groupplot}%
	\xlabel%
	\ylabel%
}

% required styles
\pgfplotsset{
	last plot/.is if = lastplot,
	last plot = false,
	no inner y ticks/.style = {ymajorticks = false, yminorticks = false},
	no inner x ticks/.style = {xmajorticks = false, xminorticks = false},
	group position/.code = {\pgfkeysalso{first plot/.style = {#1}}},
	groupplot labels/.code 2 args = {%
		\gpgfplotsset{first plot/.style = {}}%
		\pgfkeysalso{last plot = false},%
		\stepcounter{groupcol}%
		\ifnum\value{groupcol}>#1%
			\stepcounter{grouprow}%
			\setcounter{groupcol}{1}
		\fi%
		\ifnum\value{groupcol}=1%
			\pgfkeysalso{ymajorticks = true, yminorticks = true}%
		\fi%
		\ifnum\value{grouprow}=#2%
			\pgfkeysalso{xmajorticks = true, xminorticks = true}%
			\ifnum\value{groupcol}=#1%
				\pgfkeysalso{last plot = true}%
			\fi%
		\fi%
	},
	last plot style/.code = {
		\iflastplot
			\pgfkeysalso{#1}
		\fi
	}
}


%%% legend setup
\pgfplotsset{
	legend image code/.code = {
		\draw [mark repeat = 2,mark phase = 2, #1]
			plot coordinates {
				(0em, 0em)
				(.75em, 0em)
				(1.5em, 0em)
			};
	},
	/pgfplots/ybar legend 1/.style = {
		/pgfplots/legend image code/.code = {
			\draw [##1, fill = gray, /tikz/.cd, bar width = 2.5pt, yshift = -0.2em, bar shift = 0pt]
			plot coordinates {
				(\pgfplotbarwidth, .4em)
			};
			\draw [##1, /tikz/.cd, bar width = 2.5pt, yshift = -0.2em, bar shift = 0pt]
			plot coordinates {
				(0em, 1em)
			};
		},
	},
	/pgfplots/ybar legend 2/.style = {
		/pgfplots/legend image code/.code = {
			\draw [##1, fill = gray, /tikz/.cd, bar width = 2.5pt, yshift = -0.2em, bar shift = 0pt]
			plot coordinates {
				(0em, .4em)
			};
			\draw [##1, /tikz/.cd, bar width = 2.5pt, yshift = -0.2em, bar shift = 0pt]
			plot coordinates {
				(\pgfplotbarwidth, 1em)
			};
		},
	},
}


%%% helper for global pgfplot style definitions
\newcommand\gpgfplotsset[1]{%
	\begingroup%
		\globaldefs=1\relax%
		\pgfqkeys{/pgfplots}{#1}%
	\endgroup%
}


%%% get number of rows/columns in a table -> stored in `\datarows`/`\datacols`
% use: \getrows{<table>}
\newcommand{\getrows}[1]{%
	\pgfplotstablegetrowsof{#1}%
	\pgfmathsetmacro{\datarows}{\pgfplotsretval}%
}

\newcommand{\gethalfrows}[1]{%
	\pgfplotstablegetrowsof{#1}%
	\pgfmathsetmacro{\halfrows}{0.5 * \pgfplotsretval}%
}

\newcommand{\getcols}[1]{%
	\pgfplotstablegetcolsof{#1}%
	\pgfmathsetmacro{\datacols}{\pgfplotsretval}%
}

%%% variants of `\pgfplotsinvokeforeach` that iterates from 1 to the number of rows/columns in the table
% use:	\foreachtablerow{<table or file>}{<command (use `#1` to get current iterator)>}
\makeatletter
	\long\def\foreachtablerow#1#2{%
		\getrows{#1}
		\long\def\pgfplotsinvokeforeach@@##1{#2}%
		\pgfplotsforeachungrouped \pgfplotsinvokeforeach@ in {1, ..., \datarows} {%
			\expandafter\pgfplotsinvokeforeach@@\expandafter{\pgfplotsinvokeforeach@}%
		}%
	}
	
	\long\def\foreachtablecol#1#2{%
		\getcols{#1}
		\long\def\pgfplotsinvokeforeach@@##1{#2}%
		\pgfplotsforeachungrouped \pgfplotsinvokeforeach@ in {1, ..., \datacols} {%
			\expandafter\pgfplotsinvokeforeach@@\expandafter{\pgfplotsinvokeforeach@}%
		}%
	}
\makeatother


%%% command to add significance indicator to plot
% usage: \signif[pgf options]{file}{first sample}{second sample}
% useful options: `raise = ...` to raise (or lower) the indicator (connectors stay at the same height; use yshift to also raise them)
%									`shorten both = ...` to increase (or decrease) distance between plot and indicator (default = 2.5)
%									`signif cutoff = ...` to set the significance cutoff
%									`signif levels = {..., ..., ...}` to set the significance levels
%									`bar only` to not draw connectors (only the horizontal bar)
%									`draw = none` to not draw any line 
\makeatletter
	\newcommand{\signif}{%
		\@ifstar
			\signifStar%
			\signifNoStar%
	}
\makeatother

\newcommand{\signifNoStar}[4][]{%
	\addplot [no marks, shorten both = 2.5, point meta = \thisrow{p.value.#3_#4}, nodes near coords = \printsignif, nodes near coords style = {node font = \figsmaller}, #1] table [x = x.#3_#4, y = y.#3_#4] {#2};
}
\newcommand{\signifStar}[4][]{%
	\addplot [no marks, shorten both = 2.5, point meta = \thisrow{p.value.#3_#4}, nodes near coords = \printsignif*, nodes near coords style = {node font = \figsmall}, #1] table [x = x.#3_#4, y = y.#3_#4] {#2};
}


%% significance for all pairs (especially usefull for half plots or plots with selected pvalues only)
% usage: \signifall(*)[pgf options]{file}
\makeatletter
	\newcommand{\signifall}{%
		\@ifstar
			\signifallStar%
			\signifallNoStar%
	}
\makeatother

\newcommand{\signifallStar}[2][]{%
	\getcols{#2}%
	\pgfmathsetmacro{\samplen}{\datacols / 3}%
	\pgfplotsinvokeforeach{1, ..., \samplen}
	{%
		\pgfmathsetmacro{\xcol}{int(##1 - 1)}%
		\pgfmathsetmacro{\ycol}{int(\xcol + \samplen)}%
		\pgfmathsetmacro{\pcol}{int(\ycol + \samplen)}%
		\addplot [no marks, shorten both = 2.5, point meta = \thisrowno{\pcol}, nodes near coords = \printsignif*, nodes near coords style = {node font = \figsmall}, #1] table [x index = \xcol, y index = \ycol] {#2};%
	}
}

\newcommand{\signifallNoStar}[2][]{%
	\getcols{#2}%
	\pgfmathsetmacro{\samplen}{\datacols / 3}%
	\pgfplotsinvokeforeach{1, ..., \samplen}
	{%
		\pgfmathsetmacro{\xcol}{int(##1 - 1)}%
		\pgfmathsetmacro{\ycol}{int(\xcol + \samplen)}%
		\pgfmathsetmacro{\pcol}{int(\ycol + \samplen)}%
		\addplot [no marks, shorten both = 2.5, point meta = \thisrowno{\pcol}, nodes near coords = \printsignif, nodes near coords style = {node font = \figsmaller}, #1] table [x index = \xcol, y index = \ycol] {#2};%
	}
}


%% significance for simple comparisons (compared to zero distribution)
% use:	\signifallsimple(*)[<pgfplots options>]{<table>}{<column with x coords>}{<column with p-values>}
\makeatletter
	\newcommand{\signifallsimple}{%
		\@ifstar
			\signifallsimpleStar%
			\signifallsimpleNoStar%
	}
\makeatother

\newcommand{\signifallsimpleStar}[4][]{%
	\addplot [no marks, draw = none, point meta = \thisrow{#4}, nodes near coords = \vphantom{A}\printsignif*, nodes near coords style = {node font = \figsmall, anchor = north}, #1] table [x = #3, y expr = \pgfkeysvalueof{/pgfplots/ymax}] {#2};%
}

\newcommand{\signifallsimpleNoStar}[4][]{%
	\addplot [no marks, draw = none, point meta = \thisrow{#4}, nodes near coords = \printsignif, nodes near coords style = {node font = \figsmaller, anchor = north}, #1] table [x = #3, y expr = \pgfkeysvalueof{/pgfplots/ymax}] {#2};%
}


%% use point meta data to print p-values (the * version prints significance levels)
% use as: point meta = \thisrow{<p-value column>}, nodes near coords = \printsignif<*>[<cutoff (one value)/signif. levels (three values separated by ",")>] 
\makeatletter
	\newcommand{\printsignif}{%
		\@ifstar
			\printsignifStar%
			\printsignifNoStar%
	}
\makeatother

\newcommand{\printsignifStar}[1][\signiflevelone,\signifleveltwo,\signiflevelthree]{%
	\pgfmathfloatifflags{\pgfplotspointmeta}{3}{}{\pgfmathfloattosci{\pgfplotspointmeta}\tolevel#1,{\pgfmathresult}}%
}
\newcommand{\printsignifNoStar}[1][\signifcutoff]{%
	\pgfmathfloatifflags{\pgfplotspointmeta}{3}{}{\pgfmathfloattosci{\pgfplotspointmeta}\ifsignif#1,{\pgfmathresult}}%
}

% abbreviation for non-significant p-values and symbol for significance level
\newcommand{\notsignif}{ns}
\newcommand{\issignif}{{\signiffont \symbol{"2217}}}

%% default significance cutoff/levels
% 1 to \signifcutoff or \signiflevelone: ns
% \signiflevelone to \signifleveltwo: *
% \signifleveltwo to \signiflevelthree: **
% \signiflevelthree to 0: ***
\def\signifcutoff{0.01}
\def\signiflevelone{0.01}
\def\signifleveltwo{0.001}
\def\signiflevelthree{0.0001}

% convert p-value to significance level
\def\tolevel#1,#2,#3,#4{%
	\pgfmathparse{#4 <= #3 ? "\issignif\issignif\issignif" : (#4 <= #2 ? "\issignif\issignif" : (#4 <= #1 ? "\issignif" : "\notsignif"))}\pgfmathresult%
}

% print \notsignif for non-significant p-values
\def\ifsignif#1,#2{%
	\pgfmathparse{#2 <= #1 ? "$p = \pgfmathprintnumber{#2}$" : "\notsignif"}\pgfmathresult%
}

% useful styles
\makeatletter
\tikzset{
	bar only/.style = jump mark mid,
	shorten both/.style = {
		shorten < = #1,
		shorten > = #1
	},
	raise/.code = {
		\pgfkeysalso{yshift = #1}
		\pgfmathaddtolength\pgf@shorten@start@additional{-#1}
		\pgfmathaddtolength\pgf@shorten@end@additional{-#1}
	},
	signif cutoff/.code = {
		\def\signifcutoff{#1}
	},
	signif levels/.code args = {#1,#2,#3}{
		\def\signiflevelone{#1}
		\def\signifleveltwo{#2}
		\def\signiflevelthree{#3}
	}
}
\makeatother


%%% style to prepare boxplot from summarized data in a table
% table layout: one row per sample; required columns and content: lw, lower whisker; lq, lower quartile; med, median; uq, upper quartile; uw, upper whisker
% usage: boxplot from table = {table macro}{row number (0-based)} 
\makeatletter
\pgfplotsset{
    boxplot prepared from table/.code={
        \def\tikz@plot@handler{\pgfplotsplothandlerboxplotprepared}%
        \pgfplotsset{
            /pgfplots/boxplot prepared from table/.cd,
            #1,
        }
    },
    /pgfplots/boxplot prepared from table/.cd,
        table/.code={
        	\pgfplotstable@isloadedtable{#1}%
        		{\pgfplotstablecopy{#1}\to\boxplot@datatable}%
        		{\pgfplotstableread{#1}\boxplot@datatable}%
        },
        row/.initial=1,
        make style readable from table/.style={
            #1/.code={
            		\pgfmathint{\pgfkeysvalueof{/pgfplots/boxplot prepared from table/row} - 1}
                \pgfplotstablegetelem{\pgfmathresult}{##1}\of\boxplot@datatable
                \pgfplotsset{boxplot/#1/.expand once={\pgfplotsretval}}
            }
        },
        make style readable from table=lower whisker,
        make style readable from table=upper whisker,
        make style readable from table=lower quartile,
        make style readable from table=upper quartile,
        make style readable from table=median
}
\makeatother

\pgfplotsset{
	boxplot from table/.style 2 args = {%
		boxplot prepared from table = {%
			table = #1,
			row = #2,
			lower whisker = lw,
			lower quartile = lq,
			median = med,
			upper quartile = uq,
			upper whisker = uw
		}
	}
}


%%% draw a combined violin and box plot
% use `violin shade (inverse) = 0` to not shade the violin plots
% use: \violinbox[<pgfplots options for violin and box plots>]{<table for boxplot>}{<file for violin plot>}
\newcommand{\violinbox}[3][]{%
	\foreachtablerow{#2}{%
		\violinplot[save row = {##1}, violin shade, #1]{#3}{##1};%
		\boxplot[save row = {##1}, boxplot/box extend = {\pgfkeysvalueof{/pgfplots/violin extend} * 0.2}, boxplot/whisker extend = 0, #1]{#2}{##1};%
	}%
}

%%% draw a combined half violin and box plot
% use: \halfviolinbox[<pgfplots options for violin and box plots>]{<table for boxplot>}{<file for violin plot>}
\newcommand{\halfviolinbox}[3][]{%
	\foreachtablerow{#2}{%
		\ifthenelse{\isodd{##1}}{%
			\halfviolinplotleft[save row = {##1}, #1]{#3}{##1};%
			\boxplot[save row = {##1}, boxplot/box extend = {\pgfkeysvalueof{/pgfplots/violin extend} * 0.15}, boxplot/whisker extend = 0, boxplot/draw position = floor(0.5 * (##1 + 1)) - 0.0225, boxplot/draw relative anchor = 1, #1]{#2}{##1};%
		}{%
			\halfviolinplotright[save row = {##1}, #1]{#3}{##1};%
			\boxplot[save row = {##1}, boxplot/box extend = {\pgfkeysvalueof{/pgfplots/violin extend} * 0.15}, boxplot/whisker extend = 0, boxplot/draw position = floor(0.5 * (##1 + 1)) + 0.0225, boxplot/draw relative anchor = 0, #1]{#2}{##1};%
		}
	}%
}

%% draw a violin plot
% use: \violinplot[<pgfplots options>]{<file>}{<sample>}
\newcommand{\violinplot}[3][]{%
	\addplot [black, fill = viocol, fill opacity = .5, #1, viostyle] table [x expr = \thisrow{x.#3} * \pgfkeysvalueof{/pgfplots/violin extend} + #3, y = y.#3] {#2} -- cycle;%
}

%% draw a half violin plot
% use: \violinplot[<pgfplots options>]{<file>}{<sample>}
\newcommand{\halfviolinplotleft}[3][]{%
	\addplot [black, fill = viocolleft, fill opacity = .5, #1, viostyle] table [x expr = -0.95 * \thisrow{x.#3} * \pgfkeysvalueof{/pgfplots/violin extend} + floor(0.5 * (#3 + 1)) - 0.0225, y = y.#3] {#2} -- cycle;%
}
\newcommand{\halfviolinplotright}[3][]{%
	\addplot [black, fill = viocolright, fill opacity = .5, #1, viostyle] table [x expr = 0.95 * \thisrow{x.#3} * \pgfkeysvalueof{/pgfplots/violin extend} + floor(0.5 * (#3 + 1)) + 0.0225, y = y.#3] {#2} -- cycle;%
}

%% draw a box plot
% use: \boxplot[<pgfplots options>]{<table>}{<row>}
% does not add outliers! use `\outliers[<pgfplots options>]{<table>}{<sample>}
\newcommand{\boxplot}[3][]{%
	\addplot [black, fill = boxcol, boxplot from table = {#2}{#3}, boxplot/draw position = #3, mark = solido, mark options = black, #1, boxstyle] coordinates {};%
}

\newcommand{\outliers}[3][]{%
	\addplot [black, only marks, mark = solido, #1] table [x expr = #3, y = outlier.#3] {#2};%
}

%%% draw boxplots from table
% use:	\boxplots[<outlier options>]{<boxplot options>}{<base name of data files>}
\newcommand{\boxplots}[3][]{%
	\foreachtablerow{#3_boxplot.tsv}{%
		\boxplot[save row = {##1}, #2]{#3_boxplot.tsv}{##1};%
		\IfFileExists{./#3_outliers.tsv}{%
			\pgfmathint{##1 - 1}%
			\pgfplotstablegetelem{\pgfmathresult}{outliers}\of{#3_boxplot.tsv}%
			\ifnum\pgfplotsretval>0%
				\outliers[#1]{#3_outliers.tsv}{##1};%
			\fi%
		}{}%
	}%
}

%%% draw two related box plots (left and right of the common label)
% use: \halfboxplots[<outlier options>]{<boxplot options>}{<base name of data files>}
\newcommand{\halfboxplots}[3][]{%
	\foreachtablerow{#3_boxplot.tsv}{%
		\ifthenelse{\isodd{##1}}{%
			\def\curshift{-.225}
		}{%
			\def\curshift{.225}
		}%
		\boxplot[save row = {##1}, boxplot/draw position = ##1 + \curshift - floor(.5 * ##1), boxplot/box extend = 0.35, #2]{#3_boxplot.tsv}{##1};%
		\IfFileExists{./#3_outliers.tsv}{%
			\pgfmathint{##1 - 1}%
			\pgfplotstablegetelem{\pgfmathresult}{outliers}\of{#3_boxplot.tsv}%
			\ifnum\pgfplotsretval>0%
				\outliers[x filter/.expression = {x + \curshift - floor(.5 * ##1)}, #1]{#3_outliers.tsv}{##1};%
			\fi%
		}{}%
	}%
}


% default colors for violin and box plots
\colorlet{viocol}{gray}
\colorlet{viocolleft}{viocol!50!black}
\colorlet{viocolright}{viocol}
\colorlet{vioshade}{black}
\colorlet{boxcol}{white}
\colorlet{boxcolleft}{boxcol!50!black}
\colorlet{boxcolright}{boxcol}
\colorlet{boxshade}{black}

% useful styles for violin and box plots
\pgfplotsset{
	save row/.code = {\def\currrow{#1}},
	viostyle/.style = {},
	violin extend/.initial = 0.9,
	violin color/.code = {\colorlet{viocol}{#1}},
	violin colors/.code = {
		\setcounter{tmpct}{1}
		\pgfplotsinvokeforeach{#1}{
			\ifnum\value{tmpct}=\currrow
				\colorlet{viocol}{##1}
			\fi
			\stepcounter{tmpct}
		}
	},
	violin colors from table/.code 2 args = {
 		\pgfmathint{\currrow - 1}
    \pgfplotstablegetelem{\pgfmathresult}{#2}\of{#1}
    \pgfplotsset{violin color/.expand once={\pgfplotsretval}}
	},
	violin color left/.code = {\colorlet{viocolleft}{#1}},
	violin color right/.code = {\colorlet{viocolright}{#1}},
	violin color half/.code = {\colorlet{viocolleft}{#1!50!black}\colorlet{viocolright}{#1}},
	violin colors half/.code = {
		\setcounter{tmpct}{1}
		\pgfmathint{(\currrow + 1)/2}
		\pgfplotsinvokeforeach{#1}{
			\ifnum\value{tmpct}=\pgfmathresult
				\pgfkeysalso{violin color half = ##1}
			\fi
			\stepcounter{tmpct}
		}
	},
		violin colors half from table/.code 2 args = {
	 		\pgfmathint{\currrow - 1}
	    \pgfplotstablegetelem{\pgfmathresult}{#2}\of{#1}
	    \pgfplotsset{violin color half/.expand once={\pgfplotsretval}}
		},
	violin color half inverse/.code = {\colorlet{viocolleft}{#1}\colorlet{viocolright}{#1!50!black}},
	violin shade color/.code = {\colorlet{vioshade}{#1}},
	violin shade inverse/.code = {\pgfmathparse{100 - (#1 * (\currrow - 1) / (\datarows - 1))}\pgfkeysalso{viostyle/.estyle = {fill = viocol!\pgfmathresult!vioshade}}},
	violin shade inverse/.default = 50,
	violin shade/.code = {\pgfmathparse{100 - (#1 * (1 - (\currrow - 1) / (\datarows - 1)))}\pgfkeysalso{viostyle/.estyle = {fill = viocol!\pgfmathresult!vioshade}}},
	violin shade/.default = 50,
	boxstyle/.style = {},
	box color/.code = {\colorlet{boxcol}{#1}},
	box colors/.code = {
		\setcounter{tmpct}{1}
		\pgfplotsinvokeforeach{#1}{
			\ifnum\value{tmpct}=\currrow
				\colorlet{boxcol}{##1}
			\fi
			\stepcounter{tmpct}
		}
	},
	box colors from table/.code 2 args = {
 		\pgfmathint{\currrow - 1}
    \pgfplotstablegetelem{\pgfmathresult}{#2}\of{#1}
    \pgfplotsset{box color/.expand once={\pgfplotsretval}}
	},
	box shade color/.code = {\colorlet{boxshade}{#1}},
	box shade inverse/.code = {\pgfmathparse{100 - (#1 * (\currrow - 1) / (\datarows - 1))}\pgfkeysalso{boxstyle/.estyle = {fill = boxcol!\pgfmathresult!boxshade}}},
	box shade inverse/.default = 50,
	box shade/.code = {\pgfmathparse{100 - (#1 * (1 - (\currrow - 1) / (\datarows - 1)))}\pgfkeysalso{boxstyle/.estyle = {fill = boxcol!\pgfmathresult!boxshade}}},
	box shade/.default = 50,
	box color half/.code = {\colorlet{boxcolleft}{#1!50!black}\colorlet{boxcolright}{#1}},
	box 2 colors half/.code n args = {2}{
		\pgfmathint{floor(0.5 * (\currrow - 1))}
		\ifcase\pgfmathresult
			\pgfkeysalso{box color half = #1}
		\or
			\pgfkeysalso{box color half = #2}
		\fi
	},
	thissamplecolor/.store in = \samplecolor,
	sample color table/.style = {visualization depends on = value \thisrow{#1} \as \samplecolor},
	sample color/.style = {visualization depends on = value #1 \as \samplecolor},
	sample 2 colors/.code n args = {2}{
		\pgfkeysalso{
			scatter/@pre marker code/.append style={
				/utils/exec = {
					\pgfmathint{floor(.5 * \id - 1)}
					\ifcase\pgfmathresult
						\pgfkeysalso{thissamplecolor = #1}
					\or
						\pgfkeysalso{thissamplecolor = #2}
					\fi
				}
			}
		}
	}
}


%%% display sample size
% use:	\samplesize[<pgfplots options>]{<table>}{<column with x coords>}{<column with sample size>}
% to shift every second sample size use:	\samplesize[scatter, no marks, visualization depends on = {mod(x, 2) * .5\baselineskip \as \shift}, scatter/@pre marker code/.append style = {/tikz/yshift = \shift}]{<table>}{<column with x coords>}{<column with sample size>}
\newcommand{\samplesize}[4][]{%
	\addplot [black, draw = none, point meta = \thisrow{#4}, sample size, #1] table [x = #3, y expr = \pgfkeysvalueof{/pgfplots/ymin}] {#2};%
}

%% for half violin/box plots
% use:	\samplesizehalf[<pgfplots options>]{<table>}{<column with x coords>}{<column with sample size>}
\newcommand{\samplesizehalf}[4][]{%
	\addplot [black, draw = none, point meta = \thisrow{#4}, sample size, #1] table [
		x expr = {\thisrow{#3} - floor(0.5 * \thisrow{#3}) + .225 - .45 * mod(\thisrow{#3}, 2)},
		y expr = \pgfkeysvalueof{/pgfplots/ymin}
	] {#2};
}

% \samplesizehalfstacked puts one label above the other and centers them under both half plots
% adjust colors with `sample color = ...`, `sample 2 colors = ...`, `sample color table = ...`
\newcommand{\samplesizehalfstacked}[4][]{%
	\addplot [black, draw = none, point meta = \thisrow{#4}, sample size, visualization depends on = \thisrow{#3} \as \id, nodes near coords = \twosamplesize, #1] table [x expr = \thisrow{#3} * .5, y expr = \pgfkeysvalueof{/pgfplots/ymin}] {#2};
}

% stacked display of two sample sizes
\newcommand{\twosamplesize}{%
	\ifthenelse{\isodd{\id}}{%
		\pgfmathfloattosci{\pgfplotspointmeta}%
		\xdef\lastn{\pgfmathresult}%
	}{%
		\textcolor{\samplecolor!50!black}{\pgfmathprintnumber{\lastn}}\\[-.25\baselineskip]%
		\textcolor{\samplecolor}{\pgfmathprintnumber{\pgfplotspointmeta}}%
	}%
}

\def\samplecolor{viocol}

% default style for sample size nodes
\pgfplotsset{
	sample size/.style = {
		nodes near coords,
		nodes near coords style = {
			node font = \figtiny,
			align = center,
			/pgf/number format/fixed,
			/pgf/number format/1000 sep = {}
		}
	}
}


%%% logo plots
% use:	\logoplot[axis options]{file}
% specify size and position in axis options (width, heigth, at, anchor, ...)
% the width of the letters can be changed with option: letter width = ...
% base colors are changed with option: base colors = {A = ..., C = ..., ...}
% axis styles are changed with opiton: logo axis = none (no axes) / default (normal pgfplot axes) / IC (y axis for information content; no x axis; the default setting) / IC and position (y axis for information content and x axis for position)
\newcommand{\logoplot}[2][]{
	\begin{axis}[
		logo axis,
		#1,
		logo plot
	]
		\addlogoplot{#2};
	\end{axis}
}

\newcommand{\addlogoplot}[1]{%
	\getrows{#1}%
	\pgfplotsinvokeforeach{1, ..., 4}{%
		\addplot[%
			scatter,
			scatter src = explicit symbolic,
			only marks,
			mark size = \pgfkeysvalueof{/pgfplots/width} / \datarows * \pgfkeysvalueof{/pgfplots/letter width},
			visualization depends on = \thisrow{IC_##1} * \pgfkeysvalueof{/pgfplots/height} / \ymax \as\baseht
		] table[x = pos,y = IC_##1, meta = base_##1] {#1};%
	}%
}

\colorlet{baseAcol}{Green4}
\colorlet{baseCcol}{Blue2}
\colorlet{baseGcol}{DarkGoldenrod2}
\colorlet{baseTcol}{Red2}
	
\pgfdeclareplotmark{baseA}{%
	\pgftransformxscale{\pgfplotmarksize}
	\pgftransformyscale{\baseht}
	\pgfpathmoveto{\pgfpoint{-.5}{-1}}%
	\pgfpathlineto{\pgfpoint{-.1}{0}}%
	\pgfpathlineto{\pgfpoint{.1}{0}}%
	\pgfpathlineto{\pgfpoint{.5}{-1}}%
	\pgfpathlineto{\pgfpoint{.3}{-1}}%
	\pgfpathlineto{\pgfpoint{.2}{-.75}}%
	\pgfpathlineto{\pgfpoint{-.2}{-.75}}%
	\pgfpathlineto{\pgfpoint{-.3}{-1}}%
	\pgfpathclose
	\pgfpathmoveto{\pgfpoint{.14}{-.6}}%
	\pgfpathlineto{\pgfpoint{0}{-.25}}%
	\pgfpathlineto{\pgfpoint{-.14}{-.6}}%
	\pgfpathclose
	\pgfusepathqfill
}

\pgfdeclareplotmark{baseC}{%
	\pgftransformxscale{\pgfplotmarksize}
	\pgftransformyscale{\baseht}
	\pgfpathmoveto{\pgfpoint{.5}{-.7}}%
	\pgfpathcurveto{\pgfpoint{.4}{-.9}}{\pgfpoint{.3}{-1}}{\pgfpoint{0}{-1}}%
	\pgfpathcurveto{\pgfpoint{-.3}{-1}}{\pgfpoint{-.5}{-.825}}{\pgfpoint{-.5}{-.5}}%
	\pgfpathcurveto{\pgfpoint{-.5}{-.175}}{\pgfpoint{-.3}{0}}{\pgfpoint{0}{0}}%
	\pgfpathcurveto{\pgfpoint{.3}{0}}{\pgfpoint{.4}{-.1}}{\pgfpoint{.5}{-.3}}%
	\pgfpathlineto{\pgfpoint{.3}{-.38}}%
	\pgfpathcurveto{\pgfpoint{.225}{-.25}}{\pgfpoint{.2}{-.15}}{\pgfpoint{0}{-.15}}%
	\pgfpathcurveto{\pgfpoint{-.2}{-.15}}{\pgfpoint{-.3}{-.3}}{\pgfpoint{-.3}{-.5}}%
	\pgfpathcurveto{\pgfpoint{-.3}{-.7}}{\pgfpoint{-.2}{-.85}}{\pgfpoint{0}{-.85}}%
	\pgfpathcurveto{\pgfpoint{.2}{-.85}}{\pgfpoint{.225}{-.75}}{\pgfpoint{.3}{-.62}}%
	\pgfpathclose
	\pgfusepathqfill
}

\pgfdeclareplotmark{baseG}{%
	\pgftransformxscale{\pgfplotmarksize}
	\pgftransformyscale{\baseht}
	\pgfpathmoveto{\pgfpoint{.5}{-.7}}%
	\pgfpathcurveto{\pgfpoint{.4}{-.9}}{\pgfpoint{.3}{-1}}{\pgfpoint{0}{-1}}%
	\pgfpathcurveto{\pgfpoint{-.3}{-1}}{\pgfpoint{-.5}{-.825}}{\pgfpoint{-.5}{-.5}}%
	\pgfpathcurveto{\pgfpoint{-.5}{-.175}}{\pgfpoint{-.3}{0}}{\pgfpoint{0}{0}}%
	\pgfpathcurveto{\pgfpoint{.3}{0}}{\pgfpoint{.4}{-.1}}{\pgfpoint{.5}{-.3}}%
	\pgfpathlineto{\pgfpoint{.3}{-.38}}%
	\pgfpathcurveto{\pgfpoint{.225}{-.25}}{\pgfpoint{.2}{-.15}}{\pgfpoint{0}{-.15}}%
	\pgfpathcurveto{\pgfpoint{-.2}{-.15}}{\pgfpoint{-.3}{-.3}}{\pgfpoint{-.3}{-.5}}%
	\pgfpathcurveto{\pgfpoint{-.3}{-.7}}{\pgfpoint{-.2}{-.85}}{\pgfpoint{0}{-.85}}%
	\pgfpathcurveto{\pgfpoint{.2}{-.85}}{\pgfpoint{.225}{-.75}}{\pgfpoint{.3}{-.62}}%
	\pgfpathclose
	\pgfpathmoveto{\pgfpoint{.5}{-.55}}%
	\pgfpathlineto{\pgfpoint{.5}{-1}}%
	\pgfpathlineto{\pgfpoint{.3}{-1}}%
	\pgfpathlineto{\pgfpoint{.3}{-.7}}%
	\pgfpathlineto{\pgfpoint{0}{-.7}}%
	\pgfpathlineto{\pgfpoint{0}{-.55}}%
	\pgfpathclose
	\pgfusepathqfill
}

\pgfdeclareplotmark{baseT}{%
	\pgftransformxscale{\pgfplotmarksize}
	\pgftransformyscale{\baseht}
	\pgfpathmoveto{\pgfpoint{-.1}{-1}}%
	\pgfpathlineto{\pgfpoint{-.1}{-.15}}%
	\pgfpathlineto{\pgfpoint{-.5}{-.15}}%
	\pgfpathlineto{\pgfpoint{-.5}{0}}%
	\pgfpathlineto{\pgfpoint{.5}{0}}%
	\pgfpathlineto{\pgfpoint{.5}{-.15}}%
	\pgfpathlineto{\pgfpoint{.1}{-.15}}%
	\pgfpathlineto{\pgfpoint{.1}{-1}}%
	\pgfpathclose
	\pgfusepathqfill
}

\pgfplotsset{
	base colors/.code = {
		\pgfkeys{
			/logo plot/.cd,
			#1
		}
	},
	/logo plot/A/.code = {\colorlet{baseAcol}{#1}},
	/logo plot/C/.code = {\colorlet{baseCcol}{#1}},
	/logo plot/G/.code = {\colorlet{baseGcol}{#1}},
	/logo plot/T/.code = {\colorlet{baseTcol}{#1}},
	letter width/.initial = 1,
	logo y axis/.style = {},
	logo x axis/.style = {},
	show IC/.style = {
		logo y axis/.style = {
			ytick = {0, 1, 2},
			ylabel = IC (bits),
			axis y line = left,
			y axis line style = {line cap = round, -},
			ytick align = outside,
			axis y line shift = \pgfkeysvalueof{/pgfplots/major tick length}	
		}
	},
	show pos/.style = {
		logo x axis/.style = {
			axis x line = bottom,
			x axis line style = {draw = none},
			xtick style = {draw = none},
			xtick align = inside
		}
	},
	show sequence/.style 2 args = {
		execute at begin axis/.append = {
			\addplot [
				only marks,
				mark = text,
				text mark as node = true,
				text mark style = {
					anchor = north,
					inner xsep = 0pt,
					node font = \figsmaller
				},
				text mark = \WTbase,
				visualization depends on = value \thisrow{#2} \as \WTbase,
			] table [x = pos, y expr = 0] {#1};
		},
		xticklabel style = {yshift = -.8\baselineskip}
	},
	logo axis/.is choice,
	logo axis/default/.style = {
		logo y axis/.style = {},
		logo x axis/.style = {}
	},
	logo axis/none/.style = {
		logo y axis/.style = {axis y line = none},
		logo x axis/.style = {axis x line = none}
	},
	logo axis/IC/.style = {
		logo x axis/.style = {axis x line = none},
		show IC
	},
	logo axis/position/.style = {
		logo y axis/.style = {axis y line = none},
		show pos
	},
	logo axis/IC and position/.style = {
		show IC,
		show pos
	},
	logo axis/.default = IC,
	logo plot/.style = {
		logo y axis,
		logo x axis,
		stack plots = y,
		ymin = 0,
		ymax = 2,
		grid = none,
		enlarge x limits = {abs = .5},
		scatter/classes = {
			A={mark = baseA, baseAcol},
			C={mark = baseC, baseCcol},
			G={mark = baseG, baseGcol},
			T={mark = baseT, baseTcol},
			a={mark = baseA, baseAcol!50},
			c={mark = baseC, baseCcol!50},
			g={mark = baseG, baseGcol!50},
			t={mark = baseT, baseTcol!50},
			-={no markers}
		}
	},
	save base width/.style = {after end axis/.code = {\pgfmathsetlength{\global\basewidth}{\pgfplotsunitxlength * 10}}}
}


%%% hexbin plots
% \addplot [hexbin] ...
% IMPORTANT: enlargelimits has to be set explicitly for this type of plot
\def\curplotwidth{\pgfkeysvalueof{/pgfplots/width}}
\def\curplotheight{\pgfkeysvalueof{/pgfplots/height}}
\def\xenlarge{\pgfkeysvalueof{/pgfplots/enlarge x limits}}
\def\yenlarge{\pgfkeysvalueof{/pgfplots/enlarge y limits}}

\pgfplotsset{
	hexbin/.style = {
		scatter,
		scatter/use mapped color = {fill = mapped color},
		scatter src = explicit,
		only marks,
		mark = hexagon,
		mark size = (\curplotwidth) / (1 + 2 * \xenlarge) / #1,
		visualization depends on = (\curplotheight) / (\curplotwidth) * (.5 + \xenlarge) / (.5 + \yenlarge) \as \yscale
	},
	hexbin/.default = 50,
	colorbar hexbin/.style = {
		colorbar style = {
	 		name = colorbar,
			anchor = south east,
			at = {(.975, .05)},
			width = .25cm,
			height = .4 * \pgfkeysvalueof{/pgfplots/parent axis height},
			yticklabel pos = left,
			yticklabel style = {node font = \figtiny, inner xsep = .1em},
			ytick = {0, 2, ..., 15},
			yticklabel = \pgfmathparse{2^\tick}\pgfmathprintnumber{\pgfmathresult},
			ylabel = count,
			ylabel style = {node font = \figsmall}
		}
	},
	colorbar hexbin horizontal/.style = {
		colorbar horizontal,
		colorbar style = {
			anchor = south east,
			at = {(.975, .025)},
			height = .25cm,
			width = .4 * \pgfkeysvalueof{/pgfplots/parent axis height},
			tick pos = upper,
			xticklabel pos = right,
			xticklabel style = {node font = \figtiny, inner ysep = .1em, text depth = 0pt},
			xtick = {0, 2, ..., 15},
			xticklabel = \pgfmathparse{2^\tick}\pgfmathprintnumber{\pgfmathresult},
			title = count\\\vphantom{1},
			title style = {fill = none, draw = none, minimum width = 0, node font = \figsmall, align = center, anchor = south east, xshift = .2 * \pgfkeysvalueof{/pgfplots/parent axis height}}
		}
	}
}


%%% commands to draw heatmaps
% first some useful styles
\pgfplotsset{
	heatmap/.style = {
		enlarge x limits = {abs = .5},
		enlarge y limits = false,
		mesh/ordering = colwise,
		colormap name = blue-red,
		symbolic y coords = {X, A, C, G, T},
		ytick = {X, A, C, G, T},
		yticklabels = {\textDelta, A, C, G, T},
		grid = none,
		axis background/.style = {fill = gray!25},
		ylabel style = {align = center},
	},
	colorbar heatmap/.style = {
		colorbar horizontal,
		colorbar shift/.style = {yshift = -1.5\baselineskip},
		colorbar style = {
			name = colorbar,
			anchor = north east,
			at = {(parent axis.south east)},
			height = .25cm,
			width = .4 * \pgfkeysvalueof{/pgfplots/parent axis width},
			ymin = {[normalized]0},
			ymax = {[normalized]1},
			plot graphics/ymin = {[normalized]0},
			plot graphics/ymax = {[normalized]1},
			xticklabel style = {node font = \figtiny, inner sep = .1em},
			after end axis/.code = {\node[anchor = base east, node font = \figsmall] at (0, 0) {$\log_2$(enhancer strength)};},
		}
	}
}

% draw a heatmap for insertion variants (\heatmapIns[<plot options>]{<file>})
\newcommand{\heatmapIns}[2][]{%
	\addplot[%
		matrix plot,%
		mesh/rows = 4,%
		point meta = explicit,%
		line width = 0pt,%
		#1,%
	] table [x = position, y = varNuc, meta = enrichment] {#2};%
}

% draw a heatmap for substitution and deletion variants (\heatmapSubDel[<plot options>]{<file>})
\newcommand{\heatmapSubDel}[2][]{%
	\addplot[%
		matrix plot,%
		mesh/rows = 5,%
		point meta = explicit,%
		line width = 0pt,%
		#1,%
	] table [x = position, y = varNuc, meta = enrichment] {#2};%
}

% annotate WT positions on heatmap (\heatmapWT[<plot options>]{<file>})
\newcommand{\heatmapWT}[2][]{%
	\addplot[%
		only marks,%
		mark = *,%
		gray,%
		mark size = .5,%
		#1,%
	] table [x = position, y = varNuc] {#2};%
}


%%% add correlation statistics to plot (\stats[<options>]{<base name of file (must end in "_stats.tsv")>})
\newcommand{\stats}[2][]{%
	\addplot [
		only marks,
		mark = text,
		text mark as node = true,
		stats position = north west,
		text mark = {
			$n = \pgfmathprintnumber[fixed, 1000 sep = {{{{,}}}}]{\n}$\\
			$\rho = \pgfmathprintnumber[fixed, fixed zerofill, precision = 2]{\spearman}$\\
			$R^2 = \pgfmathprintnumber[fixed, fixed zerofill, precision = 2]{\rsquare}$
		},
		visualization depends on = \thisrow{n} \as \n,
		visualization depends on = \thisrow{spearman} \as \spearman,
		visualization depends on = \thisrow{rsquare} \as \rsquare,
		#1,
	] table [x expr = 0, y expr = 0] {#2_stats.tsv};
}

\pgfplotsset{
	stats position/.style = {
		text mark style = {
			align = left,
			anchor = #1,
			at = (current axis.#1)
		}
	}
}


%%% commands to draw points & mean plots
% \hmeanline[<spread of the line>]{<tikz options>}{<file>}
\newcommand{\hmeanline}[3][.8]{%
	\addplot[very thick, quiver = {v = 0, u = #1}, #2] table [x expr = \thisrow{id} - 0.5 * #1, y = mean] {#3};%
}
% \hjitter[<tikz options>]{<spread of points>}{<file>}{<sample number>}
\newcommand{\hjitter}[4][]{%
	\addplot[gray, only marks, mark = solido, mark size = 1.25, x jitter = 0.4 * #2, #1] table [x expr = #4, y = points.#4] {#3};%
}
% \hpoints[<tikz options>]{<spread of points>}{<file>}{<sample number>}
\newcommand{\hpoints}[4][]{%
	\addplot[gray, only marks, mark = solido, mark size = 1.25, #1] table [x expr = #4 + #2 / \npoints * (\lineno - \npoints / 2 - .5), y = points.#4] {#3};%
}
% \hmandp(*)[<spread of line and points>]{<tikz options>}{<base name of files>}
% the starred version randomizes the x coordinate of the points; the unstarred version orders the points sequentially
\makeatletter
	\newcommand{\hmandp}{%
		\@ifstar
			\hmandpStar%
			\hmandpNoStar%
	}
\makeatother

\newcommand{\hmandpStar}[3][.8]{%
	\hmeanline[#1]{#2}{#3_mean.tsv}%
	\foreachtablerow{#3_mean.tsv}{%
		\hjitter[#2]{#1}{#3_points.tsv}{##1}%
	}%
}
\newcommand{\hmandpNoStar}[3][.8]{%
	\hmeanline[#1]{#2}{#3_mean.tsv}%
	\getrows{#3_points.tsv}%
	\edef\npoints{\datarows}%
	\foreachtablerow{#3_mean.tsv}{%
		\hpoints[#2]{#1}{#3_points.tsv}{##1}%
	}%
}

% \vmeanline[<spread of the line>]{<tikz options>}{<file>}
\newcommand{\vmeanline}[3][.8]{%
	\addplot[very thick, quiver = {u = 0, v = #1}, #2] table [y expr = \thisrow{id} - 0.5 * #1, x = mean] {#3};%
}
% \vjitter[<tikz options>]{<spread of points>}{<file>}{<sample number>}
\newcommand{\vjitter}[4][]{%
	\addplot[gray, only marks, mark = solido, mark size = 1.25, y jitter = 0.4 * #2, #1] table [y expr = #4, x = points.#4] {#3};%
}
% \vpoints[<tikz options>]{<spread of points>}{<file>}{<sample number>}
\newcommand{\vpoints}[4][]{%
	\addplot[gray, only marks, mark = solido, mark size = 1.25, #1] table [y expr = #4 + #2 / \npoints * (\lineno - \npoints / 2 - .5), x = points.#4] {#3};%
}
% \hmandp(*)[<spread of line and points>]{<tikz options>}{<base name of files>}
% the starred version randomizes the x coordinate of the points; the unstarred version orders the points sequentially
\makeatletter
	\newcommand{\vmandp}{%
		\@ifstar
			\vmandpStar%
			\vmandpNoStar%
	}
\makeatother

\newcommand{\vmandpStar}[3][.8]{%
	\vmeanline[#1]{#2}{#3_mean.tsv}%
	\foreachtablerow{#3_mean.tsv}{%
		\vjitter[#2]{#1}{#3_points.tsv}{##1}%
	}%
}
\newcommand{\vmandpNoStar}[3][.8]{%
	\vmeanline[#1]{#2}{#3_mean.tsv}%
	\getrows{#3_points.tsv}%
	\edef\npoints{\datarows}%
	\foreachtablerow{#3_mean.tsv}{%
		\vpoints[#2]{#1}{#3_points.tsv}{##1}%
	}%
}

%%% draw histograms
% \histogram[<plot options>]{<base name of file>}{<count column>}
% use "count" as <count column> for histograms without groups; use the name of a group for histograms with groups
% `histogram type = ybar|xbar|ybar stacked|xbar stacked` has to be called in the axis options (this is usually included in the axes file)
\newcommand{\histogram}[3][]{
	\addplot[histogram = #3, #1] table {#2_hist.tsv};
}

% histogram styles
\pgfplotsset{
	histogram/.style = {thin, draw = black, fill = gray, fill opacity = 0.5, histogram data = #1},
	histogram/.default = count,
	histogram type/.style = {histogram/.append style = #1, histogram coords = #1},
	histogram coords/.is choice,
	histogram coords/ybar/.style = {
		histogram data/.style = {x filter/.expression = \thisrow{x}, y filter/.expression = \thisrow{##1}}
	},
	histogram coords/xbar/.style = {
		histogram data/.style = {y filter/.expression = \thisrow{y}, x filter/.expression = \thisrow{##1}}
	},
	histogram coords/ybar stacked/.style = {histogram coords = ybar},
	histogram coords/xbar stacked/.style = {histogram coords = xbar}
}