\begin{tikzpicture}

	%%% mutation sensitivity (positional mean) AB80
	\coordinate (AB80) at (0, 0);
	
	\pgfplotstableread{rawData/PEV_ld_TF-scan_AB80.tsv}{\TFscan}
	
	\begin{hgroupplot}[%
		group position = {anchor = above north west, at = {(AB80)}, xshift = \plotylabelwidth},
		axis limits from table = {rawData/PEV_ld_mutSens_AB80_axes.tsv},
		enlarge y limits = {value = .125, upper},
		ytick = {-10, ..., 10},
		ylabel = {$\log_2$(enhancer strength)},
		title color = AB80,
		zero line,
		legend style = {anchor = south west, at = {(0, 0)}},
		legend image post style = {very thick},
		legend plot pos = left,
		legend cell align = left,
	]{\textwidth}{2}{position}
	
		\nextgroupplot[
			title = \enhancer{AB80} enhancer \segment{A} segment,
			axis limits = AB80 A,
			shade overlap A = AB80
		]
			
			\addplot[line plot, dark] table [x = position, y = dark] {rawData/PEV_ld_mutSens_AB80_A_lines.tsv};
			\addplot[line plot, light] table [x = position, y = light] {rawData/PEV_ld_mutSens_AB80_A_lines.tsv};
			
			\foreachtablerow{\TFscan}{
				\pgfmathint{#1 - 1}
				\pgfplotstableforeachcolumn{\TFscan}\as{\col}{
					\pgfplotstablegetelem{\pgfmathresult}{\col}\of{\TFscan}
					\expandafter\edef\csname this\col\endcsname{\pgfplotsretval}
				}
				\def\xshift{0pt}
				\ifcase\pgfmathresult\relax
				\or\or
					\def\xshift{.3\baselineskip}
				\or
					\def\xshift{-.3\baselineskip}
				\or\or\or
					\def\xshift{.125\baselineskip}
				\or
					\def\xshift{-.125\baselineskip}
				\or\or
					\def\xshift{.225\baselineskip}
				\or
					\def\xshift{.025\baselineskip}
				\or\or
					\def\xshift{-.5\baselineskip}
				\or\or
					\def\xshift{-.5\baselineskip}
				\fi
				\edef\thisTF{
					\noexpand\fill[black, fill opacity = .1] (\thisstart, \noexpand\ymax) rectangle (\thisstop, \noexpand\ymin);
					\noexpand\node[anchor = east, rotate = 90, node font = \noexpand\figsmaller, yshift = \xshift] at (\thiscenter, \noexpand\ymax) {\thisid};
				}
				\thisTF
			}
						
			\legend{dark, light}
			
			
		\nextgroupplot[
			title = \enhancer{AB80} enhancer \segment{B} segment,
			axis limits = AB80 B,
			shade overlap B = AB80
		]
		
			\addplot[line plot, dark] table [x = position, y = dark] {rawData/PEV_ld_mutSens_AB80_B_lines.tsv};
			\addplot[line plot, light] table [x = position, y = light] {rawData/PEV_ld_mutSens_AB80_B_lines.tsv};
			
			\foreachtablerow{\TFscan}{
				\pgfmathint{#1 - 1}
				\pgfplotstableforeachcolumn{\TFscan}\as{\col}{
					\pgfplotstablegetelem{\pgfmathresult}{\col}\of{\TFscan}
					\expandafter\edef\csname this\col\endcsname{\pgfplotsretval}
				}
				\def\xshift{0pt}
				\ifcase\pgfmathresult\relax
				\or\or\or\or\or\or
					\def\xshift{.125\baselineskip}
				\or
					\def\xshift{-.125\baselineskip}
				\or\or
					\def\xshift{.225\baselineskip}
				\or
					\def\xshift{.025\baselineskip}
				\or\or
					\def\xshift{-.5\baselineskip}
				\or\or\or\or
					\def\xshift{.025\baselineskip}
				\or
					\def\xshift{-.025\baselineskip}
				\fi
				\edef\thisTF{
					\noexpand\fill[black, fill opacity = .1] (\thisstart, \noexpand\ymax) rectangle (\thisstop, \noexpand\ymin);
					\noexpand\node[anchor = east, rotate = 90, node font = \noexpand\figsmaller, yshift = \xshift] at (\thiscenter, \noexpand\ymax) {\thisid};
				}
				\thisTF
			}
		
	\end{hgroupplot}
	
	
	%%% mutation sensitivity (positional mean) Cab-1
	\coordinate[yshift = -.6\columnsep] (Cab-1) at (AB80 |- xlabel.south);
	
	\pgfplotstableread{rawData/PEV_ld_TF-scan_Cab-1.tsv}{\TFscan}
	
	\begin{hgroupplot}[%
		group position = {anchor = above north west, at = {(Cab-1)}, xshift = \plotylabelwidth},
		axis limits from table = {rawData/PEV_ld_mutSens_Cab-1_axes.tsv},
		enlarge y limits = {value = .125, upper},
		ytick = {-10, ..., 10},
		ylabel = {$\log_2$(enhancer strength)},
		title color = Cab-1,
		zero line,
		legend style = {anchor = south west, at = {(0, 0)}},
		legend image post style = {very thick},
		legend plot pos = left,
		legend cell align = left,
	]{\textwidth}{2}{position}
	
		\nextgroupplot[
			title = \enhancer{Cab-1} enhancer \segment{A} segment,
			axis limits = Cab-1 A,
			shade overlap A = Cab-1
		]
		
			\addplot[line plot, dark] table [x = position, y = dark] {rawData/PEV_ld_mutSens_Cab-1_A_lines.tsv};
			\addplot[line plot, light] table [x = position, y = light] {rawData/PEV_ld_mutSens_Cab-1_A_lines.tsv};
			
			\foreachtablerow{\TFscan}{
				\pgfmathint{#1 - 1}
				\pgfplotstableforeachcolumn{\TFscan}\as{\col}{
					\pgfplotstablegetelem{\pgfmathresult}{\col}\of{\TFscan}
					\expandafter\edef\csname this\col\endcsname{\pgfplotsretval}
				}
				\def\xshift{0pt}
				\ifcase\pgfmathresult\relax
				\or\or
					\def\xshift{.075\baselineskip}
				\or
					\def\xshift{-.075\baselineskip}
				\or\or
					\def\xshift{-.5\baselineskip}
				\fi
				\edef\thisTF{
					\noexpand\fill[black, fill opacity = .1] (\thisstart, \noexpand\ymax) rectangle (\thisstop, \noexpand\ymin);
					\noexpand\node[anchor = east, rotate = 90, node font = \noexpand\figsmaller, yshift = \xshift] at (\thiscenter, \noexpand\ymax) {\thisid};
				}
				\thisTF
			}
						
			\legend{dark, light}
			
			
		\nextgroupplot[
			title = \enhancer{Cab-1} enhancer \segment{B} segment,
			axis limits = Cab-1 B,
			shade overlap B = Cab-1
		]
		
			\addplot[line plot, dark] table [x = position, y = dark] {rawData/PEV_ld_mutSens_Cab-1_B_lines.tsv};
			\addplot[line plot, light] table [x = position, y = light] {rawData/PEV_ld_mutSens_Cab-1_B_lines.tsv};
			
			\foreachtablerow{\TFscan}{
				\pgfmathint{#1 - 1}
				\pgfplotstableforeachcolumn{\TFscan}\as{\col}{
					\pgfplotstablegetelem{\pgfmathresult}{\col}\of{\TFscan}
					\expandafter\edef\csname this\col\endcsname{\pgfplotsretval}
				}
				\edef\thisTF{
					\noexpand\fill[black, fill opacity = .1] (\thisstart, \noexpand\ymax) rectangle (\thisstop, \noexpand\ymin);
					\noexpand\node[anchor = east, rotate = 90, node font = \noexpand\figsmaller] at (\thiscenter, \noexpand\ymax) {\thisid};
				}
				\thisTF
			}
		
	\end{hgroupplot}
	
	
	%%% mutation sensitivity (positional mean) rbcS-E9
	\coordinate[yshift = -.6\columnsep] (rbcS-E9) at (AB80 |- xlabel.south);
	
	\pgfplotstableread{rawData/PEV_ld_TF-scan_rbcS-E9.tsv}{\TFscan}
	
	\begin{hgroupplot}[%
		group position = {anchor = above north west, at = {(rbcS-E9)}, xshift = \plotylabelwidth},
		axis limits from table = {rawData/PEV_ld_mutSens_rbcS-E9_axes.tsv},
		enlarge y limits = {value = .125, upper},
		ytick = {-10, ..., 10},
		ylabel = {$\log_2$(enhancer strength)},
		title color = rbcS-E9,
		zero line,
		legend style = {anchor = south west, at = {(0, 0)}},
		legend image post style = {very thick},
		legend plot pos = left,
		legend cell align = left,
	]{\textwidth}{2}{position}
	
		\nextgroupplot[
			title = \enhancer{rbcS-E9} enhancer \segment{A} segment,
			axis limits = rbcS-E9 A,
			shade overlap A = rbcS-E9
		]
			
			\addplot[line plot, dark] table [x = position, y = dark] {rawData/PEV_ld_mutSens_rbcS-E9_A_lines.tsv};
			\addplot[line plot, light] table [x = position, y = light] {rawData/PEV_ld_mutSens_rbcS-E9_A_lines.tsv};
			
			\foreachtablerow{\TFscan}{
				\pgfmathint{#1 - 1}
				\pgfplotstableforeachcolumn{\TFscan}\as{\col}{
					\pgfplotstablegetelem{\pgfmathresult}{\col}\of{\TFscan}
					\expandafter\edef\csname this\col\endcsname{\pgfplotsretval}
				}
				\def\xshift{0pt}
				\ifcase\pgfmathresult\relax
				\or
					\def\xshift{.6\baselineskip}
				\or\or
					\def\xshift{-.4\baselineskip}
				\or\or\or\or\or\or
					\def\xshift{.1\baselineskip}
				\or
					\def\xshift{-.25\baselineskip}
				\fi
				\edef\thisTF{
					\noexpand\fill[black, fill opacity = .1] (\thisstart, \noexpand\ymax) rectangle (\thisstop, \noexpand\ymin);
					\noexpand\node[anchor = east, rotate = 90, node font = \noexpand\figsmaller, yshift = \xshift] at (\thiscenter, \noexpand\ymax) {\thisid};
				}
				\thisTF
			}
						
			\legend{dark, light}
			
			
		\nextgroupplot[
			title = \enhancer{rbcS-E9} enhancer \segment{B} segment,
			axis limits = rbcS-E9 B,
			shade overlap B = rbcS-E9
		]
			
			\addplot[line plot, dark] table [x = position, y = dark] {rawData/PEV_ld_mutSens_rbcS-E9_B_lines.tsv};
			\addplot[line plot, light] table [x = position, y = light] {rawData/PEV_ld_mutSens_rbcS-E9_B_lines.tsv};
			
			\foreachtablerow{\TFscan}{
				\pgfmathint{#1 - 1}
				\pgfplotstableforeachcolumn{\TFscan}\as{\col}{
					\pgfplotstablegetelem{\pgfmathresult}{\col}\of{\TFscan}
					\expandafter\edef\csname this\col\endcsname{\pgfplotsretval}
				}
				\def\xshift{0pt}
				\ifcase\pgfmathresult\relax
				\or\or\or\or\or\or\or\or\or
					\def\xshift{.1\baselineskip}
				\or
					\def\xshift{-.25\baselineskip}
				\fi
				\edef\thisTF{
					\noexpand\fill[black, fill opacity = .1] (\thisstart, \noexpand\ymax) rectangle (\thisstop, \noexpand\ymin);
					\noexpand\node[anchor = east, rotate = 90, node font = \noexpand\figsmaller, yshift = \xshift] at (\thiscenter, \noexpand\ymax) {\thisid};
				}
				\thisTF
			}
		
	\end{hgroupplot}
	
	
	%%% transcription factor legend
	\coordinate[yshift = -.6\columnsep] (TF legend) at (AB80 |- xlabel.south);
	
	\node[anchor = north west, xshift = \columnsep, inner sep = 0pt] at (TF legend) {%
		\fontspec{Arial}%
		\rowcolors{2}{white}{gray!20}%
		\setlength{\tabcolsep}{3pt}%
		\pgfplotstabletypeset[
			begin table = {\begin{tabularx}{\textwidth - \columnsep}},
			end table = {\end{tabularx}},
			every head row/.style = {before row = \toprule, after row = \midrule},
			every last row/.style = {after row = \bottomrule},
			extra space/.style = {column type = X},
			string type,
			columns = {ID, family, consensus, ID, family, consensus},
			display columns/0/.style = {select equal part entry of = {0}{2}},
			display columns/1/.style = {select equal part entry of = {0}{2}},
			display columns/2/.style = {select equal part entry of = {0}{2}, extra space},
			display columns/3/.style = {select equal part entry of = {1}{2}},
			display columns/4/.style = {select equal part entry of = {1}{2}},
			display columns/5/.style = {select equal part entry of = {1}{2}},
			assign column name/.style = {/pgfplots/table/column name={\textbf{#1}}},
			column type = l,
		]{rawData/PEV_ld_TF-scan_legend.tsv}%
	};
	
	
	%%% subfigure labels
	\subfiglabel{AB80}
	\subfiglabel{Cab-1}
	\subfiglabel{rbcS-E9}
	\subfiglabel{TF legend}

\end{tikzpicture}