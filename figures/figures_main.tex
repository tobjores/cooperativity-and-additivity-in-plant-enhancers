% !TeX spellcheck = en_US
\documentclass[10pt]{article}

\usepackage{geometry}
\usepackage[no-math]{fontspec}
\usepackage{mathfont}
\usepackage{amsmath}
\usepackage[autostyle]{csquotes}
\usepackage{graphicx}
\usepackage[x11names, table]{xcolor}
\usepackage{tikz}
\usepackage{pgfplots}
\usepackage{pgfplotstable}
\usepackage{xifthen}
\usepackage{caption}
\usepackage{newfloat}
\usepackage[hidelinks]{hyperref}
\usepackage{booktabs}
\usepackage{tabularx}
\usepackage{xstring}
\usepackage{polyglossia}
\setdefaultlanguage[variant = american]{english}
\usepackage[noabbrev]{cleveref}

%%% font definitions
\defaultfontfeatures{Ligatures = TeX}
\setsansfont{Arial}
\renewcommand{\familydefault}{\sfdefault}
\mathfont{Arial}
\newfontfamily{\signiffont}{Arial Unicode MS}
\newfontfamily{\symbolfont}{Symbol}

%%% font sizes
\newcommand{\fignormal}{\footnotesize}
\newcommand{\figsmall}{\scriptsize}
\newcommand{\figsmaller}{\fontsize{6}{7}\selectfont}
\newcommand{\figtiny}{\tiny}
\newcommand{\figlarge}{\small}
\newcommand{\fighuge}{\normalsize}

%%% page layout
\geometry{letterpaper, textwidth = 165mm, textheight = 220mm, marginratio = 1:1}
\renewcommand{\textfraction}{0}
\makeatletter
	\setlength{\@fptop}{0pt}
\makeatother
    
%%% header and footer
\pagestyle{empty}

% define page style for supplemental data (normal font & no page number)
% set header text with `\markboth{<left head>}{<right head>}` or `\markright{<head>}`
\makeatletter
	\def\ps@SuppData{%
		\let\@oddfoot\@empty\let\@evenfoot\@empty
		\def\@evenhead{\hfil\leftmark}%
		\def\@oddhead{\rightmark\hfil}%
		\let\@mkboth\@gobbletwo
		\let\sectionmark\@gobble
		\let\subsectionmark\@gobble
	}
\makeatother


%%% no indentation
\setlength\parindent{0pt}

%%% useful lengths
\setlength{\columnsep}{4mm}
\newlength{\onecolumnwidth}
\setlength{\onecolumnwidth}{\textwidth}
\newlength{\twocolumnwidth}
\setlength{\twocolumnwidth}{0.5\onecolumnwidth}
\addtolength{\twocolumnwidth}{-.5\columnsep}
\newlength{\threecolumnwidth}
\setlength{\threecolumnwidth}{.333\onecolumnwidth}
\addtolength{\threecolumnwidth}{-.666\columnsep}
\newlength{\twothirdcolumnwidth}
\setlength{\twothirdcolumnwidth}{2\threecolumnwidth}
\addtolength{\twothirdcolumnwidth}{\columnsep}
\newlength{\fourcolumnwidth}
\setlength{\fourcolumnwidth}{.25\onecolumnwidth}
\addtolength{\fourcolumnwidth}{-.75\columnsep}
\newlength{\threequartercolumnwidth}
\setlength{\threequartercolumnwidth}{3\fourcolumnwidth}
\addtolength{\threequartercolumnwidth}{2\columnsep}

\newlength{\templength}
\newlength{\basewidth}

\newlength{\xdistance}
\newlength{\ydistance}

%%% useful commands
\renewcommand{\textprime}{\char"2032{}}
\newcommand{\textalpha}{\char"03B1{}}
\renewcommand{\textbeta}{\char"03B2{}}
\newcommand{\textDelta}{\char"0394{}}
\newcommand{\textlambda}{\char"03BB{}}
\renewcommand{\textle}{\char"2264{}}
\renewcommand{\textge}{\char"2265{}}


\newcommand{\light}{light}
\newcommand{\dark}{dark}
\newcommand{\lightResp}{light-responsiveness}

\newrobustcmd{\enhancer}[1]{%
	\IfStrEqCase{#1}{%
		{none}{none}%
		{35S}{35S}%
		{AB80}{\textit{AB80}}%
		{Cab-1}{\textit{Cab\nobreakdash-1}}%
		{rbcS-E9}{\textit{rbcS\nobreakdash-E9}}% removing this percentage sign causes an error
	}[\textcolor{Firebrick1}{\bfseries unknown enhancer}]%
}
\newrobustcmd{\segment}[1]{%
	\IfStrEqCase{#1}{%
		{FL}{FL}%
		{A}{5\textprime{}}%
		{B}{3\textprime{}}% removing this percentage sign causes an error
	}[\textcolor{Firebrick1}{\bfseries unknown segment}]%
}
\newrobustcmd{\timepoint}[1]{%
	\pgfmathint{#1 + 8}%
	ZT \pgfmathresult%
}


\makeatletter
	\newcommand{\distance}[2]{
		\path (#1);
		\pgfgetlastxy{\xa}{\ya} 
		\path (#2);
		\pgfgetlastxy{\xb}{\yb}   
		\pgfpointdiff{\pgfpoint{\xa}{\ya}}{\pgfpoint{\xb}{\yb}}%
		\setlength{\xdistance}{\pgf@x}
		\setlength{\ydistance}{\pgf@y}
	}
	
	\def\convertto#1#2{\strip@pt\dimexpr #2*65536/\number\dimexpr 1#1\relax\,#1}
\makeatother

%%% tikz setup
\pgfmathsetseed{928}

%%% general
\usetikzlibrary{calc, positioning, arrows.meta, arrows, bending, external, backgrounds, topaths, shapes.arrows, shapes.geometric, shapes.symbols, decorations.markings}

\tikzexternalize[prefix = extFigures/, only named = true]

%%% layers
\pgfdeclarelayer{background}
\pgfdeclarelayer{semi foreground}
\pgfdeclarelayer{foreground}
\pgfsetlayers{background, main, semi foreground, foreground}


%%% colors
\colorlet{35S promoter}{DarkSeaGreen3}
\colorlet{35S enhancer}{DodgerBlue1}

\colorlet{none}{black}
\colorlet{35S}{35S enhancer}
\colorlet{AB80}{OrangeRed1}
\colorlet{Cab-1}{Gold3}
\colorlet{rbcS-E9}{DarkOrchid2}

\colorlet{light}{DarkGoldenrod1}
\colorlet{dark}{black}

\definecolor{t0}{HTML}{F8766D}
\definecolor{t6}{HTML}{A3A500}
\definecolor{t12}{HTML}{00BF7D}
\definecolor{t18}{HTML}{00B0F6}
\definecolor{t24}{HTML}{E76BF3}


%%% save and use jpeg for externalization
\tikzset{
	% Defines a custom style which generates BOTH, .pdf and .jpg export
	% but prefers the .jpg on inclusion.
	jpeg export/.style = {
		external/system call/.add = {}{%
			&& pdftocairo -jpeg -r #1 -singlefile "\image.pdf" "\image" %
		},
		/pgf/images/external info,
		/pgf/images/include external/.code = {%
			\includegraphics[width=\pgfexternalwidth,height=\pgfexternalheight]{##1.jpg}%
		},
	},
	jpeg export/.default = 300
}


%%%%%%%%%%%%%%%%%%%%%%%%
%%% drawing commands %%%
%%%%%%%%%%%%%%%%%%%%%%%%

%%% Plant STARR-seq constructs
% use:	\PSconstruct(*)[<barcode color>]{<coordinate (no parentheses)>}{<enhancer color>}
\makeatletter
	\newcommand{\PSconstruct}{%
		\@ifstar
			\PSconstructStar%
			\PSconstructNoStar%
	}
\makeatother

\newcommand{\PSconstructNoStar}[3][Orchid1]{
	\draw[dashed, thick] (#2) -- ++(.5, 0) coordinate (c1) {};
	
	\draw[line width = .25cm, -Triangle Cap, #3]  (c1) ++(.1, 0) -- ++(.9, 0) coordinate (c2);
	
	\draw[line width = .2cm, 35S promoter] (c2) ++(.1, 0) -- ++(.4, 0) coordinate (c3);
	\draw[-{Stealth[round]}, thick] (c3) ++(.5\pgflinewidth, 0) |- ++(.4,.3);
	
	\node[draw = black, thin, anchor = west] (ORF) at ($(c3) + (.4, 0)$) {~~GFP~};
	
	\node[draw = black, thin, anchor = west, text depth = 0pt] (polyA) at ($(ORF.east) + (.2, 0)$) {pA};
	
	\coordinate[xshift = .1cm] (c4) at (polyA.east);
	
	\begin{pgfonlayer} {background}
		\fill[#1] (ORF.north west) ++(.15, 0) rectangle ($(ORF.south west) + (.05, 0)$);
		\draw[thick] (c1) -- (ORF.west) (ORF.east) -- (polyA.west) (polyA.east) -- (c4);
	\end{pgfonlayer}

	\draw[dashed, thick] (c4) -- ++(.5, 0) coordinate (construct end);
}

\newcommand{\PSconstructStar}[3][Orchid1]{
	\draw[dashed, thick] (#2) ++(.3, 0) -- ++(.5, 0) coordinate (c1) {};
	
	\draw[line width = .25cm, -Triangle Cap, #3!50]  (c1) ++(.1, 0) -- ++(.6, 0) coordinate (c2);
	
	\draw[line width = .2cm, 35S promoter] (c2) ++(.1, 0) -- ++(.4, 0) coordinate (c3);
	\draw[-{Stealth[round]}, thick] (c3) ++(.5\pgflinewidth, 0) |- ++(.4,.3);
	
	\node[draw = black, thin, anchor = west] (ORF) at ($(c3) + (.4, 0)$) {~~GFP~};
	
	\node[draw = black, thin, anchor = west, text depth = 0pt] (polyA) at ($(ORF.east) + (.2, 0)$) {pA};
	
	\coordinate[xshift = .1cm] (c4) at (polyA.east);
	
	\begin{pgfonlayer} {background}
		\fill[#1] (ORF.north west) ++(.15, 0) rectangle ($(ORF.south west) + (.05, 0)$);
		\draw[thick] (c1) -- (ORF.west) (ORF.east) -- (polyA.west) (polyA.east) -- (c4);
	\end{pgfonlayer}

	\draw[dashed, thick] (c4) -- ++(.5, 0) coordinate (construct end);
}


%% Plant STARR-seq constructs with enhancer fragment combinations
% use:	\PEFconstruct(*)[<barcode color>]{<coordinate (no parentheses)>}{<fragment specification>}
% the non-starred version generates constructs with random fragments (the <fragment specification> argument should be omitted);
% the starred version generates constructs with the specified fragments
% format for <fragment specification>: <enhancer 1>/<fragment 1>, <enhancer 2>/<fragment 2>, ...
\makeatletter
	\newcommand{\PEFconstruct}{%
		\@ifstar
			\PEFconstructStar%
			\PEFconstructNoStar%
	}
\makeatother

\pgfplotstableread{../data/sequence_files/enhancer-fragments.tsv}{\enhFragsTable}

\newcommand{\PEFconstructNoStar}[2][Orchid1]{
	\draw[dashed, thick] (#2) -- ++(.5, 0) coordinate (c1) {};
	
	\draw[line width = .15cm, gray] (c1) ++(.1, 0) -- ++(.05, 0) coordinate (enh);
	
	\pgfmathint{random(3)}
	
	\foreach \n in {1, ..., \pgfmathresult}{
		\pgfmathint{random(0, 20)}
		\pgfplotstablegetelem{\pgfmathresult}{enhancer}\of\enhFragsTable
		\edef\thiscolor{\pgfplotsretval}
		\pgfplotstablegetelem{\pgfmathresult}{length}\of\enhFragsTable
		\edef\thislength{\pgfplotsretval}
		\pgfplotstablegetelem{\pgfmathresult}{fragment}\of\enhFragsTable
		\edef\thisfrag{\pgfplotsretval}
		\draw[line width = .25cm, \thiscolor!50] (enh) -- ++(.015 * \thislength, 0) node[pos = .5, text = black, node font = \figsmall\bfseries] {\vphantom{abcdectrl}\thisfrag} coordinate (enh end);
		\draw[line width = .15cm, gray] (enh end) -- ++(.05, 0) coordinate (enh);
	}
	
	\draw[line width = .2cm, 35S promoter] (enh) ++(.1, 0) coordinate (c2) -- ++(.4, 0) coordinate (c3);
	\draw[-{Stealth[round]}, thick] (c3) ++(.5\pgflinewidth, 0) |- ++(.4,.3);
	
	\node[draw = black, thin, anchor = west] (ORF) at ($(c3) + (.4, 0)$) {~~GFP~};
	
	\node[draw = black, thin, anchor = west, text depth = 0pt] (polyA) at ($(ORF.east) + (.2, 0)$) {pA};
	
	\coordinate[xshift = .1cm] (c4) at (polyA.east);
	
	\begin{pgfonlayer} {background}
		\fill[#1] (ORF.north west) ++(.15, 0) rectangle ($(ORF.south west) + (.05, 0)$);
		\draw[thick] (c1) -- (ORF.west) (ORF.east) -- (polyA.west) (polyA.east) -- (c4);
	\end{pgfonlayer}

	\draw[dashed, thick] (c4) -- ++(.5, 0) coordinate (construct end);
}

\newcommand{\PEFconstructStar}[3][Orchid1]{
	\draw[dashed, thick] (#2) -- ++(.5, 0) coordinate (c1) {};
	
	\draw[line width = .15cm, gray] (c1) ++(.1, 0) -- ++(.1, 0) coordinate (enh);
	
	\pgfmathint{random(3)}
	
	\foreach \thiscolor/\thisfrag in {#3}{
		
		\pgfplotstableforeachcolumnelement{enhancer}\of\enhFragsTable\as\thisenh{
			\ifx\thiscolor\thisenh\relax
				\pgfplotstablegetelem{\pgfplotstablerow}{fragment}\of\enhFragsTable
				\ifx\thisfrag\pgfplotsretval\relax
					\pgfplotstablegetelem{\pgfplotstablerow}{length}\of\enhFragsTable
					\edef\thislength{\pgfplotsretval}
				\fi
			\fi
		}
		
		\draw[line width = .25cm, \thiscolor!50] (enh) -- ++(.015 * \thislength, 0) node[pos = .5, text = black, node font = \figsmall\bfseries] {\vphantom{abcdectrl}\thisfrag} coordinate (enh end);
		\draw[line width = .15cm, gray] (enh end) -- ++(.1, 0) coordinate (enh);
	}
	
	\draw[line width = .2cm, 35S promoter] (enh) ++(.1, 0) coordinate (c2) -- ++(.4, 0) coordinate (c3);
	\draw[-{Stealth[round]}, thick] (c3) ++(.5\pgflinewidth, 0) |- ++(.4,.3);
	
	\node[draw = black, thin, anchor = west] (ORF) at ($(c3) + (.4, 0)$) {~~GFP~};
	
	\node[draw = black, thin, anchor = west, text depth = 0pt] (polyA) at ($(ORF.east) + (.2, 0)$) {pA};
	
	\coordinate[xshift = .1cm] (c4) at (polyA.east);
	
	\begin{pgfonlayer} {background}
		\fill[#1] (ORF.north west) ++(.15, 0) rectangle ($(ORF.south west) + (.05, 0)$);
		\draw[thick] (c1) -- (ORF.west) (ORF.east) -- (polyA.west) (polyA.east) -- (c4);
	\end{pgfonlayer}

	\draw[dashed, thick] (c4) -- ++(.5, 0) coordinate (construct end);
}

%%% infiltrated leaf
% use:	\leaf[<scale factor>]{<coordinate (no parentheses)>}
% default size: 4cm x 4cm
\newcommand{\leaf}[2][1]{%
	\coordinate (leaf center) at (#2);
	\begin{scope}[scale = #1, rotate = -45]
		\useasboundingbox[rotate = -45] (leaf center) +(-2, -2) rectangle +(2, 2);
		
		\path (leaf center) +(-.275, 2.225) coordinate (tip);
	
		\path (tip) ++(-1.512, -2.036) coordinate (c1) ++(1.495, -1.482) coordinate (c2) ++(0.115, -0.077) coordinate (c3) ++(0.091, -0.648) coordinate (c4) ++(0.197, 0) coordinate (c5) ++(-0.088, 0.624) coordinate (c6) ++(0.144, 0.107) coordinate (c7) ++(1.58, 1.514) coordinate (c8);
		
		\filldraw[draw = Chartreuse4, fill = Chartreuse3] (c5)
			.. controls +(-0.052, 0.182) and ($(c5) + (-0.088, 0.354)$) .. (c6)
			.. controls +(0, 0.076) and ($(c6) + (0.028, 0.149)$) .. (c7)
			.. controls +(0.508, -0.162) and ($(c7) + (1.957, -0.273)$) .. (c8)
			.. controls +(-0.424, 2.007) and ($(c8) + (-1.637, 1.595)$) .. (tip)
			.. controls +(-0.255, -0.410) and ($(tip) + (-1.446, -0.729)$) .. (c1) 
			.. controls +(-0.064, -1.265) and ($(c1) + (0.359, -1.908)$) .. (c2)
			.. controls +(0.057, 0.022) and ($(c2) + (0.115, 0.012)$) .. (c3)
			.. controls +(0, -0.203) and ($(c3) + (0.025, -0.458)$) .. (c4)
		;
		
		\coordinate (stem) at ($(c4)!.5!(c5)$);
			
		\path (stem) ++(-0.079, 0.78) coordinate (c9) ++(0.014, 0.148) coordinate (c10) ++(0.017, 0.548) coordinate (c11) ++(0, 0.15) coordinate (c12) ++(-0.036, 0.942) coordinate (c13) ++(-0.015, 0.21) coordinate (c14) ++(-0.13, 1.023) coordinate (c15);
		
		\draw[Chartreuse2, thin, line cap = round] (c15) .. controls +(0.071, -0.366) and ($(c15) + (0.106, -0.707)$) .. (c14);
		\draw[Chartreuse2] (c14) .. controls +(0.005, -0.072)  and ($(c14) + (0.01, -0.142)$) .. (c13);
		\draw[Chartreuse2] (c13) .. controls +(0.02, -0.315) and ($(c13) + (0.034, -0.632)$) .. (c12);
		\draw[Chartreuse2, thick] (c12) .. controls +(0, -0.051) and ($(c12) + (0, -0.101)$) .. (c11);
		\draw[Chartreuse2, thick] (c11) .. controls +(-0.002, -0.185) and ($(c11) + (-0.006, -0.378)$) .. (c10);
		\draw[Chartreuse2, very thick] (c10) .. controls +(-0.003, -0.05) and ($(c10) + (-0.009, -0.099)$) .. (c9);
		\draw[Chartreuse2, very thick, line cap = round] (c9) .. controls +(-0.025, -0.249) and ($(c9) + (0.013, -0.542)$) .. (stem);
		
		\draw[Chartreuse2, line cap = round, thick] (stem) ++(1.462, 2.097) coordinate (c16) .. controls +(-0.2, -1.143) and ($(c16) + (-1.525, -0.933)$) .. ++(-1.527, -1.169) ++(1.064, 1.937) coordinate (c17);
		\draw[Chartreuse2, line cap = round] (c17) .. controls +(-0.312, -1.002) and ($(c17) + (-1.047, -0.998)$) .. ++(-1.047, -1.239) ++(0.4807, 1.856) coordinate (c18);
		\draw[Chartreuse2, line cap = round, thin] (c18) .. controls +(-0.242, -0.562) and ($(c18) + (-0.544, -0.553)$) .. ++(-0.531, -0.704) ++(-0.563, 0.624) coordinate (c19);
		\draw[Chartreuse2, line cap = round, thin] (c19) .. controls +(0.082, -0.422) and ($(c19) + (0.564, -0.624)$) .. ++(0.578, -0.834) ++(-0.975, 0.286) coordinate (c20);
		\draw[Chartreuse2, line cap = round] (c20) .. controls +(0.019, -0.88) and ($(c20) + (1.006, -1.128)$) .. ++(1.011, -1.378) ++(-1.487, 0.529) coordinate (c21);
		\draw[Chartreuse2, line cap = round, thick] (c21) .. controls +(0.034, -1.107) and ($(c21) + (1.484, -0.824)$) .. ++(1.456, -1.225);
		
		
		\fill[Chartreuse4, opacity = .5, decoration={random steps, segment length = #1 * 0.075cm, amplitude = #1 * 0.075cm}, decorate] (tip)  ++(-0.2, -2) circle (0.8);
		
	\end{scope}
}


%%% pea pod
% use:	\pea[<scale factor>]{<coordinate (no parentheses)>}
% default size: 1cm x 1cm
\newcommand{\pea}[2][1]{
	\begin{scope}[shift = {($(#2) + (-.5 * #1, .2834 * #1)$)}, scale = .547 * #1]
		\path[draw = black, fill = Chartreuse4, line join = round, very thin] (1.64133, -1.06530) .. controls (1.50484, -.73549) and (1.13383, -.88057) .. (.65445, -.36124) .. controls (.48949, -.18253) and (.39816, -.12003) .. (.14922, .16505) .. controls (.14008, .24513) and (.17965, .31314) .. (.19047, .33626) .. controls (.20129, .35935) and (.16246, .36982) .. (.15027, .35318) .. controls (.13809, .33654) and (.10078, .30825) .. (.09266, .19607) .. controls (.03051, .07232) and (-.04875, -.05624) .. (.05148, -.13432) .. controls (-.00844, -.24253) and (.04625, -.50120) .. (.12754, -.52249) .. controls (.06695, -.71124) and (.30070, -.93651) .. (.37570, -.91444) .. controls (.40480, -1.08382) and (.61578, -1.30385) .. (.76152, -1.20616) .. controls (.84117, -1.37628) and (1.11062, -1.37358) .. (1.13133, -1.34143) .. controls (1.23582, -1.50467) and (1.57453, -1.25952) .. (1.81715, -1.21698) .. controls (1.82456, -1.21636) and (1.86125, -1.17568) .. cycle;
		
		\path[draw = black, line cap = round, line join = round, very thin] (1.69902,-1.14710) .. controls (1.65699, -1.09284) and (1.47121, -1.10471) .. (1.21238, -1.05214) .. controls (.97582, -1.00405) and (.74457, -.85714) .. (.61602, -.73799) .. controls (.33668, -.47901) and (.15789, -.21350) .. (.12238, .06275) .. controls (.10871, .16404) and (.11816, .15466) .. (.17121, .08775) .. controls (.39191, -.19065) and (.66066, -.46753) .. (.88508, -.64229) .. controls (1.13840, -.83956) and (1.47039, -.84413) .. (1.59676, -1.05991);
		
		\path[draw = black, fill = Chartreuse2, very thin] (.12691, .03286) .. controls (.14422, -.07124) and (.18172, -.17382) .. (.23719, -.27510) .. controls (.35104, -.27747) and (.44564, -.16291) .. (.40254, -.04237) .. controls (.36174, .07192) and (.21888, .11122) .. (.12691, .03286) -- cycle;
		
		\path[draw = black, fill = Chartreuse2, very thin] (.21379, -.23007) .. controls (.26512, -.33425) and (.33562, -.43702) .. (.42285, -.53866) .. controls (.53415, -.51721) and (.63570, -.39831) .. (.58934, -.26858) .. controls (.53325, -.11172) and (.31238, -.09543) .. (.21379, -.23007) -- cycle;
		
		\path[draw = black, fill = Chartreuse2, very thin] (.39465, -.50503) .. controls (.42419, -.54108) and (.56460, -.70820) .. (.71621, -.82022) .. controls (.88242, -.73974) and (.89641, -.51663) .. (.75585, -.41383) .. controls (.63335, -.32424) and (.45921, -.37011) .. (.39465, -.50503) -- cycle;
		
		\path[draw = black, fill = Chartreuse2, very thin] (.60930, -.73167) .. controls (.64250, -.76323) and (.78804, -.89433) .. (1.00750, -.98721) .. controls (1.12937, -.87645) and (1.10961, -.68930) .. (.97408, -.60984) .. controls (.84271, -.53283) and (.66773, -.59239) .. (.60930, -.73167) -- cycle;
		
		\path[draw = black, fill = Chartreuse2, very thin] (.86129, -.91456) .. controls (1.06227, -1.02913) and (1.20794, -1.05346) .. (1.30973, -1.06967) .. controls (1.35277, -.96335) and (1.30637, -.83909) .. (1.20156, -.77995) .. controls (1.06515, -.70314) and (.89673, -.77134) .. (.86129, -.91456) -- cycle;
		
		\path[draw = black, fill = Chartreuse2, very thin] (1.20465, -1.05046) .. controls (1.38008, -1.08718) and (1.53193, -1.09354) .. (1.61988, -1.11167) .. controls (1.60806, -1.04567) and (1.56715, -.97220) .. (1.49355, -.93198) .. controls (1.37850, -.86910) and (1.24451, -.92324) .. (1.20465, -1.05046) -- cycle;
		
		\path[draw = black, line cap = round, line join = round, very thin] (1.09938, -1.19639) .. controls (1.10091, -1.55882) and (1.43296, -1.33730) .. (1.71002, -1.24524)(.74102, -1.06917) .. controls (.71219, -1.37608) and (1.10613, -1.38057) .. (1.13129, -1.34143)(.41137, -.77948) .. controls (.26355, -.91108) and (.56926, -1.33507) .. (.76152, -1.20616)(.18801, -.42905) .. controls (-.01895, -.62565) and (.28770, -.94034) .. (.37570, -.91452)(.08121, -.10030) .. controls (-.02230, -.17061) and (.03559, -.49842) .. (.12754, -.52249);
	\end{scope}
}


%%% wheat ears
% use:	\wheat[<scale factor>]{<coordinate (no parentheses)>}
% default size: 1cm x 1cm
\newcommand{\wheat}[2][1]{
	\begin{scope}[shift = {($(#2) + (-.468486 * #1, .23621 * #1)$)}, scale = .2648 * #1]
		\path[draw = DarkGoldenrod4!50!black, fill = Goldenrod2, line join = round, very thin] (3.5026, -2.3344) .. controls (2.5296, -2.4228) and (2.2226, -1.1610) .. (2.0018, -0.2584) .. controls (2.1170, -1.2034) and (2.5064, -2.5332) .. (3.5381, -2.4212) -- cycle;
		
		\path[draw = DarkGoldenrod4!50!black, fill = Gold2, line join = round, very thin]
			(1.9563, -1.9194) .. controls (2.1187, -2.1561) and (2.4252, -2.2377) .. (2.6805, -2.1236) .. controls (2.5187, -1.8875) and (2.2125, -1.8050) .. cycle
			(1.7596, -1.4931) .. controls (1.8824, -1.7529) and (2.1725, -1.8816) .. (2.4422, -1.8096) .. controls (2.3194, -1.5498) and (2.0293, -1.4211) .. cycle
			(1.6360, -1.0376) .. controls (1.7179, -1.3125) and (1.9844, -1.4843) .. (2.2628, -1.4540) .. controls (2.1810, -1.1797) and (1.9148, -1.0073) .. cycle
			(1.5570, -0.5496) .. controls (1.6058, -0.8322) and (1.8506, -1.0344) .. (2.1302, -1.0370) .. controls (2.0814, -0.7544) and (1.8368, -0.5523) .. cycle
			(1.5493, -0.0767) .. controls (1.5497, -0.3633) and (1.7564, -0.6039) .. (2.0320, -0.6539) .. controls (2.0315, -0.3673) and (1.8251, -0.1267) .. cycle
			(1.6028, 0.4259) .. controls (1.5589, 0.1425) and (1.7262, -0.1272) .. (1.9904, -0.2190) .. controls (2.0344, 0.0647) and (1.8668, 0.3342) .. cycle
			(2.6720, -1.1997) .. controls (2.4919, -1.4232) and (2.5016, -1.7402) .. (2.6844, -1.9520) .. controls (2.8643, -1.7287) and (2.8548, -1.4117) .. cycle
			(2.5583, -0.7359) .. controls (2.3358, -0.9170) and (2.2792, -1.2291) .. (2.4139, -1.4744) .. controls (2.6363, -1.2935) and (2.6929, -0.9813) .. cycle
			(2.3974, -0.2748) .. controls (2.1723, -0.4523) and (2.1108, -0.7636) .. (2.2419, -1.0110) .. controls (2.4667, -0.8336) and (2.5285, -0.5227) .. cycle
			(2.3066, 0.1865) .. controls (2.0681, 0.0273) and (1.9825, -0.2782) .. (2.0936, -0.5351) .. controls (2.3321, -0.3758) and (2.4176, -0.0702) .. cycle
			(2.3027, 0.6979) .. controls (2.0530, 0.5571) and (1.9446, 0.2588) .. (2.0361, -0.0057) .. controls (2.2858, 0.1353) and (2.3942, 0.4335) .. cycle
			(1.8764, 0.9961) .. controls (1.7075, 0.7642) and (1.7323, 0.4481) .. (1.9252, 0.2451) .. controls (2.0941, 0.4770) and (2.0692, 0.7932) .. cycle;
		
		\path[draw = DarkGoldenrod4!50!black, fill = Goldenrod2, line join = round, very thin] (3.2539, -2.1854) .. controls (2.5102, -2.8189) and (1.5314, -1.9655) .. (0.8303, -1.3557) .. controls (1.4696, -2.0611) and (2.5549, -2.9225) .. (3.3329, -2.2358) -- cycle;
		
		\path[draw = DarkGoldenrod4!50!black, fill = Gold2, line join = round, very thin]
			(0.0033, -0.4040) .. controls (-0.0007, -0.6903) and (0.2019, -0.9343) .. (0.4771, -0.9887) .. controls (0.4811, -0.7012) and (0.2772, -0.4580) .. cycle
			(1.7515, -2.7386) .. controls (2.0201, -2.8380) and (2.3179, -2.7284) .. (2.4608, -2.4875) .. controls (2.1918, -2.3878) and (1.8940, -2.4981) .. cycle
			(1.3449, -2.5040) .. controls (1.5944, -2.6447) and (1.9054, -2.5838) .. (2.0850, -2.3685) .. controls (1.8353, -2.2275) and (1.5240, -2.2889) .. cycle
			(0.9812, -2.2033) .. controls (1.2069, -2.3807) and (1.5241, -2.3667) .. (1.7332, -2.1817) .. controls (1.5078, -2.0045) and (1.1911, -2.0178) .. cycle
			(0.6350, -1.8502) .. controls (0.8381, -2.0529) and (1.1545, -2.0767) .. (1.3844, -1.9176) .. controls (1.1815, -1.7150) and (0.8651, -1.6909) .. cycle
			(0.3558, -1.4684) .. controls (0.5214, -1.7019) and (0.8289, -1.7797) .. (1.0832, -1.6613) .. controls (0.9171, -1.4271) and (0.6092, -1.3505) .. cycle
			(0.1096, -1.0271) .. controls (0.2372, -1.2837) and (0.5292, -1.4076) .. (0.7983, -1.3300) .. controls (0.6706, -1.0735) and (0.3787, -0.9495) .. cycle
			(1.9208, -1.7379) .. controls (1.9026, -2.0233) and (2.0927, -2.2772) .. (2.3650, -2.3452) .. controls (2.3831, -2.0590) and (2.1923, -1.8055) .. cycle
			(1.5604, -1.4246) .. controls (1.4828, -1.7015) and (1.6176, -1.9888) .. (1.8685, -2.1111) .. controls (1.9456, -1.8351) and (1.8122, -1.5477) .. cycle
			(1.1628, -1.1409) .. controls (1.0815, -1.4157) and (1.2109, -1.7053) .. (1.4606, -1.8319) .. controls (1.5420, -1.5576) and (1.4130, -1.2676) .. cycle
			(0.8224, -0.8165) .. controls (0.7197, -1.0841) and (0.8259, -1.3830) .. (1.0649, -1.5288) .. controls (1.1677, -1.2614) and (1.0616, -0.9624) .. cycle
			(0.5243, -0.4010) .. controls (0.4017, -0.6597) and (0.4847, -0.9658) .. (0.7124, -1.1296) .. controls (0.8354, -0.8698) and (0.7510, -0.5638) .. cycle;
	\end{scope}
}

%%% Arabidopsis
% use:	\arabidopsis[<scale factor>]{<coordinate (no parentheses)>}
% default size: 2cm x 2cm
\newcommand{\arabidopsis}[2][1]{
	\begin{scope}[shift = {($(#2) + (-.901 * #1, .929 * #1)$)}, scale = .696 * #1]
		\fill[white] (1.110976, -.7490396) .. controls (1.059388, -.7421765) and (1.06516, -.6995117) .. (1.090398, -.7049062) .. controls (1.094612, -.6646722) and (1.140522, -.6731404) .. (1.139822, -.6900455) .. controls (1.161616, -.6755084) and (1.179327, -.7053998) .. (1.137018, -.7378842) -- cycle;
		\fill[white] (1.516505, -.2226796) .. controls (1.532256, -.1671951) and (1.569578, -.1908112) .. (1.558339, -.2140427) .. controls (1.585799, -.220477) and (1.583996, -.2564119) .. (1.56507, -.2595871) .. controls (1.574659, -.2856564) and (1.518744, -.2955267) .. (1.521593, -.258699) -- cycle;
		\fill[white] (1.497206, 0.03760345) .. controls (1.512569, .1178339) and (1.570914, 0.09925079) .. (1.561346, 0.06547394) .. controls (1.597664, 0.08219116) and (1.6078, 0.04001312) .. (1.58766, 0.02977478) .. controls (1.634637, 0.01435913) and (1.561761, -0.04352264) .. (1.518353, 0.00133087) -- cycle;
		\fill[white] (1.436943, 0.01749329) .. controls (1.40656, 0.00903809) and (1.392565, 0.04726685) .. (1.418312, 0.04548859) .. controls (1.406998, 0.07590271) and (1.443938, 0.09044678) .. (1.455336, 0.05982086) .. controls (1.454122, 0.08362213) and (1.500802, 0.06858459) .. (1.465237, 0.03310974) -- cycle;
		\draw[black, line width = #1 * 0.1pt, line cap = round] (1.110976, -.7490396) .. controls (1.059388, -.7421765) and (1.06516, -.6995117) .. (1.090398, -.7049062) (1.129261, -.7287445) .. controls (1.125241, -.7125012) and (1.127808, -.6989293) .. (1.139822, -.6900455) .. controls (1.161616, -.6755084) and (1.179327, -.7053998) .. (1.137018, -.7378844) (1.111843, -.7332196) .. controls (1.100274, -.726212) and (1.089398, -.7152493) .. (1.090398, -.7049063) .. controls (1.094612, -.6646724) and (1.140522, -.6731406) .. (1.139822, -.6900455);
		\draw[black, line width = #1 * 0.1pt, line cap = round] (1.418312, 0.04548859) .. controls (1.406998, 0.07590271) and (1.443938, 0.09044678) .. (1.455336, 0.05982086) .. controls (1.454122, 0.08362213) and (1.500802, 0.06858459) .. (1.465237, 0.03310974) (1.436943, 0.01749329) .. controls (1.40656, 0.00903809) and (1.392565, 0.04726685) .. (1.418312, 0.04548859) .. controls (1.424161, 0.04474426) and (1.427797, 0.03803925) .. (1.430103, 0.03448517) (1.418312, 0.04548859) .. controls (1.406998, 0.07590271) and (1.443938, 0.09044678) .. (1.455336, 0.05982086) .. controls (1.457083, 0.05492767) and (1.457702, 0.04872772) .. (1.457172, 0.04489319);
		\draw[black, line width = #1 * 0.1pt, line cap = round] (1.548882, 0.01051208) .. controls (1.561193, 0.02702515) and (1.576233, 0.0333728) .. (1.58766, 0.02977478) .. controls (1.634637, 0.01435913) and (1.561761, -0.04352264) .. (1.518353, 0.00133087) (1.561346, 0.06547394) .. controls (1.597664, 0.08219116) and (1.6078, 0.04001312) .. (1.58766, 0.02977478) (1.497206, 0.03760345) .. controls (1.512569, .1178339) and (1.570914, 0.09925079) .. (1.561346, 0.06547394) .. controls (1.55772, 0.05133911) and (1.550039, 0.03949677) .. (1.526977, 0.03753906);
		\draw[black, line width = #1 * 0.1pt, line cap = round] (1.516505, -.2226796) .. controls (1.532256, -.1671951) and (1.569578, -.1908112) .. (1.558339, -.2140427) .. controls (1.585799, -.220477) and (1.583996, -.2564119) .. (1.56507, -.2595871) .. controls (1.574659, -.2856564) and (1.518744, -.2955267) .. (1.521593, -.258699) (1.558339, -.2140427) .. controls (1.585799, -.220477) and (1.583996, -.2564119) .. (1.56507, -.2595871) .. controls (1.555155, -.2609512) and (1.548795, -.2583966) .. (1.541025, -.2528477) (1.516505, -.2226796) .. controls (1.532256, -.1671951) and (1.569578, -.1908112) .. (1.558339, -.2140427) .. controls (1.553725, -.2223553) and (1.5473, -.2299008) .. (1.537186, -.2301865);
		\filldraw[Green4, draw = Green4!50!black, line width = #1 * 0.15pt] (1.260802, -1.65163) .. controls (1.252702, -1.480435) and (1.191644, -1.440478) .. (1.1176, -1.357702) .. controls (1.091667, -1.328711) and (1.103697, -1.291315) .. (1.079069, -1.257097) .. controls (1.054261, -1.222629) and (1.074713, -1.205013) .. (1.064732, -1.171028) .. controls (1.03889, -1.08303) and (1.090716, -1.026298) .. (1.12365, -1.018192) .. controls (1.238643, -.9898877) and (1.299704, -1.226189) .. (1.282453, -1.368344) .. controls (1.271076, -1.462095) and (1.289273, -1.394524) .. (1.281664, -1.634375);
		\draw[Chartreuse3!50!Chartreuse4, line width = #1 * 0.4pt, line cap = round]	(1.267139, -1.630478) .. controls (1.279164, -1.400917) and (1.175734, -1.352484) .. (1.150742, -1.09394);
		\filldraw[Green4, draw = Green4!50!black, line width = #1 * 0.15pt] (1.257073, -1.613271) .. controls (1.451349, -1.077845) and (1.454698, -1.134954) .. (1.274401, -.9786447) .. controls (1.257414, -.9639175) and (1.264737, -.9443961) .. (1.262147, -.9079227) .. controls (1.257073, -.836456) and (1.420537, -.7034793) .. (1.497307, -.7365228) .. controls (1.606083, -.7833429) and (1.795836, -.9226257) .. (1.57517, -1.080421) .. controls (1.415191, -1.194819) and (1.469509, -1.280213) .. (1.289362, -1.599571);
		\draw[Chartreuse3!50!Chartreuse4, line width = #1 * 0.4pt, line cap = round] (1.483121, -.8405951) .. controls (1.497032, -1.209887) and (1.358083, -1.30055) .. (1.274401, -1.644367);
		\filldraw[Green4, draw = Green4!50!black, line width = #1 * 0.15pt] (1.289362, -1.599571) .. controls (1.458887, -1.47267) and (1.620601, -1.449775) .. (1.563134, -1.352484) .. controls (1.508087, -1.259292) and (1.652702, -1.27478) .. (1.707603, -1.295272) .. controls (1.760986, -1.315199) and (1.789353, -1.299651) .. (1.807231, -1.316943) .. controls (1.989878, -1.493602) and (1.718136, -1.520414) .. (1.661592, -1.457041) .. controls (1.633508, -1.425565) and (1.526314, -1.476194) .. (1.306812, -1.620024);
		\draw[Chartreuse3!50!Chartreuse4, line width = #1 * 0.4pt, line cap = round] (1.747994, -1.364964) .. controls (1.686613, -1.374907) and (1.374055, -1.543023) .. (1.281664, -1.634375);
		\filldraw[Green4, draw = Green4!50!black, line width = #1 * 0.15pt] (1.260802, -1.65163) .. controls (1.306902, -1.466121) and (1.465397, -1.351467) .. (1.45448, -1.293419) .. controls (1.443801, -1.236643) and (1.505927, -1.188484) .. (1.516539, -1.120541) .. controls (1.536722, -.9913243) and (1.909822, -.7345792) .. (2.015351, -.857966) .. controls (2.13048, -.9925774) and (1.870889, -1.225336) .. (1.822996, -1.220363) .. controls (1.757843, -1.213598) and (1.747973, -1.250014) .. (1.740287, -1.264849) .. controls (1.721344, -1.301416) and (1.546288, -1.224848) .. (1.306812, -1.620024);
		\draw[Chartreuse3!50!Chartreuse4, line width = #1 * 0.4pt, line cap = round] (1.83517, -.9151849) .. controls (1.593072, -1.241134) and (1.313334, -1.493684) .. (1.281664, -1.634375);
		\filldraw[Green4, draw = Green4!50!black, line width = #1 * 0.15pt] (1.059839, -1.39177) .. controls (.8660852, -1.161855) and (.9119225, -1.142813) .. (.7467854, -1.070244) .. controls (.6747658, -1.038596) and (.6873651, -1.009138) .. (.6157742, -.9597348) .. controls (.4563691, -.8497337) and (.3192791, -.5931989) .. (.3987961, -.5698506) .. controls (.5415633, -.5279306) and (.5556416, -.6061438) .. (.6570277, -.682739) .. controls (.7109188, -.7234526) and (.7943625, -.8114293) .. (.8311678, -.8861511) .. controls (.8644155, -.9536507) and (.9227099, -1.007779) .. (.9256161, -1.032485) .. controls (.9447453, -1.195104) and (1.001012, -1.181425) .. (1.150742, -1.426253);
		\draw[Chartreuse3!50!Chartreuse4, line width = #1 * 0.4pt, line cap = round] (1.110338, -1.401564) .. controls (.7178429, -.9009685) and (.5840659, -.6409457) .. (.5168233, -.6260461);
		\filldraw[Green4, draw = Green4!50!black, line width = #1 * 0.15pt] (1.212983, -1.66448) .. controls (1.128811, -1.648619) and (.9914369, -1.639774) .. (.8671743, -1.60557) .. controls (.7066595, -1.561388) and (.567846, -1.541183) .. (.5390547, -1.585572) .. controls (.4807458, -1.675469) and (.4048689, -1.690349) .. (.3602141, -1.695491) .. controls (.1792303, -1.716333) and (0.03555548, -1.567241) .. (0.03629992, -1.489242) .. controls (0.03763966, -1.348873) and (.4340899, -1.334304) .. (.506601, -1.378525) .. controls (.6591374, -1.471549) and (.7797838, -1.572827) .. (1.158642, -1.624502);
		\draw[Chartreuse3!50!Chartreuse4, line width = #1 * 0.4pt, line cap = round] (1.24544, -1.659935) .. controls (1.215259, -1.636665) and (1.010394, -1.635278) .. (.8360976, -1.57254) .. controls (.6305917, -1.498569) and (.379686, -1.492437) .. (.2470609, -1.505955);
		\filldraw[Green4, draw = Green4!50!black, line width = #1 * 0.15pt] (1.212983, -1.66448) .. controls (1.141422, -1.611898) and (1.061812, -1.550873) .. (.9890236, -1.519809) .. controls (.9714784, -1.512321) and (.930235, -1.559643) .. (.9154265, -1.541404) .. controls (.8945015, -1.515631) and (.8544054, -1.5341) .. (.8507917, -1.532798) .. controls (.8284225, -1.524738) and (.8587304, -1.510681) .. (.7436899, -1.483481) .. controls (.6169164, -1.453508) and (.3003641, -1.413666) .. (.3610103, -1.326288) .. controls (.4904286, -1.139825) and (.8754481, -1.356553) .. (.947956, -1.400779) .. controls (1.042372, -1.458368) and (.9522707, -1.425929) .. (1.128074, -1.552754);
		\draw[Chartreuse3!50!Chartreuse4, line width = #1 * 0.4pt, line cap = round] (1.215259, -1.636665) .. controls (.9978675, -1.482232) and (.8968261, -1.391116) .. (.5706844, -1.324205);
		\filldraw[Green4, draw = Green4!50!black, line width = #1 * 0.15pt] (1.899221, -1.585572) .. controls (2.028201, -1.511157) and (1.919623, -1.471103) .. (2.260023, -1.51714) .. controls (2.42255, -1.539121) and (2.756279, -1.712312) .. (2.718552, -1.804291) .. controls (2.691785, -1.86955) and (2.685695, -1.894108) .. (2.297149, -1.848499) .. controls (2.270155, -1.845331) and (2.268071, -1.824809) .. (2.2437, -1.814969) .. controls (2.229655, -1.809298) and (2.195565, -1.821826) .. (2.174404, -1.805122) .. controls (2.087753, -1.736722) and (1.953629, -1.644367) .. (1.886529, -1.64242);
		\draw[Chartreuse3!50!Chartreuse4, line width = #1 * 0.4pt, line cap = round] (2.481512, -1.64242) .. controls (2.393123, -1.621897) and (2.033024, -1.577676) .. (1.937092, -1.607863);
		\filldraw[Green4, draw = Green4!50!black, line width = #1 * 0.15pt] (1.260802, -1.65163) .. controls (1.322766, -1.598464) and (1.351099, -1.586127) .. (1.59872, -1.489175) .. controls (1.70599, -1.447175) and (1.908458, -1.425225) .. (1.990233, -1.585083) .. controls (2.023482, -1.650079) and (1.99824, -1.705139) .. (1.650557, -1.695491) .. controls (1.471962, -1.690535) and (1.409731, -1.598754) .. (1.301233, -1.65928);
		\draw[Chartreuse3!50!Chartreuse4, line width = #1 * 0.4pt, line cap = round] (1.788574, -1.501598) .. controls (1.655662, -1.49362) and (1.551567, -1.562188) .. (1.464878, -1.572967) .. controls (1.37123, -1.584611) and (1.293361, -1.63106) .. (1.26941, -1.670197);
		\filldraw[Green4, draw = Green4!50!black, line width = #1 * 0.15pt] (1.212983, -1.66448) .. controls (1.143732, -1.540031) and (1.022274, -1.478925) .. (1.034623, -1.404825) .. controls (1.042691, -1.356416) and (1.314234, -1.443328) .. (1.260803, -1.65163);
		\draw[Chartreuse3!50!Chartreuse4, line width = #1 * 0.4pt, line cap = round] (1.24544, -1.695491) .. controls (1.23669, -1.518158) and (1.150742, -1.519989) .. (1.082413, -1.437705);
		\filldraw[Green4, draw = Green4!25!black, line width = #1 * 0.1pt] (1.313275, -.4124908) .. controls (1.379494, -.2955698) and (1.505575, -.308222) .. (1.512805, -.3482965) .. controls (1.527054, -.427265) and (1.3549, -.3991324) .. (1.30592, -.4367551);
		\draw[Chartreuse3!50!Chartreuse4, line width = #1 * 0.3pt, line cap = round] (1.309138, -.4215445) .. controls (1.32873, -.40599) and (1.396374, -.3604324) .. (1.472182, -.3580121);
		\filldraw[Green4, draw = Green4!25!black, line width = #1 * 0.1pt] (1.289641, -.4302139) .. controls (1.225633, -.3151486) and (1.104749, -.3562045) .. (1.150111, -.2730673) .. controls (1.192384, -.1955902) and (1.297454, -.3393903) .. (1.293926, -.4092657);
		\draw[Chartreuse3!50!Chartreuse4, line width = #1 * 0.3pt, line cap = round] (1.180296, -.294657) .. controls (1.196421, -.2848813) and (1.270843, -.3604895) .. (1.292616, -.4155617);
		\filldraw[Green4, draw = Green4!25!black, line width = #1 * 0.1pt] (1.425318, -0.0920694) .. controls (1.444892, -0.04842621) and (1.492004, -0.046474) .. (1.492889, -0.06289307) .. controls (1.494425, -0.09138336) and (1.472398, -0.09816925) .. (1.419542, -.101889);
		\filldraw[Green4, draw = Green4!25!black, line width = #1 * 0.15pt] (1.237375, -1.709747) .. controls (.8686243, -1.839455) and (.6684191, -1.850224) .. (.5996783, -2.128251) .. controls (.5773847, -2.218419) and (.4769325, -2.231359) .. (.4478499, -2.285979) .. controls (.388618, -2.397223) and (.3040061, -2.415909) .. (.1629469, -2.422675) .. controls (0.05425596, -2.427888) and (-.09978406, -2.42783) .. (-.1143554, -2.407839) .. controls (-.1665305, -2.336259) and (-.1432023, -2.014742) .. (.3181223, -1.871277) .. controls (.6899586, -1.755642) and (.523928, -1.909868) .. (.8051375, -1.791136) .. controls (.9724944, -1.720474) and (1.197205, -1.662085) .. (1.24544, -1.659935);
		\filldraw[Green4, draw = Green4!25!black, line width = #1 * 0.1pt] (1.226325, -.9933966) .. controls (1.17401, -.9590718) and (.8519284, -1.018405) .. (.8712038, -.9149969) .. controls (.8929716, -.7982172) and (1.157684, -.8687085) .. (1.219093, -.9686751);
		\draw[Chartreuse3!50!Chartreuse4, line width = #1 * 0.3pt, line cap = round] (.9389383, -.9178856) .. controls (1.049261, -.9433618) and (1.092403, -.9060446) .. (1.222533, -.9796097);
		\filldraw[Green4, draw = Green4!25!black, line width = #1 * 0.1pt] (1.24544, -1.659935) .. controls (1.223125, -1.55415) and (1.211655, -1.289246) .. (1.226325, -.9933966) .. controls (1.22655, -.988876) and (1.175696, -.8292197) .. (1.137576, -.7771216) .. controls (1.135419, -.7741731) and (1.115179, -.7717822) .. (1.110976, -.7490396) .. controls (1.109699, -.7421284) and (1.134517, -.7347879) .. (1.137018, -.7378844) .. controls (1.14891, -.7526132) and (1.143283, -.7628717) .. (1.146008, -.766196) .. controls (1.198115, -.8297755) and (1.228403, -.9559714) .. (1.22873, -.9501501) .. controls (1.248771, -.5928473) and (1.306514, -.2115433) .. (1.426494, -0.05867249) .. controls (1.43056, -0.05349274) and (1.439863, -0.00869873) .. (1.444617, 0.0025119) .. controls (1.444855, 0.00307312) and (1.43773, 0.00764954) .. (1.436943, 0.01749329) .. controls (1.436425, 0.0239679) and (1.46399, 0.03710449) .. (1.465237, 0.03310974) .. controls (1.471722, 0.01233826) and (1.458201, 0.00470825) .. (1.457364, 0.00170136) .. controls (1.454663, -0.00800415) and (1.446422, -0.03197693) .. (1.448757, -0.02911621) .. controls (1.45763, -0.01824585) and (1.466809, -0.00804169) .. (1.476302, 0.00145142) .. controls (1.481416, 0.00656555) and (1.473293, 0.02633392) .. (1.497207, 0.03760345) .. controls (1.509271, 0.04328888) and (1.526673, 0.00691986) .. (1.518354, 0.00133057) .. controls (1.504202, -0.00817688) and (1.499005, -0.00235596) .. (1.491119, -0.00946045) .. controls (1.415642, -0.0774585) and (1.360388, -.2086484) .. (1.321188, -.3690387) .. controls (1.319473, -.3760574) and (1.379332, -.279348) .. (1.484626, -.2389552) .. controls (1.49042, -.2367325) and (1.490185, -.224389) .. (1.516505, -.2226796) .. controls (1.524115, -.2221854) and (1.529831, -.2564153) .. (1.521593, -.258699) .. controls (1.500187, -.2646336) and (1.49176, -.2523102) .. (1.49127, -.252475) .. controls (1.376579, -.2910558) and (1.310965, -.4130669) .. (1.309139, -.4215446) .. controls (1.215139, -.8579851) and (1.228697, -1.470561) .. (1.274402, -1.644367);
		\draw[Chartreuse3!50!Chartreuse4, line width = #1 * 0.4pt, line cap = round] (1.237375, -1.67746) .. controls (.707674, -1.819383) and (.6752964, -1.901088) .. (.3569763, -2.065095);
		\filldraw[Green4, draw = Green4!50!black, line width = #1 * 0.15pt] (1.242588, -1.770743) .. controls (1.018009, -2.019354) and (1.185388, -1.876744) .. (1.174349, -2.126797) .. controls (1.166429, -2.306182) and (.9223582, -2.302882) .. (.8859967, -2.251906) .. controls (.8172271, -2.155494) and (.8728653, -2.011261) .. (.9441631, -1.96511) .. controls (1.094147, -1.868024) and (1.02482, -2.016552) .. (1.235724, -1.697611);
		\draw[Chartreuse3!50!Chartreuse4, line width = #1 * 0.4pt, line cap = round] (1.275191, -1.704479) .. controls (.9988309, -1.99817) and (1.04323, -2.079486) .. (.9638289, -2.18092);
		\filldraw[Green4, draw = Green4!50!black, line width = #1 * 0.15pt] (1.301233, -1.65928) .. controls (2.04463, -1.981619) and (1.891531, -1.721116) .. (2.032323, -1.753348) .. controls (2.093732, -1.767406) and (2.188408, -1.880858) .. (2.312403, -1.924456) .. controls (2.638488, -2.039112) and (2.589329, -2.165254) .. (2.519076, -2.252216) .. controls (2.43377, -2.357812) and (2.018468, -2.357669) .. (1.901249, -2.118594) .. controls (1.822595, -1.958175) and (1.665798, -1.849493) .. (1.307997, -1.714651);
		\draw[Chartreuse3!50!Chartreuse4, line width = #1 * 0.4pt, line cap = round] (1.299704, -1.684722) .. controls (1.824951, -1.931137) and (1.810116, -1.768883) .. (2.268672, -2.057832);
		\filldraw[Green4, draw = Green4!50!black, line width = #1 * 0.15pt] (1.301233, -1.65928) .. controls (1.341332, -2.075482) and (1.550518, -1.941286) .. (1.506694, -2.322522) .. controls (1.497513, -2.402393) and (1.434476, -2.516491) .. (1.407827, -2.595714) .. controls (1.386474, -2.659193) and (1.289363, -2.777413) .. (1.19791, -2.763916) .. controls (1.068081, -2.744754) and (1.036797, -2.571601) .. (1.009442, -2.532555) .. controls (.9668889, -2.471815) and (.9704975, -2.252596) .. (1.031327, -2.104443) .. controls (1.099186, -1.939168) and (1.280252, -1.985923) .. (1.235724, -1.697611) .. controls (1.224776, -1.652777) and (1.296776, -1.630253) .. (1.301233, -1.65928) ;
		\draw[Chartreuse3!50!Chartreuse4, line width = #1 * 0.4pt, line cap = round] (1.260802, -1.65163) .. controls (1.309579, -1.724783) and (1.269409, -2.351198) .. (1.192651, -2.651588);
	\end{scope}
}
%%% set compatibility level
\pgfplotsset{compat = 1.18}

%%% length, counters, booleans, ...
\newlength{\plotwidth}
\newlength{\plotheight}
\newlength{\plotylabelwidth}
\newlength{\plotxlabelheight}
\newlength{\groupplotsep}

\setlength{\plotylabelwidth}{2.8em}
\setlength{\plotxlabelheight}{2.8em}
\setlength{\groupplotsep}{1mm}

\newcounter{groupplot}
\newcounter{groupcol}
\newcounter{grouprow}
\newcounter{tmpct}

\newif\iflastplot

%%% project-specific settings and macros
\pgfplotsset{
	axis limits/.is choice,
	axis limits/AB80 A/.style = {xmin = 1, xmax = 169, xtick = {1, 20, 40, ..., 170}},
	axis limits/AB80 B/.style = {xmin = 79, xmax = 247, xtick = {79, 100, 120, ..., 250}},
	axis limits/Cab-1 A/.style = {xmin = 1, xmax = 169, xtick = {1, 20, 40, ..., 170}},
	axis limits/Cab-1 B/.style = {xmin = 100, xmax = 268, xtick = {100, 120, ..., 270}},
	axis limits/rbcS-E9 A/.style = {xmin = 1, xmax = 169, xtick = {1, 20, 40, ..., 170}},
	axis limits/rbcS-E9 B/.style = {xmin = 66, xmax = 234, xtick = {66, 80, 100, ..., 240}},
	shade right/.style 2 args = {execute at begin axis/.append = {\fill[#2, fill opacity = .1] (#1, \ymin) rectangle (\xmax, \ymax);}},
	shade left/.style 2 args = {execute at begin axis/.append = {\fill[#2, fill opacity = .1] (\xmin, \ymin) rectangle (#1, \ymax);}},
	shade overlap A/.is choice,
	shade overlap A/AB80/.style = {shade right = {79}{AB80}},
	shade overlap A/Cab-1/.style = {shade right = {100}{Cab-1}},
	shade overlap A/rbcS-E9/.style = {shade right = {66}{rbcS-E9}},
	shade overlap B/.style = {shade left = {169}{#1}}
}

%% draw enhancer fragments
% \enhFrag[<yshift (multiples of \baselineskip)>]{<center>}{<width>}{<color>}{<name>}
\newcommand{\enhFrag}[5][1]{
	\node[anchor = north, font = \bfseries\vphantom{abcdectrl}, inner sep = .15em, yshift = #1 * \baselineskip, fill = #4, fill opacity = .5, text opacity = 1, minimum width = #3 * \pgfplotsunitxlength] (#4 #5) at (#2, \ymin) {#5};
}

%% colorbar for distance between deletions in PEVdouble
	\pgfplotsset{
		colorbar distance/.style = {
			colorbar horizontal,
			colorbar style = {
				anchor = south west,
				at = {(.025, .025)},
				height = .25cm,
				width = .4 * \pgfkeysvalueof{/pgfplots/parent axis height},
				tick pos = upper,
				xticklabel pos = right,
				xticklabel style = {node font = \figtiny, inner ysep = .1em, text depth = 0pt},
				title = distance (bp)\\\vphantom{1},
				title style = {fill = none, draw = none, minimum width = 0, node font = \figsmall, align = center, anchor = south},
				extra x ticks = {8},
			},
		}
	}

%%% general setup
\usepgfplotslibrary{statistics, groupplots, colormaps}

\colorlet{titlecol}{gray}

\pgfplotsset{
	table/col sep = tab,
	every axis/.append style = {
		alias = last plot,
		thin,
		scale only axis = true,
		width = \fourcolumnwidth - \plotylabelwidth,
		height = 3.5cm,
		legend style = {node font = \figsmall, fill = none, draw = none, /tikz/every even column/.append style = {column sep = \groupplotsep}},
		tick label style = {node font = \figsmall},
		tick align = outside,
		label style = {node font = \fignormal, text depth = 0pt, align = center},
		title style = {node font = \fignormal},
		every axis title shift = 0pt,
		max space between ticks = 20,
		ticklabel style = {
			/pgf/number format/fixed,
		},
		every tick/.append style = {black, thin},
		tickwidth = .75mm,
		every x tick label/.append style = {align = center, inner xsep = 0pt},
		every y tick label/.append style = {inner ysep = 0pt},
		scaled ticks = false,
		mark/.default = solido,
		mark size = .75,
		grid = both,
		grid style = {very thin, gray!20},
		tick pos = lower,
		x grids/.is choice,
		x grids/true/.style = {xmajorgrids = true, xminorgrids = true},
		x grids/false/.style = {xmajorgrids = false, xminorgrids = false},
		y grids/.is choice,
		y grids/true/.style = {ymajorgrids = true, yminorgrids = true},
		y grids/false/.style = {ymajorgrids = false, yminorgrids = false},
		boxplot/whisker extend = {\pgfkeysvalueof{/pgfplots/boxplot/box extend}*0.5},
		boxplot/every median/.style = thick,
		title style = {minimum width = \plotwidth, fill = titlecol!20, draw = black, yshift = -.5\pgflinewidth, name = title, text depth = 0pt, align = center},
		legend cell align = left,
		axis background/.style = {fill = white},
		line plot/.style = {thick, line join = round},
		unbounded coords = jump,
		boxplot/draw direction = y,
		xticklabel style = {align = center}
	},
	x decimals/.style = {x tick label style = {/pgf/number format/.cd, fixed, zerofill, precision = #1}},
	y decimals/.style = {y tick label style = {/pgf/number format/.cd, fixed, zerofill, precision = #1}},
	zero line/.style = {execute at begin axis/.append = {\draw[thin, #1] (\xmin, 0) -- (\xmax, 0);}},
	zero line/.default = black,
	show diagonal/.style = {execute at begin axis/.append = {\draw[thin, #1] (-10, -10) -- (10, 10);}},
	show diagonal/.default = {gray, dashed},
	colormap = {blue-red}{color(-1) = (RoyalBlue1) color(0) = (white), color(1) = (IndianRed1)},
	title color/.code = {\colorlet{titlecol}{#1}}
}

%%% extra markers
% round marker without border
\pgfdeclareplotmark{solido}{%
  \pgfpathcircle{\pgfpointorigin}{\pgfplotmarksize + .5\pgflinewidth}%
  \pgfusepathqfill
}

% hexagonal marker without border
\pgfdeclareplotmark{hexagon}
{%
	\pgftransformyscale{\yscale}
	\pgfmathsetlength{\templength}{0.5\pgfplotmarksize / cos(30)}
  \pgfpathmoveto{\pgfqpoint{0pt}{\templength}}
  \pgfpathlineto{\pgfqpointpolar{150}{\templength}}
  \pgfpathlineto{\pgfqpointpolar{210}{\templength}}
  \pgfpathlineto{\pgfqpointpolar{270}{\templength}}
  \pgfpathlineto{\pgfqpointpolar{330}{\templength}}
  \pgfpathlineto{\pgfqpointpolar{30}{\templength}}
  \pgfpathclose
	\pgfusepathqfill
}
\def\yscale{1}

% hexagonal marker with border
\pgfdeclareplotmark{hexagonb}
{%
	\pgftransformyscale{\yscale}
	\pgfmathsetlength{\templength}{0.5\pgfplotmarksize / cos(30)}
  \pgfpathmoveto{\pgfqpoint{0pt}{\templength}}
  \pgfpathlineto{\pgfqpointpolar{150}{\templength}}
  \pgfpathlineto{\pgfqpointpolar{210}{\templength}}
  \pgfpathlineto{\pgfqpointpolar{270}{\templength}}
  \pgfpathlineto{\pgfqpointpolar{330}{\templength}}
  \pgfpathlineto{\pgfqpointpolar{30}{\templength}}
  \pgfpathclose
	\pgfusepathqfillstroke
}


%%% commands for min and max coordinates
\def\xmin{\pgfkeysvalueof{/pgfplots/xmin}}
\def\xmax{\pgfkeysvalueof{/pgfplots/xmax}}
\def\ymin{\pgfkeysvalueof{/pgfplots/ymin}}
\def\ymax{\pgfkeysvalueof{/pgfplots/ymax}}


%%% add jitter to points of a scatter plot
\pgfplotsset{
	x jitter/.style = {
		x filter/.expression = {x + rand * #1}
	},
	y jitter/.style = {
		y filter/.expression = {y + rand * #1}
	},
	x jitter/.default = 0.25,
	y jitter/.default = 0.25
}


%%% set x axis limits and tick labels (depending on sample number)
% use:	x limits = {<number of samples>}
%				x tick table = {<table>}{<column>}
% for half violin plots use:	x tick table half = {<table>}{<column>}
\pgfplotsset{
	x limits/.style = {xmin = 1 - 0.6, xmax = #1 + 0.6},
	y limits/.style = {ymin = 1 - 0.6, ymax = #1 + 0.6},
	x tick table/.code 2 args = {\getrows{#1}},
	x tick table/.append style = {%
		x limits = {\datarows},
		xtick = {1, ..., \datarows},
		xticklabels from table = {#1}{#2}
	},
	x tick table segment/.code 2 args = {\getrows{#1}},
	x tick table segment/.append style = {%
		x limits = {\datarows},
		xtick = {1, ..., \datarows},
		xticklabel = {\pgfmathint{\tick - 1}\pgfplotstablegetelem{\pgfmathresult}{#2}\of{#1}\segment{\pgfplotsretval}}
	},
	x tick table enhancer/.code 2 args = {\getrows{#1}},
	x tick table enhancer/.append style = {%
		x limits = {\datarows},
		xtick = {1, ..., \datarows},
		xticklabel = {\pgfmathint{\tick - 1}\pgfplotstablegetelem{\pgfmathresult}{#2}\of{#1}\enhancer{\pgfplotsretval}}
	},
	x tick table half/.code 2 args = {\gethalfrows{#1}},
	x tick table half/.append style = {%
		x limits = {\halfrows},
		xtick = {1, ..., \halfrows},
		xticklabel = {\pgfmathint{2 * (\tick - 1)}\pgfplotstablegetelem{\pgfmathresult}{#2}\of{#1}\pgfplotsretval}
	},
	x tick table half segment/.code 2 args = {\gethalfrows{#1}},
	x tick table half segment/.append style = {%
		x limits = {\halfrows},
		xtick = {1, ..., \halfrows},
		xticklabel = {\pgfmathint{2 * (\tick - 1)}\pgfplotstablegetelem{\pgfmathresult}{#2}\of{#1}\segment{\pgfplotsretval}}
	},
	x tick table half enhancer/.code 2 args = {\gethalfrows{#1}},
	x tick table half enhancer/.append style = {%
		x limits = {\halfrows},
		xtick = {1, ..., \halfrows},
		xticklabel = {\pgfmathint{2 * (\tick - 1)}\pgfplotstablegetelem{\pgfmathresult}{#2}\of{#1}\enhancer{\pgfplotsretval}}
	},
	y tick table/.code 2 args= {\getrows{#1}},
	y tick table/.append style = {%
		y limits = {\datarows},
		ytick = {1, ..., \datarows},
		yticklabels from table = {#1}{#2}
	},
	y tick table half/.code 2 args= {\gethalfrows{#1}},
	y tick table half/.append style = {%
		y limits = {\halfrows},
		ytick = {1, ..., \halfrows},
		yticklabel = {\pgfmathint{2 * (\tick - 1)}\pgfplotstablegetelem{\pgfmathresult}{#2}\of{#1}\pgfplotsretval}
	},
	xymin/.style = {xmin = #1, ymin = #1},
	xymax/.style = {xmax = #1, ymax = #1},
	xytick/.style = {xtick = {#1}, ytick = {#1}},
}
\makeatletter
	\pgfplotsset{
		set limit from table/.code = {
			\pgfplotstablegetelem{0}{#1}\of{\axes@datatable}%
			\pgfplotsset{#1/.expand once={\pgfplotsretval}}%
		},
		axis limits from table/.code = {
			\pgfplotstable@isloadedtable{#1}%
				{\pgfplotstablecopy{#1}\to\axes@datatable}%
				{\pgfplotstableread{#1}\axes@datatable}%
			\pgfplotstableforeachcolumn{\axes@datatable}\as{\colname}{%
				\pgfplotsset{set limit from table/.expand once = {\colname}}%
			}%
		},
		option from table/.code 2 args = {
			\pgfplotstablegetelem{0}{#1}\of{#2}%
			\pgfplotsset{#1/.expand once={\pgfplotsretval}}%
		},
		option from table row/.code n args = {3}{
			\pgfplotstablegetelem{#3}{#1}\of{#2}%
			\pgfplotsset{#1/.expand once={\pgfplotsretval}}%
		},
	}
\makeatother


%%% horizontal groupplot
% use: \begin{hgroupplot}[<axis options>]{<total width>}{<number of columns>}{<xlabel>} <plot commands> \end{hgroupplot}
\newenvironment{hgroupplot}[4][]{%
	\setcounter{groupcol}{0}\setcounter{grouprow}{1}%
	\pgfmathsetlength{\plotwidth}{(#2 - \plotylabelwidth - ((#3 - 1) * \groupplotsep) ) / #3}%
	\ifthenelse{\isempty{#4}}{%
		\def\xlabel{}%
	}{%
		\def\xlabel{%
			\path let \p1 = (group c1r1.west), \p2 = (group c#3r1.east), \p3 = (xlabel.base) in node[anchor = base, align = center] (plot xlabel) at ($(\x1, \y3)!.5!(\x2, \y3)$) {#4};%
		}%
	}
	\begin{groupplot}[
		width = \plotwidth,
		group style = {
			horizontal sep = \groupplotsep,
			y descriptions at = edge left,
			x descriptions at = edge bottom,
			columns = #3,
			rows = 1,
			every plot/.append style = {no inner x ticks, no inner y ticks, first plot, groupplot labels = {#3}{1}}
		},
		xlabel = \vphantom{#4},
		xlabel style = {name = xlabel},
		#1
	]
}{%
	\end{groupplot}%
	\xlabel%
}


%%% vertical groupplot
% use: \begin{vgroupplot}[<axis options>]{<total width>}{<number of rows>}{<ylabel>} <plot commands> \end{vgroupplot}
\newenvironment{vgroupplot}[4][]{%
	\setcounter{groupcol}{0}\setcounter{grouprow}{1}%
	\pgfmathsetlength{\plotheight}{(#2 - \plotxlabelheight - ((#3 - 1) * \groupplotsep) ) / #3}%
	\ifthenelse{\isempty{#4}}{%
		\def\ylabel{}%
	}{%
		\def\ylabel{%
			\path let \p1 = (group c1r1.north), \p2 = (group c1r#3.south), \p3 = (ylabel.base) in node[anchor = base, rotate = 90, align = center] (plot ylabel) at ($(\x3, \y1)!.5!(\x3, \y2)$) {#4};%
		}%
	}
	\setcounter{groupplot}{0}
	\begin{groupplot}[
		height = \plotheight,
		group style = {
			vertical sep = \groupplotsep,
			y descriptions at = edge left,
			x descriptions at = edge bottom,
			columns = 1,
			rows = #3,
			every plot/.append style = {no inner x ticks, no inner y ticks, first plot, groupplot labels = {1}{#3}}
		},
		ylabel = \vphantom{#4},
		ylabel style = {name = ylabel},
		#1
	]
}{%
	\end{groupplot}%
	\ylabel%
}


%%% horizontal and vertical groupplot
% use: \begin{hvgroupplot}[<axis options>]{<total width>}{<total height>}{<number of columns>}{<number of rows>}{<xlabel>}{<ylabel>} <plot commands> \end{hvgroupplot}
\newenvironment{hvgroupplot}[7][]{%
	\setcounter{groupcol}{0}\setcounter{grouprow}{1}%
	\pgfmathsetlength{\plotwidth}{(#2 - \plotylabelwidth - ((#4 - 1) * \groupplotsep) ) / #4}%
	\pgfmathsetlength{\plotheight}{(#3 - \plotxlabelheight - ((#5 - 1) * \groupplotsep) ) / #5}%
	\ifthenelse{\isempty{#6}}{%
		\def\xlabel{}%
	}{%
		\def\xlabel{%
			\path let \p1 = (group c1r1.west), \p2 = (group c#4r1.east), \p3 = (xlabel.base) in node[anchor = base, align = center] (plot xlabel) at ($(\x1, \y3)!.5!(\x2, \y3)$) {#6};%
		}%
	}
	\ifthenelse{\isempty{#7}}{%
		\def\ylabel{}%
	}{%
		\def\ylabel{%
			\path let \p1 = (group c1r1.north), \p2 = (group c1r#5.south), \p3 = (ylabel.base) in node[anchor = base, rotate = 90, align = center] (plot ylabel) at ($(\x3, \y1)!.5!(\x3, \y2)$) {#7};%
		}%
	}
	\begin{groupplot}[
		width = \plotwidth,
		height = \plotheight,
		group style = {
			horizontal sep = \groupplotsep,
			vertical sep = \groupplotsep,
			y descriptions at = edge left,
			x descriptions at = edge bottom,
			columns = #4,
			rows = #5,
			every plot/.append style = {no inner x ticks, no inner y ticks, first plot, groupplot labels = {#4}{#5}}
		},
		xlabel = \vphantom{#6},
		xlabel style = {name = xlabel},
		ylabel = \vphantom{#7},
		ylabel style = {name = ylabel},
		#1
	]
}{%
	\end{groupplot}%
	\xlabel%
	\ylabel%
}

% required styles
\pgfplotsset{
	last plot/.is if = lastplot,
	last plot = false,
	no inner y ticks/.style = {ymajorticks = false, yminorticks = false},
	no inner x ticks/.style = {xmajorticks = false, xminorticks = false},
	group position/.code = {\pgfkeysalso{first plot/.style = {#1}}},
	groupplot labels/.code 2 args = {%
		\gpgfplotsset{first plot/.style = {}}%
		\pgfkeysalso{last plot = false},%
		\stepcounter{groupcol}%
		\ifnum\value{groupcol}>#1%
			\stepcounter{grouprow}%
			\setcounter{groupcol}{1}
		\fi%
		\ifnum\value{groupcol}=1%
			\pgfkeysalso{ymajorticks = true, yminorticks = true}%
		\fi%
		\ifnum\value{grouprow}=#2%
			\pgfkeysalso{xmajorticks = true, xminorticks = true}%
			\ifnum\value{groupcol}=#1%
				\pgfkeysalso{last plot = true}%
			\fi%
		\fi%
	},
	last plot style/.code = {
		\iflastplot
			\pgfkeysalso{#1}
		\fi
	}
}


%%% legend setup
\pgfplotsset{
	legend image code/.code = {
		\draw [mark repeat = 2,mark phase = 2, #1]
			plot coordinates {
				(0em, 0em)
				(.75em, 0em)
				(1.5em, 0em)
			};
	},
	/pgfplots/ybar legend 1/.style = {
		/pgfplots/legend image code/.code = {
			\draw [##1, fill = gray, /tikz/.cd, bar width = 2.5pt, yshift = -0.2em, bar shift = 0pt]
			plot coordinates {
				(\pgfplotbarwidth, .4em)
			};
			\draw [##1, /tikz/.cd, bar width = 2.5pt, yshift = -0.2em, bar shift = 0pt]
			plot coordinates {
				(0em, 1em)
			};
		},
	},
	/pgfplots/ybar legend 2/.style = {
		/pgfplots/legend image code/.code = {
			\draw [##1, fill = gray, /tikz/.cd, bar width = 2.5pt, yshift = -0.2em, bar shift = 0pt]
			plot coordinates {
				(0em, .4em)
			};
			\draw [##1, /tikz/.cd, bar width = 2.5pt, yshift = -0.2em, bar shift = 0pt]
			plot coordinates {
				(\pgfplotbarwidth, 1em)
			};
		},
	},
}


%%% helper for global pgfplot style definitions
\newcommand\gpgfplotsset[1]{%
	\begingroup%
		\globaldefs=1\relax%
		\pgfqkeys{/pgfplots}{#1}%
	\endgroup%
}


%%% get number of rows/columns in a table -> stored in `\datarows`/`\datacols`
% use: \getrows{<table>}
\newcommand{\getrows}[1]{%
	\pgfplotstablegetrowsof{#1}%
	\pgfmathsetmacro{\datarows}{\pgfplotsretval}%
}

\newcommand{\gethalfrows}[1]{%
	\pgfplotstablegetrowsof{#1}%
	\pgfmathsetmacro{\halfrows}{0.5 * \pgfplotsretval}%
}

\newcommand{\getcols}[1]{%
	\pgfplotstablegetcolsof{#1}%
	\pgfmathsetmacro{\datacols}{\pgfplotsretval}%
}

%%% variants of `\pgfplotsinvokeforeach` that iterates from 1 to the number of rows/columns in the table
% use:	\foreachtablerow{<table or file>}{<command (use `#1` to get current iterator)>}
\makeatletter
	\long\def\foreachtablerow#1#2{%
		\getrows{#1}
		\long\def\pgfplotsinvokeforeach@@##1{#2}%
		\pgfplotsforeachungrouped \pgfplotsinvokeforeach@ in {1, ..., \datarows} {%
			\expandafter\pgfplotsinvokeforeach@@\expandafter{\pgfplotsinvokeforeach@}%
		}%
	}
	
	\long\def\foreachtablecol#1#2{%
		\getcols{#1}
		\long\def\pgfplotsinvokeforeach@@##1{#2}%
		\pgfplotsforeachungrouped \pgfplotsinvokeforeach@ in {1, ..., \datacols} {%
			\expandafter\pgfplotsinvokeforeach@@\expandafter{\pgfplotsinvokeforeach@}%
		}%
	}
\makeatother


%%% command to add significance indicator to plot
% usage: \signif[pgf options]{file}{first sample}{second sample}
% useful options: `raise = ...` to raise (or lower) the indicator (connectors stay at the same height; use yshift to also raise them)
%									`shorten both = ...` to increase (or decrease) distance between plot and indicator (default = 2.5)
%									`signif cutoff = ...` to set the significance cutoff
%									`signif levels = {..., ..., ...}` to set the significance levels
%									`bar only` to not draw connectors (only the horizontal bar)
%									`draw = none` to not draw any line 
\makeatletter
	\newcommand{\signif}{%
		\@ifstar
			\signifStar%
			\signifNoStar%
	}
\makeatother

\newcommand{\signifNoStar}[4][]{%
	\addplot [no marks, shorten both = 2.5, point meta = \thisrow{p.value.#3_#4}, nodes near coords = \printsignif, nodes near coords style = {node font = \figsmaller}, #1] table [x = x.#3_#4, y = y.#3_#4] {#2};
}
\newcommand{\signifStar}[4][]{%
	\addplot [no marks, shorten both = 2.5, point meta = \thisrow{p.value.#3_#4}, nodes near coords = \printsignif*, nodes near coords style = {node font = \figsmall}, #1] table [x = x.#3_#4, y = y.#3_#4] {#2};
}


%% significance for all pairs (especially usefull for half plots or plots with selected pvalues only)
% usage: \signifall(*)[pgf options]{file}
\makeatletter
	\newcommand{\signifall}{%
		\@ifstar
			\signifallStar%
			\signifallNoStar%
	}
\makeatother

\newcommand{\signifallStar}[2][]{%
	\getcols{#2}%
	\pgfmathsetmacro{\samplen}{\datacols / 3}%
	\pgfplotsinvokeforeach{1, ..., \samplen}
	{%
		\pgfmathsetmacro{\xcol}{int(##1 - 1)}%
		\pgfmathsetmacro{\ycol}{int(\xcol + \samplen)}%
		\pgfmathsetmacro{\pcol}{int(\ycol + \samplen)}%
		\addplot [no marks, shorten both = 2.5, point meta = \thisrowno{\pcol}, nodes near coords = \printsignif*, nodes near coords style = {node font = \figsmall}, #1] table [x index = \xcol, y index = \ycol] {#2};%
	}
}

\newcommand{\signifallNoStar}[2][]{%
	\getcols{#2}%
	\pgfmathsetmacro{\samplen}{\datacols / 3}%
	\pgfplotsinvokeforeach{1, ..., \samplen}
	{%
		\pgfmathsetmacro{\xcol}{int(##1 - 1)}%
		\pgfmathsetmacro{\ycol}{int(\xcol + \samplen)}%
		\pgfmathsetmacro{\pcol}{int(\ycol + \samplen)}%
		\addplot [no marks, shorten both = 2.5, point meta = \thisrowno{\pcol}, nodes near coords = \printsignif, nodes near coords style = {node font = \figsmaller}, #1] table [x index = \xcol, y index = \ycol] {#2};%
	}
}


%% significance for simple comparisons (compared to zero distribution)
% use:	\signifallsimple(*)[<pgfplots options>]{<table>}{<column with x coords>}{<column with p-values>}
\makeatletter
	\newcommand{\signifallsimple}{%
		\@ifstar
			\signifallsimpleStar%
			\signifallsimpleNoStar%
	}
\makeatother

\newcommand{\signifallsimpleStar}[4][]{%
	\addplot [no marks, draw = none, point meta = \thisrow{#4}, nodes near coords = \vphantom{A}\printsignif*, nodes near coords style = {node font = \figsmall, anchor = north}, #1] table [x = #3, y expr = \pgfkeysvalueof{/pgfplots/ymax}] {#2};%
}

\newcommand{\signifallsimpleNoStar}[4][]{%
	\addplot [no marks, draw = none, point meta = \thisrow{#4}, nodes near coords = \printsignif, nodes near coords style = {node font = \figsmaller, anchor = north}, #1] table [x = #3, y expr = \pgfkeysvalueof{/pgfplots/ymax}] {#2};%
}


%% use point meta data to print p-values (the * version prints significance levels)
% use as: point meta = \thisrow{<p-value column>}, nodes near coords = \printsignif<*>[<cutoff (one value)/signif. levels (three values separated by ",")>] 
\makeatletter
	\newcommand{\printsignif}{%
		\@ifstar
			\printsignifStar%
			\printsignifNoStar%
	}
\makeatother

\newcommand{\printsignifStar}[1][\signiflevelone,\signifleveltwo,\signiflevelthree]{%
	\pgfmathfloatifflags{\pgfplotspointmeta}{3}{}{\pgfmathfloattosci{\pgfplotspointmeta}\tolevel#1,{\pgfmathresult}}%
}
\newcommand{\printsignifNoStar}[1][\signifcutoff]{%
	\pgfmathfloatifflags{\pgfplotspointmeta}{3}{}{\pgfmathfloattosci{\pgfplotspointmeta}\ifsignif#1,{\pgfmathresult}}%
}

% abbreviation for non-significant p-values and symbol for significance level
\newcommand{\notsignif}{ns}
\newcommand{\issignif}{{\signiffont \symbol{"2217}}}

%% default significance cutoff/levels
% 1 to \signifcutoff or \signiflevelone: ns
% \signiflevelone to \signifleveltwo: *
% \signifleveltwo to \signiflevelthree: **
% \signiflevelthree to 0: ***
\def\signifcutoff{0.01}
\def\signiflevelone{0.01}
\def\signifleveltwo{0.001}
\def\signiflevelthree{0.0001}

% convert p-value to significance level
\def\tolevel#1,#2,#3,#4{%
	\pgfmathparse{#4 <= #3 ? "\issignif\issignif\issignif" : (#4 <= #2 ? "\issignif\issignif" : (#4 <= #1 ? "\issignif" : "\notsignif"))}\pgfmathresult%
}

% print \notsignif for non-significant p-values
\def\ifsignif#1,#2{%
	\pgfmathparse{#2 <= #1 ? "$p = \pgfmathprintnumber{#2}$" : "\notsignif"}\pgfmathresult%
}

% useful styles
\makeatletter
\tikzset{
	bar only/.style = jump mark mid,
	shorten both/.style = {
		shorten < = #1,
		shorten > = #1
	},
	raise/.code = {
		\pgfkeysalso{yshift = #1}
		\pgfmathaddtolength\pgf@shorten@start@additional{-#1}
		\pgfmathaddtolength\pgf@shorten@end@additional{-#1}
	},
	signif cutoff/.code = {
		\def\signifcutoff{#1}
	},
	signif levels/.code args = {#1,#2,#3}{
		\def\signiflevelone{#1}
		\def\signifleveltwo{#2}
		\def\signiflevelthree{#3}
	}
}
\makeatother


%%% style to prepare boxplot from summarized data in a table
% table layout: one row per sample; required columns and content: lw, lower whisker; lq, lower quartile; med, median; uq, upper quartile; uw, upper whisker
% usage: boxplot from table = {table macro}{row number (0-based)} 
\makeatletter
\pgfplotsset{
    boxplot prepared from table/.code={
        \def\tikz@plot@handler{\pgfplotsplothandlerboxplotprepared}%
        \pgfplotsset{
            /pgfplots/boxplot prepared from table/.cd,
            #1,
        }
    },
    /pgfplots/boxplot prepared from table/.cd,
        table/.code={
        	\pgfplotstable@isloadedtable{#1}%
        		{\pgfplotstablecopy{#1}\to\boxplot@datatable}%
        		{\pgfplotstableread{#1}\boxplot@datatable}%
        },
        row/.initial=1,
        make style readable from table/.style={
            #1/.code={
            		\pgfmathint{\pgfkeysvalueof{/pgfplots/boxplot prepared from table/row} - 1}
                \pgfplotstablegetelem{\pgfmathresult}{##1}\of\boxplot@datatable
                \pgfplotsset{boxplot/#1/.expand once={\pgfplotsretval}}
            }
        },
        make style readable from table=lower whisker,
        make style readable from table=upper whisker,
        make style readable from table=lower quartile,
        make style readable from table=upper quartile,
        make style readable from table=median
}
\makeatother

\pgfplotsset{
	boxplot from table/.style 2 args = {%
		boxplot prepared from table = {%
			table = #1,
			row = #2,
			lower whisker = lw,
			lower quartile = lq,
			median = med,
			upper quartile = uq,
			upper whisker = uw
		}
	}
}


%%% draw a combined violin and box plot
% use `violin shade (inverse) = 0` to not shade the violin plots
% use: \violinbox[<pgfplots options for violin and box plots>]{<table for boxplot>}{<file for violin plot>}
\newcommand{\violinbox}[3][]{%
	\foreachtablerow{#2}{%
		\violinplot[save row = {##1}, violin shade, #1]{#3}{##1};%
		\boxplot[save row = {##1}, boxplot/box extend = {\pgfkeysvalueof{/pgfplots/violin extend} * 0.2}, boxplot/whisker extend = 0, #1]{#2}{##1};%
	}%
}

%%% draw a combined half violin and box plot
% use: \halfviolinbox[<pgfplots options for violin and box plots>]{<table for boxplot>}{<file for violin plot>}
\newcommand{\halfviolinbox}[3][]{%
	\foreachtablerow{#2}{%
		\ifthenelse{\isodd{##1}}{%
			\halfviolinplotleft[save row = {##1}, #1]{#3}{##1};%
			\boxplot[save row = {##1}, boxplot/box extend = {\pgfkeysvalueof{/pgfplots/violin extend} * 0.15}, boxplot/whisker extend = 0, boxplot/draw position = floor(0.5 * (##1 + 1)) - 0.0225, boxplot/draw relative anchor = 1, #1]{#2}{##1};%
		}{%
			\halfviolinplotright[save row = {##1}, #1]{#3}{##1};%
			\boxplot[save row = {##1}, boxplot/box extend = {\pgfkeysvalueof{/pgfplots/violin extend} * 0.15}, boxplot/whisker extend = 0, boxplot/draw position = floor(0.5 * (##1 + 1)) + 0.0225, boxplot/draw relative anchor = 0, #1]{#2}{##1};%
		}
	}%
}

%% draw a violin plot
% use: \violinplot[<pgfplots options>]{<file>}{<sample>}
\newcommand{\violinplot}[3][]{%
	\addplot [black, fill = viocol, fill opacity = .5, #1, viostyle] table [x expr = \thisrow{x.#3} * \pgfkeysvalueof{/pgfplots/violin extend} + #3, y = y.#3] {#2} -- cycle;%
}

%% draw a half violin plot
% use: \violinplot[<pgfplots options>]{<file>}{<sample>}
\newcommand{\halfviolinplotleft}[3][]{%
	\addplot [black, fill = viocolleft, fill opacity = .5, #1, viostyle] table [x expr = -0.95 * \thisrow{x.#3} * \pgfkeysvalueof{/pgfplots/violin extend} + floor(0.5 * (#3 + 1)) - 0.0225, y = y.#3] {#2} -- cycle;%
}
\newcommand{\halfviolinplotright}[3][]{%
	\addplot [black, fill = viocolright, fill opacity = .5, #1, viostyle] table [x expr = 0.95 * \thisrow{x.#3} * \pgfkeysvalueof{/pgfplots/violin extend} + floor(0.5 * (#3 + 1)) + 0.0225, y = y.#3] {#2} -- cycle;%
}

%% draw a box plot
% use: \boxplot[<pgfplots options>]{<table>}{<row>}
% does not add outliers! use `\outliers[<pgfplots options>]{<table>}{<sample>}
\newcommand{\boxplot}[3][]{%
	\addplot [black, fill = boxcol, boxplot from table = {#2}{#3}, boxplot/draw position = #3, mark = solido, mark options = black, #1, boxstyle] coordinates {};%
}

\newcommand{\outliers}[3][]{%
	\addplot [black, only marks, mark = solido, #1] table [x expr = #3, y = outlier.#3] {#2};%
}

%%% draw boxplots from table
% use:	\boxplots[<outlier options>]{<boxplot options>}{<base name of data files>}
\newcommand{\boxplots}[3][]{%
	\foreachtablerow{#3_boxplot.tsv}{%
		\boxplot[save row = {##1}, #2]{#3_boxplot.tsv}{##1};%
		\IfFileExists{./#3_outliers.tsv}{%
			\pgfmathint{##1 - 1}%
			\pgfplotstablegetelem{\pgfmathresult}{outliers}\of{#3_boxplot.tsv}%
			\ifnum\pgfplotsretval>0%
				\outliers[#1]{#3_outliers.tsv}{##1};%
			\fi%
		}{}%
	}%
}

%%% draw two related box plots (left and right of the common label)
% use: \halfboxplots[<outlier options>]{<boxplot options>}{<base name of data files>}
\newcommand{\halfboxplots}[3][]{%
	\foreachtablerow{#3_boxplot.tsv}{%
		\ifthenelse{\isodd{##1}}{%
			\def\curshift{-.225}
		}{%
			\def\curshift{.225}
		}%
		\boxplot[save row = {##1}, boxplot/draw position = ##1 + \curshift - floor(.5 * ##1), boxplot/box extend = 0.35, #2]{#3_boxplot.tsv}{##1};%
		\IfFileExists{./#3_outliers.tsv}{%
			\pgfmathint{##1 - 1}%
			\pgfplotstablegetelem{\pgfmathresult}{outliers}\of{#3_boxplot.tsv}%
			\ifnum\pgfplotsretval>0%
				\outliers[x filter/.expression = {x + \curshift - floor(.5 * ##1)}, #1]{#3_outliers.tsv}{##1};%
			\fi%
		}{}%
	}%
}


% default colors for violin and box plots
\colorlet{viocol}{gray}
\colorlet{viocolleft}{viocol!50!black}
\colorlet{viocolright}{viocol}
\colorlet{vioshade}{black}
\colorlet{boxcol}{white}
\colorlet{boxcolleft}{boxcol!50!black}
\colorlet{boxcolright}{boxcol}
\colorlet{boxshade}{black}

% useful styles for violin and box plots
\pgfplotsset{
	save row/.code = {\def\currrow{#1}},
	viostyle/.style = {},
	violin extend/.initial = 0.9,
	violin color/.code = {\colorlet{viocol}{#1}},
	violin colors/.code = {
		\setcounter{tmpct}{1}
		\pgfplotsinvokeforeach{#1}{
			\ifnum\value{tmpct}=\currrow
				\colorlet{viocol}{##1}
			\fi
			\stepcounter{tmpct}
		}
	},
	violin colors from table/.code 2 args = {
 		\pgfmathint{\currrow - 1}
    \pgfplotstablegetelem{\pgfmathresult}{#2}\of{#1}
    \pgfplotsset{violin color/.expand once={\pgfplotsretval}}
	},
	violin color left/.code = {\colorlet{viocolleft}{#1}},
	violin color right/.code = {\colorlet{viocolright}{#1}},
	violin color half/.code = {\colorlet{viocolleft}{#1!50!black}\colorlet{viocolright}{#1}},
	violin colors half/.code = {
		\setcounter{tmpct}{1}
		\pgfmathint{(\currrow + 1)/2}
		\pgfplotsinvokeforeach{#1}{
			\ifnum\value{tmpct}=\pgfmathresult
				\pgfkeysalso{violin color half = ##1}
			\fi
			\stepcounter{tmpct}
		}
	},
		violin colors half from table/.code 2 args = {
	 		\pgfmathint{\currrow - 1}
	    \pgfplotstablegetelem{\pgfmathresult}{#2}\of{#1}
	    \pgfplotsset{violin color half/.expand once={\pgfplotsretval}}
		},
	violin color half inverse/.code = {\colorlet{viocolleft}{#1}\colorlet{viocolright}{#1!50!black}},
	violin shade color/.code = {\colorlet{vioshade}{#1}},
	violin shade inverse/.code = {\pgfmathparse{100 - (#1 * (\currrow - 1) / (\datarows - 1))}\pgfkeysalso{viostyle/.estyle = {fill = viocol!\pgfmathresult!vioshade}}},
	violin shade inverse/.default = 50,
	violin shade/.code = {\pgfmathparse{100 - (#1 * (1 - (\currrow - 1) / (\datarows - 1)))}\pgfkeysalso{viostyle/.estyle = {fill = viocol!\pgfmathresult!vioshade}}},
	violin shade/.default = 50,
	boxstyle/.style = {},
	box color/.code = {\colorlet{boxcol}{#1}},
	box colors/.code = {
		\setcounter{tmpct}{1}
		\pgfplotsinvokeforeach{#1}{
			\ifnum\value{tmpct}=\currrow
				\colorlet{boxcol}{##1}
			\fi
			\stepcounter{tmpct}
		}
	},
	box colors from table/.code 2 args = {
 		\pgfmathint{\currrow - 1}
    \pgfplotstablegetelem{\pgfmathresult}{#2}\of{#1}
    \pgfplotsset{box color/.expand once={\pgfplotsretval}}
	},
	box shade color/.code = {\colorlet{boxshade}{#1}},
	box shade inverse/.code = {\pgfmathparse{100 - (#1 * (\currrow - 1) / (\datarows - 1))}\pgfkeysalso{boxstyle/.estyle = {fill = boxcol!\pgfmathresult!boxshade}}},
	box shade inverse/.default = 50,
	box shade/.code = {\pgfmathparse{100 - (#1 * (1 - (\currrow - 1) / (\datarows - 1)))}\pgfkeysalso{boxstyle/.estyle = {fill = boxcol!\pgfmathresult!boxshade}}},
	box shade/.default = 50,
	box color half/.code = {\colorlet{boxcolleft}{#1!50!black}\colorlet{boxcolright}{#1}},
	box 2 colors half/.code n args = {2}{
		\pgfmathint{floor(0.5 * (\currrow - 1))}
		\ifcase\pgfmathresult
			\pgfkeysalso{box color half = #1}
		\or
			\pgfkeysalso{box color half = #2}
		\fi
	},
	thissamplecolor/.store in = \samplecolor,
	sample color table/.style = {visualization depends on = value \thisrow{#1} \as \samplecolor},
	sample color/.style = {visualization depends on = value #1 \as \samplecolor},
	sample 2 colors/.code n args = {2}{
		\pgfkeysalso{
			scatter/@pre marker code/.append style={
				/utils/exec = {
					\pgfmathint{floor(.5 * \id - 1)}
					\ifcase\pgfmathresult
						\pgfkeysalso{thissamplecolor = #1}
					\or
						\pgfkeysalso{thissamplecolor = #2}
					\fi
				}
			}
		}
	}
}


%%% display sample size
% use:	\samplesize[<pgfplots options>]{<table>}{<column with x coords>}{<column with sample size>}
% to shift every second sample size use:	\samplesize[scatter, no marks, visualization depends on = {mod(x, 2) * .5\baselineskip \as \shift}, scatter/@pre marker code/.append style = {/tikz/yshift = \shift}]{<table>}{<column with x coords>}{<column with sample size>}
\newcommand{\samplesize}[4][]{%
	\addplot [black, draw = none, point meta = \thisrow{#4}, sample size, #1] table [x = #3, y expr = \pgfkeysvalueof{/pgfplots/ymin}] {#2};%
}

%% for half violin/box plots
% use:	\samplesizehalf[<pgfplots options>]{<table>}{<column with x coords>}{<column with sample size>}
\newcommand{\samplesizehalf}[4][]{%
	\addplot [black, draw = none, point meta = \thisrow{#4}, sample size, #1] table [
		x expr = {\thisrow{#3} - floor(0.5 * \thisrow{#3}) + .225 - .45 * mod(\thisrow{#3}, 2)},
		y expr = \pgfkeysvalueof{/pgfplots/ymin}
	] {#2};
}

% \samplesizehalfstacked puts one label above the other and centers them under both half plots
% adjust colors with `sample color = ...`, `sample 2 colors = ...`, `sample color table = ...`
\newcommand{\samplesizehalfstacked}[4][]{%
	\addplot [black, draw = none, point meta = \thisrow{#4}, sample size, visualization depends on = \thisrow{#3} \as \id, nodes near coords = \twosamplesize, #1] table [x expr = \thisrow{#3} * .5, y expr = \pgfkeysvalueof{/pgfplots/ymin}] {#2};
}

% stacked display of two sample sizes
\newcommand{\twosamplesize}{%
	\ifthenelse{\isodd{\id}}{%
		\pgfmathfloattosci{\pgfplotspointmeta}%
		\xdef\lastn{\pgfmathresult}%
	}{%
		\textcolor{\samplecolor!50!black}{\pgfmathprintnumber{\lastn}}\\[-.25\baselineskip]%
		\textcolor{\samplecolor}{\pgfmathprintnumber{\pgfplotspointmeta}}%
	}%
}

\def\samplecolor{viocol}

% default style for sample size nodes
\pgfplotsset{
	sample size/.style = {
		nodes near coords,
		nodes near coords style = {
			node font = \figtiny,
			align = center,
			/pgf/number format/fixed,
			/pgf/number format/1000 sep = {}
		}
	}
}


%%% logo plots
% use:	\logoplot[axis options]{file}
% specify size and position in axis options (width, heigth, at, anchor, ...)
% the width of the letters can be changed with option: letter width = ...
% base colors are changed with option: base colors = {A = ..., C = ..., ...}
% axis styles are changed with opiton: logo axis = none (no axes) / default (normal pgfplot axes) / IC (y axis for information content; no x axis; the default setting) / IC and position (y axis for information content and x axis for position)
\newcommand{\logoplot}[2][]{
	\begin{axis}[
		logo axis,
		#1,
		logo plot
	]
		\addlogoplot{#2};
	\end{axis}
}

\newcommand{\addlogoplot}[1]{%
	\getrows{#1}%
	\pgfplotsinvokeforeach{1, ..., 4}{%
		\addplot[%
			scatter,
			scatter src = explicit symbolic,
			only marks,
			mark size = \pgfkeysvalueof{/pgfplots/width} / \datarows * \pgfkeysvalueof{/pgfplots/letter width},
			visualization depends on = \thisrow{IC_##1} * \pgfkeysvalueof{/pgfplots/height} / \ymax \as\baseht
		] table[x = pos,y = IC_##1, meta = base_##1] {#1};%
	}%
}

\colorlet{baseAcol}{Green4}
\colorlet{baseCcol}{Blue2}
\colorlet{baseGcol}{DarkGoldenrod2}
\colorlet{baseTcol}{Red2}
	
\pgfdeclareplotmark{baseA}{%
	\pgftransformxscale{\pgfplotmarksize}
	\pgftransformyscale{\baseht}
	\pgfpathmoveto{\pgfpoint{-.5}{-1}}%
	\pgfpathlineto{\pgfpoint{-.1}{0}}%
	\pgfpathlineto{\pgfpoint{.1}{0}}%
	\pgfpathlineto{\pgfpoint{.5}{-1}}%
	\pgfpathlineto{\pgfpoint{.3}{-1}}%
	\pgfpathlineto{\pgfpoint{.2}{-.75}}%
	\pgfpathlineto{\pgfpoint{-.2}{-.75}}%
	\pgfpathlineto{\pgfpoint{-.3}{-1}}%
	\pgfpathclose
	\pgfpathmoveto{\pgfpoint{.14}{-.6}}%
	\pgfpathlineto{\pgfpoint{0}{-.25}}%
	\pgfpathlineto{\pgfpoint{-.14}{-.6}}%
	\pgfpathclose
	\pgfusepathqfill
}

\pgfdeclareplotmark{baseC}{%
	\pgftransformxscale{\pgfplotmarksize}
	\pgftransformyscale{\baseht}
	\pgfpathmoveto{\pgfpoint{.5}{-.7}}%
	\pgfpathcurveto{\pgfpoint{.4}{-.9}}{\pgfpoint{.3}{-1}}{\pgfpoint{0}{-1}}%
	\pgfpathcurveto{\pgfpoint{-.3}{-1}}{\pgfpoint{-.5}{-.825}}{\pgfpoint{-.5}{-.5}}%
	\pgfpathcurveto{\pgfpoint{-.5}{-.175}}{\pgfpoint{-.3}{0}}{\pgfpoint{0}{0}}%
	\pgfpathcurveto{\pgfpoint{.3}{0}}{\pgfpoint{.4}{-.1}}{\pgfpoint{.5}{-.3}}%
	\pgfpathlineto{\pgfpoint{.3}{-.38}}%
	\pgfpathcurveto{\pgfpoint{.225}{-.25}}{\pgfpoint{.2}{-.15}}{\pgfpoint{0}{-.15}}%
	\pgfpathcurveto{\pgfpoint{-.2}{-.15}}{\pgfpoint{-.3}{-.3}}{\pgfpoint{-.3}{-.5}}%
	\pgfpathcurveto{\pgfpoint{-.3}{-.7}}{\pgfpoint{-.2}{-.85}}{\pgfpoint{0}{-.85}}%
	\pgfpathcurveto{\pgfpoint{.2}{-.85}}{\pgfpoint{.225}{-.75}}{\pgfpoint{.3}{-.62}}%
	\pgfpathclose
	\pgfusepathqfill
}

\pgfdeclareplotmark{baseG}{%
	\pgftransformxscale{\pgfplotmarksize}
	\pgftransformyscale{\baseht}
	\pgfpathmoveto{\pgfpoint{.5}{-.7}}%
	\pgfpathcurveto{\pgfpoint{.4}{-.9}}{\pgfpoint{.3}{-1}}{\pgfpoint{0}{-1}}%
	\pgfpathcurveto{\pgfpoint{-.3}{-1}}{\pgfpoint{-.5}{-.825}}{\pgfpoint{-.5}{-.5}}%
	\pgfpathcurveto{\pgfpoint{-.5}{-.175}}{\pgfpoint{-.3}{0}}{\pgfpoint{0}{0}}%
	\pgfpathcurveto{\pgfpoint{.3}{0}}{\pgfpoint{.4}{-.1}}{\pgfpoint{.5}{-.3}}%
	\pgfpathlineto{\pgfpoint{.3}{-.38}}%
	\pgfpathcurveto{\pgfpoint{.225}{-.25}}{\pgfpoint{.2}{-.15}}{\pgfpoint{0}{-.15}}%
	\pgfpathcurveto{\pgfpoint{-.2}{-.15}}{\pgfpoint{-.3}{-.3}}{\pgfpoint{-.3}{-.5}}%
	\pgfpathcurveto{\pgfpoint{-.3}{-.7}}{\pgfpoint{-.2}{-.85}}{\pgfpoint{0}{-.85}}%
	\pgfpathcurveto{\pgfpoint{.2}{-.85}}{\pgfpoint{.225}{-.75}}{\pgfpoint{.3}{-.62}}%
	\pgfpathclose
	\pgfpathmoveto{\pgfpoint{.5}{-.55}}%
	\pgfpathlineto{\pgfpoint{.5}{-1}}%
	\pgfpathlineto{\pgfpoint{.3}{-1}}%
	\pgfpathlineto{\pgfpoint{.3}{-.7}}%
	\pgfpathlineto{\pgfpoint{0}{-.7}}%
	\pgfpathlineto{\pgfpoint{0}{-.55}}%
	\pgfpathclose
	\pgfusepathqfill
}

\pgfdeclareplotmark{baseT}{%
	\pgftransformxscale{\pgfplotmarksize}
	\pgftransformyscale{\baseht}
	\pgfpathmoveto{\pgfpoint{-.1}{-1}}%
	\pgfpathlineto{\pgfpoint{-.1}{-.15}}%
	\pgfpathlineto{\pgfpoint{-.5}{-.15}}%
	\pgfpathlineto{\pgfpoint{-.5}{0}}%
	\pgfpathlineto{\pgfpoint{.5}{0}}%
	\pgfpathlineto{\pgfpoint{.5}{-.15}}%
	\pgfpathlineto{\pgfpoint{.1}{-.15}}%
	\pgfpathlineto{\pgfpoint{.1}{-1}}%
	\pgfpathclose
	\pgfusepathqfill
}

\pgfplotsset{
	base colors/.code = {
		\pgfkeys{
			/logo plot/.cd,
			#1
		}
	},
	/logo plot/A/.code = {\colorlet{baseAcol}{#1}},
	/logo plot/C/.code = {\colorlet{baseCcol}{#1}},
	/logo plot/G/.code = {\colorlet{baseGcol}{#1}},
	/logo plot/T/.code = {\colorlet{baseTcol}{#1}},
	letter width/.initial = 1,
	logo y axis/.style = {},
	logo x axis/.style = {},
	show IC/.style = {
		logo y axis/.style = {
			ytick = {0, 1, 2},
			ylabel = IC (bits),
			axis y line = left,
			y axis line style = {line cap = round, -},
			ytick align = outside,
			axis y line shift = \pgfkeysvalueof{/pgfplots/major tick length}	
		}
	},
	show pos/.style = {
		logo x axis/.style = {
			axis x line = bottom,
			x axis line style = {draw = none},
			xtick style = {draw = none},
			xtick align = inside
		}
	},
	show sequence/.style 2 args = {
		execute at begin axis/.append = {
			\addplot [
				only marks,
				mark = text,
				text mark as node = true,
				text mark style = {
					anchor = north,
					inner xsep = 0pt,
					node font = \figsmaller
				},
				text mark = \WTbase,
				visualization depends on = value \thisrow{#2} \as \WTbase,
			] table [x = pos, y expr = 0] {#1};
		},
		xticklabel style = {yshift = -.8\baselineskip}
	},
	logo axis/.is choice,
	logo axis/default/.style = {
		logo y axis/.style = {},
		logo x axis/.style = {}
	},
	logo axis/none/.style = {
		logo y axis/.style = {axis y line = none},
		logo x axis/.style = {axis x line = none}
	},
	logo axis/IC/.style = {
		logo x axis/.style = {axis x line = none},
		show IC
	},
	logo axis/position/.style = {
		logo y axis/.style = {axis y line = none},
		show pos
	},
	logo axis/IC and position/.style = {
		show IC,
		show pos
	},
	logo axis/.default = IC,
	logo plot/.style = {
		logo y axis,
		logo x axis,
		stack plots = y,
		ymin = 0,
		ymax = 2,
		grid = none,
		enlarge x limits = {abs = .5},
		scatter/classes = {
			A={mark = baseA, baseAcol},
			C={mark = baseC, baseCcol},
			G={mark = baseG, baseGcol},
			T={mark = baseT, baseTcol},
			a={mark = baseA, baseAcol!50},
			c={mark = baseC, baseCcol!50},
			g={mark = baseG, baseGcol!50},
			t={mark = baseT, baseTcol!50},
			-={no markers}
		}
	},
	save base width/.style = {after end axis/.code = {\pgfmathsetlength{\global\basewidth}{\pgfplotsunitxlength * 10}}}
}


%%% hexbin plots
% \addplot [hexbin] ...
% IMPORTANT: enlargelimits has to be set explicitly for this type of plot
\def\curplotwidth{\pgfkeysvalueof{/pgfplots/width}}
\def\curplotheight{\pgfkeysvalueof{/pgfplots/height}}
\def\xenlarge{\pgfkeysvalueof{/pgfplots/enlarge x limits}}
\def\yenlarge{\pgfkeysvalueof{/pgfplots/enlarge y limits}}

\pgfplotsset{
	hexbin/.style = {
		scatter,
		scatter/use mapped color = {fill = mapped color},
		scatter src = explicit,
		only marks,
		mark = hexagon,
		mark size = (\curplotwidth) / (1 + 2 * \xenlarge) / #1,
		visualization depends on = (\curplotheight) / (\curplotwidth) * (.5 + \xenlarge) / (.5 + \yenlarge) \as \yscale
	},
	hexbin/.default = 50,
	colorbar hexbin/.style = {
		colorbar style = {
	 		name = colorbar,
			anchor = south east,
			at = {(.975, .05)},
			width = .25cm,
			height = .4 * \pgfkeysvalueof{/pgfplots/parent axis height},
			yticklabel pos = left,
			yticklabel style = {node font = \figtiny, inner xsep = .1em},
			ytick = {0, 2, ..., 15},
			yticklabel = \pgfmathparse{2^\tick}\pgfmathprintnumber{\pgfmathresult},
			ylabel = count,
			ylabel style = {node font = \figsmall}
		}
	},
	colorbar hexbin horizontal/.style = {
		colorbar horizontal,
		colorbar style = {
			anchor = south east,
			at = {(.975, .025)},
			height = .25cm,
			width = .4 * \pgfkeysvalueof{/pgfplots/parent axis height},
			tick pos = upper,
			xticklabel pos = right,
			xticklabel style = {node font = \figtiny, inner ysep = .1em, text depth = 0pt},
			xtick = {0, 2, ..., 15},
			xticklabel = \pgfmathparse{2^\tick}\pgfmathprintnumber{\pgfmathresult},
			title = count\\\vphantom{1},
			title style = {fill = none, draw = none, minimum width = 0, node font = \figsmall, align = center, anchor = south east, xshift = .2 * \pgfkeysvalueof{/pgfplots/parent axis height}}
		}
	}
}


%%% commands to draw heatmaps
% first some useful styles
\pgfplotsset{
	heatmap/.style = {
		enlarge x limits = {abs = .5},
		enlarge y limits = false,
		mesh/ordering = colwise,
		colormap name = blue-red,
		symbolic y coords = {X, A, C, G, T},
		ytick = {X, A, C, G, T},
		yticklabels = {\textDelta, A, C, G, T},
		grid = none,
		axis background/.style = {fill = gray!25},
		ylabel style = {align = center},
	},
	colorbar heatmap/.style = {
		colorbar horizontal,
		colorbar shift/.style = {yshift = -1.5\baselineskip},
		colorbar style = {
			name = colorbar,
			anchor = north east,
			at = {(parent axis.south east)},
			height = .25cm,
			width = .4 * \pgfkeysvalueof{/pgfplots/parent axis width},
			ymin = {[normalized]0},
			ymax = {[normalized]1},
			plot graphics/ymin = {[normalized]0},
			plot graphics/ymax = {[normalized]1},
			xticklabel style = {node font = \figtiny, inner sep = .1em},
			after end axis/.code = {\node[anchor = base east, node font = \figsmall] at (0, 0) {$\log_2$(enhancer strength)};},
		}
	}
}

% draw a heatmap for insertion variants (\heatmapIns[<plot options>]{<file>})
\newcommand{\heatmapIns}[2][]{%
	\addplot[%
		matrix plot,%
		mesh/rows = 4,%
		point meta = explicit,%
		line width = 0pt,%
		#1,%
	] table [x = position, y = varNuc, meta = enrichment] {#2};%
}

% draw a heatmap for substitution and deletion variants (\heatmapSubDel[<plot options>]{<file>})
\newcommand{\heatmapSubDel}[2][]{%
	\addplot[%
		matrix plot,%
		mesh/rows = 5,%
		point meta = explicit,%
		line width = 0pt,%
		#1,%
	] table [x = position, y = varNuc, meta = enrichment] {#2};%
}

% annotate WT positions on heatmap (\heatmapWT[<plot options>]{<file>})
\newcommand{\heatmapWT}[2][]{%
	\addplot[%
		only marks,%
		mark = *,%
		gray,%
		mark size = .5,%
		#1,%
	] table [x = position, y = varNuc] {#2};%
}


%%% add correlation statistics to plot (\stats[<options>]{<base name of file (must end in "_stats.tsv")>})
\newcommand{\stats}[2][]{%
	\addplot [
		only marks,
		mark = text,
		text mark as node = true,
		stats position = north west,
		text mark = {
			$n = \pgfmathprintnumber[fixed, 1000 sep = {{{{,}}}}]{\n}$\\
			$\rho = \pgfmathprintnumber[fixed, fixed zerofill, precision = 2]{\spearman}$\\
			$R^2 = \pgfmathprintnumber[fixed, fixed zerofill, precision = 2]{\rsquare}$
		},
		visualization depends on = \thisrow{n} \as \n,
		visualization depends on = \thisrow{spearman} \as \spearman,
		visualization depends on = \thisrow{rsquare} \as \rsquare,
		#1,
	] table [x expr = 0, y expr = 0] {#2_stats.tsv};
}

\pgfplotsset{
	stats position/.style = {
		text mark style = {
			align = left,
			anchor = #1,
			at = (current axis.#1)
		}
	}
}


%%% commands to draw points & mean plots
% \hmeanline[<spread of the line>]{<tikz options>}{<file>}
\newcommand{\hmeanline}[3][.8]{%
	\addplot[very thick, quiver = {v = 0, u = #1}, #2] table [x expr = \thisrow{id} - 0.5 * #1, y = mean] {#3};%
}
% \hjitter[<tikz options>]{<spread of points>}{<file>}{<sample number>}
\newcommand{\hjitter}[4][]{%
	\addplot[gray, only marks, mark = solido, mark size = 1.25, x jitter = 0.4 * #2, #1] table [x expr = #4, y = points.#4] {#3};%
}
% \hpoints[<tikz options>]{<spread of points>}{<file>}{<sample number>}
\newcommand{\hpoints}[4][]{%
	\addplot[gray, only marks, mark = solido, mark size = 1.25, #1] table [x expr = #4 + #2 / \npoints * (\lineno - \npoints / 2 - .5), y = points.#4] {#3};%
}
% \hmandp(*)[<spread of line and points>]{<tikz options>}{<base name of files>}
% the starred version randomizes the x coordinate of the points; the unstarred version orders the points sequentially
\makeatletter
	\newcommand{\hmandp}{%
		\@ifstar
			\hmandpStar%
			\hmandpNoStar%
	}
\makeatother

\newcommand{\hmandpStar}[3][.8]{%
	\hmeanline[#1]{#2}{#3_mean.tsv}%
	\foreachtablerow{#3_mean.tsv}{%
		\hjitter[#2]{#1}{#3_points.tsv}{##1}%
	}%
}
\newcommand{\hmandpNoStar}[3][.8]{%
	\hmeanline[#1]{#2}{#3_mean.tsv}%
	\getrows{#3_points.tsv}%
	\edef\npoints{\datarows}%
	\foreachtablerow{#3_mean.tsv}{%
		\hpoints[#2]{#1}{#3_points.tsv}{##1}%
	}%
}

% \vmeanline[<spread of the line>]{<tikz options>}{<file>}
\newcommand{\vmeanline}[3][.8]{%
	\addplot[very thick, quiver = {u = 0, v = #1}, #2] table [y expr = \thisrow{id} - 0.5 * #1, x = mean] {#3};%
}
% \vjitter[<tikz options>]{<spread of points>}{<file>}{<sample number>}
\newcommand{\vjitter}[4][]{%
	\addplot[gray, only marks, mark = solido, mark size = 1.25, y jitter = 0.4 * #2, #1] table [y expr = #4, x = points.#4] {#3};%
}
% \vpoints[<tikz options>]{<spread of points>}{<file>}{<sample number>}
\newcommand{\vpoints}[4][]{%
	\addplot[gray, only marks, mark = solido, mark size = 1.25, #1] table [y expr = #4 + #2 / \npoints * (\lineno - \npoints / 2 - .5), x = points.#4] {#3};%
}
% \hmandp(*)[<spread of line and points>]{<tikz options>}{<base name of files>}
% the starred version randomizes the x coordinate of the points; the unstarred version orders the points sequentially
\makeatletter
	\newcommand{\vmandp}{%
		\@ifstar
			\vmandpStar%
			\vmandpNoStar%
	}
\makeatother

\newcommand{\vmandpStar}[3][.8]{%
	\vmeanline[#1]{#2}{#3_mean.tsv}%
	\foreachtablerow{#3_mean.tsv}{%
		\vjitter[#2]{#1}{#3_points.tsv}{##1}%
	}%
}
\newcommand{\vmandpNoStar}[3][.8]{%
	\vmeanline[#1]{#2}{#3_mean.tsv}%
	\getrows{#3_points.tsv}%
	\edef\npoints{\datarows}%
	\foreachtablerow{#3_mean.tsv}{%
		\vpoints[#2]{#1}{#3_points.tsv}{##1}%
	}%
}

%%% draw histograms
% \histogram[<plot options>]{<base name of file>}{<count column>}
% use "count" as <count column> for histograms without groups; use the name of a group for histograms with groups
% `histogram type = ybar|xbar|ybar stacked|xbar stacked` has to be called in the axis options (this is usually included in the axes file)
\newcommand{\histogram}[3][]{
	\addplot[histogram = #3, #1] table {#2_hist.tsv};
}

% histogram styles
\pgfplotsset{
	histogram/.style = {thin, draw = black, fill = gray, fill opacity = 0.5, histogram data = #1},
	histogram/.default = count,
	histogram type/.style = {histogram/.append style = #1, histogram coords = #1},
	histogram coords/.is choice,
	histogram coords/ybar/.style = {
		histogram data/.style = {x filter/.expression = \thisrow{x}, y filter/.expression = \thisrow{##1}}
	},
	histogram coords/xbar/.style = {
		histogram data/.style = {y filter/.expression = \thisrow{y}, x filter/.expression = \thisrow{##1}}
	},
	histogram coords/ybar stacked/.style = {histogram coords = ybar},
	histogram coords/xbar stacked/.style = {histogram coords = xbar}
}

%%% table layout
\renewcommand{\arraystretch}{1.33}

\setlength{\aboverulesep}{0pt}
\setlength{\belowrulesep}{0pt}

\newcounter{tblerows}
\expandafter\let\csname c@tblerows\endcsname\rownum % to restore proper behaviour of \rowcolors in tabularx environments

\newcolumntype{L}[1]{>{\raggedright\let\newline\\\arraybackslash\hspace{0pt}}p{#1}}
\newcolumntype{C}[1]{>{\centering\let\newline\\\arraybackslash\hspace{0pt}}p{#1}}
\newcolumntype{R}[1]{>{\raggedleft\let\newline\\\arraybackslash\hspace{0pt}}p{#1}}

%%% declare figure types
\newif\ifnpc

\newif\ifmain
\newcounter{fig}

\newenvironment{fig}{%
	\begin{figure}[p]%
		\stepcounter{fig}%
		\pdfbookmark{\figurename\ \thefig}{figure\thefig}
		\tikzsetnextfilename{figure\thefig}%
		\fignormal%
		\centering%
}{%
	\end{figure}%
	\clearpage%
	\ifnpc%
		\makenextpagecaption%
		\global\npcfalse%
	\fi%
}

\newif\ifsupp
\newcounter{sfig}

\DeclareFloatingEnvironment[fileext = losf, name = Supplemental \figurename]{suppfigure}

\newenvironment{sfig}{%
	\begin{suppfigure}[p]%
		\stepcounter{sfig}%
		\setcounter{subfig}{0}%
		\pdfbookmark{\suppfigurename\ S\thesfig}{supp_figure\thesfig}
		\tikzsetnextfilename{supp_fig\thesfig}%
		\fignormal%
		\centering%
}{%
	\end{suppfigure}%
	\clearpage%
	\ifnpc%
		\makenextpagecaption%
		\global\npcfalse%
	\fi%
}


\newif\ifrev
\newcounter{rfig}

\DeclareFloatingEnvironment[fileext = lorf, name = Rebuttal \figurename]{revfigure}

\newenvironment{rfig}{%
	\begin{revfigure}[p]%
		\stepcounter{rfig}%
		\setcounter{subfig}{0}%
		\pdfbookmark{\revfigurename\ R\therfig}{rebuttal_figure\therfig}
		\tikzsetnextfilename{rebuttal_fig\therfig}%
		\tikzset{jpeg export}%
		\fignormal%
		\centering%
}{%
	\end{revfigure}%
	\clearpage%
	\ifnpc%
		\makenextpagecaption%
		\global\npcfalse%
	\fi%
}

%% command to put the caption on the next page
% use instead of a normal caption: \nextpagecaption{<caption text>}
\makeatletter
	\newcommand{\nextpagecaption}[1]{%
		\global\npctrue%
		\xdef\@npctype{\@currenvir}%
		\captionlistentry{#1}%
		\long\gdef\makenextpagecaption{%
				\csname\@npctype\endcsname%
					\ContinuedFloat%
					\caption{#1}%
				\csname end\@npctype\endcsname%
				\clearpage%
		}%
	}
\makeatother


%%% caption format
\DeclareCaptionJustification{nohyphen}{\hyphenpenalty = 10000}

\captionsetup{
	labelsep = space,
	justification = nohyphen,
	singlelinecheck = false,
	labelfont = bf,
	font = small,
	figureposition = below,
	tableposition = above
}
\captionsetup[figure]{skip = .5\baselineskip}

\newcommand{\titleend}{. }
\newcommand{\nextentry}{ }
\newcommand{\captiontitle}[2][]{#2\titleend #1}


%%% subfigure labels
\newif\ifsubfigupper
\subfiguppertrue

\newcounter{subfig}[figure]

\tikzset{
	subfig label/.style = {anchor = north west, inner sep = 0pt, font = \normalsize\bfseries}
}

\newcommand{\subfiglabel}[2][]{
	\node[anchor = north west, inner sep = 0pt, font = \large\bfseries, #1] at (#2) {\strut\stepcounter{subfig}\ifsubfigupper\Alph{subfig}\else\alph{subfig}\fi};
}

\newcommand{\subfigrefsep}{,}
\newcommand{\subfigrefand}{~and~}
\newcommand{\subfigrefrange}{\textendash}

\newcommand{\subfigunformatted}[1]{\ifsubfigupper\uppercase{#1}\else\lowercase{#1}\fi}
\newcommand{\plainsubfigref}[1]{\textbf{\subfigunformatted{#1}}}
\newcommand{\subfig}[1]{\plainsubfigref{#1}\subfigrefsep}
\newcommand{\subfigtwo}[2]{\plainsubfigref{#1}\subfigrefand\plainsubfigref{#2}\subfigrefsep}
\newcommand{\subfigrange}[2]{\plainsubfigref{#1}\subfigrefrange\plainsubfigref{#2}\subfigrefsep}
\newcommand{\parensubfig}[2][]{(#1\plainsubfigref{#2})}
\newcommand{\parensubfigtwo}[3][]{(#1\plainsubfigref{#2}\subfigrefand\plainsubfigref{#3})}
\newcommand{\parensubfigrange}[3][]{(#1\plainsubfigref{#2}\subfigrefrange\plainsubfigref{#3})}

\newcommand{\supports}[1]{(Supports #1) }


%%% cross-reference setup
\newcommand{\crefrangeconjunction}{\subfigrefrange}
\newcommand{\crefpairgroupconjunction}{\subfigrefand}
\newcommand{\crefmiddlegroupconjunction}{, }
\newcommand{\creflastgroupconjunction}{,\subfigrefand}

\crefname{figure}{Figure}{Figures}
\crefname{suppfigure}{Supplemental Figure}{Supplemental Figures}
\crefname{table}{Table}{Tables}
\crefname{supptable}{Supplemental Table}{Supplemental Tables}
\crefname{suppdata}{Supplemental Data Set}{Supplemental Data Sets}


%%% reference for this paper
% update once we have a DOI (e.g. "Plant Cell (2015). 10.1105/tpc.18.00001.")
\newcommand{\paperref}{}


%%% what to include
\maintrue
\supptrue
\revfalse


%%%%%%%%%%%%%%%%%%%%%%%%%%%%%%%%%%%%
%%%   ||    preamble end    ||   %%%
%%%--\\//------------------\\//--%%%
%%%   \/   begin document   \/   %%%
%%%%%%%%%%%%%%%%%%%%%%%%%%%%%%%%%%%%

\begin{document}
	\sffamily
	\frenchspacing
	
	%%% change names of figures
	\renewcommand\thesuppfigure{S\arabic{suppfigure}}
	\renewcommand\therevfigure{R\arabic{revfigure}}
	
	
	%%% Main figures start
	\ifmain
		
		\begin{fig}
			\begin{tikzpicture}

	%%% scheme of the assay
	%% enhancer diagrams
	\coordinate (enhancers) at (0, 0);
	
	\setlength{\templength}{2cm/250}
	
	% AB80
	\coordinate[shift = {(.4, -1.25)}] (AB80) at (enhancers);
	
	\draw[very thick] (AB80) -- ++(3.25cm, 0) coordinate (TSS) -- ++(.5cm, 0) node[anchor = west, fill = black, text = white, node font = \figsmall\bfseries, minimum width = 1.33cm, outer sep = 0pt] (gene) {\enhancer{AB80}};

	\fill[black] (gene.north east) -- ++(.075, 0) coordinate (gene end) -- ($(gene.north east)!.4!(gene.south east)$) -- ($(gene.north east -| gene end)!.6!(gene.south east -| gene end)$) -- (gene.south east) -- cycle;
	
	\draw[thick, -{Stealth[round]}] (TSS) ++(0, -.2) node[anchor = north, node font = \figtiny] {$+1$} |- ++(.4,.5);
	
	\node[anchor = east, fill = AB80, signal, node font = \figsmall\bfseries, minimum width = 247\templength, outer sep = 0pt, xshift = -101\templength] (FL) at (TSS) {FL};
	
	\draw[thick] (FL.west) ++(0, .2) coordinate (c1) -- ++(0, -.4) node[anchor = north, node font = \figtiny] (p1) {$-347$}
		(FL.west) ++(170\templength, .2) coordinate (c2) -- ++(0, -.4) node[anchor = north, node font = \figtiny] {$-179$}
		(FL.east) ++(-170\templength, .2) coordinate (c3) -- ++(0, -.4) node[anchor = north, node font = \figtiny] {$-269$}
		(FL.east) ++(0, .2) coordinate (c4) -- ++(0, -.4) node[anchor = north, node font = \figtiny] (p2) {$-101$};
		
	\draw[|<->|] (p1.south) ++(0, -.1) -- ($(p2.south) + (0, -.1)$) node[pos = .5, fill = white, text depth = 0pt, node font = \figsmaller] (size) {247 bp};
	
	\node[anchor = east, fill = AB80!75, signal, node font = \figsmall\bfseries, minimum width = 169\templength, outer sep = 0pt, shift = {(0, 1)}] (part A) at (FL) {\segment{A}};
	\node[below = 0pt of part A, node font = \figsmaller] {169 bp};
	
	\node[anchor = west, fill = AB80!75, signal, node font = \figsmall\bfseries, minimum width = 169\templength, outer sep = 0pt, shift = {(.2, 1)}] (part B) at (FL) {\segment{B}};
	\node[below = 0pt of part B, node font = \figsmaller] {169 bp};
	
	\draw[thin, gray] (c1) -- (part A.south west) (c2) -- (part A.east) (c3) -- (part B.south west) (c4) -- (part B.east);
	
	\coordinate[shift = {(.4, .6)}] (pea) at (gene.north west);
	\pea{pea}


	% Cab-1
	\coordinate[yshift = -1.25cm -.5\columnsep] (Cab-1) at (AB80 |- size.south);
	
	\draw[very thick] (Cab-1) -- ++(3.25cm, 0) coordinate (TSS) -- ++(.5cm, 0) node[anchor = west, fill = black, text = white, node font = \figsmall\bfseries, minimum width = 1.33cm, outer sep = 0pt] (gene) {\enhancer{Cab-1}};

	\fill[black] (gene.north east) -- ++(.075, 0) coordinate (gene end) -- ($(gene.north east)!.4!(gene.south east)$) -- ($(gene.north east -| gene end)!.6!(gene.south east -| gene end)$) -- (gene.south east) -- cycle;
	
	\draw[thick, -{Stealth[round]}] (TSS) ++(0, -.2) node[anchor = north, node font = \figtiny] {$+1$} |- ++(.4,.5);
	
	\node[anchor = east, fill = Cab-1, signal, node font = \figsmall\bfseries, minimum width = 268\templength, outer sep = 0pt, xshift = -90\templength] (FL) at (TSS) {FL};
	
	\draw[thick] (FL.west) ++(0, .2) coordinate (c1) -- ++(0, -.4) node[anchor = north, node font = \figtiny] (p1) {$-357$}
		(FL.west) ++(170\templength, .2) coordinate (c2) -- ++(0, -.4) node[anchor = north, node font = \figtiny] {$-189$}
		(FL.east) ++(-170\templength, .2) coordinate (c3) -- ++(0, -.4) node[anchor = north, node font = \figtiny] {$-258$}
		(FL.east) ++(0, .2) coordinate (c4) -- ++(0, -.4) node[anchor = north, node font = \figtiny] (p2) {$-90$};
	
	\draw[|<->|] (p1.south) ++(0, -.1) -- ($(p2.south) + (0, -.1)$) node[pos = .5, fill = white, text depth = 0pt, node font = \figsmaller] (size) {268 bp};
	
	\node[anchor = east, fill = Cab-1!75, signal, node font = \figsmall\bfseries, minimum width = 169\templength, outer sep = 0pt, shift = {(-.1, 1)}] (part A) at (FL) {\segment{A}};
	\node[below = 0pt of part A, node font = \figsmaller] {169 bp};
	
	\node[anchor = west, fill = Cab-1!75, signal, node font = \figsmall\bfseries, minimum width = 169\templength, outer sep = 0pt, shift = {(.1, 1)}] (part B) at (FL) {\segment{B}};
	\node[below = 0pt of part B, node font = \figsmaller] {169 bp};
	
	\draw[thin, gray] (c1) -- (part A.south west) (c2) -- (part A.east) (c3) -- (part B.south west) (c4) -- (part B.east);
	
	\coordinate[shift = {(.4, .6)}] (wheat) at (gene.north west);
	\wheat{wheat}


	% rbcS-E9
	\coordinate[yshift = -1.25cm -.5\columnsep] (rbcS-E9) at (AB80 |- size.south);
	
	\draw[very thick] (rbcS-E9) -- ++(3.25cm, 0) coordinate (TSS) -- ++(.5cm, 0) node[anchor = west, fill = black, text = white, node font = \figsmall\bfseries, minimum width = 1.33cm, outer sep = 0pt] (gene) {\enhancer{rbcS-E9}};

	\fill[black] (gene.north east) -- ++(.075, 0) coordinate (gene end) -- ($(gene.north east)!.4!(gene.south east)$) -- ($(gene.north east -| gene end)!.6!(gene.south east -| gene end)$) -- (gene.south east) -- cycle;
	
	\draw[thick, -{Stealth[round]}] (TSS) ++(0, -.2) node[anchor = north, node font = \figtiny] {$+1$} |- ++(.4,.5);
	
	\node[anchor = east, fill = rbcS-E9, signal, node font = \figsmall\bfseries, minimum width = 234\templength, outer sep = 0pt, xshift = -83\templength] (FL) at (TSS) {FL};
	
	\draw[thick] (FL.west) ++(0, .2) coordinate (c1) -- ++(0, -.4) node[anchor = north, node font = \figtiny] (p1) {$-316$}
		(FL.west) ++(170\templength, .2) coordinate (c2) -- ++(0, -.4) node[anchor = north, node font = \figtiny] {$-148$}
		(FL.east) ++(-170\templength, .2) coordinate (c3) -- ++(0, -.4) node[anchor = north, node font = \figtiny] {$-251$}
		(FL.east) ++(0, .2) coordinate (c4) -- ++(0, -.4) node[anchor = north, node font = \figtiny] (p2) {$-83$};
	
	\draw[|<->|] (p1.south) ++(0, -.1) -- ($(p2.south) + (0, -.1)$) node[pos = .5, fill = white, text depth = 0pt, node font = \figsmaller] (size) {234 bp};
	
	\node[anchor = east, fill = rbcS-E9!75, signal, node font = \figsmall\bfseries, minimum width = 169\templength, outer sep = 0pt, shift = {(-.1, 1)}] (part A) at (FL) {\segment{A}};
	\node[below = 0pt of part A, node font = \figsmaller] {169 bp};
	
	\node[anchor = west, fill = rbcS-E9!75, signal, node font = \figsmall\bfseries, minimum width = 169\templength, outer sep = 0pt, shift = {(.1, 1)}] (part B) at (FL) {\segment{B}};
	\node[below = 0pt of part B, node font = \figsmaller] {169 bp};
	
	\draw[thin, gray] (c1) -- (part A.south west) (c2) -- (part A.east) (c3) -- (part B.south west) (c4) -- (part B.east);
	
	\coordinate[shift = {(.4, .6)}] (pea 2) at (gene.north west);
	\pea{pea 2}
	
	
	%% Plant STARR-seq constructs
	\draw[decorate, decoration = {brace, amplitude = .33cm, raise= .33cm}] ($(pea -| gene end) + (0, .5)$) -- (size -| gene end) coordinate[pos = .5, xshift = .66cm] (curly);
	
	\draw[ultra thick, -Latex] (curly) ++(-.75\pgflinewidth, 0) -- ++(1, 0) coordinate[xshift = .5\columnsep] (constructs);
	
	\coordinate[yshift = 1.75cm] (construct 1) at (constructs);
	\coordinate[yshift = 1.05cm] (construct 2) at (constructs);
	\coordinate[yshift = .35cm] (construct 3) at (constructs);
	\coordinate[yshift = -.35cm] (construct 4) at (constructs);
	\coordinate[yshift = -1.05cm] (construct 5) at (constructs);
	\coordinate[yshift = -1.75cm] (construct 6) at (constructs);
	
	\PSconstruct{construct 1}{AB80}
	
	\draw[Latex-] (c3) ++(-.2, .15) -- ++(0, .4) node[anchor = south, text depth = 0pt] {35S minimal promoter};
	\coordinate (constructs end) at (construct end);
	
	\PSconstruct*[MediumPurple1]{construct 2}{AB80}
	\PSconstruct[HotPink1]{construct 3}{Cab-1}
	\PSconstruct*[MediumOrchid3]{construct 4}{Cab-1}
	\PSconstruct[Plum1]{construct 5}{rbcS-E9}
	\PSconstruct*[SlateBlue3]{construct 6}{rbcS-E9}

	\draw[Latex-] (c2) ++(-.3, -.175) -- ++(0, -.4) node[anchor = north, text depth = 0pt] (enh) {enhancer};
	\node[anchor = base, text depth = 0pt, xshift = .1cm] (bc) at (ORF.west |- enh.base) {barcode};
	\draw[Latex-] (ORF.south west) ++(.1, -.05) -- (bc.north);
	
	
	%% infiltration
	\draw[decorate, decoration = {brace, amplitude = .33cm, raise= .33cm}] ($(construct 1 -| constructs end) + (0, .33)$) -- ($(construct 6 -| constructs end) +(0, -.33)$) coordinate[pos = .5, xshift = .66cm] (curly 2);
	
	% light
	\fill[light] (curly 2 -| \textwidth, 0) ++(-2.2, 2.4) coordinate (light bar bottom left) rectangle ++(2.2, \columnsep) coordinate (light bar top right);
	
	\distance{light bar bottom left}{light bar top right}
	
	\fill[dark] (light bar bottom left) ++(16/48*\xdistance, 0) coordinate (dark bl) rectangle ++(8/48*\xdistance, \ydistance) coordinate (dark tr) ++(16/48*\xdistance, -\ydistance) coordinate (dark 2 bl) rectangle ++(8/48*\xdistance, \ydistance);
	
	\path[node font = \figsmall] ($(light bar bottom left |- light bar top right)!.5!(dark bl)$) node {16h}
		($(dark bl)!.5!(dark tr)$) node[text = white] {8h}
		($(dark tr)!.5!(dark 2 bl)$) node {16h}
		($(dark 2 bl)!.5!(light bar top right)$) node[text = white, inner xsep = 0pt] {8h};
		
	\node[anchor = south east, node font = \bfseries, minimum width = \xdistance, text depth = 0pt] (light) at (light bar top right) {\light};
	
	\coordinate[yshift = -2.2cm] (light bottom) at (light bar bottom left);
	\fill[light, fill opacity = .2] (light bar bottom left) rectangle (light bottom -| dark bl)
		(light bar bottom left -| dark tr) rectangle (light bottom -| dark 2 bl);
	\fill[dark, fill opacity = .2] (dark bl) rectangle (light bottom -| dark tr)
		(dark 2 bl) rectangle (light bottom -| light bar top right);
	
	\coordinate[shift = {(1.1, -1.1)}] (leaf 1) at (light bar bottom left);
	\leaf[.5]{leaf 1}
	
	% dark
	\fill[dark] (curly 2 -| light bar bottom left) ++(0, -2.4) coordinate (dark bar top left) rectangle ++(2.2, -\columnsep) coordinate (dark bar bottom right);
	
	\path[node font = \figsmall] ($(dark bar top left)!.5!(dark bar bottom right)$) node[text = white] {48h};
		
	\node[anchor = north east, node font = \bfseries, minimum width = \xdistance] at (dark bar bottom right) {\dark};
	
	\coordinate[yshift = 2.2cm] (dark top) at (dark bar top left);
	\fill[dark, fill opacity = .2] (dark bar top left) rectangle (dark top -| dark bar bottom right);
	
	\coordinate[shift = {(1.1, 1.1)}] (leaf 2) at (dark bar top left);
	\leaf[.5]{leaf 2}
	
	% arrows
	\draw[ultra thick, -Latex] (curly 2) ++(-.75\pgflinewidth, 0) to[out = 0, in = 180] ($(leaf 1) + (-1.2, 0)$);
	\draw[ultra thick, -Latex] (curly 2) ++(-.75\pgflinewidth, 0) to[out = 0, in = 180] ($(leaf 2) + (-1.2, 0)$);

	
	%%% plant enhancer strength in light/dark
	\coordinate[yshift = -\columnsep] (light/dark) at (enhancers |- size.south);

	\def\shadeplot{%
		\fill[light, fill opacity = .1] (\xmin, \ymin) rectangle (1, \ymax) (1.5, \ymin) rectangle (2, \ymax) (2.5, \ymin) rectangle (3, \ymax);
		\fill[dark, fill opacity = .1] (1, \ymin) rectangle (1.5, \ymax) (2, \ymin) rectangle (2.5, \ymax) (3, \ymin) rectangle (\xmax, \ymax);
		\draw (1.5, \ymin) -- (1.5, \ymax) (2.5, \ymin) -- (2.5, \ymax);
	}

	\begin{hgroupplot}[%
		group position = {anchor = above north west, at = {(light/dark)}, xshift = \plotylabelwidth},
		axis limits from table = {rawData/PEfl_strength_fwd_axes.tsv},
		enlarge y limits = {value = .125, lower},
		ytick = {-2, 0, ..., 10},
		group/every plot/.append style = {x grids = false},
		zero line,
		ylabel = $\log_2$(enhancer strength),
		ylabel style = {xshift = .5em}
	]{\twothirdcolumnwidth}{4}{enhancer segment}
		
		\nextgroupplot[
			title = \enhancer{35S},
			width = 4.8/10.8*\plotwidth,
			title style = {minimum width = 4.8/10.8*\plotwidth},
			title color = 35S,
			x tick table half segment = {rawData/PEfl_strength_fwd_35S_boxplot.tsv}{part},
			execute at begin axis/.append = {
				\fill[light, fill opacity = .1] (\xmin, \ymin) rectangle (1, \ymax);
				\fill[black, fill opacity = .1] (1, \ymin) rectangle (\xmax, \ymax);
			}
		]
		
			% boxplots
			\halfboxplots[]{%
				box color = 35S,
				fill opacity = .5
			}{rawData/PEfl_strength_fwd_35S}
			
			% add sample size
			\samplesizehalf[nodes near coords style = {name = samplesize}]{rawData/PEfl_strength_fwd_35S_boxplot.tsv}{id}{n}
			
			% save coordinates
			\coordinate (35S) at (1, 0);
			\coordinate (35S dist) at (1.225, 0);
		
		
		\nextgroupplot[
			title = \enhancer{AB80},
			width = 12.8/10.8*\plotwidth,
			title style = {minimum width = 12.8/10.8*\plotwidth},
			title color = AB80,
			x tick table half segment = {rawData/PEfl_strength_fwd_AB80_boxplot.tsv}{part},
			execute at begin axis/.append = {\shadeplot}
		]
		
			% boxplots
			\halfboxplots[]{%
				box color = AB80,
				fill opacity = .5
			}{rawData/PEfl_strength_fwd_AB80}
			
			% add sample size
			\samplesizehalf{rawData/PEfl_strength_fwd_AB80_boxplot.tsv}{id}{n}
			
			% save coordinates
			\pgfplotsinvokeforeach{1, ..., 3}{		
				\coordinate (AB80_#1) at (#1, 0);
			}
		
		
		\nextgroupplot[
			title = \enhancer{Cab-1},
			width = 12.8/10.8*\plotwidth,
			title style = {minimum width = 12.8/10.8*\plotwidth},
			title color = Cab-1,
			x tick table half segment = {rawData/PEfl_strength_fwd_Cab-1_boxplot.tsv}{part},
			execute at begin axis/.append = {\shadeplot}
		]
		
			% boxplots
			\halfboxplots[]{%
				box color = Cab-1,
				fill opacity = .5
			}{rawData/PEfl_strength_fwd_Cab-1}
			
			% add sample size
			\samplesizehalf{rawData/PEfl_strength_fwd_Cab-1_boxplot.tsv}{id}{n}
			
			% save coordinates
			\pgfplotsinvokeforeach{1, ..., 3}{		
				\coordinate (Cab-1_#1) at (#1, 0);
			}
		
		
		\nextgroupplot[
			title = \enhancer{rbcS-E9},
			width = 12.8/10.8*\plotwidth,
			title style = {minimum width = 12.8/10.8*\plotwidth},
			title color = rbcS-E9,
			x tick table half segment = {rawData/PEfl_strength_fwd_rbcS-E9_boxplot.tsv}{part},
			execute at begin axis/.append = {\shadeplot}
		]
		
			% boxplots
			\halfboxplots[]{%
				box color = rbcS-E9,
				fill opacity = .5
			}{rawData/PEfl_strength_fwd_rbcS-E9}
			
			% add sample size
			\samplesizehalf{rawData/PEfl_strength_fwd_rbcS-E9_boxplot.tsv}{id}{n}
			
			% save coordinates
			\pgfplotsinvokeforeach{1, ..., 3}{		
				\coordinate (rbcS-E9_#1) at (#1, 0);
			}
			
	\end{hgroupplot}
	
	% light labels
	\distance{35S}{35S dist}
	
	\node[anchor = south, font = \bfseries\vphantom{A}, xshift = -\xdistance, text depth = 0pt] (lplus) at (group c1r1.south -| 35S) {+};
	\node[anchor = south, font = \bfseries\vphantom{A}, xshift = \xdistance, text depth = 0pt] at (group c1r1.south -| 35S) {\textminus};		
	
	\foreach \enh in {AB80, Cab-1, rbcS-E9}{
		\foreach \x in {1, ..., 3}{
			\node[anchor = south, font = \bfseries\vphantom{A}, xshift = -\xdistance, text depth = 0pt] at (group c1r1.south -| \enh_\x) {+};
			\node[anchor = south, font = \bfseries\vphantom{A}, xshift = \xdistance, text depth = 0pt] at (group c1r1.south -| \enh_\x) {\textminus};		
		}
	}
	
	\node[anchor = base east, node font = \figsmall, text depth = 0pt] at (group c1r1.west |- lplus.base) {light};
	
	\foreach \x in {1, ..., 4}{
		\draw[dashed] (group c\x r1.west |- samplesize.south) -- (group c\x r1.east |- samplesize.south);
	}
	
	%%% light-responsiveness
	\coordinate (light-resp) at (light/dark -| \textwidth - \threecolumnwidth, 0);
	
	\node[anchor = north west, node font = \figsmall] (penh) at (light-resp) {\phantom{enhancer}};
	
	\distance{penh.north east}{\textwidth, 0}
	
	\begin{axis}[%
		anchor = above north west,
		at = {(penh.north east)},
		width = \xdistance,
		axis limits from table = {rawData/PEfl_lightResp_fwd_axes.tsv},
		ytick = {-10, -8, ..., 10},
		x tick table segment = {rawData/PEfl_LightResp_fwd_boxplot.tsv}{part},
		xticklabel style = {name = xticklabel},
		x grids = false,
		zero line,
		title = light-responsiveness,
		title style= {minimum width = \xdistance, left color = light!20, right color = dark!20},
		ylabel = $\log_2$(light-responsiveness)
	]
	
			% boxplots
			\boxplots[]{%
				box colors from table = {rawData/PEfl_lightResp_fwd_boxplot.tsv}{enhancer},
				fill opacity = .5
			}{rawData/PEfl_lightResp_fwd}
			
			% add sample size
			\samplesize{rawData/PEfl_LightResp_fwd_boxplot.tsv}{id}{n}
			
			% save coordinates
			\coordinate (c1) at (1, 0);
			\coordinate (c2) at (2, 0);
	
	\end{axis}
	
	\node[anchor = base east, node font = \figsmall] (lpart) at (last plot.west |- xticklabel.base) {segment};
	\node[anchor = north east, node font = \figsmall] (lenh) at (lpart.south east) {enhancer};
	
	\distance{c1}{c2}
	
	\foreach \x/\len in {0/1, 1/3, 4/3, 7/3}{
		\draw[serif cm-serif cm, shorten both = .1\xdistance] (xticklabel.south -| c1) ++(\xdistance * \x  - .5\xdistance, 0) -- ++(\len * \xdistance, 0) coordinate[pos = .5] (mid); 
		\node[anchor = base, node font = \figsmall] at (mid |- lenh.base) {\pgfplotstablegetelem{\x}{enhancer}\of{rawData/PEfl_lightResp_fwd_boxplot.tsv}\enhancer{\pgfplotsretval}};
	}
	
	
	%%% dual-luciferase construct
	\coordinate[yshift = -\columnsep] (DL construct) at (enhancers |- xlabel.south);
	
	\coordinate[shift = {(.1, -1)}] (construct) at (DL construct);
	
	\draw[dashed, thick] (construct) -- ++(.5, 0) coordinate (c1);
	
	\node[draw = black, thin, anchor = west, text depth = 0pt] (Luc) at ($(c1) + (.1, 0)$) {Luc};
	
	\draw[line width = .2cm, DarkOliveGreen3] (Luc.east) ++(.4, 0) coordinate (c2) -- ++(.5, 0) coordinate[pos = .5] (pro1) coordinate (c3);
	\draw[-{Stealth[round]}, thick] (c2) ++(-.5\pgflinewidth, 0) |- ++(-.4, .3);
	
	\node[draw = black, thin, anchor = west, text depth = 0pt] (BlpR) at ($(c3) + (.3, 0)$) {BlpR};
	
	\draw[line width = .2cm, PaleGreen3] (BlpR.east) ++(.4, 0) coordinate (c4) -- ++(.5, 0) coordinate (c5);
	\draw[-{Stealth[round]}, thick] (c4) ++(-.5\pgflinewidth, 0) |- ++(-.4, .3);
	
	\draw[line width = .25cm, -Triangle Cap, AB80!75]  (c5) ++(.4, 0) -- ++(.6, 0) coordinate[pos = .5] (enh) coordinate (c6);
	
	\draw[line width = .2cm, 35S promoter] (c6) ++(.1, 0) -- ++(.3, 0) coordinate[pos = .5] (pro2) coordinate (c7);
	\draw[-{Stealth[round]}, thick] (c7) ++(.5\pgflinewidth, 0) |- ++(.4, .3);
	
	\node[draw = black, thin, anchor = west, text depth = 0pt] (NLuc) at ($(c7) + (.4, 0)$) {NanoLuc};
	
	\draw[dashed, thick] (NLuc.east) ++(.1, 0) coordinate (c8) -- ++(.5, 0) coordinate (construct end);
	
	\begin{pgfonlayer} {background}
		\draw[thick] (c1) -- (Luc.west) (Luc.east) -- (BlpR.west) (BlpR.east) -- (NLuc.west) (NLuc.east) -- (c8);
	\end{pgfonlayer}
	
	\draw[Latex-] (enh) ++(0, .175) -- ++(0, .4) node[anchor = south, text depth = 0pt] {enhancer};
	\draw[Latex-] (pro1) ++(0, -.15) -- ++(0, -.4) node[anchor = north, text depth = 0pt] {\textit{AtUBQ10} promoter};
	\draw[Latex-] (pro2) ++(0, -.15) -- ++(0, -.4) node[anchor = north, text depth = 0pt] (l35Spr) {35S minimal promoter};

	\draw[decorate, decoration = {brace, amplitude = .4cm, aspect = .825}] (l35Spr.south -| construct end) -- (l35Spr.south -| construct) coordinate[pos = .825, yshift = -.4cm] (curly);
	
	\draw[ultra thick, -Latex] (curly) ++(0, .75\pgflinewidth) -- ++(0, -.3) arc[start angle = 180, end angle = 270, radius = .8cm] -- ++(.3, 0) coordinate[shift = {(1, .1)}] (Arabidopsis);
	\arabidopsis[.9]{Arabidopsis};
	\node [anchor = north east, align = center, inner xsep = 0pt, xshift = .1cm] at (curly) {integrate\\into\\Arabidopsis\\genome};
	
	\draw[ultra thick, -Latex] (Arabidopsis) ++(1, -.1) -- ++(.8, 0) node[anchor = west, align = center] {measure\\luciferase (Luc)\\and nanoluciferase\\(NanoLuc) activity};
		
	
	%%% dual-luciferase plot
	\coordinate (DL plot) at (DL construct -| \textwidth - \twocolumnwidth, 0);
	
	\begin{axis}[%
		anchor = north west,
		at = {(DL plot)},
		width = \fourcolumnwidth - \plotylabelwidth,
		xshift = \plotylabelwidth,
		x grids = false,
		axis limits from table = {rawData/DL_enhancers_axes.tsv},
		zero line,
		ylabel = $\log_2$(NanoLuc/Luc),
		x tick table enhancer = {rawData/DL_enhancers_boxplot.tsv}{sample},
		xticklabel style = {rotate = 45, anchor = north east}
	]

		% boxplots
		\boxplots{%
			box colors from table = {rawData/DL_enhancers_boxplot.tsv}{sample},
			fill opacity = .5
		}{rawData/DL_enhancers}
		
		% add sample size
		\samplesize[nodes near coords style = {name = samplesize}]{rawData/DL_enhancers_boxplot.tsv}{id}{n}

	\end{axis}
	
	
	%%% correlation between Plant STARR-seq and dual-luciferase assay
	\coordinate (DL cor) at (DL construct -| \textwidth - \fourcolumnwidth, 0);
	
	\begin{axis}[
		anchor = north west,
		at = {(DL cor)},
		width = \fourcolumnwidth - \plotylabelwidth,
		xshift = \plotylabelwidth,
		axis limits from table = {rawData/DL_enhancers_cor_axes.tsv},
		enlargelimits = .1,
		xlabel = {\vphantom{$\log_2$(enhancer strength)}},
		xlabel style = {name = xlabel},
		ylabel = {$\log_2$(NanoLuc/Luc)},
		scatter/classes = {
			none={black},
			35S={35S enhancer},
			AB80={AB80},
			Cab-1={Cab-1},
			rbcS-E9={rbcS-E9}
		},
		legend style = {anchor = south east, at = {(1, 0)}},
		legend image post style = {fill opacity = 1, mark size = 1.25},
		legend plot pos = right,
		legend cell align = right,
	]
	
		% regression line
		\addplot [draw = none, forget plot] table [x = enrichment, y = {create col/linear regression = { y = l2ratio}}] {rawData/DL_enhancers_cor_points.tsv};
		
		\draw[gray, dashed] (\xmin, \xmin * \pgfplotstableregressiona + \pgfplotstableregressionb) -- (\xmax, \xmax * \pgfplotstableregressiona + \pgfplotstableregressionb);
	
		% scatter plot
		\addplot [
			scatter,
			scatter src = explicit symbolic,
			only marks,
			mark = solido,
			mark size = 2,
			error bars/y dir = both,
			error bars/y explicit,
		] table[x = enrichment, y = l2ratio, meta = enhancer, y error = CI] {rawData/DL_enhancers_cor_points.tsv};
		
		% correlation
		\stats{rawData/DL_enhancers_cor}
		
		% legend
		\legend{\enhancer{none}, \enhancer{35S}, \enhancer{AB80}, \enhancer{Cab-1}, \enhancer{rbcS-E9}}
	
	\end{axis}
	
	\node[anchor = north east, inner xsep = 0pt] at (\textwidth, 0 |- xlabel.north) {$\log_2$(enhancer strength)};
	
	
	%%% subfigure labels
	\subfiglabel{enhancers |- light.north}
	\subfiglabel{light/dark}
	\subfiglabel{light-resp}
	\subfiglabel{DL construct}
	\subfiglabel{DL plot}
	\subfiglabel{DL cor}
	
\end{tikzpicture}%
			\nextpagecaption{%
				\captiontitle{Enhancers from photosynthesis genes show light-responsive activity}%
				\subfigtwo{A}{B} Full-length (FL) enhancers, as well as 169-bp long segments from their 5\textprime{} or 3\textprime{} end, of the \textit{Pisum sativum} \enhancer{AB80} and \enhancer{rbcS-E9} genes and the \textit{Triticum aestivum} \enhancer{Cab-1} gene were cloned upstream of the 35S minimal promoter driving the expression of a barcoded GFP reporter gene \parensubfig{A}. All constructs were pooled and the viral 35S enhancer was added as an internal control. The pooled enhancer library was subjected to Plant STARR-seq in tobacco leaves with plants grown for 2 days in normal light/dark cycles ($+$ light) or completely in the dark (\textminus{} light) prior to RNA extraction \parensubfig{B}. Enhancer strength was normalized to a control construct without an enhancer ($\log_2$ set to 0).\nextentry
				\subfig{C} Light-responsiveness ($\log_2$[enhancer strength\textsuperscript{\light}/enhancer strength\textsuperscript{\dark}]) was determined for the indicated enhancer segments.\nextentry
				\subfig{D} Transgenic \textit{Arabidopsis} lines were generated with T-DNAs harboring a constitutively expressed luciferase (Luc) gene and a nanoluciferase (NanoLuc) gene under control of a 35S minimal promoter coupled to the 35S enhancer or the \segment{B} segments of the \enhancer{AB80}, \enhancer{Cab-1}, or \enhancer{rbcS-E9} enhancers.\nextentry
				\subfig{E} Nanoluciferase activity was measured in 5 T2 plants from these lines and normalized to the activity of luciferase. The NanoLuc/Luc ratio was normalized to a control construct without an enhancer (none; $\log_2$ set to 0).\nextentry
				\subfig{F} The mean NanoLuc/Luc ratio was compared to the mean enhancer strength determined by STARR-seq. Pearson's $R^2$, Spearman's $\rho$, and number ($n$) of enhancers are indicated. A linear regression line is shown as a dashed line.  Error bars represent the 95\% confidence interval.\nextentry
				Box plots in \plainsubfigref{B}, \plainsubfigref{C}, and \plainsubfigref{E} represent the median (center line), upper and lower quartiles (box limits), 1.5$\times$ interquartile range (whiskers), and outliers (points) for all corresponding barcodes \parensubfigtwo{B}{C} or plant lines \parensubfig{E} from two \parensubfigtwo{B}{C} or three \parensubfig{E} independent replicates. Numbers at the bottom of each box plot indicate the number of samples in each group.
			}%
			\label{fig:PEfl_fwd}%
		\end{fig}
		
		\begin{fig}
			\begin{tikzpicture}

	%%% mutation sensitivity (positional mean) AB80
	\coordinate (AB80) at (0, 0);
	
	\begin{hgroupplot}[%
		group position = {anchor = above north west, at = {(AB80)}, xshift = \plotylabelwidth},
		axis limits from table = {rawData/PEV_ld_mutSens_AB80_axes.tsv},
		enlarge y limits = {value = .2, lower},
		ytick = {-10, ..., 10},
		ylabel = {$\log_2$(enhancer strength)},
		title color = AB80,
		zero line,
		legend style = {anchor = south west, at = {(0, 0)}},
		legend image post style = {very thick},
		legend plot pos = left,
		legend cell align = left,
	]{\textwidth}{2}{position}
	
		\nextgroupplot[
			title = \enhancer{AB80} enhancer \segment{A} segment,
			axis limits = AB80 A,
			shade overlap A = AB80
		]
			
			\addplot[line plot, dark] table [x = position, y = dark] {rawData/PEV_ld_mutSens_AB80_A_lines.tsv};
			\addplot[line plot, light] table [x = position, y = light] {rawData/PEV_ld_mutSens_AB80_A_lines.tsv};
						
			\legend{dark, light}
			
			
		\nextgroupplot[
			title = \enhancer{AB80} enhancer \segment{B} segment,
			axis limits = AB80 B,
			shade overlap B = AB80
		]
		
			\enhFrag[1]{168}{23}{AB80}{a}
			\enhFrag[1]{193}{19}{AB80}{b}
			\enhFrag[2]{205}{19}{AB80}{c}
			\enhFrag[1]{228}{23}{AB80}{d}
		
			\addplot[line plot, dark] table [x = position, y = dark] {rawData/PEV_ld_mutSens_AB80_B_lines.tsv};
			\addplot[line plot, light] table [x = position, y = light] {rawData/PEV_ld_mutSens_AB80_B_lines.tsv};
		
	\end{hgroupplot}
	
	
	%%% mutation sensitivity (positional mean) Cab-1
	\coordinate[yshift = -\columnsep] (Cab-1) at (AB80 |- xlabel.south);
	
	\begin{hgroupplot}[%
		group position = {anchor = above north west, at = {(Cab-1)}, xshift = \plotylabelwidth},
		axis limits from table = {rawData/PEV_ld_mutSens_Cab-1_axes.tsv},
		enlarge y limits = {value = .125, lower},
		ytick = {-10, ..., 10},
		ylabel = {$\log_2$(enhancer strength)},
		title color = Cab-1,
		zero line,
		legend style = {anchor = south west, at = {(0, 0)}},
		legend image post style = {very thick},
		legend plot pos = left,
		legend cell align = left,
	]{\textwidth}{2}{position}
	
		\nextgroupplot[
			title = \enhancer{Cab-1} enhancer \segment{A} segment,
			axis limits = Cab-1 A,
			shade overlap A = Cab-1
		]
		
			\addplot[line plot, dark] table [x = position, y = dark] {rawData/PEV_ld_mutSens_Cab-1_A_lines.tsv};
			\addplot[line plot, light] table [x = position, y = light] {rawData/PEV_ld_mutSens_Cab-1_A_lines.tsv};
						
			\legend{dark, light}
			
			
		\nextgroupplot[
			title = \enhancer{Cab-1} enhancer \segment{B} segment,
			axis limits = Cab-1 B,
			shade overlap B = Cab-1
		]
		
			\enhFrag[1]{149}{19}{Cab-1}{a}
			\enhFrag[1]{179}{21}{Cab-1}{b}
			\enhFrag[2]{193}{19}{Cab-1}{c}
			\enhFrag[1]{220}{25}{Cab-1}{d}
			\enhFrag[2]{239}{19}{Cab-1}{e}
		
			\addplot[line plot, dark] table [x = position, y = dark] {rawData/PEV_ld_mutSens_Cab-1_B_lines.tsv};
			\addplot[line plot, light] table [x = position, y = light] {rawData/PEV_ld_mutSens_Cab-1_B_lines.tsv};
		
	\end{hgroupplot}
	
	
	%%% mutation sensitivity (positional mean) rbcS-E9
	\coordinate[yshift = -\columnsep] (rbcS-E9) at (AB80 |- xlabel.south);
	
	\begin{hgroupplot}[%
		group position = {anchor = above north west, at = {(rbcS-E9)}, xshift = \plotylabelwidth},
		axis limits from table = {rawData/PEV_ld_mutSens_rbcS-E9_axes.tsv},
		enlarge y limits = {value = .125, lower},
		ytick = {-10, ..., 10},
		ylabel = {$\log_2$(enhancer strength)},
		title color = rbcS-E9,
		zero line,
		legend style = {anchor = south west, at = {(0, 0)}},
		legend image post style = {very thick},
		legend plot pos = left,
		legend cell align = left,
	]{\textwidth}{2}{position}
	
		\nextgroupplot[
			title = \enhancer{rbcS-E9} enhancer \segment{A} segment,
			axis limits = rbcS-E9 A,
			shade overlap A = rbcS-E9
		]
			
			\addplot[line plot, dark] table [x = position, y = dark] {rawData/PEV_ld_mutSens_rbcS-E9_A_lines.tsv};
			\addplot[line plot, light] table [x = position, y = light] {rawData/PEV_ld_mutSens_rbcS-E9_A_lines.tsv};
						
			\legend{dark, light}
			
			
		\nextgroupplot[
			title = \enhancer{rbcS-E9} enhancer \segment{B} segment,
			axis limits = rbcS-E9 B,
			shade overlap B = rbcS-E9
		]
			
			\enhFrag[1]{109}{17}{rbcS-E9}{a}
			\enhFrag[2]{120}{17}{rbcS-E9}{b}
			\enhFrag[1]{141}{21}{rbcS-E9}{c}
			\enhFrag[1]{173}{25}{rbcS-E9}{d}
			\enhFrag[2]{193}{21}{rbcS-E9}{e}
			
			\addplot[line plot, dark] table [x = position, y = dark] {rawData/PEV_ld_mutSens_rbcS-E9_B_lines.tsv};
			\addplot[line plot, light] table [x = position, y = light] {rawData/PEV_ld_mutSens_rbcS-E9_B_lines.tsv};
		
	\end{hgroupplot}
	
	
	%%% subfigure labels
	\subfiglabel{AB80}
	\subfiglabel{Cab-1}
	\subfiglabel{rbcS-E9}
	
\end{tikzpicture}%
			\caption{%
				\captiontitle{The \enhancer{AB80}, \enhancer{Cab-1}, and \enhancer{rbcS-E9} enhancers contain multiple mutation-sensitive regions}%
				\subfigrange{A}{C} All possible single-nucleotide substitution, deletion, and insertion variants of the \segment{A} and \segment{B} segments of the \enhancer{AB80} \parensubfig{A}, \enhancer{Cab-1} \parensubfig{B}, and \enhancer{rbcS-E9} \parensubfig{C} enhancers were subjected to Plant STARR-seq in tobacco plants grown in normal light/dark cycles (\light) or completely in the dark (\dark) for two days prior to RNA extraction. Enhancer strength was normalized to the wild-type variant ($\log_2$ set to 0). A sliding average (window size = 6 bp) of the mean enhancer strength for all variants at a given position is shown. The shaded area indicates the region where the \segment{A} and \segment{B} segments overlap. Mutation-sensitive regions in the \segment{B} enhancer segments are indicated by shaded rectangles labeled a\textendash e.
			}%
			\label{fig:PEV_ld_MutSens}%
		\end{fig}
		
		\begin{fig}
			%%% command to draw grouped logo plots
% \logogroupplot[<plot options>]{<enhancer>}{<region>}{<condition>}{<relative plot width>}
\newcommand{\logogroupplot}[5][]{
	\nextgroupplot[
		width = #5*\plotwidth,
		title = {\enhancer{#2} region \textbf{#3}},
		title style = {minimum width = #5*\plotwidth},
		show sequence = {rawData/PEV_ld_seqLogo_#2_#3_#4.tsv}{WT},
		alias = {logo #2 #3},
		#1,
	]
	
		\addlogoplot{rawData/PEV_ld_seqLogo_#2_#3_#4.tsv};
}

\begin{tikzpicture}
	\setlength{\groupplotsep}{\columnsep}

	%%% fragment overview
	\coordinate (overview) at (0, 0);
	
	\begin{hgroupplot}[%
		group position = {anchor = above north west, at = {(overview)}, xshift = \columnsep},
		height = 1.5cm,
		axis lines = none,
		title style = {signal, anchor = south west, at = {(0, 1)}, minimum width = \plotwidth - 1.25\pgflinewidth},
		zero line,
		clip = false,
		legend style = {anchor = south west, at = {(0, 0)}},
		legend image post style = {very thick},
		legend plot pos = left,
		legend cell align = left,
		enlarge y limits = .05
	]{\textwidth + \plotylabelwidth - \columnsep}{3}{}
		
		\nextgroupplot[
			title = \enhancer{AB80} enhancer \segment{B} segment,
			title color = AB80,
			axis limits from table = {rawData/PEV_ld_mutSens_AB80_recap_axes.tsv},
			axis limits = AB80 B,
		]
		
			\enhFrag[-1]{168}{23}{AB80}{a}
			\enhFrag[-1]{193}{19}{AB80}{b}
			\enhFrag[0]{205}{19}{AB80}{c}
			\enhFrag[-1]{228}{23}{AB80}{d}
			
			\pgfplotsinvokeforeach{a, b, c, d}{
				\fill[AB80, fill opacity = .1] (AB80 #1.north west) rectangle (AB80 #1.east |- 0, \ymax);
			}
			
			\addplot[line plot, dark] table [x = position, y = dark] {rawData/PEV_ld_mutSens_AB80_B_lines.tsv};
			\addplot[line plot, light] table [x = position, y = light] {rawData/PEV_ld_mutSens_AB80_B_lines.tsv};
			
			\legend{dark, light}
			
			
		\nextgroupplot[
			title = \enhancer{Cab-1} enhancer \segment{B} segment,
			title color = Cab-1,
			axis limits from table = {rawData/PEV_ld_mutSens_Cab-1_recap_axes.tsv},
			axis limits = Cab-1 B,
		]
		
			\enhFrag[-1]{149}{19}{Cab-1}{a}
			\enhFrag[-1]{179}{21}{Cab-1}{b}
			\enhFrag[0]{193}{19}{Cab-1}{c}
			\enhFrag[-1]{220}{25}{Cab-1}{d}
			\enhFrag[0]{239}{19}{Cab-1}{e}
			
			\pgfplotsinvokeforeach{a, b, c, d, e}{
				\fill[Cab-1, fill opacity = .1] (Cab-1 #1.north west) rectangle (Cab-1 #1.east |- 0, \ymax);
			}
		
			\addplot[line plot, dark] table [x = position, y = dark] {rawData/PEV_ld_mutSens_Cab-1_B_lines.tsv};
			\addplot[line plot, light] table [x = position, y = light] {rawData/PEV_ld_mutSens_Cab-1_B_lines.tsv};
			
			
		\nextgroupplot[
			title = \enhancer{rbcS-E9} enhancer \segment{B} segment,
			title color = rbcS-E9,
			axis limits from table = {rawData/PEV_ld_mutSens_rbcS-E9_recap_axes.tsv},
			axis limits = rbcS-E9 B,
		]
			
			\enhFrag[-1]{109}{17}{rbcS-E9}{a}
			\enhFrag[0]{120}{17}{rbcS-E9}{b}
			\enhFrag[-1]{141}{21}{rbcS-E9}{c}
			\enhFrag[-1]{173}{25}{rbcS-E9}{d}
			\enhFrag[0]{193}{21}{rbcS-E9}{e}
			
			\pgfplotsinvokeforeach{a, b, c, d, e}{
				\fill[rbcS-E9, fill opacity = .1] (rbcS-E9 #1.north west) rectangle (rbcS-E9 #1.east |- 0, \ymax);
			}
		
			\addplot[line plot, dark] table [x = position, y = dark] {rawData/PEV_ld_mutSens_rbcS-E9_B_lines.tsv};
			\addplot[line plot, light] table [x = position, y = light] {rawData/PEV_ld_mutSens_rbcS-E9_B_lines.tsv};
		
	\end{hgroupplot}
	
	
	%%% AB80 sequence logos
	\coordinate[yshift = -\columnsep] (AB80 logos) at (overview |- AB80 a.south);

	\begin{hgroupplot}[%
		group position = {anchor = above north west, at = {(AB80 logos)}, yshift = -\columnsep},
		height = .67cm,
		title style = {fill = AB80!50, draw = none},
		axis background/.style = {fill = AB80!10},
		logo axis = position,
		logo plot,
		xlabel style = {anchor = south}
	]{\textwidth + \plotylabelwidth}{4}{}
		
		\logogroupplot{AB80}{a}{light}{23/21}
		\logogroupplot{AB80}{b}{light}{19/21}
		\logogroupplot{AB80}{c}{light}{19/21}
		\logogroupplot[save base width]{AB80}{d}{light}{23/21}
		
	\end{hgroupplot}
	
	\fill[light!20] (AB80 logos) rectangle (group c4r1.above north east) node[pos = .5, text = black, font = \bfseries, text depth = 0pt] {\light};
	
	% matching transcription factors
	\pgfplotstableread{rawData/PEV_ld_TF-matches_AB80_light.tsv}{\TFmatches}
	\getrows{\TFmatches}
	\foreach \x [evaluate = \x as \xx using int(\x - 1)] in {1, ..., \datarows}{
		\pgfplotstableforeachcolumn{\TFmatches}\as{\col}{
			\pgfplotstablegetelem{\xx}{\col}\of{\TFmatches}
			\expandafter\edef\csname this\col\endcsname{\pgfplotsretval}
		}
		\logoplot[%
			anchor = north west,
			at = {(logo AB80 \thisregion.below south west)},
			xshift = \thisoffset\basewidth,
			width = \thislength\basewidth,
			height = .67cm,
			logo axis = none,
			axis background/.style = {fill = none}
		]{rawData/seqLogo_\thisTF_\thisorientation.tsv}
		\ifnum\thisoffset=0\relax
			\node[anchor = east, align = right, font = \bfseries] at (last plot.west) {\textbf{\thisTFfamily}};
		\else
			\node[anchor = west, align = left, font = \bfseries] at (last plot.east) {\textbf{\thisTFfamily}};
		\fi
	}
	
	\begin{pgfonlayer}{background}
		\fill[light!10] (AB80 logos) rectangle (group c4r1.east |- last plot.south);
	\end{pgfonlayer}
	
	
	%%% Cab-1 sequence logos
	\coordinate[yshift = -\columnsep] (Cab-1 logos) at (overview |- last plot.below south);

	\begin{hgroupplot}[%
		group position = {anchor = above north west, at = {(Cab-1 logos)}, yshift = -\columnsep},
		height = .67cm,
		title style = {fill = Cab-1!50, draw = none},
		axis background/.style = {fill = Cab-1!10},
		logo axis = position,
		logo plot,
		xlabel style = {anchor = south}
	]{\textwidth + \plotylabelwidth}{5}{}
		
		\logogroupplot[xtick = {145, 150, 155}]{Cab-1}{a}{light}{19/20.6}
		\logogroupplot{Cab-1}{b}{light}{21/20.6}
		\logogroupplot{Cab-1}{c}{light}{19/20.6}
		\logogroupplot{Cab-1}{d}{light}{25/20.6}
		\logogroupplot[save base width]{Cab-1}{e}{light}{19/20.6}
		
	\end{hgroupplot}
	
	\fill[light!20] (Cab-1 logos) rectangle (group c5r1.above north east) node[pos = .5, text = black, font = \bfseries, text depth = 0pt] {\light};
	
	% matching transcription factors
	\pgfplotstableread{rawData/PEV_ld_TF-matches_Cab-1_light.tsv}{\TFmatches}
	\getrows{\TFmatches}
	\foreach \x [evaluate = \x as \xx using int(\x - 1)] in {1, ..., \datarows}{
		\pgfplotstableforeachcolumn{\TFmatches}\as{\col}{
			\pgfplotstablegetelem{\xx}{\col}\of{\TFmatches}
			\expandafter\edef\csname this\col\endcsname{\pgfplotsretval}
		}
		\logoplot[%
			anchor = north west,
			at = {(logo Cab-1 \thisregion.below south west)},
			xshift = \thisoffset\basewidth,
			width = \thislength\basewidth,
			height = .67cm,
			logo axis = none,
			axis background/.style = {fill = none}
		]{rawData/seqLogo_\thisTF_\thisorientation.tsv}
		\node[anchor = west, align = left, font = \bfseries] at (last plot.east) {\textbf{\thisTFfamily}};
	}
	
	\begin{pgfonlayer}{background}
		\fill[light!10] (Cab-1 logos) rectangle (group c5r1.east |- last plot.south);
	\end{pgfonlayer}
	
	
	%%% rbcS-E9 sequence logos
	\coordinate[yshift = -\columnsep] (rbcS-E9 logos) at (overview |- last plot.below south);

	\begin{hgroupplot}[%
		group position = {anchor = above north west, at = {(rbcS-E9 logos)}, yshift = -\columnsep},
		height = .67cm,
		title style = {fill = rbcS-E9!50, draw = none},
		axis background/.style = {fill = rbcS-E9!10},
		logo axis = position,
		logo plot,
		xlabel style = {anchor = south}
	]{\textwidth + \plotylabelwidth}{5}{}
		
		\logogroupplot{rbcS-E9}{a}{light}{17/20.2}
		\logogroupplot{rbcS-E9}{b}{light}{17/20.2}
		\logogroupplot{rbcS-E9}{c}{light}{21/20.2}
		\logogroupplot{rbcS-E9}{d}{light}{25/20.2}
		\logogroupplot[save base width]{rbcS-E9}{e}{light}{21/20.2}
		
	\end{hgroupplot}
	
	\fill[light!20] (rbcS-E9 logos) rectangle (group c5r1.above north east) node[pos = .5, text = black, font = \bfseries, text depth = 0pt] {\light};
	
	% matching transcription factors
	\pgfplotstableread{rawData/PEV_ld_TF-matches_rbcS-E9_light.tsv}{\TFmatches}
	\getrows{\TFmatches}
	\foreach \x [evaluate = \x as \xx using int(\x - 1)] in {1, ..., \datarows}{
		\pgfplotstableforeachcolumn{\TFmatches}\as{\col}{
			\pgfplotstablegetelem{\xx}{\col}\of{\TFmatches}
			\expandafter\edef\csname this\col\endcsname{\pgfplotsretval}
		}
		\logoplot[%
			anchor = north west,
			at = {(logo rbcS-E9 \thisregion.below south west)},
			xshift = \thisoffset\basewidth,
			width = \thislength\basewidth,
			height = .67cm,
			logo axis = none,
			axis background/.style = {fill = none}
		]{rawData/seqLogo_\thisTF_\thisorientation.tsv}
		\def\regionE{e}
		\ifx\thisregion\regionE\relax
			\node[anchor = east, align = right, font = \bfseries] at (last plot.west) {\textbf{\thisTFfamily}};
		\else
			\node[anchor = west, align = left, font = \bfseries] at (last plot.east) {\textbf{\thisTFfamily}};
		\fi
	}
	
	\begin{pgfonlayer}{background}
		\fill[light!10] (rbcS-E9 logos) rectangle (group c5r1.east |- last plot.south);
	\end{pgfonlayer}
	
	
	%%% rbcS-E9 sequence logos in dark
	\coordinate[yshift = -\columnsep] (rbcS-E9 logos dark) at (overview |- last plot.below south);
	
	\begin{hgroupplot}[%
		group position = {anchor = above north west, at = {(rbcS-E9 logos dark)}, yshift = -\columnsep},
		height = .67cm,
		title style = {fill = rbcS-E9!50, draw = none},
		axis background/.style = {fill = rbcS-E9!10},
		logo axis = position,
		logo plot,
		xlabel style = {anchor = south},
	]{55\basewidth + 2\groupplotsep + \plotylabelwidth}{3}{}
		
		\logogroupplot{rbcS-E9}{a}{dark}{17/55*3}
		\logogroupplot{rbcS-E9}{b}{dark}{17/55*3}
		\logogroupplot{rbcS-E9}{c}{dark}{21/55*3}
		
	\end{hgroupplot}
	
	\fill[dark!20] (rbcS-E9 logos dark) rectangle (group c3r1.above north east) node[pos = .5, text = black, font = \bfseries] {\dark};
	
	% matching transcription factors
	\pgfplotstableread{rawData/PEV_ld_TF-matches_rbcS-E9_dark.tsv}{\TFmatches}
	\getrows{\TFmatches}
	\foreach \x [evaluate = \x as \xx using int(\x - 1)] in {1, ..., \datarows}{
		\pgfplotstableforeachcolumn{\TFmatches}\as{\col}{
			\pgfplotstablegetelem{\xx}{\col}\of{\TFmatches}
			\expandafter\edef\csname this\col\endcsname{\pgfplotsretval}
		}
		\logoplot[%
			anchor = north west,
			at = {(logo rbcS-E9 \thisregion.below south west)},
			xshift = \thisoffset\basewidth,
			width = \thislength\basewidth,
			height = .67cm,
			logo axis = none,
			axis background/.style = {fill = none}
		]{rawData/seqLogo_\thisTF_\thisorientation.tsv}
		\node[anchor = west, align = left, font = \bfseries] at (last plot.east) {\textbf{\thisTFfamily}};
	}
	
	\begin{pgfonlayer}{background}
		\fill[dark!10] (rbcS-E9 logos dark) rectangle (group c3r1.east |- last plot.south);
	\end{pgfonlayer}
	
	
	%%% subfigure labels
	\subfiglabel{overview}
	\subfiglabel{AB80 logos}
	\subfiglabel{Cab-1 logos}
	\subfiglabel{rbcS-E9 logos}
	\subfiglabel{rbcS-E9 logos dark}
	
\end{tikzpicture}%
			\caption{%
				\captiontitle{Mutation-sensitive regions harbor transcription factor binding sites}%
				\subfig{A} 4 to 5 mutation-sensitive regions (shaded rectangles; labeled a\textendash e) were defined for the \segment{B} segments of the \enhancer{AB80}, \enhancer{Cab-1}, and \enhancer{rbcS-E9} enhancers. The mutational sensitivity plots are reproduced from \cref{fig:PEV_ld_MutSens}.\nextentry
				\subfigrange{B}{E} Sequence logo plots were generated from the enhancer strength in the \light{} \parensubfigrange{B}{D} or \dark{} \parensubfig{E} of all possible single-nucleotide substitution variants within the indicated mutation-sensitive regions of the \enhancer{AB80} \parensubfig{B}, \enhancer{Cab-1} \parensubfig{C}, or \enhancer{rbcS-E9} \parensubfigtwo{D}{E} enhancers. The sequence of the wild-type enhancer and the position along it is shown on the x axis. Letters with dark colors in the logo plot represent wild-type bases. The sequence logos for each region were compared to known transcription factor binding motifs and significant matches are shown below the plots.\nextentry
				\subfig{F} For each transcription factor binding motif matching a sequence logo plots derived from the saturation mutagenesis data in the light \parensubfigrange[mutagenesis; see ]{B}{D} or identified by the motif-scanning approach (scanning; see \cref{sfig:PEV_ld_fimo}), the correlation (Pearson's $r$) between the strength of an enhancer variant and the score of how well the variant sequence matches this motif is plotted as points. The lines represent the average correlation for all motifs of a given enhancer.
			}%
			\label{fig:PEV_ld_motifs}
		\end{fig}
		
		\begin{fig}
			\begin{tikzpicture}

	%%% scheme of circadian rhythm experiment
	\coordinate (cr scheme) at (0, 0);
	
	% will be drawn after next plot
	
	
	%%% circadian rhythm of enhancer activity
	\coordinate (circadian rhythm) at (cr scheme -| \textwidth - \threequartercolumnwidth, 0);
	
	\begin{hgroupplot}[%
		group position = {anchor = above north west, at = {(circadian rhythm)}, xshift = \plotylabelwidth},
		axis limits from table = {rawData/PEV_cr_WT_axes.tsv},
		enlarge x limits = false,
		enlarge y limits = {value = .25, upper},
		xtick = {8, 12, ..., 32},
		ytick = {-1, -.5, ..., 3},
		y decimals = 1,
		ylabel = $\log_2$(enhancer strength),
		execute at begin axis/.append = {\fill[dark, fill opacity = .1] (16, \ymin) rectangle (24, \ymax);},
		xticklabel style = {name = xticklabel},
		zero line,
	]{\threequartercolumnwidth}{3}{time in constant light (ZT)}	
		
		\nextgroupplot[
			title = \enhancer{AB80} enhancer\\\segment{B} segment,
			title color = AB80
		]
			
			% amplitude & stats
			\addplot[
				quiver = {u = 0, v = \thisrow{amplitude}},
				<->,
				x filter/.code = {\def\tempA{AB80}\edef\tempB{\thisrow{enhancer}}\ifx\tempA\tempB\relax\else\def\pgfmathresult{}\fi},
				visualization depends on = \thisrow{amplitude} \as \amp,
				visualization depends on = \thisrow{peak} \as \peak,
				visualization depends on = \thisrow{rsquare} \as \rsquare,
				nodes near coords = {%
					$\text{amplitude} = \pgfmathprintnumber[fixed, fixed zerofill, precision = 2]{\amp}$\\%
					$\text{peak time} = \text{ZT }\pgfmathprintnumber[fixed, precision = 0]{\peak}$\\%
					$R^2 = \pgfmathprintnumber[fixed, fixed zerofill, precision = 2]{\rsquare}$
				},
				nodes near coords style = {anchor = north west, at = {(current axis.north west)}, align = left}
			] table [x = peak, y expr = 0] {rawData/PEV_cr_WT_stats.tsv};

			
			% fitted line
			\addplot[line plot, AB80!50] table [x = ZT, y = AB80] {rawData/PEV_cr_WT_lines.tsv};
					
			% points
			\addplot [
				only marks,
				mark = solido,
				mark size = 2,
				AB80
			] table[x = ZT, y = AB80] {rawData/PEV_cr_WT_points.tsv};
		
		
		\nextgroupplot[
			title = \enhancer{Cab-1} enhancer\\\segment{B} segment,
			title color = Cab-1
		]
		
			% amplitude & stats
			\addplot[
				quiver = {u = 0, v = \thisrow{amplitude}},
				<->,
				x filter/.code = {\def\tempA{Cab-1}\edef\tempB{\thisrow{enhancer}}\ifx\tempA\tempB\relax\else\def\pgfmathresult{}\fi},
				visualization depends on = \thisrow{amplitude} \as \amp,
				visualization depends on = \thisrow{peak} \as \peak,
				visualization depends on = \thisrow{rsquare} \as \rsquare,
				nodes near coords = {%
					$\text{amplitude} = \pgfmathprintnumber[fixed, fixed zerofill, precision = 2]{\amp}$\\%
					$\text{peak time} = \text{ZT }\pgfmathprintnumber[fixed, precision = 0]{\peak}$\\%
					$R^2 = \pgfmathprintnumber[fixed, fixed zerofill, precision = 2]{\rsquare}$
				},
				nodes near coords style = {anchor = north west, at = {(current axis.north west)}, align = left}
			] table [x = peak, y expr = 0] {rawData/PEV_cr_WT_stats.tsv};
		
			% fitted line
			\addplot[line plot, Cab-1!50] table [x = ZT, y = Cab-1] {rawData/PEV_cr_WT_lines.tsv};
					
			% points
			\addplot [
				only marks,
				mark = solido,
				mark size = 2,
				Cab-1
			] table[x = ZT, y = Cab-1] {rawData/PEV_cr_WT_points.tsv};
		
		
		\nextgroupplot[
			title = \enhancer{rbcS-E9} enhancer\\\segment{B} segment,
			title color = rbcS-E9
		]
		
			% amplitude & stats
			\addplot[
				quiver = {u = 0, v = \thisrow{amplitude}},
				<->,
				x filter/.code = {\def\tempA{rbcS-E9}\edef\tempB{\thisrow{enhancer}}\ifx\tempA\tempB\relax\else\def\pgfmathresult{}\fi},
				visualization depends on = \thisrow{amplitude} \as \amp,
				visualization depends on = \thisrow{peak} \as \peak,
				visualization depends on = \thisrow{rsquare} \as \rsquare,
				nodes near coords = {%
					$\text{amplitude} = \pgfmathprintnumber[fixed, fixed zerofill, precision = 2]{\amp}$\\%
					$\text{peak time} = \text{ZT }\pgfmathprintnumber[fixed, precision = 0]{\peak}$\\%
					$R^2 = \pgfmathprintnumber[fixed, fixed zerofill, precision = 2]{\rsquare}$
				},
				nodes near coords style = {anchor = north west, at = {(current axis.north west)}, align = left}
			] table [x = peak, y expr = 0] {rawData/PEV_cr_WT_stats.tsv};
		
			% fitted line
			\addplot[line plot, rbcS-E9!50] table [x = ZT, y = rbcS-E9] {rawData/PEV_cr_WT_lines.tsv};
					
			% points
			\addplot [
				only marks,
				mark = solido,
				mark size = 2,
				rbcS-E9
			] table[x = ZT, y = rbcS-E9] {rawData/PEV_cr_WT_points.tsv};
			
	\end{hgroupplot}
	
	
	%%% scheme of circadian rhythm experiment (continued)
	\distance{group c1r1.south}{group c1r1.above north}
	\pgfmathsetlength{\templength}{\ydistance/80}
	
	\coordinate[xshift = \columnsep] (light bar top left) at (cr scheme);
	
	\fill[light] (light bar top left) rectangle ++(\columnsep, -80\templength) coordinate (light bar bottom right);
	\fill[dark] (light bar top left) ++(0, -16\templength) rectangle ++(\columnsep, -8\templength) ++(-\columnsep, -16\templength) rectangle ++(\columnsep, -8\templength) coordinate (ZT0);
	
	\draw[Latex-, thick] (light bar top left -| light bar bottom right) ++(.05, -4\templength) -- ++(.5, 0) node[anchor = west, text depth = 0pt, inner sep = .15em] (infiltration) {infiltration};
	
	\coordinate[yshift = -.67cm] (leaf) at (infiltration.south);
	\leaf[.33]{leaf}
	
	\foreach \x in {-8, 0, 6, ..., 24}{
		\draw[Latex-, thick] (ZT0) ++(.05, {(-8 - \x) * \templength}) -- ++(.5, 0) node[anchor = west, text depth = 0pt, inner sep = .15em] (t\x) {\timepoint{\x}};
	}
	
	\draw[decorate, decoration = {brace, amplitude = .25cm}] (t0.north -| t24.east) -- (t24.south east) node[pos = .5, xshift = .25cm, anchor = west, align = left] {sampled\\timepoints};
	
	
	%%% histograms of variant effects (amplitude and enhancer strength)
	\coordinate[yshift = -\columnsep] (amplitude histograms) at (cr scheme |- xlabel.south);
	
	\distance{title.south}{title.north}
	\pgfmathsetlength{\plotwidth}{(\textwidth - 2\plotylabelwidth - \columnsep - \groupplotsep - 2\ydistance - .5\baselineskip) / 3}
	
	\node[anchor = north west, draw, fill = gray!20, text depth = 0pt, minimum width = \plotwidth, xshift = \plotylabelwidth] (tamplitude) at (amplitude histograms) {\vphantom{(\textDelta lhpg)}\textDelta{} amplitude};
	
	\begin{hvgroupplot}[%
		group position = {anchor = north west, at = {(tamplitude.south west)}, shift = {(.5\pgflinewidth, .5\pgflinewidth)}},
		enlarge x limits = {value = .05, upper},
		enlarge y limits = .05,
		ytick = {-5, -4, ..., 5},
		zero line,
		title style = {
			anchor = south,
			at = {(1, .5)},
			rotate = -90,
			minimum width = \plotheight,
			fill = titlecol!20,
			draw = black,
			shift = {(-.5\pgflinewidth, -.5\pgflinewidth)}
		}
	]{2\plotwidth + \plotylabelwidth + \groupplotsep}{10.5cm + \plotxlabelheight + 2\groupplotsep}{2}{3}{number of variants}{$\log_2$(enhancer strength)}
		
		\nextgroupplot[
			axis limits from table = {rawData/PEV_cr_amplitude_axes.tsv}
		]
		
			\histogram[fill = AB80]{rawData/PEV_cr_amplitude_AB80}{count}
			
		\nextgroupplot[
			axis limits from table = {rawData/PEV_cr_enrichment_axes.tsv},
			title = \enhancer{AB80} enhancer\\\segment{B} segment,
			title color = AB80
		]
		
			\histogram[fill = AB80]{rawData/PEV_cr_enrichment_AB80}{count}
			
			
		\nextgroupplot[
			axis limits from table = {rawData/PEV_cr_amplitude_axes.tsv}
		]
		
			\histogram[fill = Cab-1]{rawData/PEV_cr_amplitude_Cab-1}{count}
			
		\nextgroupplot[
			axis limits from table = {rawData/PEV_cr_enrichment_axes.tsv},
			title = \enhancer{Cab-1} enhancer\\\segment{B} segment,
			title color = Cab-1
		]
		
			\histogram[fill = Cab-1]{rawData/PEV_cr_enrichment_Cab-1}{count}
			
			
		\nextgroupplot[
			axis limits from table = {rawData/PEV_cr_amplitude_axes.tsv}
		]
		
			\histogram[fill = rbcS-E9]{rawData/PEV_cr_amplitude_rbcS-E9}{count}
			
		\nextgroupplot[
			axis limits from table = {rawData/PEV_cr_enrichment_axes.tsv},
			title = \enhancer{rbcS-E9} enhancer\\\segment{B} segment,
			title color = rbcS-E9
		]
		
			\histogram[fill = rbcS-E9]{rawData/PEV_cr_enrichment_rbcS-E9}{count}
			
	\end{hvgroupplot}
	
	\node[anchor = south, draw, fill = gray!20, text depth = 0pt, minimum width = \plotwidth, yshift = -.5\pgflinewidth] (tstrength) at (group c2r1.north) {\textDelta{} enhancer strength (\timepoint{0})};
	
	
	%%% histograms of variant effects (peak)
	\coordinate[xshift = \columnsep] (peak histograms) at (amplitude histograms -| title.north);
	
	\node[anchor = north west, draw, fill = gray!20, text depth = 0pt, minimum width = \plotwidth, xshift = \plotylabelwidth + .5\baselineskip] (tpeak) at (peak histograms) {\vphantom{(lhpg)}\textDelta{} peak time};
	
	\begin{vgroupplot}[%
		group position = {anchor = north west, at = {(tpeak.south west)}, shift = {(.5\pgflinewidth, .5\pgflinewidth)}},
		width = \plotwidth,
		xlabel = {number of variants},
		ytick = {-12, -8, ..., 12},
		enlarge x limits = {value = .05, upper},
		enlarge y limits = .05,
		axis limits from table = {rawData/PEV_cr_peak_axes.tsv},
		zero line,
		title style = {
			anchor = south,
			at = {(1, .5)},
			rotate = -90,
			minimum width = \plotheight,
			fill = titlecol!20,
			draw = black,
			shift = {(-.5\pgflinewidth, -.5\pgflinewidth)}
		}
	]{10.5cm + \plotxlabelheight + 2\groupplotsep}{3}{peak shift (h)}
	
		\nextgroupplot[
			title = \enhancer{AB80} enhancer\\\segment{B} segment,
			title color = AB80
		]
		
			\histogram[fill = AB80!25!gray]{rawData/PEV_cr_peak_AB80}{badFit}
			\histogram[fill = AB80]{rawData/PEV_cr_peak_AB80}{goodFit}
			
	
		\nextgroupplot[
			title = \enhancer{Cab-1} enhancer\\\segment{B} segment,
			title color = Cab-1
		]
		
			\histogram[fill = Cab-1!25!gray]{rawData/PEV_cr_peak_Cab-1}{badFit}
			\histogram[fill = Cab-1]{rawData/PEV_cr_peak_Cab-1}{goodFit}
			
	
		\nextgroupplot[
			title = \enhancer{rbcS-E9} enhancer\\\segment{B} segment,
			title color = rbcS-E9
		]
		
			\histogram[fill = rbcS-E9!25!gray]{rawData/PEV_cr_peak_rbcS-E9}{badFit}
			\histogram[fill = rbcS-E9]{rawData/PEV_cr_peak_rbcS-E9}{goodFit}
	
	\end{vgroupplot}
	
	
	%%% subfigure labels
	\subfiglabel{cr scheme}
	\subfiglabel[xshift = -.5\baselineskip]{circadian rhythm}
	\subfiglabel{amplitude histograms}
	\subfiglabel{peak histograms}
	
\end{tikzpicture}%
			\caption{%
				\captiontitle{Circadian oscillation is robustly encoded in the \enhancer{AB80}, \enhancer{Cab-1}, and \enhancer{rbcS-E9} enhancers}%
				\subfig{A} All possible single-nucleotide variants of the \enhancer{AB80}, \enhancer{Cab-1}, and \enhancer{rbcS-E9} enhancers were subjected to Plant STARR-seq in tobacco leaves. On the morning of the third day after transformation (\timepoint{-8}), the plants were shifted to constant light. Leaves were harvested for RNA extraction starting at mid-day (\timepoint{0}) and in 6 hour intervals (\timepoint{6}, 20, 26, and 32) afterwards.\nextentry
				\subfig{B} A sine wave with a period of 24 h was fitted to the enhancer strength of a given variant across all sampled time points. The fitted line is plotted together with the measured data points for the wild-type enhancers. The equilibrium point of the curves was set to 0. The amplitude is shown as a two-sided arrow at the time of highest enhancer strength (peak time). The goodness-of-fit ($R^2$) is indicated. The shaded gray area represents the timing of the dark period if the plants had not been shifted to constant light.\nextentry
				\subfigtwo{C}{D} Histograms of the difference between the amplitude \parensubfig{C} and peak time \parensubfig{D} of each single-nucleotide variant relative to the wild-type enhancer. For comparison, the difference in enhancer strength at \timepoint{0} is also shown in \plainsubfigref{C}. Variants with a below average goodness-of-fit are grayed out in \plainsubfigref{D}.\nextentry
				Only data for the \segment{B} enhancer segments is shown.
			}%
			\label{fig:PEV_cr}
		\end{fig}
		
		\begin{fig}
			\begin{tikzpicture}
	\addtolength{\plotylabelwidth}{\baselineskip}
	
	\pgfplotsset{
		colorbar distance/.style = {
			colorbar horizontal,
			colorbar style = {
				anchor = south west,
				at = {(.025, .025)},
				height = .25cm,
				width = .4 * \pgfkeysvalueof{/pgfplots/parent axis height},
				tick pos = upper,
				xticklabel pos = right,
				xticklabel style = {node font = \figtiny, inner ysep = .1em, text depth = 0pt},
				title = distance (bp)\\\vphantom{1},
				title style = {fill = none, draw = none, minimum width = 0, node font = \figsmall, align = center, anchor = south},
				extra x ticks = {8},
			},
		}
	}
	
	%%% effect of two deletions in light
	\coordinate[yshift = -\columnsep] (prediction light) at (0, 0);
	
	\node[anchor = north west, text depth = 0pt, font = \hphantom{A}] (ptitle) at (prediction light) {\vphantom{\light}};
	
	\distance{ptitle.south}{ptitle.north}
	
	\pgfmathsetlength{\templength}{(\textwidth + \pgflinewidth - 2\plotylabelwidth - 2\groupplotsep - 2\ydistance - \columnsep) / 4}
	
	\begin{hgroupplot}[%
		group position = {anchor = above north west, at = {(prediction light)}, xshift = \plotylabelwidth},
		axis limits from table = {rawData/PEVdouble_prediction_light_axes.tsv},
		enlargelimits = .05,
		ylabel = \textbf{measurement}:\\$\log_2$(enhancer strength),
		zero line,
		show diagonal,
		colormap name = viridis,
		colorbar distance,
		point meta min = 8
	]{3\templength + \plotylabelwidth + 2\groupplotsep}{3}{\textbf{additive model}: $\log_2$(enhancer strength)}
		
		\nextgroupplot[
			title = \enhancer{AB80},
			title color = AB80
		]
		
			% scatter plot
			\addplot [
				scatter,
				only marks,
				mark = solido,
				fill opacity = .5,
				point meta = explicit
			] table[x = prediction, y = enrichment, meta = distance] {rawData/PEVdouble_prediction_AB80_light_points.tsv};
			
			% correlation
			\stats[stats position = south east]{rawData/PEVdouble_prediction_AB80_light}
			
			
		\nextgroupplot[
			title = \enhancer{Cab-1},
			title color = Cab-1
		]
		
			% scatter plot
			\addplot [
				scatter,
				only marks,
				mark = solido,
				fill opacity = .5,
				point meta = explicit
			] table[x = prediction, y = enrichment, meta = distance] {rawData/PEVdouble_prediction_Cab-1_light_points.tsv};
			
			% correlation
			\stats[stats position = south east]{rawData/PEVdouble_prediction_Cab-1_light}
			
			
		\nextgroupplot[
			title = \enhancer{rbcS-E9},
			title color = rbcS-E9
		]
		
			% scatter plot
			\addplot [
				scatter,
				only marks,
				mark = solido,
				fill opacity = .5,
				point meta = explicit
			] table[x = prediction, y = enrichment, meta = distance] {rawData/PEVdouble_prediction_rbcS-E9_light_points.tsv};
			
			% correlation
			\stats[stats position = south east]{rawData/PEVdouble_prediction_rbcS-E9_light}
		
	\end{hgroupplot}
	
	\distance{group c3r1.south}{group c3r1.north}
	
	\node[anchor = south west, rotate = -90, draw = black, fill = light!20, minimum width = \ydistance, shift = {(-.5\pgflinewidth, -.5\pgflinewidth)}, text depth = 0pt, font = \hphantom{A}] (tlight) at (group c3r1.north east) {\light};
	
	
	%%% effect of two deletions in dark
	\coordinate[xshift = \columnsep] (prediction dark) at (group c3r1.above north -| tlight.north);
	
	\begin{axis}[%
		name = dark plot,
		anchor = above north west,
		at = {(prediction dark)},
		xshift = \plotylabelwidth,
		width = \templength,
		axis limits from table = {rawData/PEVdouble_prediction_dark_axes.tsv},
		enlargelimits = .05,
		ylabel = \textbf{measurement}:\\$\log_2$(enhancer strength),
		zero line,
		show diagonal,
		title = \enhancer{rbcS-E9},
		title color = rbcS-E9,
		title style = {minimum width = \templength},
		xlabel = \textbf{additive model}:\\$\log_2$(enhancer strength),
		colormap name = viridis,
		colorbar distance,
		point meta min = 8
	]
		
			% scatter plot
			\addplot [
				scatter,
				only marks,
				mark = solido,
				fill opacity = .5,
				point meta = explicit
			] table[x = prediction, y = enrichment, meta = distance] {rawData/PEVdouble_prediction_rbcS-E9_dark_points.tsv};
			
			% correlation
			\stats[stats position = south east]{rawData/PEVdouble_prediction_rbcS-E9_dark}
		
	\end{axis}
	
	\distance{dark plot.south}{dark plot.north}
	
	\node[anchor = south west, rotate = -90, draw = black, fill = dark!20, minimum width = \ydistance, shift = {(-.5\pgflinewidth, -.5\pgflinewidth)}, text depth = 0pt, font = \hphantom{A}] at (dark plot.north east) {\dark};
	
	
	%%% subfigure labels
	\subfiglabel{prediction light}
	\subfiglabel{prediction dark}
	
\end{tikzpicture}%
			\caption{%
				\captiontitle{Epistatic interactions between single-nucleotide deletions}%
				\subfigrange{A}{C} Selected single-nucleotide deletion variants (\plainsubfigref{A}; \textDelta1, \textDelta2, \textDelta3, \dots) of the \segment{B} segment of the \enhancer{AB80}, \enhancer{Cab-1}, and \enhancer{rbcS-E9} enhancers and all possible combinations with two of these deletions (\plainsubfigref{A}; \textDelta1+\textDelta2, \textDelta1+\textDelta3, \textDelta2+\textDelta3, \dots) were subjected to Plant STARR-seq in tobacco plants grown in normal light/dark cycles \parensubfig{B} or completely in the dark \parensubfig{C} for two days prior to RNA extraction. For each pair of deletions, the expected enhancer strength based on the sum of the effects of the individual deletions (additive model) is plotted against the measured enhancer strength. The color of the points represents the distance between the two deletions in a pair.
			}%
			\label{fig:PEVdouble}
		\end{fig}
		
		\begin{fig}
			\begin{tikzpicture}
	\pgfmathsetseed{1}

	%%% fragment overview
	\coordinate (overview) at (0, 0);
	
	\begin{scope}
		\setlength{\groupplotsep}{\columnsep}
		
		\begin{hgroupplot}[%
			group position = {anchor = above north west, at = {(overview)}, xshift = \columnsep},
			height = 1.5cm,
			axis lines = none,
			title style = {signal, anchor = south west, at = {(0, 1)}, minimum width = \plotwidth - 1.25\pgflinewidth},
			zero line,
			clip = false,
			legend style = {anchor = south west, at = {(0, 0)}},
			legend image post style = {very thick},
			legend plot pos = left,
			legend cell align = left,
			enlarge y limits = .05
		]{\textwidth + \plotylabelwidth - \columnsep}{3}{}
			
			\nextgroupplot[
				title = \enhancer{AB80} enhancer \segment{B} segment,
				title color = AB80,
				axis limits from table = {rawData/PEV_ld_mutSens_AB80_recap_axes.tsv},
				axis limits = AB80 B,
			]
			
				\addplot[line plot, dark] table [x = position, y = dark] {rawData/PEV_ld_mutSens_AB80_B_lines.tsv};
				\addplot[line plot, light] table [x = position, y = light] {rawData/PEV_ld_mutSens_AB80_B_lines.tsv};
				
				\enhFrag[-1]{168}{23}{AB80}{a}
				\enhFrag[-1]{193}{19}{AB80}{b}
				\enhFrag[0]{205}{19}{AB80}{c}
				\enhFrag[-2]{198}{33}{AB80}{bc}
				\enhFrag[-1]{228}{23}{AB80}{d}
				
				\legend{dark, light}
				
			\nextgroupplot[
				title = \enhancer{Cab-1} enhancer \segment{B} segment,
				title color = Cab-1,
				axis limits from table = {rawData/PEV_ld_mutSens_Cab-1_recap_axes.tsv},
				axis limits = Cab-1 B,
			]
			
				\addplot[line plot, dark] table [x = position, y = dark] {rawData/PEV_ld_mutSens_Cab-1_B_lines.tsv};
				\addplot[line plot, light] table [x = position, y = light] {rawData/PEV_ld_mutSens_Cab-1_B_lines.tsv};
				
				\enhFrag[-1]{119}{21}{Cab-1}{ctrl}
				\enhFrag[-1]{149}{19}{Cab-1}{a}
				\enhFrag[-1]{179}{21}{Cab-1}{b}
				\enhFrag[0]{193}{19}{Cab-1}{c}
				\enhFrag[-2]{184}{35}{Cab-1}{bc}
				\enhFrag[-1]{220}{25}{Cab-1}{d}
				\enhFrag[0]{239}{19}{Cab-1}{e}
				\enhFrag[-2]{229}{41}{Cab-1}{de}
				
				
			\nextgroupplot[
				title = \enhancer{rbcS-E9} enhancer \segment{B} segment,
				title color = rbcS-E9,
				axis limits from table = {rawData/PEV_ld_mutSens_rbcS-E9_recap_axes.tsv},
				axis limits = rbcS-E9 B,
			]
			
				\addplot[draw = none, fill = dark, fill opacity = .33, restrict x to domain = 131:151] table [x = position, y = dark] {rawData/PEV_ld_mutSens_rbcS-E9_B_lines.tsv} -- (151, 0) -| cycle;
				\addplot[draw = none, fill = light, fill opacity = .33, restrict x to domain = 161:185] table [x = position, y = light] {rawData/PEV_ld_mutSens_rbcS-E9_B_lines.tsv} -- (185, 0) -| cycle;
			
				\addplot[line plot, dark] table [x = position, y = dark] {rawData/PEV_ld_mutSens_rbcS-E9_B_lines.tsv};
				\addplot[line plot, light] table [x = position, y = light] {rawData/PEV_ld_mutSens_rbcS-E9_B_lines.tsv};
				
				\node[node font = \figsmall, inner sep = .15em] (AUC) at (156, -1.6) {AUC};
				\draw[thin] (AUC) -- (140, -.4) (AUC) -- (175, -.75);
				
				\enhFrag[0]{109}{17}{rbcS-E9}{a}
				\enhFrag[1]{120}{17}{rbcS-E9}{b}
				\enhFrag[-1]{115}{31}{rbcS-E9}{ab}
				\enhFrag[0]{141}{21}{rbcS-E9}{c}
				\enhFrag[-2]{124}{47}{rbcS-E9}{abc}
				\enhFrag[0]{173}{25}{rbcS-E9}{d}
				\enhFrag[1]{193}{21}{rbcS-E9}{e}
				\enhFrag[-1]{186}{39}{rbcS-E9}{de}
			
		\end{hgroupplot}
	\end{scope}

	\draw[decorate, decoration = {brace, amplitude = .25cm, raise = .25cm}] (rbcS-E9 abc.south -| AB80 a.west) -- (rbcS-E9 b.north -| AB80 a.west) coordinate[pos = .5, xshift = -.5cm] (curly);
	
	\draw[ultra thick, -Latex] (curly) ++(.75\pgflinewidth, 0) -- ++(-.25, 0) arc[start angle = 90, end angle = 270, radius = .8cm] node[pos = .5, anchor = east, align = right] {mix \&\\match} -- ++(.25, 0) coordinate[xshift = \columnsep] (constructs);
	
	\coordinate[yshift = .35cm] (construct 1) at (constructs);
	\coordinate[yshift = -.35cm] (construct 3) at (constructs);
	
	\PEFconstruct{construct 1}
	\coordinate[xshift = \columnsep] (construct 2) at (construct end);
	
	\PEFconstruct[MediumPurple1]{construct 2}
	
	\PEFconstruct[HotPink1]{construct 3}
	\coordinate[xshift = \columnsep] (construct 4) at (construct end);

	\PEFconstruct[MediumOrchid3]{construct 4}
	
	\draw[ultra thick, -Latex] (constructs -| construct end) ++(\columnsep, 0) -- ++(1, 0) coordinate[shift = {(.667cm + \columnsep, .167)}] (leaf);
	\leaf[.33]{leaf}
	
	
	%%% PEF: enhancer strength by number of fragments
	\coordinate[yshift = -\columnsep] (n frags) at (overview |- ORF.south);
	
	\begin{hgroupplot}[%
		group position = {anchor = above north west, at = {(n frags)}, xshift = \plotylabelwidth},
		ylabel = $\log_2$(enhancer strength),
		ytick = {-10, -8, ..., 10},
		axis limits from table = {rawData/PEF_strength_by_frags_axes.tsv},
		group/every plot/.append style = {x grids = false},
		zero line,
	]{\twocolumnwidth}{2}{number of enhancer fragments}
		
		\nextgroupplot[
			title = \light,
			title color = light,
			x tick table = {rawData/PEF_strength_by_frags_light_boxplot.tsv}{sample}
		]
		
			% violin and box plot
			\violinbox[%
				violin color = {light},
				violin shade inverse
			]{rawData/PEF_strength_by_frags_light_boxplot.tsv}{rawData/PEF_strength_by_frags_light_violin.tsv}
				
			% add sample size
			\samplesize{rawData/PEF_strength_by_frags_light_boxplot.tsv}{id}{n}
		
		
		\nextgroupplot[
			title = \dark,
			title color = dark,
			x tick table = {rawData/PEF_strength_by_frags_dark_boxplot.tsv}{sample}
		]
		
			% violin and box plot
			\violinbox[%
				violin color = {dark},
				violin shade color = white
			]{rawData/PEF_strength_by_frags_dark_boxplot.tsv}{rawData/PEF_strength_by_frags_dark_violin.tsv}
				
			% add sample size
			\samplesize{rawData/PEF_strength_by_frags_dark_boxplot.tsv}{id}{n}
		
	\end{hgroupplot}
	
	
	%%% PEF: single fragments vs. DMS AUCs
	\coordinate (AUC) at (n frags -| \textwidth - \twocolumnwidth, 0);

	\begin{hgroupplot}[%
		group position = {anchor = above north west, at = {(AUC)}, xshift = \plotylabelwidth},
		enlargelimits = .15,
		ytick = {-10, -8, ..., 10},
		zero line,
		legend style = {anchor = north east, at = {(1, 1)}},
		legend plot pos = right,
		legend cell align = right,
		scatter/classes = {
			AB80={AB80},
			Cab-1={Cab-1},
			rbcS-E9={rbcS-E9}
		},
		ylabel = {$\log_2$(enhancer strength)}
	]{\twocolumnwidth}{2}{$-$AUC}
	
		\nextgroupplot[
			title = \light,
			title color = light,
			axis limits from table = {rawData/PEF_cor_AUC_light_axes.tsv},
		]
		
			% regression line
			\addplot [draw = none, forget plot] table [x = AUC, y = {create col/linear regression = { y = enrichment}}] {rawData/PEF_cor_AUC_light_points.tsv};
			
			\draw[gray, dashed] (\xmin, \xmin * \pgfplotstableregressiona + \pgfplotstableregressionb) -- (\xmax, \xmax * \pgfplotstableregressiona + \pgfplotstableregressionb);
			
			% scatter plot
			\addplot [
				scatter,
				scatter src = explicit symbolic,
				only marks,
				mark = solido,
				mark size = 1.25
			] table[x = AUC, y = enrichment, meta = enhancer] {rawData/PEF_cor_AUC_light_points.tsv};
			
			% correlation
			\stats{rawData/PEF_cor_AUC_light}
			
			% legend
			\legend{\enhancer{AB80}, \enhancer{Cab-1}, \enhancer{rbcS-E9}}
			
			
		\nextgroupplot[
			title = \dark,
			title color = dark,
			axis limits from table = {rawData/PEF_cor_AUC_dark_axes.tsv},
		]
		
			% regression line
			\addplot [draw = none, forget plot] table [x = AUC, y = {create col/linear regression = { y = enrichment}}] {rawData/PEF_cor_AUC_dark_points.tsv};
			
			\draw[gray, dashed] (\xmin, \xmin * \pgfplotstableregressiona + \pgfplotstableregressionb) -- (\xmax, \xmax * \pgfplotstableregressiona + \pgfplotstableregressionb);
			
			% scatter plot
			\addplot [
				scatter,
				scatter src = explicit symbolic,
				only marks,
				mark = solido,
				mark size = 1.25
			] table[x = AUC, y = enrichment, meta = enhancer] {rawData/PEF_cor_AUC_dark_points.tsv};
			
			% correlation
			\stats{rawData/PEF_cor_AUC_dark}
			
	\end{hgroupplot}
	
	
	%%% schematic of fragments/fragment combinations with different spacing
	% native spacing
	\coordinate[yshift = -\columnsep] (scheme spacing 1) at (n frags |- xlabel.south);
	
	\coordinate[shift = {(\columnsep, -.5)}] (construct bc) at (scheme spacing 1);
	\node[anchor = west] (label bc) at (construct bc) {\textbf{bc} (native spacing):};
	\PEFconstruct*{label bc.east}{AB80/bc}
	
	\draw[line width = \baselineskip, AB80!25, -Triangle Cap] ($(c1)!.5!(c2)$) ++(-2, -.66) coordinate (enh) -- ++(3, 0) node[pos = 0, anchor = west, text = black] {AB80};
	\draw[line width = \baselineskip, AB80, opacity = .25] (enh) ++(1.5, 0) coordinate (b start) -- ++(.58, 0) coordinate (b end) node[pos = .5, text = black, font = \bfseries\vphantom{abcdectrl}, text opacity = 1]  {b};
	\draw[line width = \baselineskip, AB80, opacity = .25] (enh) ++(2.5, 0) coordinate (c end) -- ++(-.58, 0) coordinate (c start) node[pos = .5, text = black, font = \bfseries\vphantom{abcdectrl}, text opacity = 1]  {c};
	
	\draw[thin, line cap = round] (c1) ++(.2, -.125) -- ($(b start) + (0, .5\baselineskip)$) (c2) ++(-.2, -.125) -- ($(c end) + (0, .5\baselineskip)$);
	
	% altered spacing
	\coordinate (scheme spacing 2) at (scheme spacing 1 -| \textwidth - \twocolumnwidth, 0);
	
	\coordinate[xshift = \columnsep] (construct b+c) at (construct bc -| scheme spacing 2);
	\node[anchor = west] (label b+c) at (construct b+c) {\textbf{b+c} (altered spacing):};
	\PEFconstruct*{label b+c.east}{AB80/b,AB80/c}
	
	\draw[line width = \baselineskip, AB80!25, -Triangle Cap] ($(c1)!.5!(c2)$) ++(-2, -.66) coordinate (enh) -- ++(3, 0) node[pos = 0, anchor = west, text = black] {AB80};
	\draw[line width = \baselineskip, AB80, opacity = .25] (enh) ++(1.5, 0) coordinate (b start) -- ++(.58, 0) coordinate (b end) node[pos = .5, text = black, font = \bfseries\vphantom{abcdectrl}, text opacity = 1]  {b};
	\draw[line width = \baselineskip, AB80, opacity = .25] (enh) ++(2.5, 0) coordinate (c end) -- ++(-.58, 0) coordinate (c start) node[pos = .5, text = black, font = \bfseries\vphantom{abcdectrl}, text opacity = 1]  {c};
	
	\draw[thin, line cap = round] (c1) ++(.2, -.125) -- ($(b start) + (0, .5\baselineskip)$) ($(c1)!.5!(c2)$) ++(-.05, -.125) -- ($(b end) + (0, .5\baselineskip)$);
	\draw[thin, line cap = round] ($(c1)!.5!(c2)$) ++(.05, -.125) -- ($(c start) + (0, .5\baselineskip)$) (c2) ++(-.2, -.125) -- ($(c end) + (0, .5\baselineskip)$);
	
	
	%%% Examples of fragments and fragment combinations (light)
	\coordinate[yshift = -\columnsep - .5\baselineskip] (frags light) at (scheme spacing 1 |- enh);
	
	\distance{title.south}{title.north}
	
	\pgfmathsetlength{\templength}{(\textwidth  - .5\threecolumnwidth + \pgflinewidth - 2\plotylabelwidth - 2\groupplotsep - 2\ydistance - 2\columnsep) / 4}
	
	\begin{hgroupplot}[%
		group position = {anchor = above north west, at = {(frags light)}, xshift = \plotylabelwidth},
		axis limits from table = {rawData/PEF_fragments_light_axes.tsv},
		enlarge y limits = .05,
		ylabel = $\log_2$(enhancer strength),
		zero line,
		xticklabel style = {font = \vphantom{abcde}}
	]{3\templength + \plotylabelwidth + 2\groupplotsep}{3}{fragments}
		
		\nextgroupplot[
			title = \enhancer{AB80},
			title color = AB80,
			x tick table = {rawData/PEF_fragments_AB80_bc_light_mean.tsv}{construct},
		]
		
			\hmandp{AB80, mark options = {AB80!50}}{rawData/PEF_fragments_AB80_bc_light};
			
			
		\nextgroupplot[
			title = \enhancer{Cab-1},
			title color = Cab-1,
			x tick table = {rawData/PEF_fragments_Cab-1_bc_light_mean.tsv}{construct},
		]
		
			\hmandp{Cab-1, mark options = {Cab-1!50}}{rawData/PEF_fragments_Cab-1_bc_light};
			
			
		\nextgroupplot[
			title = \enhancer{rbcS-E9},
			title color = rbcS-E9,
			x tick table = {rawData/PEF_fragments_rbcS-E9_ab_light_mean.tsv}{construct},
		]
		
			\hmandp{rbcS-E9, mark options = {rbcS-E9!50}}{rawData/PEF_fragments_rbcS-E9_ab_light};
		
	\end{hgroupplot}
	
	\distance{group c3r1.south}{group c3r1.north}
	
	\node[anchor = south west, rotate = -90, draw = black, fill = light!20, minimum width = \ydistance, shift = {(-.5\pgflinewidth, -.5\pgflinewidth)}, text depth = 0pt, font = \hphantom{A}] (tlight) at (group c3r1.north east) {\light};
	
	
	%%% Examples of fragments and fragment combinations (dark)
	\coordinate[xshift = \columnsep] (frags dark) at (group c3r1.above north -| tlight.north);
	
	\begin{axis}[%
		anchor = above north west,
		at = {(frags dark)},
		xshift = \plotylabelwidth,
		width = \templength,
		axis limits from table = {rawData/PEF_fragments_dark_axes.tsv},
		enlarge y limits = .05,
		ylabel = $\log_2$(enhancer strength),
		zero line,
		xticklabel style = {font = \vphantom{abcde}},
		title = \enhancer{rbcS-E9},
		title color = rbcS-E9,
		title style = {minimum width = \templength},
		x tick table = {rawData/PEF_fragments_rbcS-E9_ab_dark_mean.tsv}{construct},
		xlabel = fragments
	]
		
			\hmandp{rbcS-E9, mark options = {rbcS-E9!50}}{rawData/PEF_fragments_rbcS-E9_ab_dark};
		
	\end{axis}
	
	\distance{last plot.south}{last plot.north}
	
	\node[anchor = south west, rotate = -90, draw = black, fill = dark!20, minimum width = \ydistance, shift = {(-.5\pgflinewidth, -.5\pgflinewidth)}, text depth = 0pt, font = \hphantom{A}] at (last plot.north east) {\dark};


	%%% influence of fragment order on enhancer strength
	\coordinate (order) at (frags light -| \textwidth - .5\threecolumnwidth, 0);
	
	\distance{last plot.south}{last plot.above north}

	\begin{axis}[%
		anchor = above north west,
		at = {(order)},
		xshift = \plotylabelwidth,
		yshift = .2\pgflinewidth,
		width = .5\threecolumnwidth - \plotylabelwidth,
		height = \ydistance - .5\pgflinewidth,
		ymin = -0.4,
		ymax = 4.2,
		ylabel = $\log_2$(enhancer strength),
		zero line,
		xticklabel style = {font = \vphantom{\light\dark}},
		x tick table = {rawData/PEF_order_diff_boxplot.tsv}{sample},
		xlabel = condition
	]
		
		% violin plots
		\violinbox[%
			violin colors from table = {rawData/PEF_order_diff_boxplot.tsv}{sample},
			violin shade = 0
		]{rawData/PEF_order_diff_boxplot.tsv}{rawData/PEF_order_diff_violin.tsv}
		
		% sample size
		\samplesize{rawData/PEF_order_diff_boxplot.tsv}{id}{n}
		
		% p-value
		\signif{rawData/PEF_order_diff_pvalues.tsv}{1}{2}
		
	\end{axis}

	
	%%% subfigure labels
	\subfiglabel{overview}
	\subfiglabel{n frags}
	\subfiglabel{AUC}
	\subfiglabel{scheme spacing 1}
	\subfiglabel{scheme spacing 2}
	\subfiglabel{frags light}
	\subfiglabel{frags dark}
	\subfiglabel{order}
	
\end{tikzpicture}%
			\nextpagecaption{%
				\captiontitle{The number, spacing, and order of mutation-sensitive regions affects enhancer strength}%
				\subfig{A} Fragments of the \enhancer{AB80}, \enhancer{Cab-1}, and \enhancer{rbcS-E9} enhancers spanning 1\textendash3 mutation-sensitive regions (shaded rectangles; labeled a\textendash e, ab, abc, bc, de) as well as a control fragment (ctrl) from a mutation-insensitive region in \enhancer{Cab-1} and a shuffled version of the \enhancer{AB80} fragment d were ordered as oligonucleotides. These fragments were randomly combined to create synthetic enhancers with up to three fragments which were then subjected to Plant STARR-seq in tobacco plants grown in normal light/dark cycles (\light) or completely in the dark (\dark) for two days prior to RNA extraction. The mutational sensitivity plots are reproduced from \cref{fig:PEV_ld_MutSens}.\nextentry
				\subfig{B} Violin plots of the strength of the synthetic enhancers grouped by the number of contained fragments.\nextentry
				\subfig{C} For each enhancer fragment, the area under the curve (AUC) in the mutational sensitivity plots was calculated and plotted against the fragment's enhancer strength. AUCs in the \dark{} or \light{} for \enhancer{rbcS-E9} fragments c and d, respectively, are shown in \plainsubfigref{A}. Pearson's $R^2$, Spearman's $\rho$, and number ($n$) of enhancer fragments are indicated. A linear regression line is shown as a dashed line.\nextentry
				\subfigrange{D}{G} Plots of the strength of enhancer fragments \parensubfig{D} or fragment combinations (separated by a + sign and shown in the order in which they appear in the construct; \plainsubfigref{E}) in three replicates (points) and the mean strength (lines). Enhancer strength was determined using tobacco plants grown in the light \parensubfig{F} or dark \parensubfig{G} prior to RNA extraction.\nextentry
				\subfig{H} Violin plots of the difference in enhancer strength between synthetic enhancers harboring the same two enhancer fragments but in different order. The $p$-value from a two-sided Wilcoxon rank-sum test comparing light and dark results is indicated ($p$).\nextentry
				Violin plots in \plainsubfigref{B} and \plainsubfigref{H} represent the kernel density distribution and the box plots inside represent the median (center line), upper and lower quartiles and 1.5$\times$ interquartile range (whiskers) for all corresponding synthetic enhancers. Numbers at the bottom of each violin indicate the number of elements in each group.
				Enhancer strength in \plainsubfigref{B}\subfigrefrange\plainsubfigref{G} was normalized to a control construct without an enhancer ($\log_2$ set to 0).
			}%
			\label{fig:PEF_cooperativity}
		\end{fig}
		
		\begin{fig}
			\begin{tikzpicture}
	%%% commands to create legend entries from a table
	\newcommand{\getlegendentries}{
		\pgfplotstableread{rawData/PEF_strength_legend.tsv}\loadedtable%
		\pgfplotstableforeachcolumn\loadedtable\as\col{%
			\pgfplotstablegetelem{0}{\col}\of\loadedtable%
			\edef\thisentry{%
				\noexpand\addlegendentry{%
					\noexpand\vphantom{($n = 1$)}\\\col\\($n = \pgfmathprintnumber{\pgfplotsretval}$)%
				}%
			}%
			\thisentry%
		}%
	}
	
	\newcommand{\getlegendentriessynenh}{
		\pgfplotstableread{rawData/DL_synEnh_cor_points.tsv}\loadedtable%
		\pgfplotstableforeachcolumnelement{enhancer}\of\loadedtable\as\thisenh{%
			\pgfplotstablegetelem{\pgfplotstablerow}{synEnh}\of\loadedtable%
			\edef\thisentry{%
				\noexpand\addlegendentry{%
					\ifx\thisenh\pgfplotsretval%
						\thisenh%
					\else%
						\noexpand\vphantom{\pgfplotsretval}\\\thisenh\\\pgfplotsretval%
					\fi%
				}%
			}%
			\thisentry%
		}%
	}
	
	
	%%% define colors for synthetic enhancers
	\pgfplotsforeachungrouped \x in {1, ..., 11}{
		\pgfmathint{max((6 - \x) * 100 / 5, 0)}
		\edef\tempA{\pgfmathresult}
		\pgfmathint{min((11 - \x) * 100 / 5, 100)}
		\edef\tempB{\pgfmathresult}
		\colorlet{syn\x}{AB80!\tempA!Cab-1!\tempB!rbcS-E9}
	}


	%%% PEF: enhancer strength in light and dark
	\coordinate[yshift = -\columnsep] (strength) at (0, 0);
	
	\begin{axis}[
		anchor = above north west,
		at ={(strength)},
		xshift = \plotylabelwidth + \baselineskip,
		width = \threecolumnwidth - \plotylabelwidth,
		axis limits from table = {rawData/PEF_strength_axes.tsv},
		enlargelimits = .05,
		xytick = {-10, -8, ..., 10},
		legend style = {
			anchor = north west,
			at = {(1, 1)},
			font = \figsmall,
			cells = {align = left},
			inner ysep = 0pt
		},
		legend image post style = {fill opacity = 1, mark size = 1.25},
		legend plot pos = left,
		legend cell align = left,
		scatter/classes = {
			inactive={gray},
			constitutive={35S enhancer},
			light-activated={light},
			dark-activated={dark}
		},
		show diagonal,
		xlabel = {\textbf{\light}: $\log_2$(enhancer strength)},
		ylabel = {\textbf{\dark}:\\$\log_2$(enhancer strength)},
		xlabel style = {name = strength xlabel}
	]
	
		% scatter plot
		\addplot[
			scatter,
			scatter src = explicit symbolic,
			only marks,
			mark = solido,
			fill opacity = .5
		] table[x = light, y = dark, meta = type] {rawData/PEF_strength_points.tsv};
		
		% legend
		\getlegendentries
	
	\end{axis}
	
	
	%%% dual-luciferase plot
	\coordinate (DL plot) at (strength -| \textwidth - \twocolumnwidth - \baselineskip, 0);
	
	\begin{axis}[%
		anchor = north west,
		at = {(DL plot)},
		width = \twocolumnwidth - \plotylabelwidth,
		xshift = \plotylabelwidth + \baselineskip,
		x grids = false,
		axis limits from table = {rawData/DL_synEnh_axes.tsv},
		zero line,
		ylabel = $\log_2$(NanoLuc/Luc),
		x tick table = {rawData/DL_synEnh_boxplot.tsv}{sample},
		xticklabel style = {rotate = 45, anchor = north east},
	]

		% boxplots
		\boxplots{%
			box colors from table = {rawData/DL_synEnh_boxplot.tsv}{sample},
			fill opacity = .5
		}{rawData/DL_synEnh}
		
		% add sample size
		\samplesize[nodes near coords style = {name = samplesize}]{rawData/DL_synEnh_boxplot.tsv}{id}{n}

	\end{axis}
	
	
	%%% correlation between Plant STARR-seq and dual-luciferase assay
	\coordinate[yshift = -\columnsep] (DL cor) at (strength |- strength xlabel.south);
	
	% will be drawn after next plot
	
	
	%%% predict enhancer strength from single fragments
	\coordinate (predict ld) at (DL cor -| \textwidth - \twocolumnwidth - \baselineskip, 0);
	
	\begin{hgroupplot}[%
		group position = {anchor = above north west, at = {(predict ld)}, xshift = \plotylabelwidth + \baselineskip},
		axis limits from table = {rawData/PEF_cor_prediction_strength_axes.tsv},
		enlargelimits = .05,
		xytick = {-10, -8, ..., 10},
		show diagonal,
		colormap name = viridis,
		colorbar hexbin,
		ylabel = {\textbf{measurement}:\\$\log_2$(enhancer strength)}
	]{\twocolumnwidth}{2}{\textbf{prediction}: $\log_2$(enhancer strength)}
	
		\nextgroupplot[
			title = {\light},
			title color = light,
		]
		
			% hexbin plot
			\addplot [hexbin] table [x = x, y = y, meta = count] {rawData/PEF_cor_prediction_strength_light_hexbin.tsv};
			
			% correlation
			\stats[stats position = south east]{rawData/PEF_cor_prediction_strength_light}
			
			
		\nextgroupplot[
			title = {\dark},
			title color = dark,
			colorbar,
		]
		
			% hexbin plot
			\addplot [hexbin] table [x = x, y = y, meta = count] {rawData/PEF_cor_prediction_strength_dark_hexbin.tsv};
			
			% correlation
			\stats{rawData/PEF_cor_prediction_strength_dark}
			
	\end{hgroupplot}
	
	
	%%% correlation between Plant STARR-seq and dual-luciferase assay (continued)
	
	\distance{group c1r1.south west}{group c1r1.above north east}
	
	\begin{axis}[
		anchor = north west,
		at = {(DL cor)},
		xshift = \plotylabelwidth,
		width = \xdistance,
		height = \ydistance,
		axis limits from table = {rawData/DL_synEnh_cor_axes.tsv},
		enlargelimits = .1,
		xlabel = {$\log_2$(enhancer strength)},
		xlabel style = {name = xlabel},
		ylabel = {$\log_2$(NanoLuc/Luc)},
		scatter/classes = {
			none={black},
			35S={35S enhancer},
			syn1={syn1},
			syn2={syn2},
			syn3={syn3},
			syn4={syn4},
			syn5={syn5},
			syn6={syn6},
			syn7={syn7},
			syn8={syn8},
			syn9={syn9},
			syn10={syn10},
			syn11={syn11}
		},
		legend style = {
			anchor = north west,
			at = {(1, 1)},
			font = \figsmall,
			cells = {align = left, row sep = -2ex},
			inner ysep = 0pt
		},
		legend image post style = {fill opacity = 1, mark size = 1.25},
		legend plot pos = left,
		legend cell align = left,
		legend columns = 2,
	]
	
		% regression line
		\addplot [draw = none, forget plot] table [x = enrichment, y = {create col/linear regression = { y = l2ratio}}] {rawData/DL_synEnh_cor_points.tsv};
		
		\draw[gray, dashed] (\xmin, \xmin * \pgfplotstableregressiona + \pgfplotstableregressionb) -- (\xmax, \xmax * \pgfplotstableregressiona + \pgfplotstableregressionb);
	
		% scatter plot
		\addplot [
			scatter,
			scatter src = explicit symbolic,
			only marks,
			mark = solido,
			mark size = 2,
			error bars/y dir = both,
			error bars/y explicit,
		] table[x = enrichment, y = l2ratio, meta = enhancer, y error = CI] {rawData/DL_synEnh_cor_points.tsv};
		
		% correlation
		\stats{rawData/DL_synEnh_cor}
		
		% legend
		\getlegendentriessynenh
	
	\end{axis}
	
	
	%%% subfigure labels
	\subfiglabel{strength}
	\subfiglabel{DL plot}
	\subfiglabel{DL cor}
	\subfiglabel{predict ld}

\end{tikzpicture}%
			\caption{%
				\captiontitle{Enhancer fragments can be used to build condition-specific synthetic enhancers}%
				\subfig{A} Plot of the strength of synthetic enhancers created by randomly combining up to three fragments derived from mutation-sensitive regions of the \enhancer{AB80}, \enhancer{Cab-1}, and \enhancer{rbcS-E9} enhancers (see \cref{fig:PEF_cooperativity}\subfigunformatted{A}) as measured by Plant STARR-seq in the \light{} or \dark. The synthetic enhancers were grouped into four categories: inactive, $\log_2$(enhancer strength) \textle{} 1 in both conditions; constitutive, similar strength in both conditions;  light-activated, at least two-fold more active in the light; dark-activated, at least two-fold more active in the dark. The number ($n$) of synthetic enhancers in each category is indicated.\nextentry
				\subfig{B} Dual-luciferase reporter constructs (see \cref{fig:PEfl_fwd}\subfigunformatted{D}) were created for 11 synthetic enhancers (syn1\textendash 11). Nanoluciferase activity was measured in at least 4 T2 plants from these lines and normalized to the activity of luciferase. The NanoLuc/Luc ratio was normalized to a control construct without an enhancer (none; $\log_2$ set to 0). Box plots are as defined in \cref{fig:PEfl_fwd}\subfigunformatted{E}.
				\subfig{C} The mean NanoLuc/Luc ratio was compared to the mean enhancer strength determined by STARR-seq. A linear regression line is shown as a dashed line. Error bars represent the 95\% confidence interval. The constituent fragments of the synthetic enhancers are indicated with fragments separated by a + sign. The first letter indicates the  enhancer from which the fragment is derived (A, \enhancer{AB80}; C, \enhancer{Cab-1}; R, \enhancer{rbcS-E9}) and the lowercase letters represent the fragment name.\nextentry
				\subfig{D} A linear model was built to predict the strength of the synthetic enhancers based on the strength of the constituent individual fragments. Hexbin plots (color represents the count of points in each hexagon) of the correlation between the model's prediction and the measured data are shown.\nextentry
				In \plainsubfigref{C}\subfigrefand\plainsubfigref{D}, Pearson's $R^2$, Spearman's $\rho$, and number ($n$) of synthetic enhancers are indicated.
			}%
			\label{fig:PEF_model}
		\end{fig}

	
	\fi
	%%% Main figures end
	
	%%% Supplementary figures start
	\ifsupp
		
		\pagestyle{SuppData}
		\markright{Supplemental Data. Jores et al. \paperref}
		
		\begin{sfig}
			\begin{tikzpicture}

	%%% plant enhancer strength by orientation in light
	\coordinate (light) at (0, 0);

	\def\shadeplot{%
		\fill[gray, fill opacity = .1] (1, \ymin) rectangle (1.5, \ymax) (2, \ymin) rectangle (2.5, \ymax) (3, \ymin) rectangle (\xmax, \ymax);
		\draw (1.5, \ymin) -- (1.5, \ymax) (2.5, \ymin) -- (2.5, \ymax);
	}
	
	\node[anchor = north west, node font = \figsmall, text depth = 0pt] (pori) at (light) {\hphantom{orientation}};
	\node[anchor = south west, text depth = 0pt, rotate = -90, font = \hphantom{A}] (ptitle) at (pori.north east) {\vphantom{\light}};
	
	\distance{ptitle.north west}{\textwidth, 0}

	\begin{hgroupplot}[%
		group position = {anchor = above north west, at = {(pori.north east)}},
		axis limits from table = {rawData/PEfl_strength_light_axes.tsv},
		enlarge y limits = {value = .125, lower},
		ytick = {-2, 0, ..., 10},
		group/every plot/.append style = {x grids = false},
		zero line,
		ylabel = $\log_2$(enhancer strength),
		ylabel style = {xshift = .5em}
	]{\xdistance + \plotylabelwidth}{4}{segment}
		
		\nextgroupplot[
			title = \enhancer{35S},
			width = 4.8/10.8*\plotwidth,
			title style = {minimum width = 4.8/10.8*\plotwidth},
			title color = 35S,
			x tick table half segment = {rawData/PEfl_strength_light_35S_boxplot.tsv}{part},
			zero line,
			execute at begin axis/.append = {
				\fill[gray, fill opacity = .1] (1, \ymin) rectangle (\xmax, \ymax);
			}
		]
		
			% boxplots
			\halfboxplots[]{%
				box color = 35S,
				fill opacity = .5
			}{rawData/PEfl_strength_light_35S}
			
			% add sample size
			\samplesizehalf[nodes near coords style = {name = samplesize}]{rawData/PEfl_strength_light_35S_boxplot.tsv}{id}{n}
			
			% save coordinates
			\coordinate (35S) at (1, 0);
			\coordinate (35S dist) at (1.225, 0);
		
		
		\nextgroupplot[
			title = \enhancer{AB80},
			width = 12.8/10.8*\plotwidth,
			title style = {minimum width = 12.8/10.8*\plotwidth},
			title color = AB80,
			x tick table half segment = {rawData/PEfl_strength_light_AB80_boxplot.tsv}{part},
			execute at begin axis/.append = {\shadeplot},
		]
		
			% boxplots
			\halfboxplots[]{%
				box color = AB80,
				fill opacity = .5
			}{rawData/PEfl_strength_light_AB80}
			
			% add sample size
			\samplesizehalf{rawData/PEfl_strength_light_AB80_boxplot.tsv}{id}{n}
			
			% save coordinates
			\pgfplotsinvokeforeach{1, ..., 3}{		
				\coordinate (AB80_#1) at (#1, 0);
			}
		
		
		\nextgroupplot[
			title = \enhancer{Cab-1},
			width = 12.8/10.8*\plotwidth,
			title style = {minimum width = 12.8/10.8*\plotwidth},
			title color = Cab-1,
			x tick table half segment = {rawData/PEfl_strength_light_Cab-1_boxplot.tsv}{part},
			execute at begin axis/.append = {\shadeplot},
		]
		
			% boxplots
			\halfboxplots[]{%
				box color = Cab-1,
				fill opacity = .5
			}{rawData/PEfl_strength_light_Cab-1}
			
			% add sample size
			\samplesizehalf{rawData/PEfl_strength_light_Cab-1_boxplot.tsv}{id}{n}
			
			% save coordinates
			\pgfplotsinvokeforeach{1, ..., 3}{		
				\coordinate (Cab-1_#1) at (#1, 0);
			}
		
		
		\nextgroupplot[
			title = \enhancer{rbcS-E9},
			width = 12.8/10.8*\plotwidth,
			title style = {minimum width = 12.8/10.8*\plotwidth},
			title color = rbcS-E9,
			x tick table half segment = {rawData/PEfl_strength_light_rbcS-E9_boxplot.tsv}{part}
		]
		
			% boxplots
			\halfboxplots[]{%
				box color = rbcS-E9,
				fill opacity = .5
			}{rawData/PEfl_strength_light_rbcS-E9}
			
			% add sample size
			\samplesizehalf{rawData/PEfl_strength_light_rbcS-E9_boxplot.tsv}{id}{n},
			execute at begin axis/.append = {\shadeplot},
			
			% save coordinates
			\pgfplotsinvokeforeach{1, ..., 3}{		
				\coordinate (rbcS-E9_#1) at (#1, 0);
			}
			
	\end{hgroupplot}
	
	\distance{group c4r1.south}{group c4r1.north}
	
	\node[anchor = south west, rotate = -90, draw = black, fill = light!20, minimum width = \ydistance, shift = {(-.5\pgflinewidth, -.5\pgflinewidth)}, text depth = 0pt, font = \hphantom{A}] at (group c4r1.north east) {\light};
	
	% orientation labels
	\distance{35S}{35S dist}
	
	\node[anchor = south, node font = \figsmall, xshift = -\xdistance, text depth = 0pt] (lfwd) at (group c1r1.south -| 35S) {fwd};
	\node[anchor = south, node font = \figsmall, xshift = \xdistance, text depth = 0pt] at (group c1r1.south -| 35S) {rev};		
	
	\foreach \enh in {AB80, Cab-1, rbcS-E9}{
		\foreach \x in {1, ..., 3}{
			\node[anchor = south, node font = \figsmall, xshift = -\xdistance, text depth = 0pt] at (group c1r1.south -| \enh_\x) {fwd};
			\node[anchor = south, node font = \figsmall, xshift = \xdistance, text depth = 0pt] at (group c1r1.south -| \enh_\x) {rev};		
		}
	}
	
	\node[anchor = base east, node font = \figsmall, text depth = 0pt] at (group c1r1.west |- lfwd.base) {orientation};
	
	\foreach \x in {1, ..., 4}{
		\draw[dashed] (group c\x r1.west |- samplesize.south) -- (group c\x r1.east |- samplesize.south);
	}
	
	
	%%% plant enhancer strength by orientation in dark
	\coordinate[yshift = -\columnsep] (dark) at (light |- xlabel.south);

	\def\shadeplot{%
		\fill[gray, fill opacity = .1] (1, \ymin) rectangle (1.5, \ymax) (2, \ymin) rectangle (2.5, \ymax) (3, \ymin) rectangle (\xmax, \ymax);
		\draw (1.5, \ymin) -- (1.5, \ymax) (2.5, \ymin) -- (2.5, \ymax);
	}
	
	\node[anchor = north west, node font = \figsmall, text depth = 0pt] (pori) at (dark) {\hphantom{orientation}};
	\node[anchor = south west, text depth = 0pt, rotate = -90, font = \hphantom{A}] (ptitle) at (pori.north east) {\vphantom{\dark}};
	
	\distance{ptitle.north west}{\textwidth, 0}

	\begin{hgroupplot}[%
		group position = {anchor = above north west, at = {(pori.north east)}},
		axis limits from table = {rawData/PEfl_strength_dark_axes.tsv},
		enlarge y limits = {value = .125, lower},
		ytick = {-2, 0, ..., 10},
		group/every plot/.append style = {x grids = false},
		zero line,
		ylabel = $\log_2$(enhancer strength),
		ylabel style = {xshift = .5em}
	]{\xdistance + \plotylabelwidth}{4}{segment}
		
		\nextgroupplot[
			title = \enhancer{35S},
			width = 4.8/10.8*\plotwidth,
			title style = {minimum width = 4.8/10.8*\plotwidth},
			title color = 35S,
			x tick table half segment = {rawData/PEfl_strength_dark_35S_boxplot.tsv}{part},
			zero line,
			execute at begin axis/.append = {
				\fill[gray, fill opacity = .1] (1, \ymin) rectangle (\xmax, \ymax);
			}
		]
		
			% boxplots
			\halfboxplots[]{%
				box color = 35S,
				fill opacity = .5
			}{rawData/PEfl_strength_dark_35S}
			
			% add sample size
			\samplesizehalf[nodes near coords style = {name = samplesize}]{rawData/PEfl_strength_dark_35S_boxplot.tsv}{id}{n}
			
			% save coordinates
			\coordinate (35S) at (1, 0);
			\coordinate (35S dist) at (1.225, 0);
		
		
		\nextgroupplot[
			title = \enhancer{AB80},
			width = 12.8/10.8*\plotwidth,
			title style = {minimum width = 12.8/10.8*\plotwidth},
			title color = AB80,
			x tick table half segment = {rawData/PEfl_strength_dark_AB80_boxplot.tsv}{part},
			execute at begin axis/.append = {\shadeplot},
		]
		
			% boxplots
			\halfboxplots[]{%
				box color = AB80,
				fill opacity = .5
			}{rawData/PEfl_strength_dark_AB80}
			
			% add sample size
			\samplesizehalf{rawData/PEfl_strength_dark_AB80_boxplot.tsv}{id}{n}
			
			% save coordinates
			\pgfplotsinvokeforeach{1, ..., 3}{		
				\coordinate (AB80_#1) at (#1, 0);
			}
		
		
		\nextgroupplot[
			title = \enhancer{Cab-1},
			width = 12.8/10.8*\plotwidth,
			title style = {minimum width = 12.8/10.8*\plotwidth},
			title color = Cab-1,
			x tick table half segment = {rawData/PEfl_strength_dark_Cab-1_boxplot.tsv}{part},
			execute at begin axis/.append = {\shadeplot},
		]
		
			% boxplots
			\halfboxplots[]{%
				box color = Cab-1,
				fill opacity = .5
			}{rawData/PEfl_strength_dark_Cab-1}
			
			% add sample size
			\samplesizehalf{rawData/PEfl_strength_dark_Cab-1_boxplot.tsv}{id}{n}
			
			% save coordinates
			\pgfplotsinvokeforeach{1, ..., 3}{		
				\coordinate (Cab-1_#1) at (#1, 0);
			}
		
		
		\nextgroupplot[
			title = \enhancer{rbcS-E9},
			width = 12.8/10.8*\plotwidth,
			title style = {minimum width = 12.8/10.8*\plotwidth},
			title color = rbcS-E9,
			x tick table half segment = {rawData/PEfl_strength_dark_rbcS-E9_boxplot.tsv}{part},
			execute at begin axis/.append = {\shadeplot},
		]
		
			% boxplots
			\halfboxplots[]{%
				box color = rbcS-E9,
				fill opacity = .5
			}{rawData/PEfl_strength_dark_rbcS-E9}
			
			% add sample size
			\samplesizehalf{rawData/PEfl_strength_dark_rbcS-E9_boxplot.tsv}{id}{n}
			
			% save coordinates
			\pgfplotsinvokeforeach{1, ..., 3}{		
				\coordinate (rbcS-E9_#1) at (#1, 0);
			}
			
	\end{hgroupplot}
	
	\distance{group c4r1.south}{group c4r1.north}
	
	\node[anchor = south west, rotate = -90, draw = black, fill = dark!20, minimum width = \ydistance, shift = {(-.5\pgflinewidth, -.5\pgflinewidth)}, text depth = 0pt, font = \hphantom{A}] at (group c4r1.north east) {\dark};
	
	% orientation labels
	\distance{35S}{35S dist}
	
	\node[anchor = south, node font = \figsmall, xshift = -\xdistance, text depth = 0pt] (lfwd) at (group c1r1.south -| 35S) {fwd};
	\node[anchor = south, node font = \figsmall, xshift = \xdistance, text depth = 0pt] at (group c1r1.south -| 35S) {rev};		
	
	\foreach \enh in {AB80, Cab-1, rbcS-E9}{
		\foreach \x in {1, ..., 3}{
			\node[anchor = south, node font = \figsmall, xshift = -\xdistance, text depth = 0pt] at (group c1r1.south -| \enh_\x) {fwd};
			\node[anchor = south, node font = \figsmall, xshift = \xdistance, text depth = 0pt] at (group c1r1.south -| \enh_\x) {rev};		
		}
	}
	
	\node[anchor = base east, node font = \figsmall, text depth = 0pt] at (group c1r1.west |- lfwd.base) {orientation};
	
	\foreach \x in {1, ..., 4}{
		\draw[dashed] (group c\x r1.west |- samplesize.south) -- (group c\x r1.east |- samplesize.south);
	}
	
	
	%%% plant enhancer light-responsiveness by orientation
	\coordinate[yshift = -\columnsep] (lightResp) at (light |- xlabel.south);

	\def\shadeplot{%
		\fill[gray, fill opacity = .1] (1, \ymin) rectangle (1.5, \ymax) (2, \ymin) rectangle (2.5, \ymax) (3, \ymin) rectangle (\xmax, \ymax);
		\draw (1.5, \ymin) -- (1.5, \ymax) (2.5, \ymin) -- (2.5, \ymax);
	}
	
	\node[anchor = north west, node font = \figsmall, text depth = 0pt] (pori) at (lightResp) {\hphantom{orientation}};
	\node[anchor = south west, text depth = 0pt, rotate = -90, font = \hphantom{A}] (ptitle) at (pori.north east) {\vphantom{\lightResp}};
	
	\distance{ptitle.north west}{\textwidth, 0}

	\begin{hgroupplot}[%
		group position = {anchor = above north west, at = {(pori.north east)}},
		axis limits from table = {rawData/PEfl_lightResp_axes.tsv},
		enlarge y limits = {value = .125, lower},
		ytick = {-6, -4, ..., 10},
		group/every plot/.append style = {x grids = false},
		zero line,
		ylabel = {$\log_2$(\lightResp)},
		ylabel style = {xshift = .75em}
	]{\xdistance + \plotylabelwidth}{4}{segment}
		
		\nextgroupplot[
			title = \enhancer{35S},
			width = 4.8/10.8*\plotwidth,
			title style = {minimum width = 4.8/10.8*\plotwidth},
			title color = 35S,
			x tick table half segment = {rawData/PEfl_lightResp_35S_boxplot.tsv}{part},
			zero line,
			execute at begin axis/.append = {
				\fill[gray, fill opacity = .1] (1, \ymin) rectangle (\xmax, \ymax);
			}
		]
		
			% boxplots
			\halfboxplots[]{%
				box color = 35S,
				fill opacity = .5
			}{rawData/PEfl_lightResp_35S}
			
			% add sample size
			\samplesizehalf[nodes near coords style = {name = samplesize}]{rawData/PEfl_lightResp_35S_boxplot.tsv}{id}{n}
			
			% save coordinates
			\coordinate (35S) at (1, 0);
			\coordinate (35S dist) at (1.225, 0);
		
		
		\nextgroupplot[
			title = \enhancer{AB80},
			width = 12.8/10.8*\plotwidth,
			title style = {minimum width = 12.8/10.8*\plotwidth},
			title color = AB80,
			x tick table half segment = {rawData/PEfl_lightResp_AB80_boxplot.tsv}{part},
			execute at begin axis/.append = {\shadeplot},
		]
		
			% boxplots
			\halfboxplots[]{%
				box color = AB80,
				fill opacity = .5
			}{rawData/PEfl_lightResp_AB80}
			
			% add sample size
			\samplesizehalf{rawData/PEfl_lightResp_AB80_boxplot.tsv}{id}{n}
			
			% save coordinates
			\pgfplotsinvokeforeach{1, ..., 3}{		
				\coordinate (AB80_#1) at (#1, 0);
			}
		
		
		\nextgroupplot[
			title = \enhancer{Cab-1},
			width = 12.8/10.8*\plotwidth,
			title style = {minimum width = 12.8/10.8*\plotwidth},
			title color = Cab-1,
			x tick table half segment = {rawData/PEfl_lightResp_Cab-1_boxplot.tsv}{part},
			execute at begin axis/.append = {\shadeplot},
		]
		
			% boxplots
			\halfboxplots[]{%
				box color = Cab-1,
				fill opacity = .5
			}{rawData/PEfl_lightResp_Cab-1}
			
			% add sample size
			\samplesizehalf{rawData/PEfl_lightResp_Cab-1_boxplot.tsv}{id}{n}
			
			% save coordinates
			\pgfplotsinvokeforeach{1, ..., 3}{		
				\coordinate (Cab-1_#1) at (#1, 0);
			}
		
		
		\nextgroupplot[
			title = \enhancer{rbcS-E9},
			width = 12.8/10.8*\plotwidth,
			title style = {minimum width = 12.8/10.8*\plotwidth},
			title color = rbcS-E9,
			x tick table half segment = {rawData/PEfl_lightResp_rbcS-E9_boxplot.tsv}{part},
			execute at begin axis/.append = {\shadeplot},
		]
		
			% boxplots
			\halfboxplots[]{%
				box color = rbcS-E9,
				fill opacity = .5
			}{rawData/PEfl_lightResp_rbcS-E9}
			
			% add sample size
			\samplesizehalf{rawData/PEfl_lightResp_rbcS-E9_boxplot.tsv}{id}{n}
			
			% save coordinates
			\pgfplotsinvokeforeach{1, ..., 3}{		
				\coordinate (rbcS-E9_#1) at (#1, 0);
			}
			
	\end{hgroupplot}
	
	\distance{group c4r1.south}{group c4r1.north}
	
	\node[anchor = south west, rotate = -90, draw = black, top color = light!20, bottom color = dark!20, minimum width = \ydistance, shift = {(-.5\pgflinewidth, -.5\pgflinewidth)}, text depth = 0pt, font = \hphantom{A}] at (group c4r1.north east) {\lightResp};
	
	% orientation labels
	\distance{35S}{35S dist}
	
	\node[anchor = south, node font = \figsmall, xshift = -\xdistance, text depth = 0pt] (lfwd) at (group c1r1.south -| 35S) {fwd};
	\node[anchor = south, node font = \figsmall, xshift = \xdistance, text depth = 0pt] at (group c1r1.south -| 35S) {rev};		
	
	\foreach \enh in {AB80, Cab-1, rbcS-E9}{
		\foreach \x in {1, ..., 3}{
			\node[anchor = south, node font = \figsmall, xshift = -\xdistance, text depth = 0pt] at (group c1r1.south -| \enh_\x) {fwd};
			\node[anchor = south, node font = \figsmall, xshift = \xdistance, text depth = 0pt] at (group c1r1.south -| \enh_\x) {rev};		
		}
	}
	
	\node[anchor = base east, node font = \figsmall, text depth = 0pt] at (group c1r1.west |- lfwd.base) {orientation};
	
	\foreach \x in {1, ..., 4}{
		\draw[dashed] (group c\x r1.west |- samplesize.south) -- (group c\x r1.east |- samplesize.south);
	}
	

	%%% subfigure labels
	\subfiglabel{light}
	\subfiglabel{dark}
	\subfiglabel{lightResp}
	
\end{tikzpicture}%
			\caption{%
				\captiontitle[\supports{\cref{fig:PEfl_fwd}}]{Enhancer strength and light-responsiveness is orientation-independent}%
				\subfigtwo{A}{B} Full-length (FL) enhancers, as well as 169-bp long segments from their 5\textprime{} or 3\textprime{} end, of the \textit{Pisum sativum} \enhancer{AB80} and \enhancer{rbcS-E9} genes and the \textit{Triticum aestivum} \enhancer{Cab-1} gene were cloned in the forward (fwd; data reproduced from \cref{fig:PEfl_fwd}, \subfigunformatted{B}\subfigrefand\subfigunformatted{C}) or reverse (rev) orientation upstream of the 35S minimal promoter driving the expression of a barcoded GFP reporter gene. All constructs were pooled and the viral 35S enhancer was added as an internal control. The pooled enhancer library was subjected to Plant STARR-seq in tobacco leaves with plants grown for 2 days in normal light/dark cycles \parensubfig{A} or completely in the dark \parensubfig{B} prior to RNA extraction. Enhancer strength was normalized to a control construct without an enhancer ($\log_2$ set to 0).\nextentry
				\subfig{C} Light-responsiveness ($\log_2$[enhancer strength\textsuperscript{\light}/enhancer strength\textsuperscript{\dark}]) was determined for the indicated enhancer segments.\nextentry
				Box plots represent the median (center line), upper and lower quartiles (box limits), 1.5$\times$ interquartile range (whiskers), and outliers (points) for all corresponding barcodes from two replicates. Numbers at the bottom of each box plot indicate the number of barcodes in each group.
			}%
			\label{sfig:PEfl_ori}
		\end{sfig}
		
		\begin{sfig}
			\begin{tikzpicture}
	\pgfplotsset{
		every axis/.append style = {
			height = 2.75cm
		}
	}
	\addtolength{\plotylabelwidth}{\baselineskip}

	%%% rep correlation PEfl
	\coordinate (PEfl reps) at (0, 0);
	
	\begin{hgroupplot}[%
		group position = {anchor = above north west, at = {(PEfl reps)}, xshift = \plotylabelwidth},
		axis limits from table = {rawData/PEfl_cor_reps_axes.tsv},
		enlargelimits = .05,
		xytick = {-10, -8, ..., 10},
		legend style = {anchor = south east, at = {(1, 0)}},
		legend image post style = {fill opacity = 1, mark size = 1.25},
		legend plot pos = right,
		legend cell align = right,
		scatter/classes = {
			none={black},
			35S={35S enhancer},
			AB80={AB80},
			Cab-1={Cab-1},
			rbcS-E9={rbcS-E9}
		},
		show diagonal,
		ylabel = {\textbf{replicate 2}:\\$\log_2$(enhancer strength)}
	]{\twocolumnwidth}{2}{\textbf{replicate 1}: $\log_2$(enhancer strength)}
	
		\nextgroupplot[
			title = \light,
			title color = light
		]
		
			% scatter plot
			\addplot [
				scatter,
				scatter src = explicit symbolic,
				only marks,
				mark = solido,
				fill opacity = .5
			] table[x = rep1, y = rep2, meta = enhancer] {rawData/PEfl_cor_reps_light_points.tsv};
			
			% correlation
			\stats{rawData/PEfl_cor_reps_light}
			
			
		\nextgroupplot[
			title = \dark,
			title color = dark
		]
		
			% scatter plot
			\addplot [
				scatter,
				scatter src = explicit symbolic,
				only marks,
				mark = solido,
				fill opacity = .5
			] table[x = rep1, y = rep2, meta = enhancer] {rawData/PEfl_cor_reps_dark_points.tsv};
			
			% correlation
			\stats{rawData/PEfl_cor_reps_dark}
			
			% legend
			\legend{\enhancer{none}, \enhancer{35S}, \enhancer{AB80}, \enhancer{Cab-1}, \enhancer{rbcS-E9}}
			
	\end{hgroupplot}
	
	
	%%% repicate correlation PEV library (light/dark)
	\coordinate (PEV ld 1vs2) at (PEfl reps -| \textwidth - \twocolumnwidth, 0);
	\coordinate[yshift = -\columnsep] (PEV ld 1vs3) at (PEfl reps |- xlabel.south);
	\coordinate (PEV ld 2vs3) at (PEV ld 1vs3 -| PEV ld 1vs2);

	\foreach \repA/\repB in {1/2, 1/3, 2/3}{
		\begin{hgroupplot}[%
			group position = {anchor = above north west, at = {(PEV ld \repA vs\repB)}, xshift = \plotylabelwidth},
			axis limits from table = {rawData/PEV_ld_cor_reps_axes.tsv},
			enlargelimits = .05,
			xytick = {-10, -8, ..., 10},
			show diagonal,
			colormap name = viridis,
			colorbar hexbin,
			ylabel = {\textbf{replicate \repB}:\\$\log_2$(enhancer strength)}
		]{\twocolumnwidth}{2}{\textbf{replicate \repA}: $\log_2$(enhancer strength)}
		
			\nextgroupplot[
				title = \light,
				title color = light
			]
			
				% hexbin plot
				\addplot [hexbin] table [x = x, y = y, meta = count] {rawData/PEV_ld_cor_reps_\repA vs\repB_light_hexbin.tsv};
				
				% correlation
				\stats{rawData/PEV_ld_cor_reps_\repA vs\repB_light}
				
				
			\nextgroupplot[
				title = \dark,
				title color = dark,
				colorbar
			]
			
				% hexbin plot
				\addplot [hexbin] table [x = x, y = y, meta = count] {rawData/PEV_ld_cor_reps_\repA vs\repB_dark_hexbin.tsv};
				
				% correlation
				\stats{rawData/PEV_ld_cor_reps_\repA vs\repB_dark}
				
		\end{hgroupplot}
	}


	%%% repicate correlation PEV library (circadian rhythm)
	\coordinate[yshift = -\columnsep] (PEV cr reps) at (PEfl reps |- xlabel.south);

	\begin{hgroupplot}[%
		group position = {anchor = above north west, at = {(PEV cr reps)}, xshift = \plotylabelwidth},
		axis limits from table = {rawData/PEV_cr_cor_reps_axes.tsv},
		enlargelimits = .05,
		xytick = {-10, -8, ..., 10},
		show diagonal,
		colormap name = viridis,
		colorbar hexbin,
		ylabel = {\textbf{replicate 2}:\\$\log_2$(enhancer strength)}
	]{\textwidth}{5}{\textbf{replicate 1}: $\log_2$(enhancer strength)}
	
		\pgfplotsinvokeforeach{0, 6, ..., 24}{
			\nextgroupplot[
				title = {\timepoint{#1}},
				title color = t#1,
				last plot style = {colorbar}
			]
			
				% hexbin plot
				\addplot [hexbin] table [x = x, y = y, meta = count] {rawData/PEV_cr_cor_reps_#1_hexbin.tsv};
				
				% correlation
				\stats{rawData/PEV_cr_cor_reps_#1};
		}

	\end{hgroupplot}
	
	
	%%% repicate correlation PEF
	\coordinate[yshift = -\columnsep] (PEF 1vs2) at (PEfl reps |- xlabel.south);
	\coordinate (PEF 1vs3) at (PEF 1vs2 -| \textwidth - \twocolumnwidth, 0);
	
	\distance{PEV cr reps}{PEF 1vs2}
	
	\coordinate[yshift = \ydistance] (PEF 2vs3) at (PEF 1vs2);

	\foreach \repA/\repB in {1/2, 1/3, 2/3}{
		\begin{hgroupplot}[%
			group position = {anchor = above north west, at = {(PEF \repA vs\repB)}, xshift = \plotylabelwidth},
			axis limits from table = {rawData/PEF_cor_reps_axes.tsv},
			enlargelimits = .05,
			xytick = {-10, -8, ..., 10},
			show diagonal,
			colormap name = viridis,
			colorbar hexbin,
			ylabel = {\textbf{replicate \repB}:\\$\log_2$(enhancer strength)}
		]{\twocolumnwidth}{2}{\textbf{replicate \repA}: $\log_2$(enhancer strength)}
		
			\nextgroupplot[
				title = \light,
				title color = light
			]
			
				% hexbin plot
				\addplot [hexbin] table [x = x, y = y, meta = count] {rawData/PEF_cor_reps_\repA vs\repB_light_hexbin.tsv};
				
				% correlation
				\stats{rawData/PEF_cor_reps_\repA vs\repB_light};
				
				
			\nextgroupplot[
				title = \dark,
				title color = dark,
				colorbar
			]
			
				% hexbin plot
				\addplot [hexbin] table [x = x, y = y, meta = count] {rawData/PEF_cor_reps_\repA vs\repB_dark_hexbin.tsv};
				
				% correlation
				\stats{rawData/PEF_cor_reps_\repA vs\repB_dark};
				
		\end{hgroupplot}
	}
	
	
	%%% replicate correlation PEVdouble library
	\coordinate (PEval reps) at (PEF 2vs3 -| \textwidth - \twocolumnwidth, 0);
	
	\begin{hgroupplot}[%
		group position = {anchor = above north west, at = {(PEval reps)}, xshift = \plotylabelwidth},
		axis limits from table = {rawData/PEval_cor_reps_axes.tsv},
		enlargelimits = .05,
		xytick = {-10, -8, ..., 10},
		show diagonal,
		colormap name = viridis,
		colorbar hexbin,
		ylabel = {\textbf{replicate 2}:\\$\log_2$(enhancer strength)}
	]{\twocolumnwidth}{2}{\textbf{replicate 1}: $\log_2$(enhancer strength)}
	
		\nextgroupplot[
			title = \light,
			title color = light
		]
		
			% hexbin plot
			\addplot [hexbin] table [x = x, y = y, meta = count] {rawData/PEval_cor_reps_light_hexbin.tsv};
			
			% correlation
			\stats{rawData/PEval_cor_reps_light};
			
			
		\nextgroupplot[
			title = \dark,
			title color = dark,
			colorbar
		]
		
			% hexbin plot
			\addplot [hexbin] table [x = x, y = y, meta = count] {rawData/PEval_cor_reps_dark_hexbin.tsv};
			
			% correlation
			\stats{rawData/PEval_cor_reps_dark};
			
	\end{hgroupplot}
	

	%%% subfigure labels
	\subfiglabel{PEfl reps}
	\subfiglabel{PEV ld 1vs2}
	\subfiglabel{PEV ld 1vs3}
	\subfiglabel{PEV ld 2vs3}
	\subfiglabel{PEV cr reps}
	\subfiglabel{PEF 1vs2}
	\subfiglabel{PEF 1vs3}
	\subfiglabel{PEF 2vs3}
	\subfiglabel{PEval reps}
	
	\end{tikzpicture}%
			\nextpagecaption{%
				\captiontitle[\supports{all figures}]{Plant STARR-seq yields highly reproducible results}%
				\subfigrange{A}{I} Correlation between biological replicates of Plant STARR-seq for the enhancer library used in \cref{fig:PEfl_fwd,sfig:PEfl_ori} \parensubfig{A}, the single-nucleotide enhancer variants library used in \cref{fig:PEV_ld_MutSens,fig:PEV_ld_motifs,fig:PEV_cr,sfig:PEV_ld_summary,sfig:PEV_ld_heatmaps,sfig:PEV_ld_fimo,sfig:PEV_cr_0vs24,sfig:PEV_cr_heatmap_AB80,sfig:PEV_cr_heatmap_Cab-1,sfig:PEV_cr_heatmap_rbcS-E9,sfig:PEV_cr_MutSens,sfig:PEval_cor_main,sfig:PEV_overlap} \parensubfigrange{B}{E}, the synthetic enhancer library used in \cref{fig:PEF_cooperativity,fig:PEF_model,sfig:PEval_cor_main,sfig:PEF_model_lr,sfig:PEF_fragments} \parensubfigrange{F}{H}, and the combined double-deletion and synthetic enhancer valdiation library used in \cref{fig:PEVdouble,sfig:PEval_cor_main} \parensubfig{I} performed under the indicated condition or at the indicated time points.\nextentry
				Pearson's $R^2$, Spearman's $\rho$, and number ($n$) of enhancer variants or enhancer fragment combinations are indicated. The color in the hexbin plots in \parensubfigrange{B}{I} represents the count of points in each hexagon.
			}%
			\label{sfig:repCor}
		\end{sfig}
		
		\begin{sfig}
			\begin{tikzpicture}

	%%% dual-luciferase assy in light and dark
	\coordinate (DL plot) at (0, 0);
	
	\def\shadeplot{%
		\fill[light, fill opacity = .1] (\xmin, \ymin) rectangle (1, \ymax) (1.5, \ymin) rectangle (2, \ymax) (2.5, \ymin) rectangle (3, \ymax) (3.5, \ymin) rectangle (4, \ymax) (4.5, \ymin) rectangle (5, \ymax);
		\fill[dark, fill opacity = .1] (1, \ymin) rectangle (1.5, \ymax) (2, \ymin) rectangle (2.5, \ymax) (3, \ymin) rectangle (3.5, \ymax) (4, \ymin) rectangle (4.5, \ymax) (5, \ymin) rectangle (\xmax, \ymax);
		\draw (1.5, \ymin) -- (1.5, \ymax) (2.5, \ymin) -- (2.5, \ymax) (3.5, \ymin) -- (3.5, \ymax) (4.5, \ymin) -- (4.5, \ymax);
	}
	
	\begin{axis}[%
		anchor = north west,
		at = {(DL plot)},
		width = \twocolumnwidth - \plotylabelwidth,
		xshift = \plotylabelwidth,
		x grids = false,
		axis limits from table = {rawData/DL_enhancers_ld_axes.tsv},
		enlarge y limits = {value = .125, lower},
		zero line,
		xlabel = enhancer,
		ylabel = $\log_2$(NanoLuc/Luc),
		x tick table half enhancer = {rawData/DL_enhancers_ld_boxplot.tsv}{enhancer},
		execute at begin axis/.append = {\shadeplot},
	]

		% boxplots
		\halfboxplots[]{%
			box colors from table = {rawData/DL_enhancers_ld_boxplot.tsv}{enhancer},
			fill opacity = .5
		}{rawData/DL_enhancers_ld}
		
		% add sample size
		\samplesizehalf[nodes near coords style = {name = samplesize}]{rawData/DL_enhancers_ld_boxplot.tsv}{id}{n}
		
		% save coordinates
		\pgfplotsinvokeforeach{1, ..., 5}{		
			\coordinate (DL #1) at (#1, 0);
		}
		\coordinate (DL dist) at (1.225, 0);

	\end{axis}
	
	% light labels
	\distance{DL 1}{DL dist}
	
	\foreach \x in {1, ..., 5}{
		\node[anchor = south, font = \bfseries\vphantom{A}, xshift = -\xdistance, text depth = 0pt] (lplus) at (last plot.south -| DL \x) {+};
		\node[anchor = south, font = \bfseries\vphantom{A}, xshift = \xdistance, text depth = 0pt] at (last plot.south -| DL \x) {\textminus};		
	}
	
	\node[anchor = base east, node font = \figsmall, text depth = 0pt] at (last plot.west |- lplus.base) {light};
	
	\draw[dashed] (last plot.west |- samplesize.south) -- (last plot.east |- samplesize.south);
	
	
	%%% subfigure labels
	\subfiglabel{DL plot}

\end{tikzpicture}%
			\caption{%
				\captiontitle[\supports{\cref{fig:PEfl_fwd}}]{The dual-luciferase assay cannot detect light-responsive enhancer activity}%
				\subfig{A} Transgenic \textit{Arabidopsis} lines were generated with constructs harboring a constitutively expressed luciferase (Luc) gene and a nanoluciferase (NanoLuc) gene under control of a 35S minimal promoter coupled to the 35S enhancer or the \segment{B} segments of the \enhancer{AB80}, \enhancer{Cab-1}, or \enhancer{rbcS-E9} enhancers (see \cref{fig:PEfl_fwd}\subfigunformatted{D}). Nanoluciferase activity was measured in 3\textendash 5 T2 plants from these lines and normalized to the activity of luciferase. Plants were either grown in normal light/dark cycles ($+$ light; data reproduced from \cref{fig:PEfl_fwd}\subfigunformatted{E}) or shifted to complete darkness (\textminus{} light) for 4 days prior to sample collection. The NanoLuc/Luc ratio was normalized to a control construct without an enhancer (none; $\log_2$ set to 0). Box plots represent the median (center line), upper and lower quartiles (box limits), 1.5$\times$ interquartile range (whiskers), and outliers (points) for all corresponding plant lines from two (\textminus{} light) or three ($+$ light) independent replicates. Numbers at the bottom of each box plot indicate the number of samples in each group.
			}%
			\label{sfig:DL_light_dark}
		\end{sfig}

		\begin{sfig}
			%%% add number of variants with a given effect to plots
\newcommand{\varEffects}[2][]{%
	\addplot [
		only marks,
		mark = text,
		text mark as node = true,
		stats position = south,
		text mark = {
			\phantom{decreasing}\llap{increasing} $= \pgfmathprintnumber[fixed, 1000 sep = {{{{,}}}}]{\increase}$\\
			\phantom{decreasing}\llap{neutral} $= \pgfmathprintnumber[fixed, 1000 sep = {{{{,}}}}]{\neutral}$\\
			\phantom{decreasing}\llap{decreasing} $= \pgfmathprintnumber[fixed, 1000 sep = {{{{,}}}}]{\decrease}$
		},
		visualization depends on = \thisrow{increase} \as \increase,
		visualization depends on = \thisrow{neutral} \as \neutral,
		visualization depends on = \thisrow{decrease} \as \decrease,
		#1,
	] table [x expr = 0, y expr = 0] {#2_stats.tsv};
}

%%% histograms with weak variants in a lighter color
\pgfplotsset{
	weak/.style = {x filter/.expression = {\thisrow{y} >= -1 && \thisrow{y} <= 1 ? \thisrow{count} : 0}},
	strong/.style = {x filter/.expression = {\thisrow{y} < -1 || \thisrow{y} > 1 ? \thisrow{count} : 0}}
}

\newcommand{\histogramStrength}[3][]{%
	\histogram[fill = #1!33, weak]{#2}{#3}%
	\histogram[fill = #1, strong]{#2}{#3}%
	\varEffects{#2}%
}

\begin{tikzpicture}

	%%% summary of variant effects for the 5' enhancer segments
	\coordinate (part A) at (0, 0);
	
	\node[anchor = north west, text depth = 0pt, align = center] (ptitle) at (part A) {\vphantom{\enhancer{AB80} enhancer}\\\vphantom{\segment{B} segment}};
	\distance{ptitle.south}{ptitle.north}
	\pgfmathsetlength{\plotwidth}{(\twocolumnwidth - \plotylabelwidth - \groupplotsep - \ydistance) / 2}
	
	\node[anchor = north west, draw, fill = light!20, text depth = 0pt, minimum width = \plotwidth, xshift = \plotylabelwidth] (tlight) at (part A) {\light};
	
	\begin{hvgroupplot}[%
		group position = {anchor = north west, at = {(tlight.south west)}, shift = {(.5\pgflinewidth, .5\pgflinewidth)}},
		enlarge x limits = {value = .05, upper},
		enlarge y limits = .05,
		ytick = {-6, -4, ..., 5},
		zero line,
		title style = {
			anchor = south,
			at = {(1, .5)},
			rotate = -90,
			minimum width = \plotheight,
			fill = titlecol!20,
			draw = black,
			shift = {(-.5\pgflinewidth, -.5\pgflinewidth)}
		}
	]{2\plotwidth + \plotylabelwidth + \groupplotsep}{10.5cm + \plotxlabelheight + 2\groupplotsep}{2}{3}{number of variants}{$\log_2$(enhancer strength)}
		
		\nextgroupplot[
			axis limits from table = {rawData/PEV_ld_summary_A_light_axes.tsv}
		]
		
			\histogramStrength[AB80]{rawData/PEV_ld_summary_AB80_A_light}{count}
			
		\nextgroupplot[
			axis limits from table = {rawData/PEV_ld_summary_A_dark_axes.tsv},
			title = \enhancer{AB80} enhancer\\\segment{A} segment,
			title color = AB80
		]
		
			\histogramStrength[AB80]{rawData/PEV_ld_summary_AB80_A_dark}{count}
			
			
		\nextgroupplot[
			axis limits from table = {rawData/PEV_ld_summary_A_light_axes.tsv}
		]
		
			\histogramStrength[Cab-1]{rawData/PEV_ld_summary_Cab-1_A_light}{count}
			
		\nextgroupplot[
			axis limits from table = {rawData/PEV_ld_summary_A_dark_axes.tsv},
			title = \enhancer{Cab-1} enhancer\\\segment{A} segment,
			title color = Cab-1
		]
		
			\histogramStrength[Cab-1]{rawData/PEV_ld_summary_Cab-1_A_dark}{count}
			
			
		\nextgroupplot[
			axis limits from table = {rawData/PEV_ld_summary_A_light_axes.tsv}
		]
		
			\histogramStrength[rbcS-E9]{rawData/PEV_ld_summary_rbcS-E9_A_light}{count}
			
		\nextgroupplot[
			axis limits from table = {rawData/PEV_ld_summary_A_dark_axes.tsv},
			title = \enhancer{rbcS-E9} enhancer\\\segment{A} segment,
			title color = rbcS-E9
		]
		
			\histogramStrength[rbcS-E9]{rawData/PEV_ld_summary_rbcS-E9_A_dark}{count}
			
	\end{hvgroupplot}
	
	\node[anchor = south, draw, fill = dark!20, text depth = 0pt, minimum width = \plotwidth, yshift = -.5\pgflinewidth] (tdark) at (group c2r1.north) {\dark};
	
	
	%%% summary of variant effects for the 3' enhancer segments
	\coordinate (part B) at (part A -| \textwidth - \twocolumnwidth, 0);
	
	\node[anchor = north west, draw, fill = light!20, text depth = 0pt, minimum width = \plotwidth, xshift = \plotylabelwidth] (tlight) at (part B) {\light};
	
	\begin{hvgroupplot}[%
		group position = {anchor = north west, at = {(tlight.south west)}, shift = {(.5\pgflinewidth, .5\pgflinewidth)}},
		enlarge x limits = {value = .05, upper},
		enlarge y limits = .05,
		ytick = {-6, -4, ..., 5},
		zero line,
		title style = {
			anchor = south,
			at = {(1, .5)},
			rotate = -90,
			minimum width = \plotheight,
			fill = titlecol!20,
			draw = black,
			shift = {(-.5\pgflinewidth, -.5\pgflinewidth)}
		}
	]{2\plotwidth + \plotylabelwidth + \groupplotsep}{10.5cm + \plotxlabelheight + 2\groupplotsep}{2}{3}{number of variants}{$\log_2$(enhancer strength)}
		
		\nextgroupplot[
			axis limits from table = {rawData/PEV_ld_summary_B_light_axes.tsv}
		]
		
			\histogramStrength[AB80]{rawData/PEV_ld_summary_AB80_B_light}{count}
			
		\nextgroupplot[
			axis limits from table = {rawData/PEV_ld_summary_B_dark_axes.tsv},
			title = \enhancer{AB80} enhancer\\\segment{B} segment,
			title color = AB80
		]
		
			\histogramStrength[AB80]{rawData/PEV_ld_summary_AB80_B_dark}{count}
			
			
		\nextgroupplot[
			axis limits from table = {rawData/PEV_ld_summary_B_light_axes.tsv}
		]
		
			\histogramStrength[Cab-1]{rawData/PEV_ld_summary_Cab-1_B_light}{count}
			
		\nextgroupplot[
			axis limits from table = {rawData/PEV_ld_summary_B_dark_axes.tsv},
			title = \enhancer{Cab-1} enhancer\\\segment{B} segment,
			title color = Cab-1
		]
		
			\histogramStrength[Cab-1]{rawData/PEV_ld_summary_Cab-1_B_dark}{count}
			
			
		\nextgroupplot[
			axis limits from table = {rawData/PEV_ld_summary_B_light_axes.tsv}
		]
		
			\histogramStrength[rbcS-E9]{rawData/PEV_ld_summary_rbcS-E9_B_light}{count}
			
		\nextgroupplot[
			axis limits from table = {rawData/PEV_ld_summary_B_dark_axes.tsv},
			title = \enhancer{rbcS-E9} enhancer\\\segment{B} segment,
			title color = rbcS-E9
		]
		
			\histogramStrength[rbcS-E9]{rawData/PEV_ld_summary_rbcS-E9_B_dark}{count}
			
	\end{hvgroupplot}
	
	\node[anchor = south, draw, fill = dark!20, text depth = 0pt, minimum width = \plotwidth, yshift = -.5\pgflinewidth] (tdark) at (group c2r1.north) {\dark};
	
	
	%%% subfigure labels
	\subfiglabel{part A}
	\subfiglabel{part B}

\end{tikzpicture}%
			\caption{%
				\captiontitle[\supports{\cref{fig:PEV_ld_MutSens,fig:PEV_ld_motifs}}]{Few single-nucleotide mutations have a strong effect on enhancer strength}%
				\subfigtwo{A}{B} All possible single-nucleotide substitution, deletion, and insertion variants of the \segment{A} \parensubfig{A} and \segment{B} \parensubfig{B} segments of the \enhancer{AB80}, \enhancer{Cab-1}, and \enhancer{rbcS-E9} enhancers were subjected to Plant STARR-seq in tobacco plants grown in normal light/dark cycles (\light) or completely in the dark (\dark) for two days prior to RNA extraction. Enhancer strength was normalized to the wild-type variant ($\log_2$ set to 0). Variants were grouped into three categories: increasing, $\log_2$(enhancer strength)~$> 1$; neutral, $\log_2$(enhancer strength) between \textminus1 and 1;  decreasing, $\log_2$(enhancer strength)~$< -1$. The number of variants in each category is indicated. Neutral variants are shown in a lighter color than increasing and decreasing variants in the histograms.
			}%
			\label{sfig:PEV_ld_summary}
		\end{sfig}
		
		\begin{sfig}
			\begin{tikzpicture}
	\addtolength{\plotylabelwidth}{.5\baselineskip}
	\setlength{\plotheight}{6.6cm}

	%%% heatmap AB80 (light/dark)
	\coordinate (AB80) at (0, 0);
	
	\node[anchor = north west, text depth = 0pt] (ptitle) at (AB80) {\vphantom{\enhancer{AB80} enhancer \segment{A} segment}};
	
	\distance{ptitle.south}{ptitle.north}

	\begin{hvgroupplot}[%
		group position = {anchor = above north west, at = {(AB80)}, xshift = \plotylabelwidth},
		axis limits from table = {rawData/PEV_ld_heatmap_AB80_axes.tsv},
		xtick = {1, 20, 40, ..., 300},
		title color = AB80,
		heatmap,
	]{\textwidth - \ydistance}{\plotheight}{2}{4}{position}{}
			
		\nextgroupplot [
			title = \enhancer{AB80} enhancer \segment{A} segment,
			axis limits = AB80 A,
			ylabel = insertion,
			height = 8/9 * \plotheight
		]
		
			\heatmapIns{rawData/PEV_ld_heatmap_AB80_A_light_ins.tsv}
			
			
		\nextgroupplot [
			title = \enhancer{AB80} enhancer \segment{B} segment,
			axis limits = AB80 B,
			height = 8/9 * \plotheight
		]
		
			\heatmapIns{rawData/PEV_ld_heatmap_AB80_B_light_ins.tsv}
			
			
		\nextgroupplot [
			axis limits = AB80 A,
			ylabel = {substitution\\[-.1\baselineskip]or deletion},
			height = 10/9 * \plotheight
		]
		
			\heatmapSubDel{rawData/PEV_ld_heatmap_AB80_A_light_sub+del.tsv}
			\heatmapWT{rawData/PEV_ld_heatmap_AB80_WT.tsv}
			
			
		\nextgroupplot [
			axis limits = AB80 B,
			height = 10/9 * \plotheight
		]
		
			\heatmapSubDel{rawData/PEV_ld_heatmap_AB80_B_light_sub+del.tsv}
			\heatmapWT{rawData/PEV_ld_heatmap_AB80_wt.tsv}
			
			
		\nextgroupplot [
			axis limits = AB80 A,
			ylabel = insertion,
			height = 8/9 * \plotheight
		]
		
			\heatmapIns{rawData/PEV_ld_heatmap_AB80_A_dark_ins.tsv}
			
			
		\nextgroupplot [
			axis limits = AB80 B,
			height = 8/9 * \plotheight
		]
		
			\heatmapIns{rawData/PEV_ld_heatmap_AB80_B_dark_ins.tsv}
			
			
		\nextgroupplot [
			axis limits = AB80 A,
			ylabel = {substitution\\[-.1\baselineskip]or deletion},
			height = 10/9 * \plotheight
		]
		
			\heatmapSubDel{rawData/PEV_ld_heatmap_AB80_A_dark_sub+del.tsv}
			\heatmapWT{rawData/PEV_ld_heatmap_AB80_WT.tsv}
			
			
		\nextgroupplot [
			axis limits = AB80 B,
			height = 10/9 * \plotheight,
			colorbar heatmap
		]
		
			\heatmapSubDel{rawData/PEV_ld_heatmap_AB80_B_dark_sub+del.tsv}
			\heatmapWT{rawData/PEV_ld_heatmap_AB80_wt.tsv}

	\end{hvgroupplot}
	
	\distance{group c2r2.south}{group c2r1.north}
	
	\node[anchor = south west, rotate = -90, draw = black, fill = light!20, minimum width = \ydistance, shift = {(-.5\pgflinewidth, -.5\pgflinewidth)}, text depth = 0pt, font = \hphantom{A}] at (group c2r1.north east) {\light};
	\node[anchor = south west, rotate = -90, draw = black, fill = dark!20, minimum width = \ydistance, shift = {(-.5\pgflinewidth, -.5\pgflinewidth)}, text depth = 0pt, font = \hphantom{A}] at (group c2r3.north east) {\dark};
	
	
	%%% heatmap Cab-1 (light/dark)
	\coordinate[yshift = -\columnsep] (Cab-1) at (AB80 |- plot xlabel.south);
	
	\distance{ptitle.south}{ptitle.north}

	\begin{hvgroupplot}[%
		group position = {anchor = above north west, at = {(Cab-1)}, xshift = \plotylabelwidth},
		axis limits from table = {rawData/PEV_ld_heatmap_Cab-1_axes.tsv},
		xtick = {1, 20, 40, ..., 300},
		title color = Cab-1,
		heatmap,
	]{\textwidth - \ydistance}{\plotheight}{2}{4}{position}{}
			
		\nextgroupplot [
			title = \enhancer{Cab-1} enhancer \segment{A} segment,
			axis limits = Cab-1 A,
			ylabel = insertion,
			height = 8/9 * \plotheight
		]
		
			\heatmapIns{rawData/PEV_ld_heatmap_Cab-1_A_light_ins.tsv}
			
			
		\nextgroupplot [
			title = \enhancer{Cab-1} enhancer \segment{B} segment,
			axis limits = Cab-1 B,
			height = 8/9 * \plotheight
		]
		
			\heatmapIns{rawData/PEV_ld_heatmap_Cab-1_B_light_ins.tsv}
			
			
		\nextgroupplot [
			axis limits = Cab-1 A,
			ylabel = {substitution\\[-.1\baselineskip]or deletion},
			height = 10/9 * \plotheight
		]
		
			\heatmapSubDel{rawData/PEV_ld_heatmap_Cab-1_A_light_sub+del.tsv}
			\heatmapWT{rawData/PEV_ld_heatmap_Cab-1_WT.tsv}
			
			
		\nextgroupplot [
			axis limits = Cab-1 B,
			height = 10/9 * \plotheight
		]
		
			\heatmapSubDel{rawData/PEV_ld_heatmap_Cab-1_B_light_sub+del.tsv}
			\heatmapWT{rawData/PEV_ld_heatmap_Cab-1_wt.tsv}
			
			
		\nextgroupplot [
			axis limits = Cab-1 A,
			ylabel = insertion,
			height = 8/9 * \plotheight
		]
		
			\heatmapIns{rawData/PEV_ld_heatmap_Cab-1_A_dark_ins.tsv}
			
			
		\nextgroupplot [
			axis limits = Cab-1 B,
			height = 8/9 * \plotheight
		]
		
			\heatmapIns{rawData/PEV_ld_heatmap_Cab-1_B_dark_ins.tsv}
			
			
		\nextgroupplot [
			axis limits = Cab-1 A,
			ylabel = {substitution\\[-.1\baselineskip]or deletion},
			height = 10/9 * \plotheight
		]
		
			\heatmapSubDel{rawData/PEV_ld_heatmap_Cab-1_A_dark_sub+del.tsv}
			\heatmapWT{rawData/PEV_ld_heatmap_Cab-1_WT.tsv}
			
			
		\nextgroupplot [
			axis limits = Cab-1 B,
			height = 10/9 * \plotheight,
			colorbar heatmap
		]
		
			\heatmapSubDel{rawData/PEV_ld_heatmap_Cab-1_B_dark_sub+del.tsv}
			\heatmapWT{rawData/PEV_ld_heatmap_Cab-1_wt.tsv}

	\end{hvgroupplot}
	
	\distance{group c2r2.south}{group c2r1.north}
	
	\node[anchor = south west, rotate = -90, draw = black, fill = light!20, minimum width = \ydistance, shift = {(-.5\pgflinewidth, -.5\pgflinewidth)}, text depth = 0pt, font = \hphantom{A}] at (group c2r1.north east) {\light};
	\node[anchor = south west, rotate = -90, draw = black, fill = dark!20, minimum width = \ydistance, shift = {(-.5\pgflinewidth, -.5\pgflinewidth)}, text depth = 0pt, font = \hphantom{A}] at (group c2r3.north east) {\dark};
	
	
	%%% heatmap rbcS-E9-1 (light/dark)
	\coordinate[yshift = -\columnsep] (rbcS-E9) at (AB80 |- plot xlabel.south);
	
	\distance{ptitle.south}{ptitle.north}

	\begin{hvgroupplot}[%
		group position = {anchor = above north west, at = {(rbcS-E9)}, xshift = \plotylabelwidth},
		axis limits from table = {rawData/PEV_ld_heatmap_rbcS-E9_axes.tsv},
		xtick = {1, 20, 40, ..., 300},
		title color = rbcS-E9,
		heatmap,
	]{\textwidth - \ydistance}{\plotheight}{2}{4}{position}{}
			
		\nextgroupplot [
			title = \enhancer{rbcS-E9} enhancer \segment{A} segment,
			axis limits = rbcS-E9 A,
			ylabel = insertion,
			height = 8/9 * \plotheight
		]
		
			\heatmapIns{rawData/PEV_ld_heatmap_rbcS-E9_A_light_ins.tsv}
			
			
		\nextgroupplot [
			title = \enhancer{rbcS-E9} enhancer \segment{B} segment,
			axis limits = rbcS-E9 B,
			height = 8/9 * \plotheight
		]
		
			\heatmapIns{rawData/PEV_ld_heatmap_rbcS-E9_B_light_ins.tsv}
			
			
		\nextgroupplot [
			axis limits = rbcS-E9 A,
			ylabel = {substitution\\[-.1\baselineskip]or deletion},
			height = 10/9 * \plotheight
		]
		
			\heatmapSubDel{rawData/PEV_ld_heatmap_rbcS-E9_A_light_sub+del.tsv}
			\heatmapWT{rawData/PEV_ld_heatmap_rbcS-E9_WT.tsv}
			
			
		\nextgroupplot [
			axis limits = rbcS-E9 B,
			height = 10/9 * \plotheight
		]
		
			\heatmapSubDel{rawData/PEV_ld_heatmap_rbcS-E9_B_light_sub+del.tsv}
			\heatmapWT{rawData/PEV_ld_heatmap_rbcS-E9_wt.tsv}
			
			
		\nextgroupplot [
			axis limits = rbcS-E9 A,
			ylabel = insertion,
			height = 8/9 * \plotheight
		]
		
			\heatmapIns{rawData/PEV_ld_heatmap_rbcS-E9_A_dark_ins.tsv}
			
			
		\nextgroupplot [
			axis limits = rbcS-E9 B,
			height = 8/9 * \plotheight
		]
		
			\heatmapIns{rawData/PEV_ld_heatmap_rbcS-E9_B_dark_ins.tsv}
			
			
		\nextgroupplot [
			axis limits = rbcS-E9 A,
			ylabel = {substitution\\[-.1\baselineskip]or deletion},
			height = 10/9 * \plotheight
		]
		
			\heatmapSubDel{rawData/PEV_ld_heatmap_rbcS-E9_A_dark_sub+del.tsv}
			\heatmapWT{rawData/PEV_ld_heatmap_rbcS-E9_WT.tsv}
			
			
		\nextgroupplot [
			axis limits = rbcS-E9 B,
			height = 10/9 * \plotheight,
			colorbar heatmap
		]
		
			\heatmapSubDel{rawData/PEV_ld_heatmap_rbcS-E9_B_dark_sub+del.tsv}
			\heatmapWT{rawData/PEV_ld_heatmap_rbcS-E9_wt.tsv}

	\end{hvgroupplot}
	
	\distance{group c2r2.south}{group c2r1.north}
	
	\node[anchor = south west, rotate = -90, draw = black, fill = light!20, minimum width = \ydistance, shift = {(-.5\pgflinewidth, -.5\pgflinewidth)}, text depth = 0pt, font = \hphantom{A}] at (group c2r1.north east) {\light};
	\node[anchor = south west, rotate = -90, draw = black, fill = dark!20, minimum width = \ydistance, shift = {(-.5\pgflinewidth, -.5\pgflinewidth)}, text depth = 0pt, font = \hphantom{A}] at (group c2r3.north east) {\dark};
	
	
	%%% subfigure labels
	\subfiglabel{AB80}
	\subfiglabel{Cab-1}
	\subfiglabel{rbcS-E9}

\end{tikzpicture}%
			\nextpagecaption{%
				\captiontitle[\supports{\cref{fig:PEV_ld_MutSens,fig:PEV_ld_motifs}}]{Saturation mutagenesis reveals mutation-sensitive patches in plant enhancers}%
				\subfigrange{A}{C} All possible single-nucleotide substitution, deletion, and insertion variants of the \segment{A} and \segment{B} segments of the \enhancer{AB80} \parensubfig{A}, \enhancer{Cab-1} \parensubfig{B}, and \enhancer{rbcS-E9} \parensubfig{C} enhancers were subjected to Plant STARR-seq in tobacco plants grown in normal light/dark cycles (\light) or completely in the dark (\dark) for two days prior to RNA extraction. Enhancer strength was normalized to the wild-type variant ($\log_2$ set to 0) and plotted as a heatmap. Missing values are shown in light gray and wild-type variants are marked with a gray dot.
			}%
			\label{sfig:PEV_ld_heatmaps}
		\end{sfig}
		
		\begin{sfig}
			\begin{tikzpicture}

	%%% correlation of variant effects in overlapping region
	\coordinate (overlap) at (0, 0);
	
	\node[anchor = north west, text depth = 0pt, font = \hphantom{A}] (ptitle) at (overlap) {\vphantom{\light}};
	
	\distance{ptitle.south}{ptitle.north}
	
	\begin{hvgroupplot}[%
		group position = {anchor = above north west, at = {(overlap)}, xshift = \plotylabelwidth},
		axis limits from table = {rawData/PEV_cor_overlap_axes.tsv},
		enlargelimits = .05,
		xytick = {-10, -8, ..., 10},
		show diagonal,
	]{\textwidth - \ydistance}{7cm + \plotxlabelheight + \groupplotsep}{3}{2}{\textbf{\segment{A} segment}: $\log_2$(enhancer strength)}{\textbf{\segment{B} segment}: $\log_2$(enhancer strength)}
	
		\nextgroupplot[
			title = \enhancer{AB80},
			title color = AB80
		]
		
			% scatter plot
			\addplot [
				AB80,
				only marks,
				mark = solido,
				fill opacity = .5
			] table[x = partA, y = partB] {rawData/PEV_cor_overlap_AB80_light_points.tsv};
			
			% correlation
			\stats{rawData/PEV_cor_overlap_AB80_light}
			
			
		\nextgroupplot[
			title = \enhancer{Cab-1},
			title color = Cab-1
		]
		
			% scatter plot
			\addplot [
				Cab-1,
				only marks,
				mark = solido,
				fill opacity = .5
			] table[x = partA, y = partB] {rawData/PEV_cor_overlap_Cab-1_light_points.tsv};
			
			% correlation
			\stats{rawData/PEV_cor_overlap_Cab-1_light}
			
	
		\nextgroupplot[
			title = \enhancer{rbcS-E9},
			title color = rbcS-E9
		]
		
			% scatter plot
			\addplot [
				rbcS-E9,
				only marks,
				mark = solido,
				fill opacity = .5
			] table[x = partA, y = partB] {rawData/PEV_cor_overlap_rbcS-E9_light_points.tsv};
			
			% correlation
			\stats{rawData/PEV_cor_overlap_rbcS-E9_light}
			
			
		\nextgroupplot
		
			% scatter plot
			\addplot [
				AB80,
				only marks,
				mark = solido,
				fill opacity = .5
			] table[x = partA, y = partB] {rawData/PEV_cor_overlap_AB80_dark_points.tsv};
			
			% correlation
			\stats{rawData/PEV_cor_overlap_AB80_dark}
			
			
		\nextgroupplot
		
			% scatter plot
			\addplot [
				Cab-1,
				only marks,
				mark = solido,
				fill opacity = .5
			] table[x = partA, y = partB] {rawData/PEV_cor_overlap_Cab-1_dark_points.tsv};
			
			% correlation
			\stats{rawData/PEV_cor_overlap_Cab-1_dark}
			
	
		\nextgroupplot
		
			% scatter plot
			\addplot [
				rbcS-E9,
				only marks,
				mark = solido,
				fill opacity = .5
			] table[x = partA, y = partB] {rawData/PEV_cor_overlap_rbcS-E9_dark_points.tsv};
			
			% correlation
			\stats{rawData/PEV_cor_overlap_rbcS-E9_dark}
			
	\end{hvgroupplot}
	
	\distance{group c3r1.south}{group c3r1.north}
	
	\node[anchor = south west, rotate = -90, draw = black, fill = light!20, minimum width = \ydistance, shift = {(-.5\pgflinewidth, -.5\pgflinewidth)}, text depth = 0pt, font = \hphantom{A}] (tlight) at (group c3r1.north east) {\light};
	\node[anchor = south west, rotate = -90, draw = black, fill = dark!20, minimum width = \ydistance, shift = {(-.5\pgflinewidth, -.5\pgflinewidth)}, text depth = 0pt, font = \hphantom{A}] (tdark) at (group c3r2.north east) {\dark};

\end{tikzpicture}%
			\caption{%
				\captiontitle[\supports{\cref{fig:PEV_ld_MutSens}}]{Effects of mutations in the overlap region of the \segment{A} and \segment{B} enhancer segments are more similar in the dark}%
				\subfig{A} All possible single-nucleotide substitution, deletion, and insertion variants of the \segment{A} and \segment{B} segments of the \enhancer{AB80}, \enhancer{Cab-1}, and \enhancer{rbcS-E9} enhancers were subjected to Plant STARR-seq in tobacco plants grown in normal light/dark cycles (\light) or completely in the dark (\dark) for two days prior to RNA extraction. Enhancer strength was normalized to the wild-type variant ($\log_2$ set to 0). For mutations located in the overlap region between the two segments (positions 79\textendash169 in \enhancer{AB80}, positions 100\textendash169 in \enhancer{Cab-1}, and positions 66\textendash169 in \enhancer{rbcS-E9}), the normalized enhancer strength measured in the context of the \segment{A} segment (x axis) is compared against the normalized enhancer strength measured in the context of the \segment{B} segment (y axis). Pearson's $R^2$, Spearman's $\rho$, and number ($n$) of enhancer variants are indicated.
			}%
			\label{sfig:PEV_overlap}
		\end{sfig}
		
		\begin{sfig}
			\begin{tikzpicture}

	%%% mutation sensitivity (positional mean) AB80
	\coordinate (AB80) at (0, 0);
	
	\pgfplotstableread{rawData/PEV_ld_TF-scan_AB80.tsv}{\TFscan}
	
	\begin{hgroupplot}[%
		group position = {anchor = above north west, at = {(AB80)}, xshift = \plotylabelwidth},
		axis limits from table = {rawData/PEV_ld_mutSens_AB80_axes.tsv},
		enlarge y limits = {value = .125, upper},
		ytick = {-10, ..., 10},
		ylabel = {$\log_2$(enhancer strength)},
		title color = AB80,
		zero line,
		legend style = {anchor = south west, at = {(0, 0)}},
		legend image post style = {very thick},
		legend plot pos = left,
		legend cell align = left,
	]{\textwidth}{2}{position}
	
		\nextgroupplot[
			title = \enhancer{AB80} enhancer \segment{A} segment,
			axis limits = AB80 A,
			shade overlap A = AB80
		]
			
			\addplot[line plot, dark] table [x = position, y = dark] {rawData/PEV_ld_mutSens_AB80_A_lines.tsv};
			\addplot[line plot, light] table [x = position, y = light] {rawData/PEV_ld_mutSens_AB80_A_lines.tsv};
			
			\foreachtablerow{\TFscan}{
				\pgfmathint{#1 - 1}
				\pgfplotstableforeachcolumn{\TFscan}\as{\col}{
					\pgfplotstablegetelem{\pgfmathresult}{\col}\of{\TFscan}
					\expandafter\edef\csname this\col\endcsname{\pgfplotsretval}
				}
				\def\xshift{0pt}
				\ifcase\pgfmathresult\relax
				\or\or\or\or\or\or\or
					\def\xshift{.3\baselineskip}
				\or
					\def\xshift{-.3\baselineskip}
				\or\or
					\def\xshift{-\baselineskip}
				\fi
				\edef\thisTF{
					\noexpand\fill[black, fill opacity = .1] (\thisstart, \noexpand\ymax) rectangle (\thisstop, \noexpand\ymin);
					\noexpand\node[anchor = east, rotate = 90, node font = \noexpand\figsmaller, yshift = \xshift] at (\thiscenter, \noexpand\ymax) {\thisid};
				}
				\thisTF
			}
						
			\legend{dark, light}
			
			
		\nextgroupplot[
			title = \enhancer{AB80} enhancer \segment{B} segment,
			axis limits = AB80 B,
			shade overlap B = AB80
		]
		
			\addplot[line plot, dark] table [x = position, y = dark] {rawData/PEV_ld_mutSens_AB80_B_lines.tsv};
			\addplot[line plot, light] table [x = position, y = light] {rawData/PEV_ld_mutSens_AB80_B_lines.tsv};
			
			\foreachtablerow{\TFscan}{
				\pgfmathint{#1 - 1}
				\pgfplotstableforeachcolumn{\TFscan}\as{\col}{
					\pgfplotstablegetelem{\pgfmathresult}{\col}\of{\TFscan}
					\expandafter\edef\csname this\col\endcsname{\pgfplotsretval}
				}
				\def\xshift{0pt}
				\ifcase\pgfmathresult\relax
				\or\or\or\or\or\or\or
					\def\xshift{.3\baselineskip}
				\or
					\def\xshift{-.3\baselineskip}
				\or\or\or\or\or
					\def\xshift{-.25\baselineskip}
				\fi
				\edef\thisTF{
					\noexpand\fill[black, fill opacity = .1] (\thisstart, \noexpand\ymax) rectangle (\thisstop, \noexpand\ymin);
					\noexpand\node[anchor = east, rotate = 90, node font = \noexpand\figsmaller, yshift = \xshift] at (\thiscenter, \noexpand\ymax) {\thisid};
				}
				\thisTF
			}
		
	\end{hgroupplot}
	
	
	%%% mutation sensitivity (positional mean) Cab-1
	\coordinate[yshift = -\columnsep] (Cab-1) at (AB80 |- xlabel.south);
	
	\pgfplotstableread{rawData/PEV_ld_TF-scan_Cab-1.tsv}{\TFscan}
	
	\begin{hgroupplot}[%
		group position = {anchor = above north west, at = {(Cab-1)}, xshift = \plotylabelwidth},
		axis limits from table = {rawData/PEV_ld_mutSens_Cab-1_axes.tsv},
		enlarge y limits = {value = .125, upper},
		ytick = {-10, ..., 10},
		ylabel = {$\log_2$(enhancer strength)},
		title color = Cab-1,
		zero line,
		legend style = {anchor = south west, at = {(0, 0)}},
		legend image post style = {very thick},
		legend plot pos = left,
		legend cell align = left,
	]{\textwidth}{2}{position}
	
		\nextgroupplot[
			title = \enhancer{Cab-1} enhancer \segment{A} segment,
			axis limits = Cab-1 A,
			shade overlap A = Cab-1
		]
		
			\addplot[line plot, dark] table [x = position, y = dark] {rawData/PEV_ld_mutSens_Cab-1_A_lines.tsv};
			\addplot[line plot, light] table [x = position, y = light] {rawData/PEV_ld_mutSens_Cab-1_A_lines.tsv};
			
			\foreachtablerow{\TFscan}{
				\pgfmathint{#1 - 1}
				\pgfplotstableforeachcolumn{\TFscan}\as{\col}{
					\pgfplotstablegetelem{\pgfmathresult}{\col}\of{\TFscan}
					\expandafter\edef\csname this\col\endcsname{\pgfplotsretval}
				}
				\edef\thisTF{
					\noexpand\fill[black, fill opacity = .1] (\thisstart, \noexpand\ymax) rectangle (\thisstop, \noexpand\ymin);
					\noexpand\node[anchor = east, rotate = 90, node font = \noexpand\figsmaller] at (\thiscenter, \noexpand\ymax) {\thisid};
				}
				\thisTF
			}
						
			\legend{dark, light}
			
			
		\nextgroupplot[
			title = \enhancer{Cab-1} enhancer \segment{B} segment,
			axis limits = Cab-1 B,
			shade overlap B = Cab-1
		]
		
			\addplot[line plot, dark] table [x = position, y = dark] {rawData/PEV_ld_mutSens_Cab-1_B_lines.tsv};
			\addplot[line plot, light] table [x = position, y = light] {rawData/PEV_ld_mutSens_Cab-1_B_lines.tsv};
			
			\foreachtablerow{\TFscan}{
				\pgfmathint{#1 - 1}
				\pgfplotstableforeachcolumn{\TFscan}\as{\col}{
					\pgfplotstablegetelem{\pgfmathresult}{\col}\of{\TFscan}
					\expandafter\edef\csname this\col\endcsname{\pgfplotsretval}
				}
				\edef\thisTF{
					\noexpand\fill[black, fill opacity = .1] (\thisstart, \noexpand\ymax) rectangle (\thisstop, \noexpand\ymin);
					\noexpand\node[anchor = east, rotate = 90, node font = \noexpand\figsmaller] at (\thiscenter, \noexpand\ymax) {\thisid};
				}
				\thisTF
			}
		
	\end{hgroupplot}
	
	
	%%% mutation sensitivity (positional mean) rbcS-E9
	\coordinate[yshift = -\columnsep] (rbcS-E9) at (AB80 |- xlabel.south);
	
	\pgfplotstableread{rawData/PEV_ld_TF-scan_rbcS-E9.tsv}{\TFscan}
	
	\begin{hgroupplot}[%
		group position = {anchor = above north west, at = {(rbcS-E9)}, xshift = \plotylabelwidth},
		axis limits from table = {rawData/PEV_ld_mutSens_rbcS-E9_axes.tsv},
		enlarge y limits = {value = .125, upper},
		ytick = {-10, ..., 10},
		ylabel = {$\log_2$(enhancer strength)},
		title color = rbcS-E9,
		zero line,
		legend style = {anchor = south west, at = {(0, 0)}},
		legend image post style = {very thick},
		legend plot pos = left,
		legend cell align = left,
	]{\textwidth}{2}{position}
	
		\nextgroupplot[
			title = \enhancer{rbcS-E9} enhancer \segment{A} segment,
			axis limits = rbcS-E9 A,
			shade overlap A = rbcS-E9
		]
			
			\addplot[line plot, dark] table [x = position, y = dark] {rawData/PEV_ld_mutSens_rbcS-E9_A_lines.tsv};
			\addplot[line plot, light] table [x = position, y = light] {rawData/PEV_ld_mutSens_rbcS-E9_A_lines.tsv};
			
			\foreachtablerow{\TFscan}{
				\pgfmathint{#1 - 1}
				\pgfplotstableforeachcolumn{\TFscan}\as{\col}{
					\pgfplotstablegetelem{\pgfmathresult}{\col}\of{\TFscan}
					\expandafter\edef\csname this\col\endcsname{\pgfplotsretval}
				}
				\edef\thisTF{
					\noexpand\fill[black, fill opacity = .1] (\thisstart, \noexpand\ymax) rectangle (\thisstop, \noexpand\ymin);
					\noexpand\node[anchor = east, rotate = 90, node font = \noexpand\figsmaller] at (\thiscenter, \noexpand\ymax) {\thisid};
				}
				\thisTF
			}
						
			\legend{dark, light}
			
			
		\nextgroupplot[
			title = \enhancer{rbcS-E9} enhancer \segment{B} segment,
			axis limits = rbcS-E9 B,
			shade overlap B = rbcS-E9
		]
			
			\addplot[line plot, dark] table [x = position, y = dark] {rawData/PEV_ld_mutSens_rbcS-E9_B_lines.tsv};
			\addplot[line plot, light] table [x = position, y = light] {rawData/PEV_ld_mutSens_rbcS-E9_B_lines.tsv};
			
			\foreachtablerow{\TFscan}{
				\pgfmathint{#1 - 1}
				\pgfplotstableforeachcolumn{\TFscan}\as{\col}{
					\pgfplotstablegetelem{\pgfmathresult}{\col}\of{\TFscan}
					\expandafter\edef\csname this\col\endcsname{\pgfplotsretval}
				}
				\edef\thisTF{
					\noexpand\fill[black, fill opacity = .1] (\thisstart, \noexpand\ymax) rectangle (\thisstop, \noexpand\ymin);
					\noexpand\node[anchor = east, rotate = 90, node font = \noexpand\figsmaller] at (\thiscenter, \noexpand\ymax) {\thisid};
				}
				\thisTF
			}
		
	\end{hgroupplot}
	
	
	%%% transcription factor legend
	\coordinate[yshift = -\columnsep] (TF legend) at (AB80 |- xlabel.south);
	
	\node[anchor = north west, xshift = \columnsep, inner sep = 0pt] at (TF legend) {%
		\rowcolors{2}{white}{gray!20}%
		\setlength{\tabcolsep}{3pt}%
		\pgfplotstabletypeset[
			begin table = {\begin{tabularx}{\textwidth - \columnsep}},
			end table = {\end{tabularx}},
			every head row/.style = {before row = \toprule, after row = \midrule},
			every last row/.style = {after row = \bottomrule},
			extra space/.style = {column type = X},
			string type,
			columns = {ID, family, consensus, ID, family, consensus},
			display columns/0/.style = {select equal part entry of = {0}{2}},
			display columns/1/.style = {select equal part entry of = {0}{2}},
			display columns/2/.style = {select equal part entry of = {0}{2}, extra space},
			display columns/3/.style = {select equal part entry of = {1}{2}},
			display columns/4/.style = {select equal part entry of = {1}{2}},
			display columns/5/.style = {select equal part entry of = {1}{2}},
			assign column name/.style = {/pgfplots/table/column name={\textbf{#1}}},
			column type = l,
		]{rawData/PEV_ld_TF-scan_legend.tsv}%
	};
	
	
	%%% subfigure labels
	\subfiglabel{AB80}
	\subfiglabel{Cab-1}
	\subfiglabel{rbcS-E9}
	\subfiglabel{TF legend}

\end{tikzpicture}%
			\nextpagecaption{%
				\captiontitle[\supports{\cref{fig:PEV_ld_motifs}}]{Mutation-sensitive regions contain few strong matches to known transcription factor binding motifs}%
				\subfigrange{A}{C} The wild-type sequences of the \segment{A} and \segment{B} segments of the \enhancer{AB80} \parensubfig{A}, \enhancer{Cab-1} \parensubfig{B}, and \enhancer{rbcS-E9} \parensubfig{C} enhancers were scanned for significant matches to known transcription factor binding motifs. Hits are shown as gray areas overlaid on the mutational sensitivity plots reproduced from \cref{fig:PEV_ld_MutSens}. The ID of the matching transcription factor motif is indicated.\nextentry
				\subfig{D} For each transcription factor motif ID, the table lists the families of transcription factors that can bind to it and the consensus binding site sequence. Ambiguous nucleotides in the consensus sequence are shown as lowercase and correspond to: n = A, C, G, or T; d = A, G, or T; r = A or G; w = A or T; m = A or C.
			}%
			\label{sfig:PEV_ld_fimo}
		\end{sfig}
		
		\begin{sfig}
			\begin{tikzpicture}
	\addtolength{\plotylabelwidth}{\baselineskip}

	%%% PEV cr: correlation between time points
	\coordinate (timepoints) at (0, 0);
	
	\node[anchor = north west, text depth = 0pt] (ptitle) at (timepoints) {\vphantom{\timepoint{24}}};
	
	\distance{ptitle.south}{ptitle.north}
	
	\begin{hgroupplot}[%
		group position = {anchor = above north west, at = {(timepoints)}, xshift = \plotylabelwidth},
		axis limits from table = {rawData/PEV_cr_cor_timepoints_axes.tsv},
		enlargelimits = .05,
		xytick = {-10, -8, ..., 10},
		show diagonal,
		colormap name = viridis,
		colorbar hexbin,
		ylabel = {\textbf{\timepoint{0}}:\\$\log_2$(enhancer strength)}
	]{\textwidth - \ydistance}{4}{\textbf{\timepoint{6}\textendash\timepoint{24}}: $\log_2$(enhancer strength)}
	
		\pgfplotsinvokeforeach{6, 12, 18, 24}{
			\nextgroupplot[
				title = {\timepoint{#1}},
				title color = t#1,
				last plot style = {colorbar}
			]
			
				% hexbin plot
				\addplot [hexbin] table [x = x, y = y, meta = count] {rawData/PEV_cr_cor_0vs#1_hexbin.tsv};
				
				% correlation
				\stats{rawData/PEV_cr_cor_0vs#1};
		}
		
	\end{hgroupplot}
	
	\distance{group c4r1.south}{group c4r1.north}
	
	\node[anchor = south west, rotate = -90, draw = black, fill = t0!20, minimum width = \ydistance, shift = {(-.5\pgflinewidth, -.5\pgflinewidth)}, text depth = 0pt, font = \hphantom{A}] at (group c4r1.north east) {\timepoint{0}};
	
	
	%%% PEV cr: correlation with PEV ld 
	\coordinate[yshift = -\columnsep] (CRvsLD) at (timepoints |- xlabel.south);
	
	\node[anchor = north west, text depth = 0pt] (ptitle) at (CRvsLD) {\vphantom{\light}};
	
	\distance{ptitle.south}{ptitle.north}
	
	\begin{hgroupplot}[%
		group position = {anchor = above north west, at = {(CRvsLD)}, xshift = \plotylabelwidth},
		axis limits from table = {rawData/PEV_cr_cor_PEV_ld_axes.tsv},
		enlargelimits = .05,
		xytick = {-10, -8, ..., 10},
		show diagonal,
		colormap name = viridis,
		colorbar hexbin,
		ylabel = {\textbf{light}:\\$\log_2$(enhancer strength)}
	]{\textwidth - \ydistance}{5}{\textbf{\timepoint{0}\textendash\timepoint{24}}: $\log_2$(enhancer strength)}
	
		\pgfplotsinvokeforeach{0, 6, ..., 24}{
			\nextgroupplot[
				title = {\timepoint{#1}},
				title color = t#1,
				last plot style = {colorbar}
			]
			
				% hexbin plot
				\addplot [hexbin] table [x = x, y = y, meta = count] {rawData/PEV_cr_cor_PEV_ld_#1_hexbin.tsv};
				
				% correlation
				\ifnum#1=6\relax
					\def\thispos{south east}
				\else
					\def\thispos{north west}
				\fi
				
				\stats[stats position = \thispos]{rawData/PEV_cr_cor_PEV_ld_#1};
			}
			
	\end{hgroupplot}
	
	\distance{group c5r1.south}{group c5r1.north}
	
	\node[anchor = south west, rotate = -90, draw = black, fill = light!20, minimum width = \ydistance, shift = {(-.5\pgflinewidth, -.5\pgflinewidth)}, text depth = 0pt, font = \hphantom{A}] at (group c5r1.north east) {\light};
	
	
	%%% subfigure labels
	\subfiglabel{timepoints}
	\subfiglabel{CRvsLD}

\end{tikzpicture}%
			\caption{%
				\captiontitle[\supports{\cref{fig:PEV_cr}}]{Strong correlation between Plant STARR-seq samples obtained 24 hours apart from each other}%
				\subfig{A} All possible single-nucleotide variants of the \enhancer{AB80}, \enhancer{Cab-1}, and \enhancer{rbcS-E9} enhancers were subjected to Plant STARR-seq in tobacco leaves. On the morning of the third day after transformation (\timepoint{-8}), the plants were shifted to constant light. Leaves were harvested for RNA extraction starting at mid-day (\timepoint{0}) and in 6 hour intervals (\timepoint{6}, 20, 26, and 32) afterwards. Hexbin plots (color represents the count of points in each hexagon) of the correlation between samples obtained at the indicated time points are shown. Pearson's $R^2$, Spearman's $\rho$, and number ($n$) of enhancer variants are indicated.\nextentry
				\subfig{B} Hexbin plots of the correlation between the samples from the time course experiment described above compared to the same library tested in normal light/dark cycles (light) as described in \cref{fig:PEV_ld_MutSens}. The "light" samples were harvested at \timepoint{0}.
			}%
			\label{sfig:PEV_cr_0vs24}
		\end{sfig}
		
		\begin{sfig}
			\begin{tikzpicture}
	\addtolength{\plotylabelwidth}{.5\baselineskip}
	\setlength{\plotheight}{16.75cm}

	%%% heatmap AB80 (circadian rhythm)
	\coordinate (AB80) at (0, 0);
	
	\node[anchor = north west, text depth = 0pt] (ptitle) at (AB80) {\vphantom{\enhancer{AB80} enhancer \segment{A} segment}};
	
	\distance{ptitle.south}{ptitle.north}

	\begin{hvgroupplot}[%
		group position = {anchor = above north west, at = {(AB80)}, xshift = \plotylabelwidth},
		axis limits from table = {rawData/PEV_cr_heatmap_AB80_axes.tsv},
		xtick = {1, 20, 40, ..., 300},
		title color = AB80,
		heatmap,
	]{\textwidth - \ydistance}{\plotheight}{2}{10}{position}{}
		
		% timepoint 0
		\nextgroupplot [
			title = \enhancer{AB80} enhancer \segment{A} segment,
			axis limits = AB80 A,
			ylabel = insertion,
			height = 8/9 * \plotheight
		]
		
			\heatmapIns{rawData/PEV_cr_heatmap_AB80_A_0_ins.tsv}
			
			
		\nextgroupplot [
			title = \enhancer{AB80} enhancer \segment{B} segment,
			axis limits = AB80 B,
			height = 8/9 * \plotheight
		]
		
			\heatmapIns{rawData/PEV_cr_heatmap_AB80_B_0_ins.tsv}
			
			
		\nextgroupplot [
			axis limits = AB80 A,
			ylabel = {substitution\\[-.1\baselineskip]or deletion},
			height = 10/9 * \plotheight
		]
		
			\heatmapSubDel{rawData/PEV_cr_heatmap_AB80_A_0_sub+del.tsv}
			\heatmapWT{rawData/PEV_cr_heatmap_AB80_WT.tsv}
			
			
		\nextgroupplot [
			axis limits = AB80 B,
			height = 10/9 * \plotheight
		]
		
			\heatmapSubDel{rawData/PEV_cr_heatmap_AB80_B_0_sub+del.tsv}
			\heatmapWT{rawData/PEV_cr_heatmap_AB80_wt.tsv}
			
			
		% timepoint 6
		\nextgroupplot [
			axis limits = AB80 A,
			ylabel = insertion,
			height = 8/9 * \plotheight
		]
		
			\heatmapIns{rawData/PEV_cr_heatmap_AB80_A_6_ins.tsv}
			
			
		\nextgroupplot [
			axis limits = AB80 B,
			height = 8/9 * \plotheight
		]
		
			\heatmapIns{rawData/PEV_cr_heatmap_AB80_B_6_ins.tsv}
			
			
		\nextgroupplot [
			axis limits = AB80 A,
			ylabel = {substitution\\[-.1\baselineskip]or deletion},
			height = 10/9 * \plotheight
		]
		
			\heatmapSubDel{rawData/PEV_cr_heatmap_AB80_A_6_sub+del.tsv}
			\heatmapWT{rawData/PEV_cr_heatmap_AB80_WT.tsv}
			
			
		\nextgroupplot [
			axis limits = AB80 B,
			height = 10/9 * \plotheight
		]
		
			\heatmapSubDel{rawData/PEV_cr_heatmap_AB80_B_6_sub+del.tsv}
			\heatmapWT{rawData/PEV_cr_heatmap_AB80_wt.tsv}
			
			
		% timepoint 12
		\nextgroupplot [
			axis limits = AB80 A,
			ylabel = insertion,
			height = 8/9 * \plotheight
		]
		
			\heatmapIns{rawData/PEV_cr_heatmap_AB80_A_12_ins.tsv}
			
			
		\nextgroupplot [
			axis limits = AB80 B,
			height = 8/9 * \plotheight
		]
		
			\heatmapIns{rawData/PEV_cr_heatmap_AB80_B_12_ins.tsv}
			
			
		\nextgroupplot [
			axis limits = AB80 A,
			ylabel = {substitution\\[-.1\baselineskip]or deletion},
			height = 10/9 * \plotheight
		]
		
			\heatmapSubDel{rawData/PEV_cr_heatmap_AB80_A_12_sub+del.tsv}
			\heatmapWT{rawData/PEV_cr_heatmap_AB80_WT.tsv}
			
			
		\nextgroupplot [
			axis limits = AB80 B,
			height = 10/9 * \plotheight
		]
		
			\heatmapSubDel{rawData/PEV_cr_heatmap_AB80_B_12_sub+del.tsv}
			\heatmapWT{rawData/PEV_cr_heatmap_AB80_wt.tsv}
			
			
		% timepoint 18
		\nextgroupplot [
			axis limits = AB80 A,
			ylabel = insertion,
			height = 8/9 * \plotheight
		]
		
			\heatmapIns{rawData/PEV_cr_heatmap_AB80_A_18_ins.tsv}
			
			
		\nextgroupplot [
			axis limits = AB80 B,
			height = 8/9 * \plotheight
		]
		
			\heatmapIns{rawData/PEV_cr_heatmap_AB80_B_18_ins.tsv}
			
			
		\nextgroupplot [
			axis limits = AB80 A,
			ylabel = {substitution\\[-.1\baselineskip]or deletion},
			height = 10/9 * \plotheight
		]
		
			\heatmapSubDel{rawData/PEV_cr_heatmap_AB80_A_18_sub+del.tsv}
			\heatmapWT{rawData/PEV_cr_heatmap_AB80_WT.tsv}
			
			
		\nextgroupplot [
			axis limits = AB80 B,
			height = 10/9 * \plotheight
		]
		
			\heatmapSubDel{rawData/PEV_cr_heatmap_AB80_B_18_sub+del.tsv}
			\heatmapWT{rawData/PEV_cr_heatmap_AB80_wt.tsv}
			
			
		% timepoint 24
		\nextgroupplot [
			axis limits = AB80 A,
			ylabel = insertion,
			height = 8/9 * \plotheight
		]
		
			\heatmapIns{rawData/PEV_cr_heatmap_AB80_A_24_ins.tsv}
			
			
		\nextgroupplot [
			axis limits = AB80 B,
			height = 8/9 * \plotheight
		]
		
			\heatmapIns{rawData/PEV_cr_heatmap_AB80_B_24_ins.tsv}
			
			
		\nextgroupplot [
			axis limits = AB80 A,
			ylabel = {substitution\\[-.1\baselineskip]or deletion},
			height = 10/9 * \plotheight
		]
		
			\heatmapSubDel{rawData/PEV_cr_heatmap_AB80_A_24_sub+del.tsv}
			\heatmapWT{rawData/PEV_cr_heatmap_AB80_WT.tsv}
			
			
		\nextgroupplot [
			axis limits = AB80 B,
			height = 10/9 * \plotheight,
			colorbar heatmap
		]
		
			\heatmapSubDel{rawData/PEV_cr_heatmap_AB80_B_24_sub+del.tsv}
			\heatmapWT{rawData/PEV_cr_heatmap_AB80_wt.tsv}

	\end{hvgroupplot}
	
	\distance{group c2r2.south}{group c2r1.north}
	
	\foreach \cond [count = \i] in {0, 6, 12, 18, 24}{
		\pgfmathint{2 * \i - 1}
		\node[anchor = south west, rotate = -90, draw = black, fill = t\cond!20, minimum width = \ydistance, shift = {(-.5\pgflinewidth, -.5\pgflinewidth)}, text depth = 0pt, font = \hphantom{A}] at (group c2r\pgfmathresult.north east) {\timepoint{\cond}};
	}

\end{tikzpicture}%
			\caption{%
				\captiontitle[\supports{\cref{fig:PEV_cr}}]{Saturation mutagenesis maps of the \enhancer{AB80} enhancer do not change much over time in constant light}%
				\subfig{A} All possible single-nucleotide variants of the \enhancer{AB80} enhancer were subjected to Plant STARR-seq in tobacco leaves. On the morning of the third day after transformation (\timepoint{-8}), the plants were shifted to constant light. Leaves were harvested for RNA extraction starting at mid-day (\timepoint{0}) and in 6 hour intervals (\timepoint{6}, 20, 26, and 32) afterwards. Enhancer strength was normalized to the wild-type variant ($\log_2$ set to 0) and plotted as a heatmap. Missing values are shown in light gray and wild-type variants are marked with a gray dot.
			}%
			\label{sfig:PEV_cr_heatmap_AB80}
		\end{sfig}
		
		\begin{sfig}
			\begin{tikzpicture}
	\addtolength{\plotylabelwidth}{.5\baselineskip}
	\setlength{\plotheight}{16.75cm}

	%%% heatmap Cab-1 (circadian rhythm)
	\coordinate (Cab-1) at (0, 0);
	
	\node[anchor = north west, text depth = 0pt] (ptitle) at (Cab-1) {\vphantom{\enhancer{Cab-1} enhancer \segment{A} segment}};
	
	\distance{ptitle.south}{ptitle.north}

	\begin{hvgroupplot}[%
		group position = {anchor = above north west, at = {(Cab-1)}, xshift = \plotylabelwidth},
		axis limits from table = {rawData/PEV_cr_heatmap_Cab-1_axes.tsv},
		xtick = {1, 20, 40, ..., 300},
		title color = Cab-1,
		heatmap,
	]{\textwidth - \ydistance}{\plotheight}{2}{10}{position}{}
		
		% timepoint 0
		\nextgroupplot [
			title = \enhancer{Cab-1} enhancer \segment{A} segment,
			axis limits = Cab-1 A,
			ylabel = insertion,
			height = 8/9 * \plotheight
		]
		
			\heatmapIns{rawData/PEV_cr_heatmap_Cab-1_A_0_ins.tsv}
			
			
		\nextgroupplot [
			title = \enhancer{Cab-1} enhancer \segment{B} segment,
			axis limits = Cab-1 B,
			height = 8/9 * \plotheight
		]
		
			\heatmapIns{rawData/PEV_cr_heatmap_Cab-1_B_0_ins.tsv}
			
			
		\nextgroupplot [
			axis limits = Cab-1 A,
			ylabel = {substitution\\[-.1\baselineskip]or deletion},
			height = 10/9 * \plotheight
		]
		
			\heatmapSubDel{rawData/PEV_cr_heatmap_Cab-1_A_0_sub+del.tsv}
			\heatmapWT{rawData/PEV_cr_heatmap_Cab-1_WT.tsv}
			
			
		\nextgroupplot [
			axis limits = Cab-1 B,
			height = 10/9 * \plotheight
		]
		
			\heatmapSubDel{rawData/PEV_cr_heatmap_Cab-1_B_0_sub+del.tsv}
			\heatmapWT{rawData/PEV_cr_heatmap_Cab-1_wt.tsv}
			
			
		% timepoint 6
		\nextgroupplot [
			axis limits = Cab-1 A,
			ylabel = insertion,
			height = 8/9 * \plotheight
		]
		
			\heatmapIns{rawData/PEV_cr_heatmap_Cab-1_A_6_ins.tsv}
			
			
		\nextgroupplot [
			axis limits = Cab-1 B,
			height = 8/9 * \plotheight
		]
		
			\heatmapIns{rawData/PEV_cr_heatmap_Cab-1_B_6_ins.tsv}
			
			
		\nextgroupplot [
			axis limits = Cab-1 A,
			ylabel = {substitution\\[-.1\baselineskip]or deletion},
			height = 10/9 * \plotheight
		]
		
			\heatmapSubDel{rawData/PEV_cr_heatmap_Cab-1_A_6_sub+del.tsv}
			\heatmapWT{rawData/PEV_cr_heatmap_Cab-1_WT.tsv}
			
			
		\nextgroupplot [
			axis limits = Cab-1 B,
			height = 10/9 * \plotheight
		]
		
			\heatmapSubDel{rawData/PEV_cr_heatmap_Cab-1_B_6_sub+del.tsv}
			\heatmapWT{rawData/PEV_cr_heatmap_Cab-1_wt.tsv}
			
			
		% timepoint 12
		\nextgroupplot [
			axis limits = Cab-1 A,
			ylabel = insertion,
			height = 8/9 * \plotheight
		]
		
			\heatmapIns{rawData/PEV_cr_heatmap_Cab-1_A_12_ins.tsv}
			
			
		\nextgroupplot [
			axis limits = Cab-1 B,
			height = 8/9 * \plotheight
		]
		
			\heatmapIns{rawData/PEV_cr_heatmap_Cab-1_B_12_ins.tsv}
			
			
		\nextgroupplot [
			axis limits = Cab-1 A,
			ylabel = {substitution\\[-.1\baselineskip]or deletion},
			height = 10/9 * \plotheight
		]
		
			\heatmapSubDel{rawData/PEV_cr_heatmap_Cab-1_A_12_sub+del.tsv}
			\heatmapWT{rawData/PEV_cr_heatmap_Cab-1_WT.tsv}
			
			
		\nextgroupplot [
			axis limits = Cab-1 B,
			height = 10/9 * \plotheight
		]
		
			\heatmapSubDel{rawData/PEV_cr_heatmap_Cab-1_B_12_sub+del.tsv}
			\heatmapWT{rawData/PEV_cr_heatmap_Cab-1_wt.tsv}
			
			
		% timepoint 18
		\nextgroupplot [
			axis limits = Cab-1 A,
			ylabel = insertion,
			height = 8/9 * \plotheight
		]
		
			\heatmapIns{rawData/PEV_cr_heatmap_Cab-1_A_18_ins.tsv}
			
			
		\nextgroupplot [
			axis limits = Cab-1 B,
			height = 8/9 * \plotheight
		]
		
			\heatmapIns{rawData/PEV_cr_heatmap_Cab-1_B_18_ins.tsv}
			
			
		\nextgroupplot [
			axis limits = Cab-1 A,
			ylabel = {substitution\\[-.1\baselineskip]or deletion},
			height = 10/9 * \plotheight
		]
		
			\heatmapSubDel{rawData/PEV_cr_heatmap_Cab-1_A_18_sub+del.tsv}
			\heatmapWT{rawData/PEV_cr_heatmap_Cab-1_WT.tsv}
			
			
		\nextgroupplot [
			axis limits = Cab-1 B,
			height = 10/9 * \plotheight
		]
		
			\heatmapSubDel{rawData/PEV_cr_heatmap_Cab-1_B_18_sub+del.tsv}
			\heatmapWT{rawData/PEV_cr_heatmap_Cab-1_wt.tsv}
			
			
		% timepoint 24
		\nextgroupplot [
			axis limits = Cab-1 A,
			ylabel = insertion,
			height = 8/9 * \plotheight
		]
		
			\heatmapIns{rawData/PEV_cr_heatmap_Cab-1_A_24_ins.tsv}
			
			
		\nextgroupplot [
			axis limits = Cab-1 B,
			height = 8/9 * \plotheight
		]
		
			\heatmapIns{rawData/PEV_cr_heatmap_Cab-1_B_24_ins.tsv}
			
			
		\nextgroupplot [
			axis limits = Cab-1 A,
			ylabel = {substitution\\[-.1\baselineskip]or deletion},
			height = 10/9 * \plotheight
		]
		
			\heatmapSubDel{rawData/PEV_cr_heatmap_Cab-1_A_24_sub+del.tsv}
			\heatmapWT{rawData/PEV_cr_heatmap_Cab-1_WT.tsv}
			
			
		\nextgroupplot [
			axis limits = Cab-1 B,
			height = 10/9 * \plotheight,
			colorbar heatmap
		]
		
			\heatmapSubDel{rawData/PEV_cr_heatmap_Cab-1_B_24_sub+del.tsv}
			\heatmapWT{rawData/PEV_cr_heatmap_Cab-1_wt.tsv}

	\end{hvgroupplot}
	
	\distance{group c2r2.south}{group c2r1.north}
	
	\foreach \cond [count = \i] in {0, 6, 12, 18, 24}{
		\pgfmathint{2 * \i - 1}
		\node[anchor = south west, rotate = -90, draw = black, fill = t\cond!20, minimum width = \ydistance, shift = {(-.5\pgflinewidth, -.5\pgflinewidth)}, text depth = 0pt, font = \hphantom{A}] at (group c2r\pgfmathresult.north east) {\timepoint{\cond}};
	}
	
	
	%%% subfigure labels
	\subfiglabel{Cab-1}

\end{tikzpicture}%
			\caption{%
				\captiontitle[\supports{\cref{fig:PEV_cr}}]{Saturation mutagenesis maps of the \enhancer{Cab-1} enhancer do not change much over time in constant light}%
				\subfig{A} The \enhancer{Cab-1} enhancer was subjected to the same experiment as the \enhancer{AB80} enhancer in \cref{sfig:PEV_cr_heatmap_AB80}.
			}%
			\label{sfig:PEV_cr_heatmap_Cab-1}
		\end{sfig}
		
		\begin{sfig}
			\begin{tikzpicture}
	\addtolength{\plotylabelwidth}{.5\baselineskip}
	\setlength{\plotheight}{16.75cm}

	%%% heatmap rbcS-E9 (circadian rhythm)
	\coordinate (rbcS-E9) at (0, 0);
	
	\node[anchor = north west, text depth = 0pt] (ptitle) at (rbcS-E9) {\vphantom{\enhancer{rbcS-E9} enhancer \segment{A} segment}};
	
	\distance{ptitle.south}{ptitle.north}

	\begin{hvgroupplot}[%
		group position = {anchor = above north west, at = {(rbcS-E9)}, xshift = \plotylabelwidth},
		axis limits from table = {rawData/PEV_cr_heatmap_rbcS-E9_axes.tsv},
		xtick = {1, 20, 40, ..., 300},
		title color = rbcS-E9,
		heatmap,
	]{\textwidth - \ydistance}{\plotheight}{2}{10}{position}{}
		
		% timepoint 0
		\nextgroupplot [
			title = \enhancer{rbcS-E9} enhancer \segment{A} segment,
			axis limits = rbcS-E9 A,
			ylabel = insertion,
			height = 8/9 * \plotheight
		]
		
			\heatmapIns{rawData/PEV_cr_heatmap_rbcS-E9_A_0_ins.tsv}
			
			
		\nextgroupplot [
			title = \enhancer{rbcS-E9} enhancer \segment{B} segment,
			axis limits = rbcS-E9 B,
			height = 8/9 * \plotheight
		]
		
			\heatmapIns{rawData/PEV_cr_heatmap_rbcS-E9_B_0_ins.tsv}
			
			
		\nextgroupplot [
			axis limits = rbcS-E9 A,
			ylabel = {substitution\\[-.1\baselineskip]or deletion},
			height = 10/9 * \plotheight
		]
		
			\heatmapSubDel{rawData/PEV_cr_heatmap_rbcS-E9_A_0_sub+del.tsv}
			\heatmapWT{rawData/PEV_cr_heatmap_rbcS-E9_WT.tsv}
			
			
		\nextgroupplot [
			axis limits = rbcS-E9 B,
			height = 10/9 * \plotheight
		]
		
			\heatmapSubDel{rawData/PEV_cr_heatmap_rbcS-E9_B_0_sub+del.tsv}
			\heatmapWT{rawData/PEV_cr_heatmap_rbcS-E9_wt.tsv}
			
			
		% timepoint 6
		\nextgroupplot [
			axis limits = rbcS-E9 A,
			ylabel = insertion,
			height = 8/9 * \plotheight
		]
		
			\heatmapIns{rawData/PEV_cr_heatmap_rbcS-E9_A_6_ins.tsv}
			
			
		\nextgroupplot [
			axis limits = rbcS-E9 B,
			height = 8/9 * \plotheight
		]
		
			\heatmapIns{rawData/PEV_cr_heatmap_rbcS-E9_B_6_ins.tsv}
			
			
		\nextgroupplot [
			axis limits = rbcS-E9 A,
			ylabel = {substitution\\[-.1\baselineskip]or deletion},
			height = 10/9 * \plotheight
		]
		
			\heatmapSubDel{rawData/PEV_cr_heatmap_rbcS-E9_A_6_sub+del.tsv}
			\heatmapWT{rawData/PEV_cr_heatmap_rbcS-E9_WT.tsv}
			
			
		\nextgroupplot [
			axis limits = rbcS-E9 B,
			height = 10/9 * \plotheight
		]
		
			\heatmapSubDel{rawData/PEV_cr_heatmap_rbcS-E9_B_6_sub+del.tsv}
			\heatmapWT{rawData/PEV_cr_heatmap_rbcS-E9_wt.tsv}
			
			
		% timepoint 12
		\nextgroupplot [
			axis limits = rbcS-E9 A,
			ylabel = insertion,
			height = 8/9 * \plotheight
		]
		
			\heatmapIns{rawData/PEV_cr_heatmap_rbcS-E9_A_12_ins.tsv}
			
			
		\nextgroupplot [
			axis limits = rbcS-E9 B,
			height = 8/9 * \plotheight
		]
		
			\heatmapIns{rawData/PEV_cr_heatmap_rbcS-E9_B_12_ins.tsv}
			
			
		\nextgroupplot [
			axis limits = rbcS-E9 A,
			ylabel = {substitution\\[-.1\baselineskip]or deletion},
			height = 10/9 * \plotheight
		]
		
			\heatmapSubDel{rawData/PEV_cr_heatmap_rbcS-E9_A_12_sub+del.tsv}
			\heatmapWT{rawData/PEV_cr_heatmap_rbcS-E9_WT.tsv}
			
			
		\nextgroupplot [
			axis limits = rbcS-E9 B,
			height = 10/9 * \plotheight
		]
		
			\heatmapSubDel{rawData/PEV_cr_heatmap_rbcS-E9_B_12_sub+del.tsv}
			\heatmapWT{rawData/PEV_cr_heatmap_rbcS-E9_wt.tsv}
			
			
		% timepoint 18
		\nextgroupplot [
			axis limits = rbcS-E9 A,
			ylabel = insertion,
			height = 8/9 * \plotheight
		]
		
			\heatmapIns{rawData/PEV_cr_heatmap_rbcS-E9_A_18_ins.tsv}
			
			
		\nextgroupplot [
			axis limits = rbcS-E9 B,
			height = 8/9 * \plotheight
		]
		
			\heatmapIns{rawData/PEV_cr_heatmap_rbcS-E9_B_18_ins.tsv}
			
			
		\nextgroupplot [
			axis limits = rbcS-E9 A,
			ylabel = {substitution\\[-.1\baselineskip]or deletion},
			height = 10/9 * \plotheight
		]
		
			\heatmapSubDel{rawData/PEV_cr_heatmap_rbcS-E9_A_18_sub+del.tsv}
			\heatmapWT{rawData/PEV_cr_heatmap_rbcS-E9_WT.tsv}
			
			
		\nextgroupplot [
			axis limits = rbcS-E9 B,
			height = 10/9 * \plotheight
		]
		
			\heatmapSubDel{rawData/PEV_cr_heatmap_rbcS-E9_B_18_sub+del.tsv}
			\heatmapWT{rawData/PEV_cr_heatmap_rbcS-E9_wt.tsv}
			
			
		% timepoint 24
		\nextgroupplot [
			axis limits = rbcS-E9 A,
			ylabel = insertion,
			height = 8/9 * \plotheight
		]
		
			\heatmapIns{rawData/PEV_cr_heatmap_rbcS-E9_A_24_ins.tsv}
			
			
		\nextgroupplot [
			axis limits = rbcS-E9 B,
			height = 8/9 * \plotheight
		]
		
			\heatmapIns{rawData/PEV_cr_heatmap_rbcS-E9_B_24_ins.tsv}
			
			
		\nextgroupplot [
			axis limits = rbcS-E9 A,
			ylabel = {substitution\\[-.1\baselineskip]or deletion},
			height = 10/9 * \plotheight
		]
		
			\heatmapSubDel{rawData/PEV_cr_heatmap_rbcS-E9_A_24_sub+del.tsv}
			\heatmapWT{rawData/PEV_cr_heatmap_rbcS-E9_WT.tsv}
			
			
		\nextgroupplot [
			axis limits = rbcS-E9 B,
			height = 10/9 * \plotheight,
			colorbar heatmap
		]
		
			\heatmapSubDel{rawData/PEV_cr_heatmap_rbcS-E9_B_24_sub+del.tsv}
			\heatmapWT{rawData/PEV_cr_heatmap_rbcS-E9_wt.tsv}

	\end{hvgroupplot}
	
	\distance{group c2r2.south}{group c2r1.north}
	
	\foreach \cond [count = \i] in {0, 6, 12, 18, 24}{
		\pgfmathint{2 * \i - 1}
		\node[anchor = south west, rotate = -90, draw = black, fill = t\cond!20, minimum width = \ydistance, shift = {(-.5\pgflinewidth, -.5\pgflinewidth)}, text depth = 0pt, font = \hphantom{A}] at (group c2r\pgfmathresult.north east) {\timepoint{\cond}};
	}

\end{tikzpicture}%
			\caption{%
				\captiontitle[\supports{\cref{fig:PEV_cr}}]{Saturation mutagenesis maps of the \enhancer{rbcS-E9} enhancer do not change much over time in constant light}%
				\subfig{A} The \enhancer{rbcS-E9} enhancer was subjected to the same experiment as the \enhancer{AB80} enhancer in \cref{sfig:PEV_cr_heatmap_AB80}.
			}%
			\label{sfig:PEV_cr_heatmap_rbcS-E9}
		\end{sfig}
		
		\begin{sfig}
			\begin{tikzpicture}

	%%% mutation sensitivity (positional mean) AB80
	\coordinate (AB80) at (0, 0);
	
	\begin{hgroupplot}[%
		group position = {anchor = above north west, at = {(AB80)}, xshift = \plotylabelwidth},
		axis limits from table = {rawData/PEV_cr_mutSens_AB80_axes.tsv},
		enlarge y limits = .05,
		ytick = {-10, ..., 10},
		ylabel = {$\log_2$(enhancer strength)},
		title color = AB80,
		zero line,
		legend style = {anchor = south west, at = {(0, 0)}},
		legend image post style = {very thick},
		legend plot pos = left,
		legend cell align = left,
		legend columns = 3,
		transpose legend,
	]{\textwidth}{2}{position}
	
		\nextgroupplot[
			title = \enhancer{AB80} enhancer \segment{A} segment,
			axis limits = AB80 A,
			shade overlap A = AB80
		]
		
			\pgfplotsinvokeforeach{0, 6, ..., 24}{
				\ifnum#1=18\relax
					\addlegendimage{empty legend}
				\fi
				\addplot[line plot, t#1] table [x = position, y = #1] {rawData/PEV_cr_mutSens_AB80_A_lines.tsv};
			}
			
			\legend{\timepoint{0}, \timepoint{6}, \timepoint{12}, ~, \timepoint{18}, \timepoint{24}}
			
			
		\nextgroupplot[
			title = \enhancer{AB80} enhancer \segment{B} segment,
			axis limits = AB80 B,
			shade overlap B = AB80
		]
		
			\pgfplotsinvokeforeach{0, 6, ..., 24}{
				\addplot[line plot, t#1] table [x = position, y = #1] {rawData/PEV_cr_mutSens_AB80_B_lines.tsv};
			}
		
	\end{hgroupplot}
	
	
	%%% mutation sensitivity (positional mean) Cab-1
	\coordinate[yshift = -\columnsep] (Cab-1) at (AB80 |- xlabel.south);
	
	\begin{hgroupplot}[%
		group position = {anchor = above north west, at = {(Cab-1)}, xshift = \plotylabelwidth},
		axis limits from table = {rawData/PEV_cr_mutSens_Cab-1_axes.tsv},
		enlarge y limits = .05,
		ytick = {-10, ..., 10},
		ylabel = {$\log_2$(enhancer strength)},
		title color = Cab-1,
		zero line,
		legend style = {anchor = south west, at = {(0, 0)}},
		legend image post style = {very thick},
		legend plot pos = left,
		legend cell align = left,
		legend columns = 3,
		transpose legend,
	]{\textwidth}{2}{position}
	
		\nextgroupplot[
			title = \enhancer{Cab-1} enhancer \segment{A} segment,
			axis limits = Cab-1 A,
			shade overlap A = Cab-1
		]
		
			\pgfplotsinvokeforeach{0, 6, ..., 24}{
				\ifnum#1=18\relax
					\addlegendimage{empty legend}
				\fi
				\addplot[line plot, t#1] table [x = position, y = #1] {rawData/PEV_cr_mutSens_Cab-1_A_lines.tsv};
			}
			
			\legend{\timepoint{0}, \timepoint{6}, \timepoint{12}, ~, \timepoint{18}, \timepoint{24}}
			
			
		\nextgroupplot[
			title = \enhancer{Cab-1} enhancer \segment{B} segment,
			axis limits = Cab-1 B,
			shade overlap B = Cab-1
		]
		
			\pgfplotsinvokeforeach{0, 6, ..., 24}{
				\addplot[line plot, t#1] table [x = position, y = #1] {rawData/PEV_cr_mutSens_Cab-1_B_lines.tsv};
			}
		
	\end{hgroupplot}
	
	
	%%% mutation sensitivity (positional mean) rbcS-E9
	\coordinate[yshift = -\columnsep] (rbcS-E9) at (AB80 |- xlabel.south);
	
	\begin{hgroupplot}[%
		group position = {anchor = above north west, at = {(rbcS-E9)}, xshift = \plotylabelwidth},
		axis limits from table = {rawData/PEV_cr_mutSens_rbcS-E9_axes.tsv},
		enlarge y limits = .05,
		ytick = {-10, ..., 10},
		ylabel = {$\log_2$(enhancer strength)},
		title color = rbcS-E9,
		zero line,
		legend style = {anchor = north west, at = {(0, 1)}},
		legend image post style = {very thick},
		legend plot pos = left,
		legend cell align = left,
		legend columns = 3,
		transpose legend,
	]{\textwidth}{2}{position}
	
		\nextgroupplot[
			title = \enhancer{rbcS-E9} enhancer \segment{A} segment,
			axis limits = rbcS-E9 A,
			shade overlap A = rbcS-E9
		]
		
			\pgfplotsinvokeforeach{0, 6, ..., 24}{
				\addplot[line plot, t#1] table [x = position, y = #1] {rawData/PEV_cr_mutSens_rbcS-E9_A_lines.tsv};
			}
			
			\legend{\timepoint{0}, \timepoint{6}, \timepoint{12}, \timepoint{18}, \timepoint{24}}
			
			
		\nextgroupplot[
			title = \enhancer{rbcS-E9} enhancer \segment{B} segment,
			axis limits = rbcS-E9 B,
			shade overlap B = rbcS-E9
		]
		
			\pgfplotsinvokeforeach{0, 6, ..., 24}{
				\addplot[line plot, t#1] table [x = position, y = #1] {rawData/PEV_cr_mutSens_rbcS-E9_B_lines.tsv};
			}
		
	\end{hgroupplot}
	
	
	%%% subfigure labels
	\subfiglabel{AB80}
	\subfiglabel{Cab-1}
	\subfiglabel{rbcS-E9}

\end{tikzpicture}%
			\caption{%
				\captiontitle[\supports{\cref{fig:PEV_cr}}]{Mutation-sensitive regions of the \enhancer{AB80}, \enhancer{Cab-1}, and \enhancer{rbcS-E9} enhancers are conserved over time in constant light}%
				\subfigrange{A}{C} All possible single-nucleotide substitution, deletion, and insertion variants of the \segment{A} and \segment{B} segments of the \enhancer{AB80} \parensubfig{A}, \enhancer{Cab-1} \parensubfig{B}, and \enhancer{rbcS-E9} \parensubfig{C} enhancers were subjected to Plant STARR-seq in tobacco leaves. On the morning of the third day after transformation (\timepoint{-8}), the plants were shifted to constant light. Leaves were harvested for RNA extraction starting at mid-day (\timepoint{0}) and in 6 hour intervals (\timepoint{6}, 20, 26, and 32) afterwards. Enhancer strength was normalized to the wild-type variant ($\log_2$ set to 0). A sliding average (window size = 6 bp) of the mean enhancer strength for all variants at a given position is shown. The shaded area indicates the region where the \segment{A} and \segment{B} segments overlap.
			}%
			\label{sfig:PEV_cr_MutSens}%
		\end{sfig}
		
		\begin{sfig}
			\begin{tikzpicture}
	\addtolength{\plotylabelwidth}{\baselineskip}

	%%% correlation between PEVdouble and PEV libraries
	\coordinate (PEVdouble vs PEV) at (0, 0);
	
	\begin{hgroupplot}[%
		group position = {anchor = above north west, at = {(PEVdouble vs PEV)}, xshift = \plotylabelwidth},
		axis limits from table = {rawData/PEVdouble_cor_PEV_axes.tsv},
		enlargelimits = .05,
		xytick = {-10, -8, ..., 10},
		legend style = {anchor = south east, at = {(1, 0)}},
		legend image post style = {fill opacity = 1, mark size = 1.25},
		legend plot pos = right,
		legend cell align = right,
		scatter/classes = {
			AB80={AB80},
			Cab-1={Cab-1},
			rbcS-E9={rbcS-E9}
		},
		show diagonal,
		ylabel = {\textbf{double-deletion library}:\\$\log_2$(enhancer strength)}
	]{\twocolumnwidth}{2}{\textbf{single-nucleotide variant library}: $\log_2$(enhancer strength)}
	
		\nextgroupplot[
			title = \light,
			title color = light,
		]
		
			% scatter plot
			\addplot [
				scatter,
				scatter src = explicit symbolic,
				only marks,
				mark = solido,
				fill opacity = .5
			] table[x = enrichment_PEV_ld, y = enrichment_PEVdouble, meta = enhancer] {rawData/PEVdouble_cor_PEV_light_points.tsv};
			
			% correlation
			\stats{rawData/PEVdouble_cor_PEV_light};
			
			
		\nextgroupplot[
			title = \dark,
			title color = dark,
		]
		
			% scatter plot
			\addplot [
				scatter,
				scatter src = explicit symbolic,
				only marks,
				mark = solido,
				fill opacity = .5
			] table[x = enrichment_PEV_ld, y = enrichment_PEVdouble, meta = enhancer] {rawData/PEVdouble_cor_PEV_dark_points.tsv};
			
			% correlation
			\stats{rawData/PEVdouble_cor_PEV_dark};
			
			% legend
			\legend{\enhancer{AB80}, \enhancer{Cab-1}, \enhancer{rbcS-E9}}
			
	\end{hgroupplot}
	
	
	%%% correlation between PEFval and PEF libraries
	\coordinate (PEFval vs PEF) at (PEVdouble vs PEV -| \textwidth - \twocolumnwidth, 0);
	
	\begin{hgroupplot}[%
		group position = {anchor = above north west, at = {(PEFval vs PEF)}, xshift = \plotylabelwidth},
		axis limits from table = {rawData/PEFval_cor_PEF_axes.tsv},
		enlargelimits = .05,
		xytick = {-10, -8, ..., 10},
		legend style = {anchor = south east, at = {(1, 0)}},
		legend image post style = {fill opacity = 1, mark size = 1.25},
		legend plot pos = right,
		legend cell align = right,
		scatter/classes = {
			1={t0},
			2={t12},
			3={t18}
		},
		show diagonal,
		ylabel = {\textbf{validation library}:\\$\log_2$(enhancer strength)}
	]{\twocolumnwidth}{2}{\textbf{large-scale library}: $\log_2$(enhancer strength)}
	
		\nextgroupplot[
			title = \light,
			title color = light,
		]
		
			% scatter plot
			\addplot [
				scatter,
				scatter src = explicit symbolic,
				only marks,
				mark = solido,
				fill opacity = .5
			] table[x = enrichment_PEF, y = enrichment_PEFval, meta = n_frags] {rawData/PEFval_cor_PEF_light_points.tsv};
			
			% correlation
			\stats{rawData/PEFval_cor_PEF_light};
			
			
		\nextgroupplot[
			title = \dark,
			title color = dark,
		]
		
			% scatter plot
			\addplot [
				scatter,
				scatter src = explicit symbolic,
				only marks,
				mark = solido,
				fill opacity = .5
			] table[x = enrichment_PEF, y = enrichment_PEFval, meta = n_frags] {rawData/PEFval_cor_PEF_dark_points.tsv};
			
			% correlation
			\stats{rawData/PEFval_cor_PEF_dark};
			
			% legend
			\legend{1 fragment, 2 fragments, 3 fragments}
			
	\end{hgroupplot}


	%%% subfigure labels
	\subfiglabel{PEVdouble vs PEV}
	\subfiglabel{PEFval vs PEF}

\end{tikzpicture}%
			\caption{%
				\captiontitle[\supports{\cref{fig:PEV_ld_MutSens,fig:PEV_ld_motifs,fig:PEF_cooperativity,fig:PEF_model,fig:PEVdouble}}]{Plant STARR-seq experiments are reproducible across libraries}%
				\subfig{A} Correlation between the enhancer strength of single-nucleotide deletion variants of the \enhancer{AB80}, \enhancer{Cab-1}, and \enhancer{rbcS-E9} enhancers present in the comprehensive single-nucleotide enhancer variants library (described in \cref{fig:PEV_ld_MutSens}) and in a second, independent library with single- and double-deletion enhancer variants (described in \cref{fig:PEVdouble}).\nextentry
				\subfig{B} Correlation between the strength of synthetic enhancers created by combining fragments of the \enhancer{AB80}, \enhancer{Cab-1}, and \enhancer{rbcS-E9} enhancers as measured in the large-scale library (described in \cref{fig:PEF_cooperativity}) and in a second, smaller validation library. Pearson's $R^2$, Spearman's $\rho$, and number ($n$) of enhancer variants are indicated.
			}%
			\label{sfig:PEval_cor_main}%
		\end{sfig}

		\begin{sfig}
			\begin{tikzpicture}

	\coordinate (frags) at (0, 0);
	
	\node[anchor = north west, text depth = 0pt, font = \hphantom{A}] (ptitle) at (frags) {\vphantom{\light}};
	
	\distance{ptitle.south}{ptitle.north}
	
	\begin{hvgroupplot}[%
		group position = {anchor = above north west, at = {(frags)}, xshift = \plotylabelwidth},
		axis limits from table = {rawData/PEF_fragments_axes.tsv},
		enlarge y limits = .05,
		zero line,
		xticklabel style = {font = \vphantom{abcde}}
	]{\textwidth - \ydistance}{7cm + \plotxlabelheight + \groupplotsep}{5}{2}{fragments}{$\log_2$(enhancer strength)}
		
		\nextgroupplot[
			title = \enhancer{AB80} fragment \textbf{bc},
			title color = AB80,
			x tick table = {rawData/PEF_fragments_AB80_bc_light_mean.tsv}{construct},
		]
		
			\hmandp{AB80, mark options = {AB80!50}}{rawData/PEF_fragments_AB80_bc_light};
			
			
		\nextgroupplot[
			title = \enhancer{Cab-1} fragment \textbf{bc},
			title color = Cab-1,
			x tick table = {rawData/PEF_fragments_Cab-1_bc_light_mean.tsv}{construct},
		]
		
			\hmandp{Cab-1, mark options = {Cab-1!50}}{rawData/PEF_fragments_Cab-1_bc_light};
			
			
		\nextgroupplot[
			title = \enhancer{Cab-1} fragment \textbf{de},
			title color = Cab-1,
			x tick table = {rawData/PEF_fragments_Cab-1_de_light_mean.tsv}{construct},
		]
		
			\hmandp{Cab-1, mark options = {Cab-1!50}}{rawData/PEF_fragments_Cab-1_de_light};
			
			
		\nextgroupplot[
			title = \enhancer{rbcS-E9} fragment \textbf{ab},
			title color = rbcS-E9,
			x tick table = {rawData/PEF_fragments_rbcS-E9_ab_light_mean.tsv}{construct},
		]
		
			\hmandp{rbcS-E9, mark options = {rbcS-E9!50}}{rawData/PEF_fragments_rbcS-E9_ab_light};
			
			
		\nextgroupplot[
			title = \enhancer{rbcS-E9} fragment \textbf{de},
			title color = rbcS-E9,
			x tick table = {rawData/PEF_fragments_rbcS-E9_de_light_mean.tsv}{construct},
		]
		
			\hmandp{rbcS-E9, mark options = {rbcS-E9!50}}{rawData/PEF_fragments_rbcS-E9_de_light};
			
		
		\nextgroupplot[
			x tick table = {rawData/PEF_fragments_AB80_bc_dark_mean.tsv}{construct},
		]
		
			\hmandp{AB80, mark options = {AB80!50}}{rawData/PEF_fragments_AB80_bc_dark};
			
			
		\nextgroupplot[
			x tick table = {rawData/PEF_fragments_Cab-1_bc_dark_mean.tsv}{construct},
		]
		
			\hmandp{Cab-1, mark options = {Cab-1!50}}{rawData/PEF_fragments_Cab-1_bc_dark};
			
			
		\nextgroupplot[
			x tick table = {rawData/PEF_fragments_Cab-1_de_dark_mean.tsv}{construct},
		]
		
			\hmandp{Cab-1, mark options = {Cab-1!50}}{rawData/PEF_fragments_Cab-1_de_dark};
			
			
		\nextgroupplot[
			x tick table = {rawData/PEF_fragments_rbcS-E9_ab_dark_mean.tsv}{construct},
		]
		
			\hmandp{rbcS-E9, mark options = {rbcS-E9!50}}{rawData/PEF_fragments_rbcS-E9_ab_dark};
			
			
		\nextgroupplot[
			x tick table = {rawData/PEF_fragments_rbcS-E9_de_dark_mean.tsv}{construct},
		]
		
			\hmandp{rbcS-E9, mark options = {rbcS-E9!50}}{rawData/PEF_fragments_rbcS-E9_de_dark};
		
	\end{hvgroupplot}
	
	\distance{group c5r1.south}{group c5r1.north}
	
	\node[anchor = south west, rotate = -90, draw = black, fill = light!20, minimum width = \ydistance, shift = {(-.5\pgflinewidth, -.5\pgflinewidth)}, text depth = 0pt, font = \hphantom{A}] (tlight) at (group c5r1.north east) {\light};
	\node[anchor = south west, rotate = -90, draw = black, fill = dark!20, minimum width = \ydistance, shift = {(-.5\pgflinewidth, -.5\pgflinewidth)}, text depth = 0pt, font = \hphantom{A}] (tdark) at (group c5r2.north east) {\dark};
	
	
	%%% subfigure labels
	\subfiglabel{frags}

\end{tikzpicture}%
			\caption{%
				\captiontitle[\supports{\cref{fig:PEF_cooperativity}}]{Correct spacing between mutation-sensitive regions is required for full activity}%
				\subfig{A} Plots of the strength in the indicated condition of enhancer fragments (\cref{fig:PEF_cooperativity}\subfigunformatted{D}) or fragment combinations (separated by a + sign and shown in the order in which they appear in the construct; \cref{fig:PEF_cooperativity}\subfigunformatted{E}) in three replicates (points) and the mean strength (lines). Enhancer strength was normalized to a control construct without an enhancer ($\log_2$ set to 0). Some plots are reproduced from \cref{fig:PEF_cooperativity}, \subfigunformatted{F}\subfigrefand\subfigunformatted{G}.
			}
			\label{sfig:PEF_fragments}
		\end{sfig}
		
		\begin{sfig}
			\begin{tikzpicture}

	%%% predict light-responsiveness from single fragments
	\coordinate (predict resp) at (0, 0);
	
	\begin{axis}[%
		anchor = above north west,
		at = {(0, 0)},
		xshift = \plotylabelwidth + \baselineskip,
		width = \fourcolumnwidth,
		axis limits from table = {rawData/PEF_cor_prediction_lightResp_axes.tsv},
		enlargelimits = .05,
		xytick = {-10, -8, ..., 10},
		show diagonal,
		colormap name = viridis,
		xlabel = {\textbf{prediction}:\\$\log_2$(\lightResp)},
		ylabel = {\textbf{measurement}:\\$\log_2$(\lightResp)},
		xlabel style = {inner xsep = 0pt, xshift = -3pt},
		title = {\lightResp},
		title style = {left color = light!20, right color = dark!20, minimum width = \fourcolumnwidth},
		colorbar style = {
	 		name = colorbar,
			anchor = north west,
			at = {(.025, .975)},
			width = .25cm,
			height = .4 * \pgfkeysvalueof{/pgfplots/parent axis height},
			ytick pos = right,
			yticklabel pos = right,
			yticklabel style = {node font = \figtiny, inner xsep = .1em},
			ytick = {0, 2, ..., 15},
			yticklabel = \pgfmathparse{2^\tick}\pgfmathprintnumber{\pgfmathresult},
			ylabel = count,
			ylabel style = {node font = \figsmall, inner ysep = 0pt}
		},
		name = blub,
		colorbar
	]

		% hexbin plot
		\addplot [hexbin] table [x = x, y = y, meta = count] {rawData/PEF_cor_prediction_lightResp_hexbin.tsv};
		
		% correlation
		\stats[stats position = south east]{rawData/PEF_cor_prediction_lightResp};
		
	\end{axis}
	
	
	%%% subfigure labels
	\subfiglabel{predict resp}
	
\end{tikzpicture}%
			\caption{%
				\captiontitle[\supports{\cref{fig:PEF_model}}]{A linear model can predict the \lightResp{} of synthetic enhancers}%
				\subfig{A} A linear model was built to predict the \lightResp{} of synthetic enhancers created by randomly combining up to three fragments derived from mutation-sensitive regions of the \enhancer{AB80}, \enhancer{Cab-1}, and \enhancer{rbcS-E9} enhancers (see \cref{fig:PEF_cooperativity}\subfigunformatted{A}) based on the \lightResp{} of the constituent individual fragments. A hexbin plot (color represents the count of points in each hexagon) of the correlation between the model's prediction and the measured data is shown. Pearson's $R^2$, Spearman's $\rho$, and number ($n$) of enhancer fragment combinations are indicated.
			}%
			\label{sfig:PEF_model_lr}
		\end{sfig}

	\fi
	%%% Supplementary figures end
	
	%%% Main figures start
	\ifrev
		
		\pagestyle{empty}
		
		\begin{rfig}
			\begin{tikzpicture}
	
	%%% effect of mutations in TFBSs
	\coordinate (effects) at (0, 0);
	
	\begin{hgroupplot}[%
		group position = {anchor = above north west, at = {(effects)}, xshift = \plotylabelwidth + .25\baselineskip},
		ymin = 0,
		ymax = 100,
		ytick = {0, 20, ..., 100},
		enlarge y limits = {upper, value = 0.3},
%		legend style = {anchor = south east, at = {(1, 0)}},
%		legend image post style = {fill opacity = 1, mark size = 1.25},
%		legend plot pos = right,
%		legend cell align = right,
		ylabel = {percent of mutations},
		xticklabel style = {rotate = 90, anchor = east, xshift = -.3333em},
		group/every plot/.append style = {x grids = false},
	]{\textwidth - .75\fourcolumnwidth - \columnsep - .25\baselineskip}{2}{}
	
		\nextgroupplot[
			title = mutagenesis,
			width = 22.4 / 39.4 * \plotwidth,
			title style = {minimum width = 22.4 / 39.4 * \plotwidth},
			x tick table = {rawData/PEV_effect_directions_mutagenesis_bars.tsv}{TF}
		]
		
			\addplot[
				ybar stacked,
				bar width = .8,
				draw = black,
				fill = RoyalBlue1,
				fill opacity = .5
			] table [x = id, y = p_decreased] {rawData/PEV_effect_directions_mutagenesis_bars.tsv};
			
			\addplot[
				ybar stacked,
				bar width = .8,
				draw = black,
				fill = IndianRed1,
				fill opacity = .5
			] table [x = id, y = p_increased] {rawData/PEV_effect_directions_mutagenesis_bars.tsv};
			
			\addplot[
				draw = none,
				visualization depends on = value \thisrow{n_decreased} \as \decreased,
				visualization depends on = value \thisrow{n_increased} \as \increased,
				nodes near coords = {\textcolor{RoyalBlue1}{\decreased}/\textcolor{IndianRed1}{\increased}},
				nodes near coords style = {anchor = west, rotate = 90, node font = \figsmall}
			] table [x = id, y expr = 100] {rawData/PEV_effect_directions_mutagenesis_bars.tsv};
			
			
		\nextgroupplot[
			title = scanning,
			width = 56.4 / 39.4 * \plotwidth,
			title style = {minimum width = 56.4 / 39.4 * \plotwidth},
			x tick table = {rawData/PEV_effect_directions_scanning_bars.tsv}{TF}
		]
		
			\addplot[
				ybar stacked,
				bar width = .8,
				draw = black,
				fill = RoyalBlue1,
				fill opacity = .5
			] table [x = id, y = p_decreased] {rawData/PEV_effect_directions_scanning_bars.tsv};
			
			\addplot[
				ybar stacked,
				bar width = .8,
				draw = black,
				fill = IndianRed1,
				fill opacity = .5
			] table [x = id, y = p_increased] {rawData/PEV_effect_directions_scanning_bars.tsv};
			
			\addplot[
				draw = none,
				visualization depends on = value \thisrow{n_decreased} \as \decreased,
				visualization depends on = value \thisrow{n_increased} \as \increased,
				nodes near coords = {\textcolor{RoyalBlue1}{\decreased}/\textcolor{IndianRed1}{\increased}},
				nodes near coords style = {anchor = west, rotate = 90, node font = \figsmall}
			] table [x = id, y expr = 100] {rawData/PEV_effect_directions_scanning_bars.tsv};
			
			\node[rotate = 90, font = \figsmaller\bfseries, red] at (1, 64.865) {increasing};
			\node[rotate = 90, font = \figsmaller\bfseries, blue] at (28, 50) {decreasing};
			
	\end{hgroupplot}
	
	
	%%% summary
	\coordinate (summary) at (effects -| \textwidth - .75\fourcolumnwidth, 0);
	
	\distance{group c1r1.south}{title.north}
	
	\begin{axis}[%
		anchor = north west,
		at = {(summary)},
		width = .75\fourcolumnwidth - \plotylabelwidth - \baselineskip,
		height = \ydistance,
		xshift = \plotylabelwidth + \baselineskip,
		x grids = false,
		axis limits from table = {rawData/PEV_effect_directions_summary_axes.tsv},
		ylabel = percent decreasing mutations,
		x tick table = {rawData/PEV_effect_directions_summary_boxplot.tsv}{sample},
		xticklabel style = {rotate = 90, anchor = east, xshift = -.3333em}
	]
	
		% lines for random samples
		\pgfplotstableforeachcolumnelement{percent}\of{rawData/PEV_effect_directions_summary_lines.tsv}\as\percent{
			\pgfplotstablegetelem{\pgfplotstablerow}{linetype}\of{rawData/PEV_effect_directions_summary_lines.tsv}
			\edef\thisline{\noexpand\draw[\pgfplotsretval, red, thick] (\noexpand\xmin, \percent) -- (\noexpand\xmax, \percent);}
			\thisline
		}

		% boxplots
		\boxplots{%
			fill = gray,
			fill opacity = .5
		}{rawData/PEV_effect_directions_summary}
		
		% add sample size
		\samplesize{rawData/PEV_effect_directions_summary_boxplot.tsv}{id}{n}

	\end{axis}
	
	
	%%% subfigure labels
	\subfiglabel{effects}
	\subfiglabel{summary}
	
\end{tikzpicture}%
			\caption{%
				\captiontitle{Effect directions for mutations in putative transcription factor binding sites}%
				\subfigtwo{A}{B} Mutations within each putative transcription factor binding site identified by the motif-scanning approach (scanning; see \cref{sfig:PEV_ld_fimo}) or by matching known transcription factor binding motifs to the sequence logo plots derived from the saturation mutagenesis data in the light (mutagenesis; see \cref{fig:PEV_ld_motifs}\subfigunformatted{B}\subfigrefrange\subfigunformatted{D}), were categorized into enhancer strength increasing (red) or decreasing (blue). The number and percentage of increasing and decreasing mutations was plotted for every binding site in the \segment{B} segments of the \enhancer{AB80}, \enhancer{Cab-1}, and \enhancer{rbcS-E9} enhancers \parensubfig{A} and summarized by the identification approach \parensubfig{B}. The solid (mean across 100 random samples) and dashed (mean +/- standard deviation) red lines in \plainsubfigref{B} indicate the median percentage of enhancer strength decreasing mutations in randomly sampled regions of the same length as the transcription factor binding sites.
			}%
			\label{rfig:effect_directions}
		\end{rfig}
		
		\begin{rfig}
			\begin{tikzpicture}

	%%% PEF: single fragments vs. DMS AUCs
	\coordinate (AUC) at (0, 0);

	\begin{hgroupplot}[%
		group position = {anchor = above north west, at = {(AUC)}, xshift = \plotylabelwidth},
		enlargelimits = .15,
		ytick = {-10, -8, ..., 10},
		zero line,
		legend style = {anchor = north east, at = {(1, .525)}},
		legend plot pos = right,
		legend cell align = right,
		scatter/classes = {
			AB80={AB80},
			Cab-1={Cab-1},
			rbcS-E9={rbcS-E9}
		},
		ylabel = {$\log_2$(enhancer strength)}
	]{\twocolumnwidth}{2}{$-$AUC}
	
		\nextgroupplot[
			title = \light,
			title color = light,
			axis limits from table = {rawData/PEF_cor_AUC_light_axes.tsv},
		]
		
			% regression lines
			\pgfplotstableread{rawData/PEF_cor_AUC_light_stats_per_enhancer.tsv}{\statstable}
			\coordinate (c1) at (rel axis cs: 0, 1);
			
			\pgfplotsinvokeforeach{AB80, Cab-1, rbcS-E9}{
				% regression line
				\pgfplotstablegetelem{0}{intercept_#1}\of\statstable
				\edef\intercept{\pgfplotsretval}
				\pgfplotstablegetelem{0}{slope_#1}\of\statstable
				\edef\slope{\pgfplotsretval}
				\edef\thisline{\noexpand\draw[#1, thick] (\noexpand\xmin, \noexpand\xmin * \slope + \intercept) -- (\noexpand\xmax, \noexpand\xmax * \slope + \intercept);}
				\thisline
				
				% correlation
				\pgfplotstablegetelem{0}{rsquare_#1}\of\statstable
				\edef\thisstat{
					\noexpand\node[anchor = north west, text = #1, inner ysep = .2em, font = \noexpand\bfseries] (stats) at (c1) {\noexpand\textit{R}\noexpand\textsuperscript{2} = \noexpand\pgfmathprintnumber[fixed, fixed zerofill, precision = 2, assume math mode = true]{\pgfplotsretval}};
					\noexpand\coordinate[yshift = -.9\noexpand\baselineskip] (c1) at (c1);
				}
				\thisstat
			}
			
			
			% scatter plot
			\addplot [
				scatter,
				scatter src = explicit symbolic,
				only marks,
				mark = solido,
				mark size = 1.25
			] table[x = AUC, y = enrichment, meta = enhancer] {rawData/PEF_cor_AUC_light_points.tsv};
			
			
		\nextgroupplot[
			title = \dark,
			title color = dark,
			axis limits from table = {rawData/PEF_cor_AUC_dark_axes.tsv},
		]
		
			% regression lines
			\pgfplotstableread{rawData/PEF_cor_AUC_dark_stats_per_enhancer.tsv}{\statstable}
			\coordinate (c1) at (rel axis cs: 0, 1);
			
			\pgfplotsinvokeforeach{AB80, Cab-1, rbcS-E9}{
				% regression line
				\pgfplotstablegetelem{0}{intercept_#1}\of\statstable
				\edef\intercept{\pgfplotsretval}
				\pgfplotstablegetelem{0}{slope_#1}\of\statstable
				\edef\slope{\pgfplotsretval}
				\edef\thisline{\noexpand\draw[#1, thick] (\noexpand\xmin, \noexpand\xmin * \slope + \intercept) -- (\noexpand\xmax, \noexpand\xmax * \slope + \intercept);}
				\thisline
				
				% correlation
				\pgfplotstablegetelem{0}{rsquare_#1}\of\statstable
				\edef\thisstat{
					\noexpand\node[anchor = north west, text = #1, inner ysep = .2em, font = \noexpand\bfseries] (stats) at (c1) {\noexpand\textit{R}\noexpand\textsuperscript{2} = \noexpand\pgfmathprintnumber[fixed, fixed zerofill, precision = 2, assume math mode = true]{\pgfplotsretval}};
					\noexpand\coordinate[yshift = -.9\noexpand\baselineskip] (c1) at (c1);
				}
				\thisstat
			}
			
			% scatter plot
			\addplot [
				scatter,
				scatter src = explicit symbolic,
				only marks,
				mark = solido,
				mark size = 1.25
			] table[x = AUC, y = enrichment, meta = enhancer] {rawData/PEF_cor_AUC_dark_points.tsv};
			
			% legend
			\legend{\enhancer{AB80}, \enhancer{Cab-1}, \enhancer{rbcS-E9}}
			
	\end{hgroupplot}
	
	
	%%% subfigure labels
	\subfiglabel{AUC}
	
\end{tikzpicture}%
			\caption{%
				\captiontitle{Correlation between enhancer fragment strength and results from the saturation mutagenesis on a per-enhancer basis}%
				\subfig{A} For each enhancer fragment, the area under the curve (AUC) in the mutational sensitivity plots was calculated and plotted against the fragment's enhancer strength. Regression lines for each enhancer are shown and the goodness-of-fit ($R^2$) is indicated (\enhancer{AB80}, red; \enhancer{Cab-1}, green; \enhancer{rbcS-E9}, purple).
			}%
			\label{rfig:AUC_per_enhancer}
		\end{rfig}
		
		\begin{rfig}
			\begin{tikzpicture}
	
	%%% effect of two deletions in light
	\coordinate (prediction light) at (0, 0);
	
	\node[anchor = north west, text depth = 0pt, font = \hphantom{A}] (ptitle) at (prediction light) {\vphantom{\light}};
	
	\distance{ptitle.south}{ptitle.north}
	
	\pgfmathsetlength{\templength}{(\textwidth + \pgflinewidth - 2\plotylabelwidth - 2\groupplotsep - 2\ydistance - \columnsep) / 4}
	
	\begin{hvgroupplot}[%
		group position = {anchor = above north west, at = {(prediction light)}, xshift = \plotylabelwidth},
		enlargelimits = .05,
		zero line,
		show diagonal,
		colormap name = viridis,
		colorbar distance,
	]{3\templength + \plotylabelwidth + 2\groupplotsep}{10.5cm + \plotxlabelheight + 2\groupplotsep}{3}{3}{\textbf{additive model}: $\log_2$(enhancer strength)}{\textbf{measurement}: $\log_2$(enhancer strength)}
		
		\nextgroupplot[
			title = \enhancer{AB80} enhancer\\\segment{B} segment,
			title color = AB80,
			axis limits from table = {rawData/PEVdouble_prediction_AB80_light_axes.tsv}
		]
		
			% scatter plot
			\addplot [
				scatter,
				only marks,
				mark = solido,
				fill opacity = .5,
				point meta = explicit,
				x filter/.expression = {\thisrow{in_TF} == 0 ? x : NaN}
			] table[x = prediction, y = enrichment, meta = distance] {rawData/PEVdouble_prediction_AB80_light_points.tsv};
			
			% correlation
			\stats[stats position = south east]{rawData/PEVdouble_prediction_AB80_light_0}
			
			
		\nextgroupplot[
			title = \enhancer{Cab-1} enhancer\\\segment{B} segment,
			title color = Cab-1,
			axis limits from table = {rawData/PEVdouble_prediction_Cab-1_light_axes.tsv}
		]
		
			% scatter plot
			\addplot [
				scatter,
				only marks,
				mark = solido,
				fill opacity = .5,
				point meta = explicit,
				x filter/.expression = {\thisrow{in_TF} == 0 ? x : NaN}
			] table[x = prediction, y = enrichment, meta = distance] {rawData/PEVdouble_prediction_Cab-1_light_points.tsv};
			
			% correlation
			\stats[stats position = south east]{rawData/PEVdouble_prediction_Cab-1_light_0}
			
			
		\nextgroupplot[
			title = \enhancer{rbcS-E9} enhancer\\\segment{B} segment,
			title color = rbcS-E9,
			axis limits from table = {rawData/PEVdouble_prediction_rbcS-E9_light_axes.tsv}
		]
		
			% scatter plot
			\addplot [
				scatter,
				only marks,
				mark = solido,
				fill opacity = .5,
				point meta = explicit,
				x filter/.expression = {\thisrow{in_TF} == 0 ? x : NaN}
			] table[x = prediction, y = enrichment, meta = distance] {rawData/PEVdouble_prediction_rbcS-E9_light_points.tsv};
			
			% correlation
			\stats[stats position = south east]{rawData/PEVdouble_prediction_rbcS-E9_light_0}
			
			
		\nextgroupplot[
			axis limits from table = {rawData/PEVdouble_prediction_AB80_light_axes.tsv}
		]
		
			% scatter plot
			\addplot [
				scatter,
				only marks,
				mark = solido,
				fill opacity = .5,
				point meta = explicit,
				x filter/.expression = {\thisrow{in_TF} == 1 ? x : NaN}
			] table[x = prediction, y = enrichment, meta = distance] {rawData/PEVdouble_prediction_AB80_light_points.tsv};
			
			% correlation
			\stats[stats position = south east]{rawData/PEVdouble_prediction_AB80_light_1}
			
			
		\nextgroupplot[
			axis limits from table = {rawData/PEVdouble_prediction_Cab-1_light_axes.tsv}
		]
		
			% scatter plot
			\addplot [
				scatter,
				only marks,
				mark = solido,
				fill opacity = .5,
				point meta = explicit,
				x filter/.expression = {\thisrow{in_TF} == 1 ? x : NaN}
			] table[x = prediction, y = enrichment, meta = distance] {rawData/PEVdouble_prediction_Cab-1_light_points.tsv};
			
			% correlation
			\stats[stats position = south east]{rawData/PEVdouble_prediction_Cab-1_light_1}
			
			
		\nextgroupplot[
			axis limits from table = {rawData/PEVdouble_prediction_rbcS-E9_light_axes.tsv}
		]
		
			% scatter plot
			\addplot [
				scatter,
				only marks,
				mark = solido,
				fill opacity = .5,
				point meta = explicit,
				x filter/.expression = {\thisrow{in_TF} == 1 ? x : NaN}
			] table[x = prediction, y = enrichment, meta = distance] {rawData/PEVdouble_prediction_rbcS-E9_light_points.tsv};
			
			% correlation
			\stats[stats position = south east]{rawData/PEVdouble_prediction_rbcS-E9_light_1}
			
			
		\nextgroupplot[
			axis limits from table = {rawData/PEVdouble_prediction_AB80_light_axes.tsv}
		]
		
			% scatter plot
			\addplot [
				scatter,
				only marks,
				mark = solido,
				fill opacity = .5,
				point meta = explicit,
				x filter/.expression = {\thisrow{in_TF} == 2 ? x : NaN}
			] table[x = prediction, y = enrichment, meta = distance] {rawData/PEVdouble_prediction_AB80_light_points.tsv};
			
			% correlation
			\stats[stats position = south east]{rawData/PEVdouble_prediction_AB80_light_2}
			
			
		\nextgroupplot[
			axis limits from table = {rawData/PEVdouble_prediction_Cab-1_light_axes.tsv}
		]
		
			% scatter plot
			\addplot [
				scatter,
				only marks,
				mark = solido,
				fill opacity = .5,
				point meta = explicit,
				x filter/.expression = {\thisrow{in_TF} == 2 ? x : NaN}
			] table[x = prediction, y = enrichment, meta = distance] {rawData/PEVdouble_prediction_Cab-1_light_points.tsv};
			
			% correlation
			\stats[stats position = south east]{rawData/PEVdouble_prediction_Cab-1_light_2}
			
			
		\nextgroupplot[
			axis limits from table = {rawData/PEVdouble_prediction_rbcS-E9_light_axes.tsv}
		]
		
			% scatter plot
			\addplot [
				scatter,
				only marks,
				mark = solido,
				fill opacity = .5,
				point meta = explicit,
				x filter/.expression = {\thisrow{in_TF} == 2 ? x : NaN}
			] table[x = prediction, y = enrichment, meta = distance] {rawData/PEVdouble_prediction_rbcS-E9_light_points.tsv};
			
			% correlation
			\stats[stats position = south east]{rawData/PEVdouble_prediction_rbcS-E9_light_2}
		
	\end{hvgroupplot}
	
	\distance{group c3r1.south}{group c3r1.north}
	
	\node[anchor = south west, rotate = -90, draw = black, fill = light!20, minimum width = \ydistance, shift = {(-.5\pgflinewidth, -.5\pgflinewidth)}, text depth = 0pt, font = \hphantom{A}] (tlight) at (group c3r1.north east) {0 deletions in TFBS};
	\node[anchor = south west, rotate = -90, draw = black, fill = light!20, minimum width = \ydistance, shift = {(-.5\pgflinewidth, -.5\pgflinewidth)}, text depth = 0pt, font = \hphantom{A}] (tlight) at (group c3r2.north east) {1 deletion in TFBS};
	\node[anchor = south west, rotate = -90, draw = black, fill = light!20, minimum width = \ydistance, shift = {(-.5\pgflinewidth, -.5\pgflinewidth)}, text depth = 0pt, font = \hphantom{A}] (tlight) at (group c3r3.north east) {2 deletions in TFBS};
	
	
	%%% effect of two deletions in dark
	\coordinate[xshift = \columnsep] (prediction dark) at (group c3r1.above north -| tlight.north);
	
	\begin{vgroupplot}[%
		group position = {anchor = above north west, at = {(prediction dark)}, xshift = \plotylabelwidth},
		width = \templength,
		axis limits from table = {rawData/PEVdouble_prediction_rbcS-E9_dark_axes.tsv},
		enlargelimits = .05,
		zero line,
		show diagonal,
		xlabel = \textbf{additive model}:\\$\log_2$(enhancer strength),
		colormap name = viridis,
		colorbar distance,
		point meta min = 8
	]{10.5cm + \plotxlabelheight + 2\groupplotsep}{3}{\textbf{measurement}: $\log_2$(enhancer strength)}
		
			\nextgroupplot[
				title = \enhancer{rbcS-E9} enhancer\\\segment{B} segment,
				title color = rbcS-E9,
				title style = {minimum width = \templength}
			]
		
				% scatter plot
				\addplot [
					scatter,
					only marks,
					mark = solido,
					fill opacity = .5,
					point meta = explicit,
					x filter/.expression = {\thisrow{in_TF} == 0 ? x : NaN}
				] table[x = prediction, y = enrichment, meta = distance] {rawData/PEVdouble_prediction_rbcS-E9_dark_points.tsv};
				
				% correlation
				\stats[stats position = south east]{rawData/PEVdouble_prediction_rbcS-E9_dark_0}
				
				
			\nextgroupplot
		
				% scatter plot
				\addplot [
					scatter,
					only marks,
					mark = solido,
					fill opacity = .5,
					point meta = explicit,
					x filter/.expression = {\thisrow{in_TF} == 1 ? x : NaN}
				] table[x = prediction, y = enrichment, meta = distance] {rawData/PEVdouble_prediction_rbcS-E9_dark_points.tsv};
				
				% correlation
				\stats[stats position = south east]{rawData/PEVdouble_prediction_rbcS-E9_dark_1}
				
				
			\nextgroupplot
		
				% scatter plot
				\addplot [
					scatter,
					only marks,
					mark = solido,
					fill opacity = .5,
					point meta = explicit,
					x filter/.expression = {\thisrow{in_TF} == 2 ? x : NaN}
				] table[x = prediction, y = enrichment, meta = distance] {rawData/PEVdouble_prediction_rbcS-E9_dark_points.tsv};
				
				% correlation
				\stats[stats position = south east]{rawData/PEVdouble_prediction_rbcS-E9_dark_2}
		
	\end{vgroupplot}
	
	\distance{group c1r1.south}{group c1r1.north}
	
	\node[anchor = south west, rotate = -90, draw = black, fill = dark!20, minimum width = \ydistance, shift = {(-.5\pgflinewidth, -.5\pgflinewidth)}, text depth = 0pt, font = \hphantom{A}] (tlight) at (group c1r1.north east) {0 deletions in TFBS};
	\node[anchor = south west, rotate = -90, draw = black, fill = dark!20, minimum width = \ydistance, shift = {(-.5\pgflinewidth, -.5\pgflinewidth)}, text depth = 0pt, font = \hphantom{A}] (tlight) at (group c1r2.north east) {1 deletion in TFBS};
	\node[anchor = south west, rotate = -90, draw = black, fill = dark!20, minimum width = \ydistance, shift = {(-.5\pgflinewidth, -.5\pgflinewidth)}, text depth = 0pt, font = \hphantom{A}] (tlight) at (group c1r3.north east) {2 deletions in TFBS};
	
	
	%%% subfigure labels
	\subfiglabel{prediction light}
	\subfiglabel{prediction dark}
	
\end{tikzpicture}%
			\caption{%
				\captiontitle{Epistatic interactions between single-nucleotide deletions within or outside of transcription factor binding sites}%
				\subfigtwo{A}{b} Selected single-nucleotide deletion variants of the \segment{B} segment of the \enhancer{AB80}, \enhancer{Cab-1}, and \enhancer{rbcS-E9} enhancers and all possible combinations with two of these deletions were subjected to Plant STARR-seq in tobacco plants grown in normal light/dark cycles \parensubfig{A} or completely in the dark \parensubfig{B} for two days prior to RNA extraction. For each pair of deletions, the expected enhancer strength based on the sum of the effects of the individual deletions (additive model) is plotted against the measured enhancer strength. The color of the points represents the distance between the two deletions in a pair. The plots are separated depending on how many deletions fell within a transcription factor binding site (TFBS) as identified in \cref{fig:PEV_ld_motifs}. The ATTGG, CCAAT, and TGTGG sequences in \enhancer{AB80}, \enhancer{Cab-1}, and \enhancer{rbcS-E9}, respectively, were also considered as transcription factor binding sites for this analysis. 
			}%
			\label{rfig:PEVdouble}
		\end{rfig}

	\fi
	%%% Main figures end

\end{document}