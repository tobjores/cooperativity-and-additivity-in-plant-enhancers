\begin{tikzpicture}
	\pgfmathsetseed{1}

	%%% fragment overview
	\coordinate (overview) at (0, 0);
	
	\begin{scope}
		\setlength{\groupplotsep}{\columnsep}
		
		\begin{hgroupplot}[%
			group position = {anchor = above north west, at = {(overview)}, xshift = \columnsep},
			height = 1.5cm,
			axis lines = none,
			title style = {signal, anchor = south west, at = {(0, 1)}, minimum width = \plotwidth - 1.25\pgflinewidth},
			zero line,
			clip = false,
			legend style = {anchor = south west, at = {(0, 0)}},
			legend image post style = {very thick},
			legend plot pos = left,
			legend cell align = left,
			enlarge y limits = .05
		]{\textwidth + \plotylabelwidth - \columnsep}{3}{}
			
			\nextgroupplot[
				title = \enhancer{AB80} enhancer \segment{B} segment,
				title color = AB80,
				axis limits from table = {rawData/PEV_ld_mutSens_AB80_recap_axes.tsv},
				axis limits = AB80 B,
			]
			
				\addplot[line plot, dark] table [x = position, y = dark] {rawData/PEV_ld_mutSens_AB80_B_lines.tsv};
				\addplot[line plot, light] table [x = position, y = light] {rawData/PEV_ld_mutSens_AB80_B_lines.tsv};
				
				\enhFrag[-1]{168}{23}{AB80}{a}
				\enhFrag[-1]{193}{19}{AB80}{b}
				\enhFrag[0]{205}{19}{AB80}{c}
				\enhFrag[-2]{198}{33}{AB80}{bc}
				\enhFrag[-1]{228}{23}{AB80}{d}
				
				\legend{dark, light}
				
			\nextgroupplot[
				title = \enhancer{Cab-1} enhancer \segment{B} segment,
				title color = Cab-1,
				axis limits from table = {rawData/PEV_ld_mutSens_Cab-1_recap_axes.tsv},
				axis limits = Cab-1 B,
			]
			
				\addplot[line plot, dark] table [x = position, y = dark] {rawData/PEV_ld_mutSens_Cab-1_B_lines.tsv};
				\addplot[line plot, light] table [x = position, y = light] {rawData/PEV_ld_mutSens_Cab-1_B_lines.tsv};
				
				\enhFrag[-1]{119}{21}{Cab-1}{ctrl}
				\enhFrag[-1]{149}{19}{Cab-1}{a}
				\enhFrag[-1]{179}{21}{Cab-1}{b}
				\enhFrag[0]{193}{19}{Cab-1}{c}
				\enhFrag[-2]{184}{35}{Cab-1}{bc}
				\enhFrag[-1]{220}{25}{Cab-1}{d}
				\enhFrag[0]{239}{19}{Cab-1}{e}
				\enhFrag[-2]{229}{41}{Cab-1}{de}
				
				
			\nextgroupplot[
				title = \enhancer{rbcS-E9} enhancer \segment{B} segment,
				title color = rbcS-E9,
				axis limits from table = {rawData/PEV_ld_mutSens_rbcS-E9_recap_axes.tsv},
				axis limits = rbcS-E9 B,
			]
			
				\addplot[draw = none, fill = dark, fill opacity = .33, restrict x to domain = 131:151] table [x = position, y = dark] {rawData/PEV_ld_mutSens_rbcS-E9_B_lines.tsv} -- (151, 0) -| cycle;
				\addplot[draw = none, fill = light, fill opacity = .33, restrict x to domain = 161:185] table [x = position, y = light] {rawData/PEV_ld_mutSens_rbcS-E9_B_lines.tsv} -- (185, 0) -| cycle;
			
				\addplot[line plot, dark] table [x = position, y = dark] {rawData/PEV_ld_mutSens_rbcS-E9_B_lines.tsv};
				\addplot[line plot, light] table [x = position, y = light] {rawData/PEV_ld_mutSens_rbcS-E9_B_lines.tsv};
				
				\node[node font = \figsmall, inner sep = .15em] (AUC) at (156, -1.6) {AUC};
				\draw[thin] (AUC) -- (140, -.4) (AUC) -- (175, -.75);
				
				\enhFrag[0]{109}{17}{rbcS-E9}{a}
				\enhFrag[1]{120}{17}{rbcS-E9}{b}
				\enhFrag[-1]{115}{31}{rbcS-E9}{ab}
				\enhFrag[0]{141}{21}{rbcS-E9}{c}
				\enhFrag[-2]{124}{47}{rbcS-E9}{abc}
				\enhFrag[0]{173}{25}{rbcS-E9}{d}
				\enhFrag[1]{193}{21}{rbcS-E9}{e}
				\enhFrag[-1]{186}{39}{rbcS-E9}{de}
			
		\end{hgroupplot}
	\end{scope}

	\draw[decorate, decoration = {brace, amplitude = .25cm, raise = .25cm}] (rbcS-E9 abc.south -| AB80 a.west) -- (rbcS-E9 b.north -| AB80 a.west) coordinate[pos = .5, xshift = -.5cm] (curly);
	
	\draw[ultra thick, -Latex] (curly) ++(.75\pgflinewidth, 0) -- ++(-.25, 0) arc[start angle = 90, end angle = 270, radius = .8cm] node[pos = .5, anchor = east, align = right] {mix \&\\match} -- ++(.25, 0) coordinate[xshift = \columnsep] (constructs);
	
	\coordinate[yshift = .35cm] (construct 1) at (constructs);
	\coordinate[yshift = -.35cm] (construct 3) at (constructs);
	
	\PEFconstruct{construct 1}
	\coordinate[xshift = \columnsep] (construct 2) at (construct end);
	
	\PEFconstruct[MediumPurple1]{construct 2}
	
	\PEFconstruct[HotPink1]{construct 3}
	\coordinate[xshift = \columnsep] (construct 4) at (construct end);

	\PEFconstruct[MediumOrchid3]{construct 4}
	
	\draw[ultra thick, -Latex] (constructs -| construct end) ++(\columnsep, 0) -- ++(1, 0) coordinate[shift = {(.667cm + \columnsep, .167)}] (leaf);
	\leaf[.33]{leaf}
	
	
	%%% PEF: enhancer strength by number of fragments
	\coordinate[yshift = -\columnsep] (n frags) at (overview |- ORF.south);
	
	\begin{hgroupplot}[%
		group position = {anchor = above north west, at = {(n frags)}, xshift = \plotylabelwidth},
		ylabel = $\log_2$(enhancer strength),
		ytick = {-10, -8, ..., 10},
		axis limits from table = {rawData/PEF_strength_by_frags_axes.tsv},
		group/every plot/.append style = {x grids = false},
		zero line,
	]{\twocolumnwidth}{2}{number of enhancer fragments}
		
		\nextgroupplot[
			title = \light,
			title color = light,
			x tick table = {rawData/PEF_strength_by_frags_light_boxplot.tsv}{sample}
		]
		
			% violin and box plot
			\violinbox[%
				violin color = {light},
				violin shade inverse
			]{rawData/PEF_strength_by_frags_light_boxplot.tsv}{rawData/PEF_strength_by_frags_light_violin.tsv}
				
			% add sample size
			\samplesize{rawData/PEF_strength_by_frags_light_boxplot.tsv}{id}{n}
		
		
		\nextgroupplot[
			title = \dark,
			title color = dark,
			x tick table = {rawData/PEF_strength_by_frags_dark_boxplot.tsv}{sample}
		]
		
			% violin and box plot
			\violinbox[%
				violin color = {dark},
				violin shade color = white
			]{rawData/PEF_strength_by_frags_dark_boxplot.tsv}{rawData/PEF_strength_by_frags_dark_violin.tsv}
				
			% add sample size
			\samplesize{rawData/PEF_strength_by_frags_dark_boxplot.tsv}{id}{n}
		
	\end{hgroupplot}
	
	
	%%% PEF: single fragments vs. DMS AUCs
	\coordinate (AUC) at (n frags -| \textwidth - \twocolumnwidth, 0);

	\begin{hgroupplot}[%
		group position = {anchor = above north west, at = {(AUC)}, xshift = \plotylabelwidth},
		enlargelimits = .15,
		ytick = {-10, -8, ..., 10},
		zero line,
		legend style = {anchor = north east, at = {(1, 1)}},
		legend plot pos = right,
		legend cell align = right,
		scatter/classes = {
			AB80={AB80},
			Cab-1={Cab-1},
			rbcS-E9={rbcS-E9}
		},
		ylabel = {$\log_2$(enhancer strength)}
	]{\twocolumnwidth}{2}{$-$AUC}
	
		\nextgroupplot[
			title = \light,
			title color = light,
			axis limits from table = {rawData/PEF_cor_AUC_light_axes.tsv},
		]
		
			% regression line
			\addplot [draw = none, forget plot] table [x = AUC, y = {create col/linear regression = { y = enrichment}}] {rawData/PEF_cor_AUC_light_points.tsv};
			
			\draw[gray, dashed] (\xmin, \xmin * \pgfplotstableregressiona + \pgfplotstableregressionb) -- (\xmax, \xmax * \pgfplotstableregressiona + \pgfplotstableregressionb);
			
			% scatter plot
			\addplot [
				scatter,
				scatter src = explicit symbolic,
				only marks,
				mark = solido,
				mark size = 1.25
			] table[x = AUC, y = enrichment, meta = enhancer] {rawData/PEF_cor_AUC_light_points.tsv};
			
			% correlation
			\stats{rawData/PEF_cor_AUC_light}
			
			% legend
			\legend{\enhancer{AB80}, \enhancer{Cab-1}, \enhancer{rbcS-E9}}
			
			
		\nextgroupplot[
			title = \dark,
			title color = dark,
			axis limits from table = {rawData/PEF_cor_AUC_dark_axes.tsv},
		]
		
			% regression line
			\addplot [draw = none, forget plot] table [x = AUC, y = {create col/linear regression = { y = enrichment}}] {rawData/PEF_cor_AUC_dark_points.tsv};
			
			\draw[gray, dashed] (\xmin, \xmin * \pgfplotstableregressiona + \pgfplotstableregressionb) -- (\xmax, \xmax * \pgfplotstableregressiona + \pgfplotstableregressionb);
			
			% scatter plot
			\addplot [
				scatter,
				scatter src = explicit symbolic,
				only marks,
				mark = solido,
				mark size = 1.25
			] table[x = AUC, y = enrichment, meta = enhancer] {rawData/PEF_cor_AUC_dark_points.tsv};
			
			% correlation
			\stats{rawData/PEF_cor_AUC_dark}
			
	\end{hgroupplot}
	
	
	%%% schematic of fragments/fragment combinations with different spacing
	% native spacing
	\coordinate[yshift = -\columnsep] (scheme spacing 1) at (n frags |- xlabel.south);
	
	\coordinate[shift = {(\columnsep, -.5)}] (construct bc) at (scheme spacing 1);
	\node[anchor = west] (label bc) at (construct bc) {\textbf{bc} (native spacing):};
	\PEFconstruct*{label bc.east}{AB80/bc}
	
	\draw[line width = \baselineskip, AB80!25, -Triangle Cap] ($(c1)!.5!(c2)$) ++(-2, -.66) coordinate (enh) -- ++(3, 0) node[pos = 0, anchor = west, text = black] {AB80};
	\draw[line width = \baselineskip, AB80, opacity = .25] (enh) ++(1.5, 0) coordinate (b start) -- ++(.58, 0) coordinate (b end) node[pos = .5, text = black, font = \bfseries\vphantom{abcdectrl}, text opacity = 1]  {b};
	\draw[line width = \baselineskip, AB80, opacity = .25] (enh) ++(2.5, 0) coordinate (c end) -- ++(-.58, 0) coordinate (c start) node[pos = .5, text = black, font = \bfseries\vphantom{abcdectrl}, text opacity = 1]  {c};
	
	\draw[thin, line cap = round] (c1) ++(.2, -.125) -- ($(b start) + (0, .5\baselineskip)$) (c2) ++(-.2, -.125) -- ($(c end) + (0, .5\baselineskip)$);
	
	% altered spacing
	\coordinate (scheme spacing 2) at (scheme spacing 1 -| \textwidth - \twocolumnwidth, 0);
	
	\coordinate[xshift = \columnsep] (construct b+c) at (construct bc -| scheme spacing 2);
	\node[anchor = west] (label b+c) at (construct b+c) {\textbf{b+c} (altered spacing):};
	\PEFconstruct*{label b+c.east}{AB80/b,AB80/c}
	
	\draw[line width = \baselineskip, AB80!25, -Triangle Cap] ($(c1)!.5!(c2)$) ++(-2, -.66) coordinate (enh) -- ++(3, 0) node[pos = 0, anchor = west, text = black] {AB80};
	\draw[line width = \baselineskip, AB80, opacity = .25] (enh) ++(1.5, 0) coordinate (b start) -- ++(.58, 0) coordinate (b end) node[pos = .5, text = black, font = \bfseries\vphantom{abcdectrl}, text opacity = 1]  {b};
	\draw[line width = \baselineskip, AB80, opacity = .25] (enh) ++(2.5, 0) coordinate (c end) -- ++(-.58, 0) coordinate (c start) node[pos = .5, text = black, font = \bfseries\vphantom{abcdectrl}, text opacity = 1]  {c};
	
	\draw[thin, line cap = round] (c1) ++(.2, -.125) -- ($(b start) + (0, .5\baselineskip)$) ($(c1)!.5!(c2)$) ++(-.05, -.125) -- ($(b end) + (0, .5\baselineskip)$);
	\draw[thin, line cap = round] ($(c1)!.5!(c2)$) ++(.05, -.125) -- ($(c start) + (0, .5\baselineskip)$) (c2) ++(-.2, -.125) -- ($(c end) + (0, .5\baselineskip)$);
	
	
	%%% Examples of fragments and fragment combinations (light)
	\coordinate[yshift = -\columnsep - .5\baselineskip] (frags light) at (scheme spacing 1 |- enh);
	
	\distance{title.south}{title.north}
	
	\pgfmathsetlength{\templength}{(\textwidth  - .5\threecolumnwidth + \pgflinewidth - 2\plotylabelwidth - 2\groupplotsep - 2\ydistance - 2\columnsep) / 4}
	
	\begin{hgroupplot}[%
		group position = {anchor = above north west, at = {(frags light)}, xshift = \plotylabelwidth},
		axis limits from table = {rawData/PEF_fragments_light_axes.tsv},
		enlarge y limits = .05,
		ylabel = $\log_2$(enhancer strength),
		zero line,
		xticklabel style = {font = \vphantom{abcde}}
	]{3\templength + \plotylabelwidth + 2\groupplotsep}{3}{fragments}
		
		\nextgroupplot[
			title = \enhancer{AB80},
			title color = AB80,
			x tick table = {rawData/PEF_fragments_AB80_bc_light_mean.tsv}{construct},
		]
		
			\hmandp{AB80, mark options = {AB80!50}}{rawData/PEF_fragments_AB80_bc_light};
			
			
		\nextgroupplot[
			title = \enhancer{Cab-1},
			title color = Cab-1,
			x tick table = {rawData/PEF_fragments_Cab-1_bc_light_mean.tsv}{construct},
		]
		
			\hmandp{Cab-1, mark options = {Cab-1!50}}{rawData/PEF_fragments_Cab-1_bc_light};
			
			
		\nextgroupplot[
			title = \enhancer{rbcS-E9},
			title color = rbcS-E9,
			x tick table = {rawData/PEF_fragments_rbcS-E9_ab_light_mean.tsv}{construct},
		]
		
			\hmandp{rbcS-E9, mark options = {rbcS-E9!50}}{rawData/PEF_fragments_rbcS-E9_ab_light};
		
	\end{hgroupplot}
	
	\distance{group c3r1.south}{group c3r1.north}
	
	\node[anchor = south west, rotate = -90, draw = black, fill = light!20, minimum width = \ydistance, shift = {(-.5\pgflinewidth, -.5\pgflinewidth)}, text depth = 0pt, font = \hphantom{A}] (tlight) at (group c3r1.north east) {\light};
	
	
	%%% Examples of fragments and fragment combinations (dark)
	\coordinate[xshift = \columnsep] (frags dark) at (group c3r1.above north -| tlight.north);
	
	\begin{axis}[%
		anchor = above north west,
		at = {(frags dark)},
		xshift = \plotylabelwidth,
		width = \templength,
		axis limits from table = {rawData/PEF_fragments_dark_axes.tsv},
		enlarge y limits = .05,
		ylabel = $\log_2$(enhancer strength),
		zero line,
		xticklabel style = {font = \vphantom{abcde}},
		title = \enhancer{rbcS-E9},
		title color = rbcS-E9,
		title style = {minimum width = \templength},
		x tick table = {rawData/PEF_fragments_rbcS-E9_ab_dark_mean.tsv}{construct},
		xlabel = fragments
	]
		
			\hmandp{rbcS-E9, mark options = {rbcS-E9!50}}{rawData/PEF_fragments_rbcS-E9_ab_dark};
		
	\end{axis}
	
	\distance{last plot.south}{last plot.north}
	
	\node[anchor = south west, rotate = -90, draw = black, fill = dark!20, minimum width = \ydistance, shift = {(-.5\pgflinewidth, -.5\pgflinewidth)}, text depth = 0pt, font = \hphantom{A}] at (last plot.north east) {\dark};


	%%% influence of fragment order on enhancer strength
	\coordinate (order) at (frags light -| \textwidth - .5\threecolumnwidth, 0);
	
	\distance{last plot.south}{last plot.above north}

	\begin{axis}[%
		anchor = above north west,
		at = {(order)},
		xshift = \plotylabelwidth,
		yshift = .2\pgflinewidth,
		width = .5\threecolumnwidth - \plotylabelwidth,
		height = \ydistance - .5\pgflinewidth,
		ymin = -0.4,
		ymax = 4.2,
		ylabel = $\log_2$(enhancer strength),
		zero line,
		xticklabel style = {font = \vphantom{\light\dark}},
		x tick table = {rawData/PEF_order_diff_boxplot.tsv}{sample},
		xlabel = condition
	]
		
		% violin plots
		\violinbox[%
			violin colors from table = {rawData/PEF_order_diff_boxplot.tsv}{sample},
			violin shade = 0
		]{rawData/PEF_order_diff_boxplot.tsv}{rawData/PEF_order_diff_violin.tsv}
		
		% sample size
		\samplesize{rawData/PEF_order_diff_boxplot.tsv}{id}{n}
		
		% p-value
		\signif{rawData/PEF_order_diff_pvalues.tsv}{1}{2}
		
	\end{axis}

	
	%%% subfigure labels
	\subfiglabel{overview}
	\subfiglabel{n frags}
	\subfiglabel{AUC}
	\subfiglabel{scheme spacing 1}
	\subfiglabel{scheme spacing 2}
	\subfiglabel{frags light}
	\subfiglabel{frags dark}
	\subfiglabel{order}
	
\end{tikzpicture}